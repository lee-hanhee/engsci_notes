\subsection{Overview of Part 1}
\begin{summary}
    \begin{itemize}
        \item \textbf{What is RL?} Learning through experience/data to make good decisions under uncertainty.
        \item Decision making is an essential part of intelligence.
        \item So far in the course, we have studied techniques to identify things.
        \item Also, in real life, we have seen a lot of progress in what is called \textbf{perceptual machine learning}, e.g., perceiving faces, cats, and dogs.
        \begin{itemize}
            \item e.g., to perceive faces, cats, dogs, digits, etc.
            \item Perceptual machine learning tries to identify something.
        \end{itemize}
        
        \item In reality, what we are trying to do is to make decisions based on the perception/information we receive.
        \begin{itemize}
            \item So, it's critical to think about how to make ``good'' decisions when it comes to intelligence.
        \end{itemize}
    \end{itemize}
\end{summary}

\subsection{How to make good decision from limited experience/data}
\begin{summary}
    \begin{itemize}
        \item These sorts of questions, particularly when faced with uncertainty, have been studied in depth at least since the 1950s.
        
        \item Reinforcement Learning (RL) builds strongly from theory and ideas starting in the 1950s with Richard Bellman.
        
        \item So, why should we study RL?
        \begin{itemize}
            \item Because understanding how to make good decisions from limited experience when faced with uncertainty is essential for any (artificial) intelligent entity.
            \item Also, because it's cool. It's practical.
        \end{itemize}
    \end{itemize}
\end{summary}

\subsection{Some impressive successes in the last decade}
\begin{summary}
    \begin{itemize}
        \item Board game Go.
        \begin{itemize}
            \item An extremely challenging board game.
            \item In 2016, a team called ``DeepMind,'' by combining Reinforcement Learning (RL) and Monte Carlo Tree Search, built an agent that could defeat the world champion.
        \end{itemize}
        \item Plasma control for fusion science.
        \item Efficient and targeted Covid-19 testing via RL. 
        \item ChatGPT
    \end{itemize}
\end{summary}

\subsection{What does RL generally involve?}
\begin{summary}
    \begin{itemize}
        \item Optimization
        \item Delayed consequences
        \item Exploration 
        \item Generalization
    \end{itemize}

\end{summary}

\subsubsection{Optimization}
\begin{summary}
    \begin{itemize}
        \item To find the "best" way to make decisions.
        \begin{itemize}
            \item Our decisions must yield the best outcomes or at least very good outcomes.
        \end{itemize}
        \item Best outcomes? How can we specify if an outcome is the best outcome?
        \begin{itemize}
            \item How to compare outcomes?
            \begin{itemize}
                \item We do it with an explicit notion of decision utility.
            \end{itemize}
        \end{itemize}
    \end{itemize}
\end{summary}

\subsubsection{Delayed Consequences}
\begin{summary}
    \begin{itemize}
        \item Decisions made now can impact things much later.
        \begin{itemize}
            \item e.g., Consider saving for retirement.
        \end{itemize}
        \item Delayed consequences introduce two challenges:
        \begin{itemize}
            \item \textbf{When planning:} Even when we know how the world works, decisions involve reasoning not only about the immediate benefit of a decision but also its long-term consequences.
            \item \textbf{When learning:} We don't know how the world works. We learn through direct world experience, but temporal credit assignment is challenging.
            \begin{itemize}
                \item You take some action now, and later you receive a good/bad outcome. How do you figure out which of your actions caused that good or bad outcome?
            \end{itemize}
        \end{itemize}
    \end{itemize}
\end{summary}

\subsubsection{Exploration}
\begin{summary}
    \begin{itemize}
        \item We learn from direct experience by interacting with our environment.
        \item You only learn about what you try out.
        \begin{itemize}
            \item We don’t know what would have happened for other decisions.
        \end{itemize}
        \item That's why it's important to sometimes explore alternative actions as it may give you valuable information.
    \end{itemize}
\end{summary}

\subsubsection{Generalization}
\begin{summary}
    \begin{itemize}
        \item Good decisions are learned from past experience.
        \begin{itemize}
            \item We need a mapping from possible states to decisions.
        \end{itemize}
        \item Why not just pre-program a decision policy/mapping?
        \begin{itemize}
            \item Because the number of possible states of the environment can be vast.
            \item From a small set of states that we have seen, we must learn a mapping that generalizes well to states that we have not seen.
        \end{itemize}
    \end{itemize}
\end{summary}

\subsubsection{RL vs. AI planning vs. (un)supervised learning}
\begin{summary}
    \customFigure[0.75]{00_Images/RL.png}{RL}
\end{summary}

\subsection{Overview of Part 2}
\begin{summary}
    \begin{itemize}
        \item Sequential decision process
        \item Observation, history, and state.
        \item Markov Decision Process (MDP).
        \begin{itemize}
            \item What is Markov about MDP? Why is the Markov assumption so common?
        \end{itemize}
        \item Dynamics Model \& Reward Model.
        \begin{itemize}
            \item Transition Graph.
        \end{itemize}
    \end{itemize}
\end{summary}

\subsection{Sequential Decision Making}
\begin{definition}
    \customFigure[0.75]{00_Images/RL1.png}{Sequential Decision Making}
    \begin{itemize}
        \item \textbf{Goal:} Select actions to maximize total expected future reward. 
        \begin{itemize}
            \item May require balancing immediate and long-term rewards.
        \end{itemize}
    \end{itemize}
\end{definition}

\begin{example}
    \customFigure[0.75]{00_Images/RL2.png}{Internet}
    \customFigure[0.75]{00_Images/RL3.png}{Diswasher Robot}
    \begin{itemize}
        \item The agent might exploit the reward system in unintended ways. The robot can break all dishes by throwing them on the floor. 
        \begin{itemize}
            \item This issue is known as Reward Hacking.
        \end{itemize}
    \end{itemize}
\end{example}

\subsubsection{Sequential Decision Making: Agent \& the World (Discrete Time)}
\begin{definition}
    \customFigure[0.75]{00_Images/RL4.png}{Sequential Decision Making}
    At each time step \( t \), the following sequence occurs:
    \begin{enumerate}
        \item \textbf{Agent takes an action} \( a_t \).
        \item \textbf{Given action} \( a_t \), the world updates and emits observation \( o_{t+1} \) and reward \( r_{t+1} \).
        \item \textbf{The agent receives} the new observation \( o_{t+1} \) and reward \( r_{t+1} \).
    \end{enumerate}
\end{definition}

\subsubsection{History: Sequence of Past Observations, Actions, \& Rewards}
\begin{definition}
    \begin{itemize}
        \item \textbf{History}:
        \[
        h_t = (a_0, o_1, r_1, a_1, o_2, r_2, \ldots, a_{t-1}, o_t, r_t)
        \]
        \begin{itemize}
            \item The agent chooses its actions based on the history \( h_t \).
        \end{itemize}
        
        \item \textbf{State}: Information assumed to determine what happens next.
        \begin{itemize}
            \item State \( S_t \) is a function of the history:
            \[
            S_t = \psi(h_t)
            \]
            \begin{itemize}
                \item The simplest state function can be:
                \[
                S_z = h_t \quad \text{or} \quad S_t = o_t
                \]
            \end{itemize}
        \end{itemize}
    \end{itemize}
\end{definition}

\subsubsection{Observation vs. History vs. State}
\begin{definition}
    \begin{itemize}
        \item \textbf{Question}: What is the difference between observation, history, and state?
        \begin{itemize}
            \item \textbf{Observation} (\( o_t \)): The latest snapshot of the environment at time \( t \).
            \item \textbf{History} (\( h_t \)): The entire sequence of past actions, observations, and rewards up to time \( t \).
            \item \textbf{State} (\( S_t \)): A function of history that summarizes the necessary information to predict future outcomes.
        \end{itemize}
    \end{itemize}
\end{definition}

\begin{example}
    \textbf{Atari Game:} Consider the design of observation, history, and state for an Atari game:
    
    \begin{itemize}
        \item \textbf{Observation}: The last image snapshot of the game.
        \item \textbf{State}: The last few observations to infer the speed and direction of moving objects, such as the ball.
        \begin{itemize}
            \item This allows the agent to determine the velocity and trajectory of objects for decision-making.
        \end{itemize}
    \end{itemize}
\end{example}

\subsection{Markov Assumption}
\begin{definition}
    \begin{itemize}
        \item Often, to make problems tractable, and because it is not a terrible assumption in reality, we will make the \textbf{Markov Assumption}.
        \[
        P(S_{t+1} \mid S_t, a_t) = P(S_{t+1} \mid S_t, a_t, S_{t-1}, a_{t-1}, S_{t-2}, a_{t-2}, \ldots, S_1, a_1)
        \]
    
        \item \textbf{Future} (\( S_{t+1} \)) is independent of \textbf{past} (\( S_{t-1}, a_{t-1}, \ldots, S_1, a_1 \)) given the \textbf{present} (\( S_t, a_t \)).
        
        \item With the Markov Assumption: STOPPED HERE.
        \begin{align*}
            P(S_{t+1} \mid S_t, S_{t-1}, \ldots, S_1, a_t, \ldots, a_1, \theta) &= P(S_{t+1} \mid S_t, a_t) \\
            &\Rightarrow P(S_t \mid S_{t-1}, S_{t-2}, \ldots, a_{t-1}) \\
            P(S_t \mid S_{t-1}, \ldots, S_1, a_{t-1}, \ldots, a_1) &= P(S_t \mid S_{t-1}, a_{t-1})
        \end{align*}
    \end{itemize}
\end{definition}

\subsubsection{Why is Markov Assumption Popular?}
\begin{definition}
    \begin{itemize}
        \item It is simple and often can be satisfied if we include some previous observations as part of the state.
        
        \item There are many problems where we can satisfy the Markov assumption by simply setting:
        \[
        S_t = o_t
        \]
    
        \item The Markov assumption significantly simplifies working with the problem.
        
        \item Just like all machine learning problems we studied so far, there is a trade-off between the expressive power of state representation and how long it takes to train your model.
    \end{itemize}    
\end{definition}

\begin{example}
    
\end{example}

\subsection{Dynamics Model \& Reward Model}

\begin{example}
    
\end{example}

\begin{example}
    
\end{example}

\subsubsection{Transition Graph}
\begin{definition}
    \begin{itemize}
        \item A \textbf{Transition Graph} is a useful way of summarizing the dynamics of a finite MDP.
        
        \item There are two types of nodes:
        \begin{itemize}
            \item \textbf{State nodes}:
            \begin{itemize}
                \item There is a state node for each possible state.
                \item Typically represented by a large circle labeled with the name of the state, e.g., \( S \).
            \end{itemize}
            
            \item \textbf{Action nodes} (also called \textbf{q-state nodes}):
            \begin{itemize}
                \item There is an action node for each state-action pair.
                \item Typically represented by a small solid circle labeled with the name of the action or the (state, action) pair, e.g., \( a \) or \( (s, a) \).
            \end{itemize}
        \end{itemize}
    \end{itemize}
\end{definition}
