\subsection{Overview of Part 1}
\begin{summary}
    \begin{itemize}
        \item \textbf{What is RL?} Learning through experience/data to make good decisions under uncertainty.
        \item Decision making is an essential part of intelligence.
        \item So far in the course, we have studied techniques to identify things.
        \item Also, in real life, we have seen a lot of progress in what is called \textbf{perceptual machine learning}, e.g., perceiving faces, cats, and dogs.
        \begin{itemize}
            \item e.g., to perceive faces, cats, dogs, digits, etc.
            \item Perceptual machine learning tries to identify something.
        \end{itemize}
        
        \item In reality, what we are trying to do is to make decisions based on the perception/information we receive.
        \begin{itemize}
            \item So, it's critical to think about how to make ``good'' decisions when it comes to intelligence.
        \end{itemize}
    \end{itemize}
\end{summary}

\subsection{How to make good decision from limited experience/data}
\begin{summary}
    \begin{itemize}
        \item These sorts of questions, particularly when faced with uncertainty, have been studied in depth at least since the 1950s.
        
        \item Reinforcement Learning (RL) builds strongly from theory and ideas starting in the 1950s with Richard Bellman.
        
        \item So, why should we study RL?
        \begin{itemize}
            \item Because understanding how to make good decisions from limited experience when faced with uncertainty is essential for any (artificial) intelligent entity.
            \item Also, because it's cool. It's practical.
        \end{itemize}
    \end{itemize}
\end{summary}

\subsection{Some impressive successes in the last decade}
\begin{summary}
    \begin{itemize}
        \item Board game Go.
        \begin{itemize}
            \item An extremely challenging board game.
            \item In 2016, a team called ``DeepMind,'' by combining Reinforcement Learning (RL) and Monte Carlo Tree Search, built an agent that could defeat the world champion.
        \end{itemize}
        \item Plasma control for fusion science.
        \item Efficient and targeted Covid-19 testing via RL. 
        \item ChatGPT
    \end{itemize}
\end{summary}

\subsection{What does RL generally involve?}
\begin{summary}
    \begin{itemize}
        \item Optimization
        \item Delayed consequences
        \item Exploration 
        \item Generalization
    \end{itemize}

\end{summary}

\subsubsection{Optimization}
\begin{summary}
    \begin{itemize}
        \item To find the "best" way to make decisions in each possible state of the environment.
        \begin{itemize}
            \item Our decisions must yield the best outcomes or at least very good outcomes.
        \end{itemize}
        \item Best outcomes? How can we specify if an outcome is the best outcome?
        \begin{itemize}
            \item How to compare outcomes?
            \begin{itemize}
                \item We do it with an explicit notion of decision utility (objective function that needs to be maximized)
            \end{itemize}
        \end{itemize}
    \end{itemize}
\end{summary}

\subsubsection{Delayed Consequences}
\begin{summary}
    \begin{itemize}
        \item Decisions made now can impact things much later.
        \begin{itemize}
            \item e.g., Consider saving for retirement.
        \end{itemize}
        \item Delayed consequences introduce two challenges:
        \begin{itemize}
            \item \textbf{When planning:} Even when we know how the world works, decisions involve reasoning not only about the immediate benefit of a decision but also its long-term consequences.
            \item \textbf{When learning:} We don't know how the world works. We learn through direct world experience, but temporal credit assignment is challenging.
            \begin{itemize}
                \item You take some action now, and later you receive a good/bad outcome. How do you figure out which of your actions caused that good or bad outcome?
            \end{itemize}
        \end{itemize}
    \end{itemize}
\end{summary}

\subsubsection{Exploration}
\begin{summary}
    \begin{itemize}
        \item We learn from direct experience by interacting with our environment.
        \item You only learn about what you try out.
        \begin{itemize}
            \item We don’t know what would have happened for other decisions.
        \end{itemize}
        \item That's why it's important to sometimes explore alternative actions as it may give you valuable information.
        \item \textbf{Exploitation:} Use the best decision learned so far.
    \end{itemize}
\end{summary}

\subsubsection{Generalization}
\begin{summary}
    \begin{itemize}
        \item Good decisions are learned from past experience.
        \begin{itemize}
            \item We need a mapping from possible states to decisions.
        \end{itemize}
        \item Why not just pre-program a decision policy/mapping?
        \begin{itemize}
            \item Because the number of possible states of the environment can be vast.
            \item From a small set of states that we have seen, we must learn a mapping that generalizes well to states that we have not seen.
        \end{itemize}
    \end{itemize}
\end{summary}

\subsubsection{RL vs. AI planning vs. (un)supervised learning}
\begin{summary}
    \customFigure[0.75]{00_Images/RL.png}{RL}
    \begin{itemize}
        \item \textbf{Supervised Learning}: A type of machine learning where the model is trained on labeled data, using input-output pairs to learn a mapping function for predictions.
        \item \textbf{Unsupervised Learning}: A method of learning where the model discovers patterns, structures, or relationships in data without predefined labels or outputs.
        \item \textbf{AI Planning}: The process of generating a sequence of actions or decisions to achieve a specific goal or solve a problem in a structured environment.
        \item \textbf{Reinforcement Learning}: A learning paradigm where an agent learns to make decisions by interacting with an environment to maximize cumulative rewards over time.
    \end{itemize}
\end{summary}

\subsection{Overview of Part 2}
\begin{summary}
    \begin{itemize}
        \item Sequential decision process
        \item Observation, history, and state.
        \item Markov Decision Process (MDP).
        \begin{itemize}
            \item What is Markov about MDP? Why is the Markov assumption so common?
        \end{itemize}
        \item Dynamics Model \& Reward Model.
        \begin{itemize}
            \item Transition Graph.
        \end{itemize}
    \end{itemize}
\end{summary}

\subsection{Sequential Decision Making}
\begin{definition}
    \customFigure[0.75]{00_Images/RL1.png}{Sequential Decision Making}
    \begin{itemize}
        \item Agent: Agent makes a decision at each time step to impact the environment. 
        \item Observation \& Reward: Based on environment's changes, it will receive observation \& reward (positive, negative, or 0)
            \begin{itemize}
                \item Observation: Latest snapshot of the environment at time $t$.
            \end{itemize}
        \item Goal: Select actions to maximize total expected future reward.
            \begin{itemize}
                \item May require balancing immediate/long-term rewards.
            \end{itemize}
    \end{itemize}
\end{definition}

\begin{example}
    \customFigure[0.75]{00_Images/RL2.png}{Internet}
    \begin{itemize}
        \item The AW Advertising Agent chooses which webpages to display ads on. Then the observation is the view time, and the reward is the click on the advertisement (i.e. money).
    \end{itemize}
    \customFigure[0.75]{00_Images/RL3.png}{Diswasher Robot}
    \begin{itemize}
        \item The robot can move certain joints to place the dishes inside the dishwasher and the observation is the camera image of the kitchen. The reward is +1 if there are no dishes on the counter and 0 otherwise. 
        \item The agent might exploit the reward system in unintended ways. The robot can break all dishes by throwing them on the floor. 
        \begin{itemize}
            \item This issue is known as Reward Hacking.
        \end{itemize}
    \end{itemize}
\end{example}

\subsubsection{Sequential Decision Making: Agent \& the World (Discrete Time)}
\begin{definition}
    \customFigure[0.75]{00_Images/RL4.png}{Sequential Decision Making}
    At each time step \( t \), the following sequence occurs:
    \begin{enumerate}
        \item \textbf{Agent takes an action} \( a_t \).
        \item \textbf{Given action} \( a_t \), the world updates and emits observation \( o_{t+1} \) and reward \( r_{t+1} \).
        \item \textbf{The agent receives} the new observation \( o_{t+1} \) and reward \( r_{t+1} \).
    \end{enumerate}
\end{definition}

\subsubsection{History: Sequence of Past Observations, Actions, \& Rewards}
\begin{definition}
    \begin{itemize}
        \item \textbf{History}:
        \[
        h_t = (a_0, o_1, r_1, a_1, o_2, r_2, \ldots, a_{t-1}, o_t, r_t)
        \]

        \item \textbf{Agent:} The agent chooses its actions based on the history \( h_t \).
        
        \item \textbf{State}: Information assumed to determine what happens next.
        \begin{itemize}
            \item State \( S_t \) is a function of the history:
            \[
            s_t = \psi(h_t)
            \]
            \begin{itemize}
                \item The simplest state function can be:
                \[
                s_t = h_t \quad \text{or} \quad s_t = o_t
                \]
                \item Sometimes the state function is a function of just the last observation, which makes sense in many cases.
                \item The state function can be more complex, e.g., a function of the last few observations.
                \item The state function can be a function of the entire history.
            \end{itemize}
        \end{itemize}
    \end{itemize}
\end{definition}

\subsubsection{Observation vs. History vs. State}
\begin{definition}
    \begin{itemize}
        \item \textbf{Question}: What is the difference between observation, history, and state?
        \begin{itemize}
            \item \textbf{Observation} (\( o_t \)): The latest snapshot of the environment at time \( t \).
            \item \textbf{History} (\( h_t \)): The entire sequence of past actions, observations, and rewards up to time \( t \).
            \item \textbf{State} (\( S_t \)): A function of history that summarizes the necessary information to predict future outcomes.
        \end{itemize}
    \end{itemize}
\end{definition}

\begin{example}
    \textbf{Atari Game:} Consider the design of observation, history, and state for an Atari game:
    
    \begin{itemize}
        \item \textbf{Observation}: The last image snapshot of the game.
        \item \textbf{State}: The last few observations to infer the speed and direction of moving objects, such as the ball.
        \begin{itemize}
            \item This allows the agent to determine the velocity and trajectory of objects for decision-making.
        \end{itemize}
    \end{itemize}
\end{example}

\subsection{Markov Assumption}
\begin{definition}
    \begin{itemize}
        \item Often, to make problems tractable, and because it is not a terrible assumption in reality, we will make the \textbf{Markov Assumption}.
        \[
        P(S_{t+1} \mid S_t, a_t) = P(S_{t+1} \mid S_t, a_t, S_{t-1}, a_{t-1}, S_{t-2}, a_{t-2}, \ldots, S_1, a_1)
        \]
    
        \item \textbf{Future} (\( S_{t+1} \)) is independent of \textbf{past} (\( S_{t-1}, a_{t-1}, \ldots, S_1, a_1 \)) given the \textbf{present} (\( S_t, a_t \)).
        
        \item With the Markov Assumption: STOPPED HERE.
        \begin{align*}
            P(S_{t+1} \mid S_t, S_{t-1}, \ldots, S_1, a_t, \ldots, a_1, \theta) &= P(S_{t+1} \mid S_t, a_t) \\
            &\Rightarrow P(S_t \mid S_{t-1}, S_{t-2}, \ldots, a_{t-1}) \\
            P(S_t \mid S_{t-1}, \ldots, S_1, a_{t-1}, \ldots, a_1) &= P(S_t \mid S_{t-1}, a_{t-1})
        \end{align*}
    \end{itemize}
\end{definition}

\subsubsection{Why is Markov Assumption Popular?}
\begin{definition}
    \begin{itemize}
        \item It is simple and often can be satisfied if we include some previous observations as part of the state.
        
        \item There are many problems where we can satisfy the Markov assumption by simply setting:
        \[
        S_t = o_t
        \]
    
        \item The Markov assumption significantly simplifies working with the problem.
        
        \item Just like all machine learning problems we studied so far, there is a trade-off between the expressive power of state representation and how long it takes to train your model.
    \end{itemize}    
\end{definition}

\begin{example}
    
\end{example}

\subsection{Dynamics Model \& Reward Model}
\begin{definition}
    \begin{itemize}
        \item The reward model can depend only on the outcome, only on the action and current state, or the output, action, and current state. 
        \item The terminal states do not have to be 0, but in the Mars example it is. 
    \end{itemize}
\end{definition}
\begin{example}
    
\end{example}

\begin{example}
    
\end{example}

\subsubsection{Transition Graph}
\begin{definition}
    \begin{itemize}
        \item A \textbf{Transition Graph} is a useful way of summarizing the dynamics of a finite MDP.
        
        \item There are two types of nodes:
        \begin{itemize}
            \item \textbf{State nodes}:
            \begin{itemize}
                \item There is a state node for each possible state.
                \item Typically represented by a large circle labeled with the name of the state, e.g., \( S \).
            \end{itemize}
            
            \item \textbf{Action nodes} (also called \textbf{q-state nodes}):
            \begin{itemize}
                \item There is an action node for each state-action pair.
                \item Typically represented by a small solid circle labeled with the name of the action or the (state, action) pair, e.g., \( a \) or \( (s, a) \).
            \end{itemize}
        \end{itemize}
    \end{itemize}
\end{definition}

\subsubsection{Expected Return}
\begin{warning}
    In reinforcement learning, maximizing the expected return is preferred over maximizing just the return because the return for a given trajectory can vary due to uncertainty or stochasticity in the environment, making it unreliable as a sole optimization target. By focusing on the expected return, the learning process accounts for all possible trajectories and their probabilities, ensuring that the policy is robust and performs well on average rather than optimizing for a single potentially rare or atypical outcome.
\end{warning}

\subsection{Lecture Part 3}
\begin{example}
    Example of Efran the gambler. 
\end{example}

\subsubsection{Expected Return}
\begin{derivation}
    % Slide 1
\section*{Formalizing for General MDPs}
\begin{itemize}
    \item $\bullet$ Let's formalize what we saw in the previous example for general MDP.
    \item $\bullet$ The \textbf{value (utility) of policy} $\pi$ \textbf{at state} $s$, denoted by $V_\pi(s)$, is the \textbf{expected return} if starting at $s$ and \textbf{following policy} $\pi$.
    \item $\bullet$ Starting at state ``$s$'' and following policy ``$\pi$'' can yield to different sequences of (state, action, reward), e.g.,

    \[
    \begin{aligned}
        &\square\ (s, \pi(s), R(s, \pi(s))), (s', \pi(s'), R(s', \pi(s'))), (s'', \pi(s''), R(s'', \pi(s''))), \dots, \\
        &\square\ (s, \pi(s), R(s, \pi(s))), (s', \pi(s'), R(s', \pi(s'))), (s''', \pi(s'''), R(s''', \pi(s'''))), \dots, \\
        &\square\ (s, \pi(s), R(s, \pi(s))), (s''', \pi(s'''), R(s''', \pi(s'''))), (s', \pi(s'), R(s', \pi(s'))), \dots, \\
        &\square\ (s, \pi(s), R(s, \pi(s))), (s'', \pi(s''), R(s'', \pi(s''))), (s', \pi(s'), R(s', \pi(s'))), \dots, \\
        &\dots
    \end{aligned}
    \]

    \item $\bullet$ Each one of the possible episodes happens with a probability.

    \item \textbf{Note:} In the above MDP, we assumed a reward model of type $R(s, a)$ for simplicity.
\end{itemize}

% Slide 2
\section*{Formulating $V_\pi(s)$}
\begin{itemize}
    \item $\bullet$ How can we formulate $V_\pi(s)$?

    \[
    V_\pi(s) = \mathbb{E}_{(s_1, s_2, \dots) | s_0=s} \left[ R(s, \pi(s)) + \gamma R(s_1, \pi(s_1)) + \gamma^2 R(s_2, \pi(s_2)) + \dots \mid (s_0 = s, \pi(s)) \right]
    \]

    Where $s_1, s_2, \dots$ is a random sequence of states yielded by following policy $\pi$. So $s_i \in \mathcal{S}$ for any $i \geq 1$.

    \item $\bullet$ What is the probability of observing the sequence $s_1, s_2, s_3, \dots$ conditioned on $s_0 = s$?

    \[
    P(s_1 \mid s, \pi(s)) \times P(s_2 \mid s_1, \pi(s_1)) \times \dots
    \]

    \item $\bullet$ We can rewrite $V_\pi(s)$ as:

    \[
    V_\pi(s) = \mathbb{E}_{(s_1, s_2, \dots) | s_0=s} \left[ R(s, \pi(s)) + \gamma R(s_1, \pi(s_1)) + \gamma^2 R(s_2, \pi(s_2)) + \dots \mid s_0 = s, \pi(s) \right]
    \]

    \[
    = R(s, \pi(s)) + \gamma \mathbb{E}_{(s_1, s_2, \dots)| s_0=s} \left[ R(s_1, \pi(s_1)) + \gamma R(s_2, \pi(s_2)) + \dots \mid s_0 = s, \pi(s) \right]
    \]
    \begin{itemize}
        \item \textbf{Note:} Can pull out $R(s, \pi(s))$ from the expectation because no randomness as it's given. 
    \end{itemize}
\end{itemize}

% Slide 3
\section*{Law of Total Probability}
\begin{itemize}
    \item $\bullet$ Recall: Law of total probability
    \[
    \mathbb{E}_x[X] = \sum_i P(A_i) \mathbb{E}_x[X \mid A_i],
    \]
    where $\{A_i\}$ is a finite partition of the sample space.

    \item $\bullet$ Recall: Let us rewrite the above recall with an additional condition, so that it is more suitable for us.
    \[
    \mathbb{E}[X \mid B] = \sum_i P(A_i \mid B) \mathbb{E}[X \mid B, A_i].
    \]

    \item $\bullet$ Thus, we can rewrite $V_\pi(s)$ as:

    \begin{align*}
        V_\pi(s) &= R(s, \pi(s)) + \gamma \sum_{s' \in \mathcal{S}} P(s_i = s' \mid s_0 = s, \pi(s)) \\
        &\mathbb{E}_{(s_1, s_2, \dots)| s_1=s',s_0=s} \left[ R(s_1, \pi(s_1)) + \gamma R(s_2, \pi(s_2)) + \dots \mid (s_0 = s, \pi(s_0)),(s_1=s',\pi(s')) \right].
    \end{align*}
    \begin{itemize}
        \item This is using the 2nd last expression. 
    \end{itemize}
\end{itemize}

\section*{Rewriting $V_\pi(s)$}
Using the Markov assumption, we can remove the condition of the past state $s$ from the expectation.

\[
\Rightarrow V_\pi(s) = R(s, \pi(s)) + 
\gamma \sum_{s' \in \mathcal{S}} T(s, \pi(s), s') 
\mathbb{E}_{(s_2, s_3, \dots)|s_1=s'} \left[ R(s', \pi(s')) + \gamma R(s_2, \pi(s_2)) + \dots \mid s_1 = s',\pi(s') \right]
\]

\[
\Rightarrow V_\pi(s) = R(s, \pi(s)) + 
\gamma \sum_{s' \in \mathcal{S}} T(s, \pi(s), s') V_\pi(s')
\]

% Slide 5
\section*{Recursion and Reward Models}
\begin{itemize}
    \item $\bullet$ We found a recursion for $V_\pi(s)$ under the \textbf{dynamics model} $T(s, a, s')$ and \textbf{reward model} $R(s, a)$:
    \[
    V_\pi(s) = R(s, \pi(s)) + \gamma \sum_{s'} T(s, \pi(s), s') V_\pi(s').
    \]

    \item $\bullet$ More generally, we can show that for the dynamics model $T(s, a, s')$ and the \textbf{reward model} $R(s, a, s')$:
    \[
    V_\pi(s) = \sum_{s' \in \mathcal{S}} T(s, \pi(s), s') \left[ R(s, \pi(s), s') + \gamma V_\pi(s') \right].
    \]
\end{itemize}

\end{derivation}