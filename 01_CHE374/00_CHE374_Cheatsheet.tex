\documentclass{article}
\usepackage{style}
\title{CHE374 Cheatsheet}
\author{Hanhee Lee}
\lhead{CHE374}
\rhead{Hanhee Lee}

\begin{document}
\maketitle

\tableofcontents

\listoffigures

\listoftables

\section{Week 1}
    \begin{terminology}
        \textbf{Interest Rate}
        \begin{enumerate}
            \item \(P\): Principle amount
            \item \(F\): Future amount 
            \item \(F_N\) : Future amount in (time unit) \(N\)
            \item \(N\): Number of periods (e.g. years)
            \item \(i\): Interest rate 
            \item \(I\): Total interest amount 
            \item \(r\) Nominal interest rate (usually for 1 year)
            \item \(m\): Number of times compounded (subperiods) per year
            \item \(i_s\): Subperiod interest rate
            \item \(i_e\): Effective interest rate, the equivalent rate if compounded only once per year.
        \end{enumerate}
    \end{terminology}

    \begin{definition}
        \textbf{Interest Rate}
        \begin{equation}
            i = \frac{I}{P}
        \end{equation}
    \end{definition}

    \begin{definition}
        \textbf{Subperiod Interest Rate}
        \begin{equation}
            i_s = \frac{r}{m}
        \end{equation}
    \end{definition}

    \begin{definition}
        \textbf{Effective Interest Rate}
        \begin{equation}
            i_e = (1 + i_s)^m - 1
        \end{equation}
    \end{definition}

    \begin{definition}
        \textbf{Simple Interest} 
        \begin{equation}
            F_N = P(1 + Ni)
        \end{equation}
    \end{definition}

    \begin{definition}
        \textbf{Compound Interest} 
        \begin{equation}
            F_N = P(1 + i)^N
        \end{equation}
    \end{definition}

    \begin{definition}
        \textbf{Compound Interest with Subperiods} 
        \begin{equation}
            F_N = P(1 + i_s)^{m} = P(1 + i_e) % Double check this equation.
        \end{equation}
    \end{definition}

    \begin{definition}
        \textbf{Continuous Compound Interest:} The finite amount of \(i_e\) as the compounding period becomes infinitesimally small.
        \begin{equation}
            i_e = \lim_{m \to \infty} \left(1 + \frac{r}{m}\right)^m - 1 = e^r - 1
        \end{equation}
        \begin{itemize}
            \item \textbf{Note:} \(i_e\) increases as the compounding period decreases.
        \end{itemize}
    \end{definition}

\section{Week 2}
    \subsection{Cash-Flow Diagrams}
        \begin{definition}
            \textbf{Cash-flow Diagrams:} A simple graph that summarizes the \textbf{timing} and \textbf{magnitude} of cash-flows.
            \begin{itemize}
                \item \textbf{X-axis:} Discrete time periods
                \item \textbf{Y-axis (implicit):} Size and direction of cash-flow.
                \item \textbf{Individual cash-flows (arrows):} 
                    \begin{itemize}
                        \item \textbf{Outflow:} Cash out of the system (downward arrow)
                        \item \textbf{Inflow:} Cash into the system (upward arrow)
                    \end{itemize}
            \end{itemize}
        \end{definition}
    
    \subsection{Equivalence Factors}
        
        

\section{Week 3}
    \subsection{Mortgage Terms}
        \begin{terminology}
            \textbf{Mortgage}
            \begin{enumerate}
                \item \textbf{Principle:} The amount of money you borrow to pay for a real property.
                \item \textbf{Down Payment:} The fraction of the cost of the real property that you pay upfront yourself. (Usually 20\%)
                \item \textbf{Loan-to-Value Ratio (LTV):} Ratio of mortgage loan to value of the property.
                \item \textbf{Mortgage Rate:} The interest rate charged on the mortgage. 
                \item \textbf{Amortization Period:} Time horizon for mortgage payment. 
                \item \textbf{Term:} Duration of time where the mortgage rate is fixed. When term ends, re-evaluate how much you still owe, then use new interest rate to calculate monthly payment based on time left in amortization period. 
            \end{enumerate}
        \end{terminology}
    
    \begin{definition}
        \textbf{Net amount owed at end of term}
        \begin{equation}
            Net = P\left(1+i\right)^{t \times N} - A \left(\frac{\left(1+i\right)^{t \times N} - 1}{i}\right)
        \end{equation}
        \begin{itemize}
            \item P (Mortgage principle)
            \item A (Regular mortgage payment (usually per month))
            \item i (Mortgage rate per annum based)
            \item N (Number of payment periods per year)
            \item t (Number of years in term)
        \end{itemize}
    \end{definition}

    \begin{definition}
        \textbf{Net monthly payment}
        \begin{equation}
            A = P(\frac{i}{1 - (1+i)^{-t \times N}})
        \end{equation}
    \end{definition}

\subsection{Bond Terms}
    \begin{terminology}
        \textbf{Bond}
        \begin{enumerate}
            \item \textbf{Bond:} A type of loan where the creditor pays a stated amount at specified intervals for a defined period (Coupon Payments), plus a final amount at a specified date (Face Value).
            \item \textbf{Coupon Rate:} The rate used to calculate coupon payments.
        \end{enumerate}
    \end{terminology}


\section{Week 4}

\section{Week 5}

\section{Week 6}

\section{Week 7}

\section{Week 8}

\section{Week 9}

\section{Week 10}

\section{Week 11}

\section{Week 12}

\section{Week 13}

\end{document}