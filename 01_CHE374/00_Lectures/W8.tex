\subsection{Liquidity ratios}
\begin{definition}

    \textbf{Current Ratio:} Measures the company's ability to meet short-term debt obligations, paying current liabilities with current assets.
    \begin{equation}
        \text{Current Ratio} = \frac{\text{Current Assets}}{\text{Current Liabilities}}
    \end{equation}
    \begin{itemize}
        \item Higher the ratio the more current assets available to pay off current debt.
        \item Numbers below 1 could be a sign of concern.
    \end{itemize}
    \vspace{1em}
    \textbf{Acid Test Ratio:} Shows company's ability to pay off debts if all of them were due immediately.
    \begin{equation}
        \text{Acid-Test Ratio} = \frac{\text{Cash} + \text{Short-term Investments} + \text{Net current receivables}}{\text{Current Liabilities}}
    \end{equation}
\end{definition}

\subsection{Efficiency ratios}
\begin{definition}

    \textbf{Inventory Turnover:} Measure of the number of times the average level of inventory is sold during the year.
    \begin{equation}
        \text{Inventory Turnover} = \frac{\text{Cost of Goods Sold}}{\text{Average Inventory over Period}}
    \end{equation}
    \begin{itemize}
        \item A high number indicates an ability to quickly sell inventory.
    \end{itemize}
    \vspace{1em}

    \textbf{Days' Inventory:} Measures speed at which inventory is sold.
    \begin{equation}
        \text{Days' Inventory} = \frac{\text{Average Inventory}}{\left(\frac{\text{Cost of Goods Sold}}{365}\right)}
    \end{equation}
    \begin{itemize}
        \item 365 is 1 year period.
        \item Lower value indicates more efficient operation.
    \end{itemize}
    \vspace{1em}

    \textbf{Accounts Receivable Turnover:} Measures how quickly a company collects money from its customers; its ability to collect cash from credit customers.
    \begin{equation}
        \text{Accounts Receivable Turnover} = \frac{\text{Net Credit Sales}}{\text{Average Net Accounts Receivable}}
    \end{equation}
    
    Alternatively, can use total sales:

    \begin{equation}
        \text{Accounts Receivable Turnover} = \frac{\text{Total Sales}}{\text{Average Net Accounts Receivables}}
    \end{equation}
    \vspace{1em}

    \textbf{Days' Receivables:} Number of days that an invoice is outstanding before payment is collected.
    \begin{itemize}
        \item Inverse of receivables turnover multiplied by number of days in period being analyzed.
    \end{itemize}

    \begin{equation}
        \text{Days' Receivables} = \frac{\text{Average Receivables}}{\left(\frac{\text{Sales}}{365}\right)}
    \end{equation}

    \begin{itemize}
        \item 365 is 1 year period.
    \end{itemize}
\end{definition}

\subsection{Leverage ratios}
\begin{definition}

    \textbf{Debt Ratio:} Proportion of assets financed with debt.
    \begin{equation}
        \text{Debt Ratio} = \frac{\text{Total Liabilities}}{\text{Total Assets}}
    \end{equation}
    \vspace{1em}
    
    \textbf{Debt to Equity Ratio:}
    \begin{equation}
        \text{Debt Equity Ratio} = \frac{\text{Total Liabilities}}{\text{Total Equity}}
    \end{equation}
    \vspace{1em}
    
    \textbf{Equity Ratio:}
    \begin{equation}
        \text{Equity Ratio} = \frac{\text{Equity}}{\text{Total Assets}}
    \end{equation}
    \vspace{1em}
    
    \textbf{Times Interest Earned:} Measures the number of times that operating income can cover interest expenses.
    \begin{itemize}
        \item Operating income is after operating expense.
    \end{itemize}
    \begin{equation}
        \text{Times Interest Earned} = \frac{\text{Operating Income}}{\text{Interest Expense}}
    \end{equation}

    Alternatively, use earnings before tax and income (EBIT) instead of operating income:

    \begin{equation}
        \text{Times Interest Earned} = \frac{\text{EBIT}}{\text{Interest Expense}}
    \end{equation}
\end{definition}

\subsection{Profitability ratios}
\begin{definition}

    \textbf{Profit Margin:} Percentage of each sales dollar earned as net income.
    \begin{equation}
        \text{Profit Margin} = \frac{\text{Net Income}}{\text{Net Sales}}
    \end{equation}
    \vspace{1em}

    \textbf{Return on Assets (ROA):} Measures how well a company is making money based on all the finance resources committed to the firm.
    \begin{equation}
        \text{ROA (First Form)} = \frac{\text{Net Income}}{\text{Average Assets}}
    \end{equation}

    \begin{equation}
        \text{ROA (Second Form)} = \frac{\text{Net Income} + \text{Interest} \cdot (1 - \text{Tax Rate})}{\text{Average Assets}}
    \end{equation}

    \begin{itemize}
        \item Asset = liabilities + equity
        \item $\text{Tax Rate} = \frac{\text{Income Tax}}{\text{Income before Tax}}$
    \end{itemize}
    \vspace{1em}

    \textbf{Return on Shareholders' Equity (ROE):} Measures how much the company has earned on funds invested by shareholders.
    \begin{equation}
        \text{ROE} = \frac{\text{Net Income}}{\text{Average Equity}}
    \end{equation}
    \vspace{1em}

    \textbf{Earnings Per Share (EPS):} Measures the profitability of a company on a per share basis.
    \begin{equation}
        \text{EPS} = \frac{\text{Net Income}}{\text{Total Shares Outstanding}}
    \end{equation}
\end{definition}

\subsection{Performance ratios}
\begin{definition}

    \textbf{Price to Earnings (P/E):} Relates a company's share price to its EPS.
    \begin{equation}
        P/E = \frac{\text{Share Price}}{\text{EPS}}
    \end{equation}
    \begin{itemize}
        \item High P/E could mean overvaluation or expectations of high growth rates.
        \item Not used for companies with no or negative earnings.
        \item Would expect higher P/E for company with more debt compared to equivalent company with less debt.
    \end{itemize}
    \vspace{1em}
    
    \textbf{Dividend Yield:} Shows how much a company pays out relative to its stock price.
    \begin{equation}
        \text{Dividend Yield} = \frac{\text{Dividend per Share}}{\text{Price per Share}}
    \end{equation}
    \begin{itemize}
        \item Mature and stable companies most likely to pay dividends.
        \item New and high-growth companies more likely to reinvest earnings instead of paying dividends.
    \end{itemize}
    \vspace{1em}
    
    \textbf{Dividend Payout Ratio:}
    \begin{equation}
        \text{Dividend Payout Ratio} = \frac{\text{Dividends}}{\text{Net Income}} = \frac{\text{Dividends per Share}}{\text{EPS}}
    \end{equation}
    \vspace{1em}
    
    \textbf{Market Capitalization:} Total dollar market value of a company's outstanding shares of stock.
    \begin{equation}
        \text{Market Cap} = \text{Price per Share} \times \text{Shares Outstanding}
    \end{equation}
\end{definition}