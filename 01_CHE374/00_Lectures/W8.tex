\begin{summary}
    \begin{itemize}
        \item Understand the purpose of accounting
        \item Know the main financial statements (emphasis on the balance sheet and income statement)
        \item Know the five basic accounts
        \begin{itemize}
            \item Permanent accounts (balance sheet)
            \begin{itemize}
                \item Assets, Liabilities, and Equity
            \end{itemize}
            \item Temporary accounts (income statement)
            \begin{itemize}
                \item Expenses and Revenues
            \end{itemize}
        \end{itemize}
        \item Know how to use double entry accounting
        \begin{itemize}
            \item Credits and debits
        \end{itemize}
        \item Construct an income statement and balance sheet given typical business transactions
        \item Financial ratio analysis
    \end{itemize}    
\end{summary}

\subsection{Purpose of Accounting}
\begin{definition}
    \begin{itemize}
        \item Measure business activity
        \item Financial reporting 
        \begin{itemize}
            \item Internal: Decision makers
            \item External: Investors, Government (tax authorities, regulators)
        \end{itemize}
    \end{itemize}
\end{definition}

\subsubsection{Types of Accounting}
\begin{definition}
    \begin{itemize}
        \item \textbf{Managerial Accounting (Internal Use)}
        \begin{itemize}
            \item \textbf{Users:}
            \begin{itemize}
                \item Internal decision makers / managers
                \item Controllers
            \end{itemize}
            \item \textbf{Method of presentation}
            \begin{itemize}
                \item Company specific
                \item Future oriented (assumptions)
            \end{itemize}
        \end{itemize}
    
        \item \textbf{Financial Accounting (External Use)}
        \begin{itemize}
            \item \textbf{Users}
            \begin{itemize}
                \item Investors / creditors
                \item Government
            \end{itemize}
            \item \textbf{Method of presentation}
            \begin{itemize}
                \item Must conform to standards / regulations (GAAP)
            \end{itemize}
        \end{itemize}
    \end{itemize}    
\end{definition}

\subsubsection{How to organize a business?}
\begin{definition}
    \begin{itemize}
        \item Proprietorship: Single owner with unlimited liability.
        \item Partnership: Two or more owners with unlimited liability
        \item Corporation: Business owned by shareholders with limited liability. 
    \end{itemize}
\end{definition}

\subsubsection{How to do accounting:}
\begin{definition}
    The rules that govern accounting are generally accepted accounting principles (GAAP).
    \begin{itemize}
        \item \textbf{Primary objective:} Provide information useful for making investment (including lending) decisions.
        \item \textbf{Useful information:}
        \begin{itemize}
            \item Relevant 
            \item Reliable 
            \item Comparable
        \end{itemize}
    \end{itemize}
\end{definition}

\subsubsection{Accounting principles}
\begin{definition}
    \begin{enumerate}
        \item Entity concept
        \item Relevance characteristic
        \item Reliability / objectivity principle
        \item Cost principle
        \item Going-concern concept
        \item Stable monetary unit concept
    \end{enumerate}    
\end{definition}

\subsection{Main Financial Statements}
\begin{definition}
    \textbf{Financial Statements:} Financial infromation about a business entity that is prepared in a systematic report format that can be used by decision makers to make decisions. 
    \vspace{1em}

    \textbf{Types}
    \begin{itemize}
        \item Income statement (Provides info of the performance of a business over a period of time)
        \begin{itemize}
            \item \textbf{Note:} Takes us from the BS at the start of the period to the end. 
        \end{itemize}
        \item Balance sheet (Snapshot in time of a summary of business' financial position)
        \item Statement of cash flow.
        \item Statement of retained earnings.
    \end{itemize}
    \customFigure[0.75]{00_Images/BSIS.png}{Balance sheet and income sheet.}
\end{definition}

\subsubsection{Accounting equation}
\begin{definition}
    \customFigure[0.75]{00_Images/TAE.png}{Accounting equation.}
    \begin{itemize}
        \item \textbf{Note:} Assets are paid for through liabilities and equity.
    \end{itemize}
\end{definition}

\subsection{Five Basic Accounts}
\begin{definition}
    \begin{itemize}
        \item Assets
        \item Liabilities
        \item Equity
        \item Revenues (sales the company had)
        \item Expenses (cost inccurred by the company)
    \end{itemize}
    \customFigure[0.75]{00_Images/BSIS1.png}{Balance sheet and income sheet.}
    \begin{itemize}
        \item \textbf{Balance Sheet:} $A = L + E$
        \begin{itemize}
            \item \textbf{BS is Permanent:} B/c not closed in at the end of each period, where they are grouped in assets, liabilities, and equity.
        \end{itemize}
        \item \textbf{Income Sheet:} $\text{Net Income} = R - E$
        \begin{itemize}
            \item \textbf{IS is temporary:} B/c closed in at the end of each period.
        \end{itemize}
    \end{itemize}
\end{definition}

\subsubsection{Components of retained earning}
\begin{definition}
    \customFigure[0.75]{00_Images/CRE.png}{Components of retained earning.}
\end{definition}

\subsubsection{The Account}
\begin{definition}
    Accounting's main summary device (record of changes)
    \begin{itemize}
        \item \textbf{Assets:} Economic resources that benefit the business now and in the future. 
        \begin{itemize}
            \item \textbf{Short term}
            \begin{itemize}
                \item Cash
                \item Accounts receivable
                \item Inventory
                \item Notes receivable
                \item Prepaid expenses
            \end{itemize}
            
            \item \textbf{Long term}
            \begin{itemize}
                \item Land
                \item Buildings
                \item Equipment, furniture, and fixtures
            \end{itemize}
        \end{itemize}
        \item \textbf{Liabilities:} Depts of the company
        \begin{itemize}
            \item \textbf{Short term}
            \begin{itemize}
                \item Bank loan
                \item Notes payable
                \item Accounts payable
                \item Accrued liabilities
                \begin{itemize}
                    \item (for expenses incurred but not paid)
                \end{itemize}
            \end{itemize}
            
            \item \textbf{Long term}
            \begin{itemize}
                \item Long-term liabilities (bonds and mortgages)
            \end{itemize}
        \end{itemize}
        \item \textbf{Equity:} Shareholders' equity is the owners' claim to the assets of a corporation. 
        \begin{itemize}
            \item Contributed capital
            \item Retained earnings
            \begin{itemize}
                \item Increases by $\text{Revenues} - \text{Expenses}$
            \end{itemize}
        \end{itemize}        
    \end{itemize}
\end{definition}

\subsubsection{Transaction}
\begin{definition}
    A transaction is any event that has a financial impact on the business.
    \begin{itemize}
        \item Transactions
        \begin{itemize}
            \item Can be measured reliably
            \item Provide objective information
            \item Must be able to assign \$ amount to the event
        \end{itemize}
    \end{itemize}    
\end{definition}

\subsubsection{The Basis}
\begin{definition}
    \textbf{ACCRUAL Accounting:}
    Transactions are to be recorded when they occur, not when the cash is actually exchanged    
        \begin{itemize}
            \item This means that you may pay tax on moneys you have not yet received, but you can also claim expenses that you have not paid for yet
            \item In both cases, however, there is evidence that something has happened
        \end{itemize}
    \vspace{1em}

    \textbf{CASH BASIS Accounting:}
    If cash is used as the time to recognise a transaction.
    \begin{itemize}
        \item While the former is used in almost all businesses, the latter is useful for small cash based business
        \item Don't confuse this with 'cash flow' however!
    \end{itemize}
\end{definition}

\subsection{Double Entry Accounting}
\begin{definition}
    Double-entry system uses \textbf{debits} and \textbf{credits} to record the dual effects of each business transaction.
\end{definition}

\subsubsection{T-Account}
\begin{definition}
    Records addition and subtraction on different sides of the line.
    \begin{itemize}
        \item \textbf{Left side accounts:} Increase in value is recorded on LS.
        \item \textbf{Right side accounts:} Increase in value is recorded on LS.
        \item \textbf{Double entry accounting:} Debits are a LHS operation and credits are a RHS operation.
        \begin{itemize}
            \item \customFigure[0.75]{00_Images/TACCT.png}{T-account, where credit is RS, and debit is LS}
            \item \textbf{Note:} For every credit, there is a debit. Vice versa.
        \end{itemize}
    \end{itemize}
\end{definition}

\subsubsection{Increases and Decreases in Accounts (LHS vs. RHS Accounts)}
\begin{definition}
    \customFigure[0.75]{00_Images/C1.png}{Increases and decreases with the accounting equation.}
    \begin{itemize}
        \item \textbf{Assets and Expenses:} LHS accounts. 
        \item \textbf{Otherwise:} RHS account.
    \end{itemize}
\end{definition}

\begin{example}
    \customFigure[0.75]{00_Images/E.png}{Example.}
    \begin{itemize}
        \item \textbf{Cash:} Asset account (LHS account), so an increase in value occurs in the LS of the T-account (i.e. debit).
        \item \textbf{Common Shares:} Equity account (RHS account), so an increase in value occurs in the RS of the T-account (i.e. credit)
    \end{itemize}
\end{example}

\subsubsection{Income Statement Accounts: Revenues and Expenses}
\begin{definition}
    Income statement accounts (temporary accounts) are closed-out at the end of a reporting period
    \begin{itemize}
        \item \textbf{Revenues}
        \begin{itemize}
            \item Sales associated with the business
            \item Increase shareholders' equity
            \item Recorded as credits
        \end{itemize}
        \item \textbf{Expenses}
        \begin{itemize}
            \item Costs incurred by the business
            \item Decrease in shareholders' equity
            \item Recorded as debits
        \end{itemize}
    \end{itemize}
\end{definition}

\subsection{How to use double entry accounting and how to construct an income statement and balance sheet?}
\begin{process}
    \begin{enumerate}
        \item Given a set of transactions.
        \item \textbf{Analyze Each Transaction:} 
        \begin{enumerate}
            \item For each transaction, identify the accounts involved.
            \item Determine whether each account is debited or credited. Use the following rules based on account types:
            \begin{itemize}
                \item \textbf{Assets and Expenses:} Increase with debits and decrease with credits.
                \begin{itemize}
                    \item \textbf{Assets:} Accounts receivable (Money owed to a business by its customers for goods or services that have been delivered but not yet paid for)
                    \item \textbf{Expenses:} Interest expense, taxes, dividend payed
                \end{itemize}
                \item \textbf{Liabilities, Equity, and Revenues:} Increase with credits and decrease with debits.
                \begin{itemize}
                    \item \textbf{Liabilities:} Accounts payable (Amount a business owes to its suppliers or vendors for goods or services it has received but not yet paid for), interest payable, taxes payable, long-term dept (e.g. loans).
                    \item \textbf{Equity:} Retained earnings (if from previous balance sheet)
                \end{itemize}
            \end{itemize}
        \end{enumerate}
    
        \item \textbf{Record Transactions in T-Accounts:} 
        \begin{enumerate}
            \item For each transaction, enter the debit and credit amounts into the respective T-accounts.
            \item \textbf{Check:} Verify that each entry in a T-account is balanced (i.e., the total debits equal the total credits).
        \end{enumerate}
    
        \item \textbf{Transfer Balances to the Income Statement}
        \begin{enumerate}
            \item Identify revenue and expense accounts from your T-accounts.
            \item Transfer the balances of these T-accounts to the income statement:
            \begin{itemize}
                \item List revenues at the top and total them.
                \item List expenses below revenues and total them.
            \end{itemize}
            \item \textbf{Operating (Income or Loss):} Subtract total expenses from total revenues.
            \item \textbf{(Income or Loss) Before Taxes:} Subtract interact expense (calculate) from Operating (Income or Loss).
            \item \textbf{Net (Income or Loss):} Subtract provision for taxes (calculate) from (Income or Loss) Before Taxes.
            \item \textbf{(Addition or Subtraction) to Equity:} Subtract dividends from Net (Income or Loss).
        \end{enumerate}
    
        \item \textbf{Transfer Balances to the Balance Sheet}
        \begin{enumerate}
            \item Identify asset, liability, and equity accounts from your T-accounts.
            \item Transfer the balances of these T-accounts to the balance sheet:
            \begin{itemize}
                \item \textbf{Equity:} include \textbf{Retained Earnings}. 
                \begin{itemize}
                    \item To find retained earnings, close out temporary accounts (i.e. revenues and expenses) by 
                    \begin{itemize}
                        \item Crediting expenses and debiting retained earnings.
                        \item Debiting revenues and crediting retained earnings.
                    \end{itemize}
                    \item \textbf{Note:} Matches \textbf{Addition to equity} from the income statement when $0$ retained earnings from previous balance sheets (i.e. new business)
                \end{itemize}
            \end{itemize}
            \item \textbf{Total Assets:} Total current assets + total non-current assets.
            \item \textbf{Total Liabilities:} Total current liabilities + total non-current liabilities.
            \item \textbf{Total equity:} ? + retained earnings.
            \item \textbf{Check:} Assets = Liabilities + Equity is balanced.
        \end{enumerate}
    \end{enumerate}    
\end{process}

\begin{warning}
    \begin{itemize}
        \item For every credit there is a debit and vice versa. 
        \item To increase the value of a LH account we debit. 
        \item To increase the value of a RH account we credit.
        \item \textbf{Note:} Retained earnings match addition to equity when it's a new business with no retained earnings to start off with. 
    \end{itemize}
\end{warning}

\begin{example}
    \customFigure[0.75]{00_Images/T.png}{Transactions.}
    \customFigure[0.75]{00_Images/TACCT1.png}{T-accounts.}
    \customFigure[0.75]{00_Images/BSIS2.png}{Balance sheet and income sheet.}
\end{example}

\begin{example}
    L3
\end{example}

\subsection{Liquidity ratios}
\begin{definition} Ability to pay current liabilities

    \textbf{Current Ratio:} Measures the company's ability to pay current liabilities with current assets.
    \begin{equation}
        \text{Current Ratio} = \frac{\text{Current Assets}}{\text{Current Liabilities}}
    \end{equation}
    \begin{itemize}
        \item Higher the ratio the more current assets available to pay off current debt.
        \item Numbers below 1 could be a sign of concern.
    \end{itemize}
    \vspace{1em}
    \textbf{Acid Test Ratio:} Shows company's ability to pay off all current liabilities if all of them were due immediately.
    \begin{equation}
        \text{Acid-Test Ratio} = \frac{\text{Cash} + \text{Short-term Investments} + \text{Net current receivables}}{\text{Current Liabilities}}
    \end{equation}
\end{definition}

\subsection{Efficiency ratios}
\begin{definition} Ability to sell inventory and collect receivables.

    \textbf{Inventory Turnover:} Measure of the number of times the average level of inventory is sold during the year.
    \begin{equation}
        \text{Inventory Turnover} = \frac{\text{Cost of Goods Sold}}{\text{Average Inventory over Period}}
    \end{equation}
    \begin{itemize}
        \item A high number indicates an ability to quickly sell inventory.
    \end{itemize}
    \vspace{1em}

    \textbf{Days' Inventory:} Measures speed at which inventory is sold.
    \begin{equation}
        \text{Days' Inventory} = \frac{\text{Average Inventory}}{\left(\frac{\text{Cost of Goods Sold}}{365}\right)}
    \end{equation}
    \begin{itemize}
        \item 365 is 1 year period.
        \item Lower value indicates more efficient operation.
        \item \textbf{Note:} Inverse of inventory turnover multiplied by the number of days in the period being analyzed.
    \end{itemize}
    \vspace{1em}

    \textbf{Accounts Receivable Turnover:} Measures a company's ability to collect cash from credit customers.
    \begin{equation}
        \text{Accounts Receivable Turnover} = \frac{\text{Net Credit Sales}}{\text{Average Net Accounts Receivable}}
    \end{equation}
    \begin{itemize}
        \item i.e. how quickly a company collects money from its customers.
    \end{itemize}
    
    Alternatively, can use total sales:

    \begin{equation}
        \text{Accounts Receivable Turnover} = \frac{\text{Total Sales}}{\text{Average Net Accounts Receivables}}
    \end{equation}
    \vspace{1em}

    \textbf{Days' Receivables:} Number of days that an invoice is outstanding before payment is collected.
    \begin{itemize}
        \item Inverse of receivables turnover multiplied by number of days in period being analyzed.
    \end{itemize}
    \begin{itemize}
        \item \textbf{Note:} Inverse of receivables turnover multiplied by the number of days in the period beign analyzed.
    \end{itemize}

    \begin{equation}
        \text{Days' Receivables} = \frac{\text{Average Receivables}}{\left(\frac{\text{Sales}}{365}\right)}
    \end{equation}

    \begin{itemize}
        \item 365 is 1 year period.
    \end{itemize}
\end{definition}

\subsection{Leverage ratios}
\begin{definition} Ability to pay long-term dept.

    \textbf{Debt Ratio:} Proportion of assets financed with debt.
    \begin{equation}
        \text{Debt Ratio} = \frac{\text{Total Liabilities}}{\text{Total Assets}}
    \end{equation}
    \vspace{1em}
    
    \textbf{Debt to Equity Ratio:}
    \begin{equation}
        \text{Debt Equity Ratio} = \frac{\text{Total Liabilities}}{\text{Total Equity}}
    \end{equation}
    \vspace{1em}
    
    \textbf{Equity Ratio:}
    \begin{equation}
        \text{Equity Ratio} = \frac{\text{Equity}}{\text{Total Assets}}
    \end{equation}
    \vspace{1em}
    
    \textbf{Times Interest Earned:} Measures the number of times that operating income can cover interest expenses.
    \begin{itemize}
        \item Operating income is after operating expense.
    \end{itemize}
    \begin{equation}
        \text{Times Interest Earned} = \frac{\text{Operating Income}}{\text{Interest Expense}}
    \end{equation}

    Alternatively, use earnings before tax and income (EBIT) instead of operating income:

    \begin{equation}
        \text{Times Interest Earned} = \frac{\text{EBIT}}{\text{Interest Expense}}
    \end{equation}
\end{definition}

\subsection{Profitability ratios}
\begin{definition}

    \textbf{Profit Margin:} Percentage of each sales dollar earned as net income.
    \begin{equation}
        \text{Profit Margin} = \frac{\text{Net Income}}{\text{Net Sales}}
    \end{equation}
    \vspace{1em}

    \textbf{Return on Assets (ROA):} Measures how profitably the company uses its assets
    \begin{equation}
        \text{ROA (First Form)} = \frac{\text{Net Income}}{\text{Average Assets}}
    \end{equation}

    \begin{equation}
        \text{ROA (Second Form)} = \frac{\text{Net Income} + \text{Interest} \cdot (1 - \text{Tax Rate})}{\text{Average Assets}}
    \end{equation}
    \begin{itemize}
        \item i.e. how well a company is making money based on all the finance resources committed to the firm.
    \end{itemize}

    \begin{itemize}
        \item Asset = liabilities + equity
        \item $\text{Tax Rate} = \frac{\text{Income Tax}}{\text{Income before Tax}}$
    \end{itemize}
    \vspace{1em}

    \textbf{Return on Shareholders' Equity (ROE):} Measures how much the company has earned on funds invested by shareholders (directly and through retained earnings)
    \begin{equation}
        \text{ROE} = \frac{\text{Net Income}}{\text{Equity}}
    \end{equation}
    \vspace{1em}

    \textbf{Earnings Per Share (EPS):} Measures the profitability of a company on a per share basis.
    \begin{equation}
        \text{EPS} = \frac{\text{Net Income}}{\text{Total Shares Outstanding}}
    \end{equation}
\end{definition}

\subsection{Performance ratios}
\begin{definition} Analysis of shares as an investment.

    \textbf{Price to Earnings (P/E):} Relates a company's share price to its EPS.
    \begin{equation}
        P/E = \frac{\text{Share Price}}{\text{EPS}}
    \end{equation}
    \begin{itemize}
        \item High P/E could mean overvaluation or expectations of high growth rates.
        \item Not used for companies with no or negative earnings since there is nothing to put in the denominator.
        \item Leverage will imapct a company's P/E ratio so be careful when comparing companies.
        \item Would expect higher P/E for company with more debt compared to equivalent company with less debt.
    \end{itemize}
    \vspace{1em}
    
    \textbf{Dividend Yield:} Shows how much a company pays out in dividends each year relative to its stock price.
    \begin{equation}
        \text{Dividend Yield} = \frac{\text{Dividend per Share}}{\text{Price per Share}}
    \end{equation}
    \begin{itemize}
        \item Mature and stable companies most likely to pay dividends.
        \item New and high-growth companies more likely to reinvest earnings instead of paying dividends.
    \end{itemize}
    \vspace{1em}
    
    \textbf{Dividend Payout Ratio:}
    \begin{equation}
        \text{Dividend Payout Ratio} = \frac{\text{Dividends}}{\text{Net Income}} = \frac{\text{Dividends per Share}}{\text{EPS}}
    \end{equation}
    \vspace{1em}
    
    \textbf{Market Capitalization:} Total dollar market value of a company's outstanding shares of stock.
    \begin{equation}
        \text{Market Cap} = \text{Price per Share} \times \text{Shares Outstanding}
    \end{equation}
\end{definition}