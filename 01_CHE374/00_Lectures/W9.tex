\subsection{Learning Objectives}
\begin{summary}
    \begin{itemize}
        \item Include the effect of taxes in PW and IRR calculations.
        \begin{itemize}
            \item Calculate corporate taxes in the income statement.
            \item Describe how corporate taxes are affected or unaffected by entries in the income statement and balance sheet.
            \item Approximate the effects of taxation on rate of returns.
            \item Describe how corporate taxes are affected by depreciation and asset sale
        \end{itemize}
        \item Describe the capital cost allowance system for calcuating Canadian taxes.
        \item Calculate PW including tax implications on a year-by-year basis.
        \begin{itemize}
            \item Taxes on revenues
            \item Tax savings from depreciation
            \item Gains/loss froma asset disposition.
        \end{itemize}
        \item Calculate PW including tax implications using the tax-benefit factor
        \begin{itemize}
            \item Define tax benefit factor and other equivalent factors (e.g. capital tax factor, capital salvage factor)
            \item Apply tax benefit factor to calculate PW (using previous example)
        \end{itemize}
    \end{itemize}
\end{summary}

\begin{intuition}
    Find the effect of PW with taxes for 
    \begin{itemize}
        \item First Cost
        \item Revenues
        \item Expenses
        \item Depreciation
        \item Salvage
    \end{itemize}

    \customFigure[0.75]{00_Images/PW.png}{PW with taxes.}
\end{intuition}

\subsection{Corporate Tax (REVENUE AND EXPENSES):}
\begin{definition}
    \textbf{Taxable Income:} Amount of taxes owed depends on profits:
    \begin{equation*}
        \text{Taxable Income} = \text{Revenue} - \text{Expenses}
    \end{equation*}
    \vspace{1em}

    \textbf{Flat Corporate Tax:} Taxes are calculated from taxable income as
    \begin{equation*}
        \text{Corporate Tax} = \text{Taxable Income} \times \text{Tax Rate}
    \end{equation*}
    \begin{itemize}
        \item Paying tax results in a negative cash flow.
        \item Tax savings result in a positive cash flow.
        \item Revenues/gains \textbf{increases} the amount of taxes owed.
        \item Expenses/deductions \textbf{decreases} the amount of taxes owed.
    \end{itemize}  
    
    \customFigure[0.75]{00_Images/PW1.png}{PW with taxes, where it is discounted using after-tax MARR}
\end{definition}

\subsubsection{Types of revenues}
\begin{terminology}
    \begin{itemize}
        \item Sales revenues
        \item Interest revenues (interest earned)
        \item Capital gains (asset sale above cost (i.e. book value)): $\text{Salvage Value} - \text{Book Value}$ (i.e. difference between selling price and book value)
    \end{itemize}
\end{terminology}

\subsubsection{Types of expenses}
\begin{terminology}
    \begin{itemize}
        \item Cost of goods sold (raw materials)
        \item General expenses/SG\&A (rent,salaries)
        \item Interest expenses (loan,debt)
        \item Depreciation expenses
        \item Capital losses (i.e. difference between selling price and book value)
        \item Not included
        \begin{itemize}
            \item Dividends 
            \item Asset purchases 
            \item Others
        \end{itemize}
    \end{itemize}            
\end{terminology}

\subsection{Assets affect taxation through depreciation (FIRST COST)}
\begin{definition}
    \begin{itemize}
        \item Purchasing an asset has no immediate effect on corporate taxes.
        \item But once your company holds an asset, this asset can affect taxes in two ways:
        \begin{itemize}
            \item Claiming depreciation
            \item Capital gains/loss
        \end{itemize}
    \end{itemize}
\end{definition}

\subsection{Effect of taxation on rate of returns}
\begin{definition} \textbf{Modified Rates:} Discounting after-tax cash flow requires lower rate of return
    \begin{equation}
        \text{MARR (after tax)} = \text{MARR (before tax)} \times (1 - \text{tax rate})
    \end{equation}
    \begin{equation}
        \text{IRR (after tax)} = \text{IRR (before tax)} \times (1 - \text{tax rate})
    \end{equation}
    \begin{itemize}
        \item \textbf{Key:} The after tax rate is used after the tax has been applied. 
    \end{itemize}
\end{definition}

\subsection{Capital cost allowance (CCA) (DEPRECIATION)}
\subsubsection{Depreciation}
\begin{intuition}
    \begin{itemize}
        \item Depreciation itself is not a cash flow. 
        \item Claiming depreciation results in tax savings (positive cash flow)
        \item \textbf{Claiming depreciation earins you money.}
    \end{itemize}
\end{intuition}

\subsubsection{CCA}
\begin{terminology}
    \begin{itemize}
        \item CCA: Canadian system for calculating depreciation and taxes.
        \item Classes: Assets are groups in classes by CCA asset class and each class has a CCA rate (depreciation rate).
        \item Terms for CCA
        \customFigure[0.75]{00_Images/TERMS.png}{Book and tax depreciation terms, but focus on the tax depreciation}
    \end{itemize}
\end{terminology}

\subsubsection{2 Important Points}
\begin{definition}
    \begin{enumerate}
        \item \textbf{CCA Pooling:} All assets in a class are pooled together and depreciation expenses are based on the UCC of all assets in that class.
        \begin{itemize}
            \item Assets are not depreciated individually.
        \end{itemize}
        \item \textbf{Half year rule:} (1) Additions in the current year are depreciated at half the CCA rate, (2) carried-over UCC is depreciated at the normal rate.
    \end{enumerate}
\end{definition}

\subsubsection{Key equations}
\begin{definition} 
    \textbf{Capital cost allowance:} Amount depreciated in a given year.
    \begin{equation}
        \text{CCA}_N = \text{CCA Rate} \times \left( \frac{1}{2} \text{ This year's addition} + \text{UCC}_{N-1} \right)
    \end{equation}
    \begin{itemize}
        \item $ \text{UCC}_{N-1} $: UCC from last year
    \end{itemize}
    \vspace{1em}

    \textbf{Undepreciated capital cost (UCC)} Book value in given year.
    \begin{equation}
        \text{UCC}_N = \text{UCC}_{N-1} + \text{This year's addition} - \text{CCA}
    \end{equation}
\end{definition}

\subsubsection{Tax savings from depreciation}
\begin{definition}
    The tax savings for year $N$ is
    \begin{equation}
        \text{Tax savings}_N = \text{CCA}_N \times t 
    \end{equation}
    \begin{itemize}
        \item $t$: Tax rate
        \item \textbf{Note:} (1) Claiming depreciation saves taxes. (2) No out-of-pocket expense from depreciation.
    \end{itemize}

    \customFigure[0.75]{00_Images/FC.png}{Tax savings from depreciation effect.}
\end{definition}

\subsubsection{CCA rules on disposition (selling asset)}
\begin{definition}
    \begin{itemize}
        \item \textbf{Key:} UCC must always be zero after the pool is closed for closed book. 
        \item \textbf{Note:} Loss means that there is a tax savings. 
        \item \textbf{Note:} Gain menas that there is tax losses.
    \end{itemize}
    \customFigure[0.75]{00_Images/RULES.png}{Rule 1}
    \customFigure[0.75]{00_Images/RULES2.png}{Rule 2}
\end{definition}

\begin{example}
    L2 on Open Book, and Two examples of Closed book.
\end{example}

\subsection{Calculating PW with taxes: explicit method FIX ONWARDS}
\begin{process}
    \begin{enumerate}
        \item \textbf{Find After-Tax MARR:}
        \begin{equation}
            \text{(After-tax) MARR} = \text{MARR (before-tax)} \times (1 - \text{tax rate})
        \end{equation}
        \vspace{1em}
    
        \item \textbf{Calculate After-Tax Revenue, Find Present Worth Over Lifespan:}
        \begin{align}
            A &= \text{Revenue} \times (1 - t) \\
            \text{PW}(A) &= A \left( P/A, i\%_{\text{MARR}}, N \right)
        \end{align}
        \vspace{1em}

        \item \textbf{Find FC}
        \vspace{1em}
    
        \item \textbf{Find tax savings from depreciation, discount appropriately to find PW of all tax savings over \(N\) years:}
        \begin{align}
            \text{CCA}_N &= \text{CCA Rate} \times \left( \frac{1}{2} \text{ This year's addition} + \text{UCC}_{N-1} \right) \\
            \text{UCC}_N &= \text{UCC}_{N-1} + \text{This year's addition} - \text{CCA}_N \\
            (P/A, i, N) &= \frac{1}{i} - \frac{1}{i(1+i)^N} = \frac{(1+i)^N - 1}{i(1+i)^N}
        \end{align}
        \vspace{1em}
    
        \item \textbf{Depend on Whether Open or Closed Book:}
        \begin{itemize}
            \item Open book:
            \begin{itemize}
                \item Discount \(S\) to PW.
            \end{itemize}
            \item Closed book:
            \begin{itemize}
                \item Claim/report gain/recapture/loss by finding \(\left| S - \text{BV} \right|\) and calculating taxed or tax savings.
            \end{itemize}
        \end{itemize}
        \vspace{1em}

        \item \textbf{Sum results of steps 1-5:}
        \begin{itemize}
            \item Evaluate like MARR evaluation
        \end{itemize}
    \end{enumerate}
\end{process}

\begin{example}
    L3
\end{example}

\subsection{Calculating PW with taxes: tax benefit factor}
\subsubsection{Tax benefit factor:}
    \begin{definition}
        For every dollar spent, the present worth (PW) of future tax savings is $\tau$ dollars:
        \begin{equation}
            \tau = \frac{\text{PW}(\text{tax savings})}{FC}
        \end{equation}
        \vspace{1em}

        \textbf{Depends on depreciation method:}
        \begin{enumerate}
            \item \textbf{Declining Balance:}
            \begin{equation}
                \tau_{db} = \frac{td}{i + d} \quad \text{(After-tax MARR)}
            \end{equation}
            \begin{itemize}
                \item \( t \): Tax rate 
                \item \( d \): Depreciation rate
                \item \( i \): After-tax MARR
            \end{itemize}
            
            \item \textbf{Declining Balance with Half-Year Rule:}
            \begin{equation}
                \tau_{\frac{1}{2}} = \frac{td}{i + d} \cdot \frac{1 + i/2}{1 + i} 
            \end{equation}
            \begin{itemize}
                \item Applies to CCA asset purchases
            \end{itemize}
        \end{enumerate}
    \end{definition}

\subsubsection{Effective first cost:}
    \begin{definition}
        Reduced first cost due to tax savings:
        \begin{equation}
            \text{PW}(\text{FC}) = - \text{FC} + \text{FC} \times \tau_{1/2} = -\text{FC}(1 - \tau_{1/2})
        \end{equation}
    \end{definition}

\subsubsection{Effective salvage value:}
    \begin{definition}
        Reduction in salvage value due to loss of tax benefits associated with disposition:
        \begin{equation}
            \text{PW}(S) = (S - R \times \tau_{db}) (P/F, i, N)
        \end{equation}
        
        \begin{itemize}
            \item \(S\): Original salvage value 
            \item \(R\): Amount reduced in asset pool
            \begin{itemize}
                \item \(R=S\): Open book 
                \item \(R=UCC\): Closed book
            \end{itemize}
        \end{itemize}
    \end{definition}

\begin{process}
    \begin{enumerate}
        \item \textbf{Find Effective Fixed Cost (FC):}
        \begin{equation}
            \text{PW}(\text{FC}) = - \text{FC} \cdot (1 - \tau_{1/2})
        \end{equation}
    
        \item \textbf{Find After-Tax Revenues:}
        \begin{equation}
            A = \text{Revenue} \cdot (1 - t)
        \end{equation}
        \begin{equation}
            \text{PW}(A) = A \cdot (P/A, i\%_{\text{MARR}}, N)
        \end{equation}
    
        \item \textbf{Find Effective Salvage Value:}
        \begin{itemize}
            \item Open Book:
            \begin{equation}
                \text{PW}(S) = S \cdot (1 - \tau_{db}) \cdot (P/F, i\%, N)
            \end{equation}
            \item Closed Book:
            \begin{equation}
                R = UCC_N, \quad T = \begin{cases} 
                +(BV-S) \cdot t & \text{if losses} \\
                -(S-BV) \cdot t & \text{if gains/recapture}
                \end{cases}
            \end{equation}
            \begin{equation}
                \text{PW}(S) = (S - R \times \tau_{db} + T) \cdot (P/F, i\%, N)
            \end{equation}
            \begin{itemize}
                \item \(T>0\): Losses
                \item \(T<0\): Gains/recapture 
            \end{itemize}
        \end{itemize}
    
        \item \textbf{Sum all values from steps 1--3:}
        \begin{itemize}
            \item Evaluate like MARR evaluation.
        \end{itemize}
    \end{enumerate}
\end{process}