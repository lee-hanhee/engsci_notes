\subsection{Taxable income (FIX)}
\begin{definition}
    \begin{equation}
        \text{Taxable Income} = \text{Revenue} - \text{Expenses}
    \end{equation}
\end{definition}

\subsection{Flat corporate tax}
\begin{definition}
    \begin{equation}
        \text{Corporate Tax Payable} = \text{Taxable Income} \times \text{Tax Rate}
    \end{equation}

    \begin{itemize}
        \item Paying tax results in a negative cash flow.
        \item Tax savings result in a positive cash flow.
    \end{itemize}    
\end{definition}

\subsection{Types of revenue}
\begin{terminology}
    \begin{itemize}
        \item Sales revenues
        \item Interest revenues (interest earned)
        \item Capital gains: $Salvage \: Value - Book \: Value$
    \end{itemize}
\end{terminology}

\subsection{Types of expenses}
\begin{terminology}
    \begin{itemize}
        \item Cost of goods sold (raw materials)
        \item General expenses/SG\&A (salaries)
        \item Interest expenses (debt)
        \item Depreciation expenses
        \item Capital losses
        \item Not included
        \begin{itemize}
            \item Dividends 
            \item Asset purchases 
            \item Others
        \end{itemize}
    \end{itemize}            
\end{terminology}

\subsection{Discounting after-tax cash flow requires lower rate of return}
\begin{definition} \textbf{Modified Rates:}
    \begin{equation}
        \text{(After tax) MARR} = \text{MARR (before tax)} \times (1 - \text{tax rate})
    \end{equation}
    \begin{equation}
        \text{(After tax) IRR} = \text{IRR (before tax)} \times (1 - \text{tax rate})
    \end{equation}
\end{definition}

\subsection{Capital cost allowance (CCA)}
\begin{definition} Amount depreciated in a given year.
    \begin{equation}
        \text{CCA}_N = \text{CCA Rate} \times \left( \frac{1}{2} \text{ This year's addition} + \text{UCC}_{N-1} \right)
    \end{equation}
    \begin{itemize}
        \item $ \text{UCC}_{N-1} $: UCC from last year
    \end{itemize}
\end{definition}

\subsection{Undepreciated capital cost (UCC)}
\begin{definition}
    Book value in given year.
    \begin{equation}
        \text{UCC}_N = \text{UCC}_{N-1} + \text{This year's addition} - \text{CCA}
    \end{equation}
\end{definition}

\subsection{CCA pooling}
\begin{definition}
    All assets in a class are pooled together, with depreciation expenses based on UCC of all assets in that class.
\end{definition}

\subsection{CCA half-year rule}
\begin{definition}
    \begin{itemize}
        \item Additions in the current year are depreciated at half the CCA rate. 
        \item Carried-over UCC is depreciated at the normal rate.
    \end{itemize}
\end{definition}

\subsection{Tax savings from depreciation}
\begin{definition}
    \begin{equation}
        \text{Tax savings} = \text{CCA} \times t 
    \end{equation}
    \begin{itemize}
        \item $t$: Tax rate
    \end{itemize}
\end{definition}

\subsection{CCA rules on disposition (selling asset)}
\begin{definition}
    \begin{itemize}
        \item If other items remain in pool: \textbf{Open Book}
        \begin{itemize}
            \item Pool not closed upon sale of asset.
            \item UCC reduced by sales proceeds (\( S \)). % sales proceed may be wrong.
        \end{itemize}
        \item If no other items in pool: \textbf{Closed Book}
        \begin{itemize}
            \item If \( S < \text{Book Value (BV)} \): \textbf{Terminal loss}, claim \( \text{BV} - S \) as expense, reduce taxable income by loss.
            \item If \( S > \text{BV} \) and \( S < \text{Cost (C)} \): \textbf{Recapture}, report \( S - \text{BV} \) as income, increase taxable income by \( S - \text{BV} \).
            \item If \( S > \text{C} \): \textbf{Capital Gain}.
            \item UCC must always be zero after the pool is closed. % after the pool is closed is different
        \end{itemize}
    \end{itemize}
\end{definition}

\subsection{Calculating PW with taxes: explicit method}
\begin{process}
    \begin{enumerate}
        \item \textbf{Find After-Tax MARR:}
        \begin{equation}
            \text{(After-tax) MARR} = \text{MARR (before-tax)} \times (1 - \text{tax rate})
        \end{equation}
        \vspace{1em}
    
        \item \textbf{Calculate After-Tax Revenue, Find Present Worth Over Lifespan:}
        \begin{align}
            A &= \text{Revenue} \times (1 - t) \\
            \text{PW}(A) &= A \left( P/A, i\%_{\text{MARR}}, N \right)
        \end{align}
        \vspace{1em}

        \item \textbf{Find FC}
        \vspace{1em}
    
        \item \textbf{Find tax savings from depreciation, discount appropriately to find PW of all tax savings over \(N\) years:}
        \begin{align}
            \text{CCA}_N &= \text{CCA Rate} \times \left( \frac{1}{2} \text{ This year's addition} + \text{UCC}_{N-1} \right) \\
            \text{UCC}_N &= \text{UCC}_{N-1} + \text{This year's addition} - \text{CCA}_N \\
            (P/A, i, N) &= \frac{1}{i} - \frac{1}{i(1+i)^N} = \frac{(1+i)^N - 1}{i(1+i)^N}
        \end{align}
        \vspace{1em}
    
        \item \textbf{Depend on Whether Open or Closed Book:}
        \begin{itemize}
            \item Open book:
            \begin{itemize}
                \item Discount \(S\) to PW.
            \end{itemize}
            \item Closed book:
            \begin{itemize}
                \item Claim/report gain/recapture/loss by finding \(\left| S - \text{BV} \right|\) and calculating taxed or tax savings.
            \end{itemize}
        \end{itemize}
        \vspace{1em}

        \item \textbf{Sum results of steps 1-5:}
        \begin{itemize}
            \item Evaluate like MARR evaluation
        \end{itemize}
    \end{enumerate}
\end{process}

\subsection{Calculating PW with taxes: tax benefit factor}
\subsubsection{Tax benefit factor:}
    \begin{definition}
        For every dollar spent, the present worth (PW) of future tax savings is $\tau$ dollars:
        \begin{equation}
            \tau = \frac{\text{PW}(\text{tax savings})}{FC}
        \end{equation}
        \vspace{1em}

        \textbf{Depends on depreciation method:}
        \begin{enumerate}
            \item \textbf{Declining Balance:}
            \begin{equation}
                \tau_{db} = \frac{td}{i + d} \quad \text{(After-tax MARR)}
            \end{equation}
            \begin{itemize}
                \item \( t \): Tax rate 
                \item \( d \): Depreciation rate
                \item \( i \): After-tax MARR
            \end{itemize}
            
            \item \textbf{Declining Balance with Half-Year Rule:}
            \begin{equation}
                \tau_{\frac{1}{2}} = \frac{td}{i + d} \cdot \frac{1 + i/2}{1 + i} 
            \end{equation}
            \begin{itemize}
                \item Applies to CCA asset purchases
            \end{itemize}
        \end{enumerate}
    \end{definition}

\subsubsection{Effective first cost:}
    \begin{definition}
        Reduced first cost due to tax savings:
        \begin{equation}
            \text{PW}(\text{FC}) = - \text{FC} + \text{FC} \times \tau_{1/2} = -\text{FC}(1 - \tau_{1/2})
        \end{equation}
    \end{definition}

\subsubsection{Effective salvage value:}
    \begin{definition}
        Reduction in salvage value due to loss of tax benefits associated with disposition:
        \begin{equation}
            \text{PW}(S) = (S - R \times \tau_{db}) (P/F, i, N)
        \end{equation}
        
        \begin{itemize}
            \item \(S\): Original salvage value 
            \item \(R\): Amount reduced in asset pool
            \begin{itemize}
                \item \(R=S\): Open book 
                \item \(R=UCC\): Closed book
            \end{itemize}
        \end{itemize}
    \end{definition}

\begin{process}
    \begin{enumerate}
        \item \textbf{Find Effective Fixed Cost (FC):}
        \begin{equation}
            \text{PW}(\text{FC}) = - \text{FC} \cdot (1 - \tau_{1/2})
        \end{equation}
    
        \item \textbf{Find After-Tax Revenues:}
        \begin{equation}
            A = \text{Revenue} \cdot (1 - t)
        \end{equation}
        \begin{equation}
            \text{PW}(A) = A \cdot (P/A, i\%_{\text{MARR}}, N)
        \end{equation}
    
        \item \textbf{Find Effective Salvage Value:}
        \begin{itemize}
            \item Open Book:
            \begin{equation}
                \text{PW}(S) = S \cdot (1 - \tau_{db}) \cdot (P/F, i\%, N)
            \end{equation}
            \item Closed Book:
            \begin{equation}
                R = UCC_N, \quad T = \begin{cases} 
                +(BV-S) \cdot t & \text{if losses} \\
                -(S-BV) \cdot t & \text{if gains/recapture}
                \end{cases}
            \end{equation}
            \begin{equation}
                \text{PW}(S) = (S - R \times \tau_{db} + T) \cdot (P/F, i\%, N)
            \end{equation}
            \begin{itemize}
                \item \(T>0\): Losses
                \item \(T<0\): Gains/recapture 
            \end{itemize}
        \end{itemize}
    
        \item \textbf{Sum all values from steps 1--3:}
        \begin{itemize}
            \item Evaluate like MARR evaluation.
        \end{itemize}
    \end{enumerate}
\end{process}