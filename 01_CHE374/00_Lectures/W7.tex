\subsection{Depreciation terms and variables}
\begin{terminology}
    \begin{itemize}
        \item \textbf{Depreciation:}
            \begin{enumerate}
                \item Diminish in value over time.
                \item Reduce the recorded value of an asset over a predetermined period.
                    \begin{itemize}
                        \item Not a cash flow!
                    \end{itemize}
            \end{enumerate}
        
        \item \textbf{Cost Basis:} The value against which depreciation is measured. Usually based on First Cost.
        
        \item \textbf{Market Value:} Actual value of an asset if sold in a free market. Usually cannot be observed until the item is actually sold.
        
        \item \textbf{Book Value:} The value calculated for accounting purposes according to an agreed-upon model.
        
        \item \textbf{$BV_t$:} Book value at time $t$ (end of year)
        
        \begin{equation}
            BV_t = BV_{t-1} - D_t = BV_0 - \sum_{k=1}^{t} D_k
        \end{equation}
        
        \item \textbf{B:} Basis, AKA first cost, original purchase price.
            \[
            B = BV_0
            \]
        
        \item \textbf{S:} Salvage value, AKA selling cost, market value (not always interchangeable).
        
        \item \textbf{$D_t$:} Depreciation in year $t$.
        
        \item \textbf{N:} Depreciable life of the asset. Not necessarily equal to the useful life.
        
        \item \textbf{d:} Proportion of asset value lost to depreciation.
            \begin{itemize}
                \item See Declining Balance Method.
            \end{itemize}
        
        \item \textbf{Loss on Disposal:} One-time additional depreciation value. Accounts for lower salvage value than predicted by depreciation.
        
        \item \textbf{Recaptured Depreciation:} One-time negative depreciation value. Accounts for higher salvage value than predicted by depreciation.
            \begin{itemize}
                \item If higher than cost basis, the difference between market value and salvage value is called capital gain, and the difference between predicted value and cost basis is recaptured depreciation.
            \end{itemize}
    
    \end{itemize}
\end{terminology}

\subsection{Reasons for depreciation}
\begin{definition}

    \textbf{Asset Deterioration}
        \begin{itemize}
            \item \textbf{Use-Related Physical Loss:} As something is used, the more it/its parts wear out. AKA "wear and tear."
            \item \textbf{Time-Related Physical Loss:} Even if not used, things will deteriorate over time due to natural or other factors.
        \end{itemize}
        
    \textbf{Asset Obsolescence}
        \begin{itemize}
            \item \textbf{Functionally-Related Loss:} Loss that occurs without physical changes.
            \begin{itemize}
                \item E.g. Car styles may change, computers become more powerful.
            \end{itemize}
        \end{itemize}
\end{definition}

\subsection{Straight line method}
\begin{definition}
    \begin{equation}
        D_t = \frac{B - S}{N}
    \end{equation}

    \begin{equation}
        BV_t = B - tD_t = B - t\left(\frac{B - S}{N}\right)
    \end{equation}
\end{definition}

\subsection{Declining balance method}
\begin{definition}
    \begin{equation}
        D_t = (BV_{t-1}) \cdot d            
    \end{equation}

    \begin{equation}
        BV_t = BV_{t-1} - (BV_{t-1}) \cdot d = B(1 - d)^t
    \end{equation}

    \textbf{Rate Selection}

    \begin{equation}
        S = B(1 - d)^N
    \end{equation}

    \begin{equation}
        d = 1 - \sqrt[N]{\frac{S}{B}}
    \end{equation}

    \textbf{Double Declining Balance:}
    Double what the straight-line method would have been.

    \begin{equation}
        d = \frac{2}{N}
    \end{equation}
\end{definition}

\subsection{Sum of years' digits (SOYD)}
\begin{definition}
    \begin{itemize}
        \item Faster than straight line during early years, slower than straight line during later years. 
        \item Arbitrarily divides depreciation into chunks.
    \end{itemize}
        
    \begin{equation}
    SOYD = \sum_{k=1}^{N} k
    \end{equation}
    
    \begin{equation}
    D_t = \frac{N - t + 1}{SOYD} \cdot (B - S)
    \end{equation}
    
    \begin{equation}
    BV_t = BV_{t-1} - D_t
    \end{equation}    
\end{definition}

\subsection{Unit of production method}
\begin{definition}
    Assumes depreciation is a function of equipment use rather than time

    \begin{equation}
    D_t = \frac{\text{production in year } t}{\text{lifetime production}} \cdot (B - S)
    \end{equation}

    \begin{equation}
    BV_t = BV_{t-1} - D_t
    \end{equation}
\end{definition}

\subsection{CCA depreciation}
\begin{definition}

    \textbf{Capital Cost Allowance (CCA):} Amount depreciated in a given year
        
    \begin{equation}
        CCA_N = \text{CCA Rate} \times \left(\frac{1}{2} \text{This year's addition} + UCC_{N-1}\right)    
    \end{equation} 
    
    \begin{itemize}
        \item $UCC_{N-1}$: UCC from last year
    \end{itemize}
    
    \textbf{Undepreciated Capital Cost (UCC):} Book value in a given year
    
    \begin{equation}
        UCC_N = UCC_{N-1} + \text{This year's addition} - CCA
    \end{equation}

\end{definition}

\subsection{Re-evaluated service life}
\begin{definition}
    If service life differs from that assumed, re-evaluate service life and depreciate at faster or slower rate from then on. 
    \begin{itemize}
        \item Do not change previous book values
    \end{itemize}
\end{definition}