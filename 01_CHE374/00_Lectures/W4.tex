\subsection{Learning objectives}
\begin{definition}
    \begin{enumerate}
        \item Understand the concept of risk and reward
        \item Understand the conceptual aspects of the Capital Asset Pricing Model (CAPM)
        \item Utilize the CAPM to price assets
        \item Understand arbitrage concepts
        \item Utilize arbitrage arguments to price assets    
    \end{enumerate}
\end{definition}

\subsection{Terms for risk and reward}
\begin{terminology}
    \begin{itemize}
        \item \textbf{Valuation:} Analytical process of determining current (or projected) worth of an asset, which has two components:
        \begin{itemize}
            \item Projected cash-flows
            \item Interest rate - determined by risk (uncertainty).
        \end{itemize}

        \item \textbf{Stock market:} The aggregation of buyers and sellers of stocks (i.e. shares), which represent ownership claims on businesses (i.e. public companies)

        \item \textbf{Financial Risk:} Uncertainty in a future payoff.
        \item \textbf{Pricing risk:} Determining an appropriate interest rate for a given uncertainty of a future cash flow. 

        
        \item \textbf{Variance of Returns:} The ith variance in the rate of return from a vector of return rates of a given ith stock, company, portfolio, etc.
        \[
        \sigma_i^2 = \text{Var}(\vec{R}_i), \quad \vec{R}_i = \begin{bmatrix}
        r_{t_1} \\
        r_{t_2} \\
        \vdots \\
        r_{t_m}
        \end{bmatrix}
        \]
        \begin{itemize}
            \item $P_{t_j} = P_{t_{j-1}} e^{r_{t_j} \Delta t}$: Money at time $j$
            \item $r_{j_j} = \frac{1}{\Delta t} \ln \frac{P_{t_j}}{P_{t_{j-1}}}$: Log returns at time $j$.
        \end{itemize}
        
        \item \textbf{Volatility ($\sigma_i$):} The standard deviation of the variance of return, which is a form of risk (i.e. uncertainty). Greater the volatility, greater the risk.
    \end{itemize}            
\end{terminology}

\subsection{Modern Portfolio Theory}
Primer for the CAPM model:
\begin{intuition}
    \begin{itemize}
        \item Framework to assemble a portfolio of assets such that the expected return ($\mathbb{E}[R]$) is maximized for a given level of risk.
        \item We want to maximize our return while minimizing volatility.
        \item Assume we can proportionally invest our wealth by an amount $x_i$ in the ith asset. 
        \[
        \vec{X} = 
        \begin{bmatrix}
        x_1 \\
        x_2 \\
        \vdots \\
        x_n
        \end{bmatrix},
        \quad
        \Sigma = 
        \begin{bmatrix}
        \sigma_{11} & \sigma_{12} & \cdots & \sigma_{1n} \\
        \sigma_{21} & \sigma_{22} & \cdots & \sigma_{2n} \\
        \vdots & \vdots & \ddots & \vdots \\
        \sigma_{n1} & \sigma_{n2} & \cdots & \sigma_{nn}
        \end{bmatrix}
        \]

        \begin{itemize}
            \item \(\sigma_{ii} = Var(\vec{R}_i)\), \(\sigma_{ij} = \sigma_{ji} = Cov(\vec{R}_i, \vec{R}_j)\)
            \item $X$: Proportions invested in each stock $i$.
        \end{itemize}
    \end{itemize}
\end{intuition}

    \subsubsection{Portfolio Return and Volatility}
    \begin{intuition}
        \textbf{Portfolio expected return}
        \[
        \mathbb{E} \left[ R_p \right] = x_1 \mathbb{E} \left[ R_1 \right] + x_2 \mathbb{E} \left[ R_2 \right] + \cdots + x_n \mathbb{E} \left[ R_n \right]
        \]
        \vspace{1em}

        \textbf{Portfolio volatility}
        \[
        \sigma_p = \sqrt{\vec{X}^T \Sigma \vec{X}}
        \]
        \vspace{1em}

        \textbf{Optimize:} Maximize \(\mathbb{E} \left[ R_p \right] \text{ for a given } \sigma_p \)
        \begin{itemize}
            \item $\mathbb{E} \left[ R_p \right]$: Expected return of the portfolio. 
            \item $\sigma_p$: Portfolio volatility
            \item \textbf{Note:} This leads to efficiency frontier as we are trying to maximum the expected return based on the voltality of these stocks. 
        \end{itemize}
    \end{intuition}

    \subsubsection{Expected return vs. voltality}
    \begin{intuition}
        \customFigure[0.5]{00_Images/ERV.png}{Expected return vs. Voltality}
        \begin{itemize}
            \item \textbf{Efficiency frontier:} Set of portfolios that offer the highest expected return for a given level of risk (standard deviation of returns), illustrating the optimal trade-off between risk and return in portfolio selection.
            \item \textbf{Capital market line:} Tangent to the efficiency frontier and intercepts the risk-free rate.  The CML depicts the risk-return combinations available by mixing the risk-free asset and the market portfolio, showing the optimal portfolios that combine these assets.
            \item \textbf{Market portfolio:} Point on the efficient frontier where the CML is tangent, representing the optimal portfolio of risky assets.
            \item \textbf{Risk-free rate:} Return on an investment with no risk (i.e. no voltality)
        \end{itemize}
    \end{intuition}

    \begin{example}
        If you invested in the MP and risk-free rate expected returns at 50\% each, then you can expect a return that follows CPL in the middle between RF and MP.
    \end{example}

    \begin{example}
        Leveraging:

        \customFigure[0.5]{00_Images/L.png}{Leveraging}
    \end{example}

\subsection{Capital Asset Pricing Model (CAPM)}
\begin{definition}
    \begin{equation}
        \mathbb{E}[R_c] = r_f + \beta_c \left(\mathbb{E}[R_{mp}] - r_f\right)
    \end{equation}
        
    \begin{itemize}
        \item $\mathbb{E}[R_c]$: Expected rate of return for a company
        \item $r_f$: Risk-free rate
        \item $\beta_c$: Measure of risk for the company, systematic risk, related to market risk
        \item $\mathbb{E}[R_{mp}]$: Expected rate of return of the market portfolio, represents the whole market
    \end{itemize}

    \vspace{1em}

    Relationships for $\beta$:
    \begin{equation}
        \beta_i = \frac{\sigma_{i, MP}}{\sigma^2_{MP}} = \rho_{i, MP} \frac{\sigma_i}{\sigma_{MP}}
    \end{equation}
    
    \begin{itemize}
        \item $\sigma_{i, MP}$: Covariance between $i^{th}$ company and market portfolio (MP)
        \item $\sigma^2_{MP}$: Variance of MP
        \item $\rho_{i, MP}$: Correlation of returns between $i^{th}$ company and MP
        \item $\sigma_i$: Volatility of $i^{th}$ company
        \item $\sigma_{MP}$: Volatility of MP
    \end{itemize}

    \customFigure[0.75]{00_Images/CAPM.png}{CAPM Visualization}
\end{definition}

    \subsubsection{Intuition for Beta}
    \begin{intuition}
        \begin{itemize}
            \item As the correlation $\rho_{i, MP}$ increases, $\beta_i$ increases.
            \item If $\rho_{i, MP}=0$, then there is 0 correlation, no systematic risk, so only the risk free rate. 
            \item ith asset voltality increases relative to market voltality. Increasing voltality increases uncertainty so it increase beta. 
        \end{itemize}
    \end{intuition}

    \subsubsection{CAPM Assumptions}
    \begin{intuition}
        \begin{itemize}
            \item Correlations between assets are fixed and constant forever
            \item All investors aim to maximize economic utility (in other words, to make as much money as possible, regardless of any other considerations)
            \item All investors are rational and risk-averse
            \item All investors have access to the same information at the same time
            \item Investors have an accurate conception of possible returns, i.e., the probability beliefs of investors match the true distribution of returns
            \item There are no taxes or transaction costs
            \item All investors are price takers, i.e., their actions do not influence prices
            \item Any investor can lend and borrow an unlimited amount at the risk-free rate of interest
            \item All securities can be divided into parcels of any size
        \end{itemize}
    \end{intuition}

    \subsubsection{Terms for CAPM}
    \begin{terminology}
        \begin{itemize}
            \item \textbf{Market Portfolio (MP):} Portfolio representing the whole market, often estimated through a stock index.
        
            \item \textbf{Systematic Risk:} Risk associated with the market as a whole, e.g. effect of economy on sales and stock value.
            
            \item \textbf{Idiosyncratic Risk:} Risk independent of the economy and specific to a company.
            \begin{itemize}
                \item \textbf{Note:} It is possible to portfolio away idiosyncratic risk by investing in many stocks (i.e. diversifying). Therefore, investors should only be rewarded for market risk as there is no "value" for idiosyncratic risk.
            \end{itemize}
        \end{itemize}
    \end{terminology}

    \begin{example}
        Difference between systematic risk and idiosyncratic risk 

        \begin{itemize}
            \item \textbf{Company return:} \( R_{C,t} = \alpha_C + \beta_C R_{MP,t} + \epsilon_{C,t} \)
            \item \textbf{Volatility in} \( R_{MP,t} \) \textbf{is systematic – "market" risk}
            \item \textbf{Volatility in} \( \epsilon_{C,t} \) \textbf{is idiosyncratic – "firm" specific risk}
        \end{itemize}
    \end{example}

\subsection{Arbitrage concepts}
\begin{terminology}
    \begin{itemize}
        \item \textbf{Financial markets:}
        \begin{itemize}
            \item Arbitrage is the practice of taking advantage of a price difference between two or more markets.
        \end{itemize}
    
        \item \textbf{Academic use}
        \begin{itemize}
            \item Arbitrage occurs when there is an opportunity to achieve a risk-free gain at a rate greater than the risk-free rate.
            \item Fundamental to finance theory is the no-arbitrage assumption.
            \begin{itemize}
                \item "No free lunch."
            \end{itemize}
        \end{itemize}
    
        \item \textbf{Statistical arbitrage}
        \begin{itemize}
            \item Occurs when a gain is achieved at a higher rate than one should for a given level of risk.
        \end{itemize}
    
        \item \textbf{The no-arbitrage concept is fundamental to valuation.}
    \end{itemize}
\end{terminology}

\begin{example}
    \customFigure[0.75]{00_Images/RIZZ.jpeg}{Arbitrage opportunity}
    \begin{itemize}
        \item Since can buy bushels for a net zero cost in the beginning at time zero, we can see how we get a profit of $\$1$ payoff, so as x goes to infinity, we can get infinite profit, which means that increased demand will increase price.
        \item Assuming the forward contract price doesn't change, then the no arbitrage price will be $10.91$
    \end{itemize}
\end{example}

\subsection{Forward vs. Future contracts}
These contracts are used to hedge against risks or to speculate. Both contracts rely on locking in a specific price for a certain asset.
\begin{warning}
    Treat them the same in this course.
\end{warning}

\begin{terminology}
    \begin{itemize}
        \item \textbf{Forward Contract (Forwards):} An obligation to buy or sell a certain asset:
        \begin{itemize}
            \item \textbf{Forward price:} At a specified price 
            \item \textbf{Contract maturity or expiration date:} At a specified time
            \item \textbf{Note:} Not traded on exchanges.
        \end{itemize}
        
        \item \textbf{Futures Contracts:} Similar to forwards except settled daily (not just at maturity), so they can be bought and sold at any time, and are also traded on exchanges.
    \end{itemize}
\end{terminology}

\begin{intuition}
    \customFigure[0.75]{00_Images/DIFF.png}{Difference between forward and future contracts.}
\end{intuition}

\begin{warning}
    For this course, use the concept associated with forward contracts.
\end{warning}

\subsection{Forward Rate}
\begin{definition}
    Given rates for investments between $t=0$ to $t=t_1$, and $t=0$ to $t=t_2$:
    \[
    r_{0,t_1}, \quad r_{0,t_2}
    \]

    \noindent The interest forward rate $t_1$ years from $t=0$ for a duration of $(t_2 - t_1)$ years is:
    
    \begin{equation}
        r_{t_1,t_2} = \frac{r_{0,t_2} \cdot t_2 - r_{0,t_1} \cdot t_1}{t_2 - t_1}
    \end{equation}            
\end{definition}

\begin{derivation}
    \customFigure[0.75]{00_Images/IRF.png}{Interest rate forward derivation.}
\end{derivation}

\begin{example}
    \customFigure[0.75]{00_Images/FEE.png}{Foreign exchange example, where we calculate the foreign exchange rate right now, and a year from now, to see what would be more beneficial.}
\end{example}

\subsection{Replication}
\begin{intuition}
    If we can replicate a portfolio that has the same risk profile of the asset that we are trying to value, then to avoid arbitrage. The value of the portfolio was equal the value of the asset.    
\end{intuition}

\begin{definition}
    \begin{itemize}
        \item Given present market price $P_{MP}$ will take on certain values if it goes up or down: $P_{MP+}$; $P_{MP-}$
            \begin{itemize}
                \item Probability of market going up is $X$, down is $(X-1)$ (interchangeable)
            \end{itemize}
        \item Given project/investment/etc. price $P_I$ will take on certain values depending on if market goes up or down: $P_{I+}$; $P_{I-}$
    \end{itemize}

    \vspace{1em}
    Want to find present value of project:

    \begin{itemize}
        \item Risk-free rate is $r_f$
    \end{itemize}

    \vspace{1em}

    The following networks are Market, $R_f$, and Project:

    \begin{tikzpicture}[->, >=stealth, auto, node distance=2.5cm, thick]

        \node (Market) {$P_{MP}$};
        \node (MarketUp) [above right=of Market] {$P_{MP+}$};
        \node (MarketDown) [below right=of Market] {$P_{MP-}$};
    
        \draw[->] (Market) -- (MarketUp) node[midway, above] {up};
        \draw[->] (Market) -- (MarketDown) node[midway, below] {down};
    
        \node (Rf) [right=4cm of Market] {$P_r = 1$};
        \node (RfPlus) [above right=of Rf] {$1 \times (1+r_f) = P_{r+}$};
        \node (RfMinus) [below right=of Rf] {$1 \times (1+r_f) = P_{r-}$};
        
        \draw[->] (Rf) -- (RfPlus);
        \draw[->] (Rf) -- (RfMinus);
        
        \node (Project) [right=4cm of Rf] {$P_I$};
        \node (ProjectUp) [above right=of Project] {$P_{I+}$};
        \node (ProjectDown) [below right=of Project] {$P_{I-}$};
    
        \draw[->] (Project) -- (ProjectUp) node[midway, above] {up};
        \draw[->] (Project) -- (ProjectDown) node[midway, below] {down};
    
    \end{tikzpicture}

    \begin{itemize}
        \item Replication wants to model a portfolio with the same risk-value as the project:
            \begin{itemize}
                \item $a$: \# of $MP$ shares
                \item $b$: \# of risk-free shares/bonds/etc.
            \end{itemize}
    \end{itemize}
    
    \begin{equation*}
        \begin{aligned}
            &\text{Up:} \quad a P_{MP+} + b P_{r+} = P_{I+} \\
            &\text{Down:} \quad a P_{MP-} + b P_{r-} = P_{I-}
        \end{aligned}
        \quad \text{Solve for } a, b
    \end{equation*}
    
    \[
    P_I = a \times P_{MP} + b \times P_r \quad \text{with} \quad P_r = 1
    \]
    
    \begin{itemize}
        \item Can also find $\beta$:
    \end{itemize}
    
    \[
    E[R_{MP}] = \frac{P_{MP+} X + P_{MP-} (X-1)}{P_{MP}} - 1 
    \]
    
    \[
    E[R_I] = \frac{P_{I+} X + P_{I-} (X-1)}{P_I} - 1
    \]
    
    \[
    \therefore \quad \beta = \frac{E[R_I] - r_f}{E[R_{MP}] - r_f} \text{ from the CAPM}
    \]

\end{definition}

\begin{process}
    \begin{enumerate}
        \item Draw the up and down diagrams for the market portfolio, risk free rate, and project.
        \item If there is probability associated with the up and down diagrams, then calculate the expected payoffs. 
        \item Identify the values for each up and down, where the project current value is usually the unknown. The goal is to replicate the portfolio so we know how to price the project. 
        \item Determine the up and down equations by buying $a$ units of market portfolio and $b$ units of risk-free portfolio with $\$1$ by convention.
        \item Find $a$ and $b$ by solving the system. 
        \begin{itemize}
            \item a and b do not have to be integers and can be negative (i.e. borrowing in the associated asset)
        \end{itemize}
        \item Determine the current project price by buying $a$ units of market portfolio and $b$ units of risk-free portfolio without the up and down. 
        \item To determine $\beta$
        \begin{itemize}
            \item Find the $E[R_{MP}]$ using the above equation, where $X$ denotes the percentage of going up or down in the market.
            \item Find $E[R_I]$ of the project using the above equation
        \end{itemize}
    \end{enumerate}
\end{process}

\begin{example}
    We are trying to replicate a portfolio s.t. the replicated portfolio has the same risk or payoff as the project. 

    \customFigure[0.75]{00_Images/R.png}{Replication example of finding the appropriate project asset value}
    \customFigure[0.75]{00_Images/R1.png}{Finding beta}
    \customFigure[0.75]{00_Images/R4.png}{(Q1) Calculating using the risk free rate and to avoid arbitage, you need to discount by the risk free rate. (Q2) Calculate the expected payoff, then since there isn't a probable state of being up market or down market, use the risk-free rate (since the $\beta$ is 0) to discount the price.}
    \begin{itemize}
        \item \textbf{Note:} If we do not price it at $\$70.48$, then there is an arbitrage opportunity exists. However, the arbitrage would be quickly reverted due to equilibrium in the supply and demand.
    \end{itemize}
\end{example}

\begin{example}
    \customFigure[0.5]{00_Images/R5.png}{The same thing but we have to calculate the expected payoffs.}
    \begin{itemize}
        \item Since we need an initial investment of 20k, then we should buy the project for 28.37k or less.
    \end{itemize}
\end{example}

\begin{example}
    Risk-neural probability is out of scope.
    \customFigure[0.75]{00_Images/R6.png}{Bond example}
\end{example}

