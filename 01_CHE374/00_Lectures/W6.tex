\subsection{Comparison methods terminology}
\begin{terminology}
    \begin{itemize}
        \item \textbf{Internal Rate of Return (IRR):} The discount rate at which the present worth of a project is equal to 0.
            \begin{itemize}
                \item More profitable projects have higher IRR.
            \end{itemize}
        
        \item \textbf{Simple Investment:} When all negative cash flows occur before all positive cash flows.
            \begin{itemize}
                \item Can have multiple IRR when cash flows are not simple.
            \end{itemize}
        
        \item \textbf{Payback Period:} Time it takes for the sum of revenues/savings to equal the initial investment.
        
        \item \textbf{Discounted Payback Period:} Time it takes for the sum of present worth (PW) of revenues/savings to equal the initial investment.
        
        \item \textbf{Incremental IRR:} Evaluates the difference (increment) between two mutually exclusive alternatives.
        
        \item \textbf{De Facto MARR:} The IRR of the project that, when summing the financial commitments (FC) of all projects by highest IRR, is the last to be taken on before exceeding the allotted budget.
    
    \end{itemize}
\end{terminology}

\subsection{IRR calculation}
\begin{process}

    \textbf{Analytical method}
        \begin{enumerate}
            \item Write PW equation for all cash flows using explicit formulas for cash flow factors, leaving discount rate as $i$.
            \item Enter the equation into Desmos, iteratively solve for $i > 0$ that makes PW equal to 0.
            \item If $IRR > MARR$, project is worthwhile.
            \item If $IRR < MARR$, project is not worthwhile.
        \end{enumerate}

    \vspace{1em}
    \textbf{Excel method}
        \begin{enumerate}
            \item Enter cash flows into Excel starting at year 0.
            \item Use IRR function: $=IRR(cashflows, i\% \quad guess)$
            \item If $IRR > MARR$, project is worthwhile.
            \item If $IRR < MARR$, project is not worthwhile.
        \end{enumerate}            
\end{process}

\subsection{Payback period calculation}
\subsubsection{Non-Discounted}
    \begin{definition}
        \begin{enumerate}
            \item Find the year when the sum of revenues/savings equals the initial investment.
            \item If between periods, interpolate:
            
            \begin{equation}
                \frac{N - y_1}{y_2 - y_1} = \frac{FC - c_1}{c_2 - c_1} \Rightarrow N = (y_2 - y_1) \frac{FC - c_1}{c_2 - c_1} + Y_1
            \end{equation}
        \end{enumerate}

        \begin{itemize}
            \item $N$: Payback period (a decimal value when interpolating)
            \item $FC$: First cost/initial investment (positive)
            \item $y_1$: Lower bound of the period of interpolation $(y_1 \leq y_2)$
            \item $y_2$: Upper bound of the period of interpolation
            \item $c_1$: Cumulative sum of revenues/savings at period $y_1 (c_1 \leq c_2)$
            \item $c_2$: Cumulative sum of revenues/savings at period $y_2$
        \end{itemize}
    \end{definition}

\subsubsection{Discounted}
    \begin{definition}
        \begin{enumerate}
            \item Discount revenue at year $y_N$ by $(P/F, i, N)$.
            \item Add discounted revenue to cumulative discounted revenue of the previous year; the sum is the cumulative discounted revenue in year $y_N$.
            \item Visually identify the discounted payback period or interpolate as seen in the Non-Discounted section.
        \end{enumerate}
    \end{definition}

\subsection{Incremental IRR}
\begin{process}
    \begin{enumerate}
        \item Order alternatives in increasing order of $FC$ (First cost).
        
        \item Start with the "do-nothing" alternative.
        
        \item Using $\Delta FC$ and $\Delta A$ between the current choice and the option being evaluated as the $FC$ and $A$, if the resultant $IRR > MARR$, switch to that alternative as the reference.
        
        \item Repeat step 3 for the rest of the options.
        
        \item Final choice is the most profitable project.
    \end{enumerate}
\end{process}