\subsection{Learning Objectives}
\begin{definition}
    \begin{itemize}
        \item Calculate the internal rate of return of a project given discrete cash flows. 
        \begin{itemize}
            \item Define internal rate of return. 
            \item Estimate IRR by engineering factors and calculate IRR by Excel
            \item Use IRR to evaluate independent projects. 
            \item Recognize limitations of IRR. 
        \end{itemize}
        \item Compare different alternatives based on incremental internal rate of return
        \begin{itemize}
            \item IRR does not work to compare mutually exclusive options
            \item Use incremental analysis for mutually exclusive options. 
        \end{itemize}
        \item Calculate the payback period and discounted payback period of a project. 
        \item Recognize pros and cons of the methods. 
    \end{itemize}
\end{definition}

\subsection{IRR}
\begin{intuition}

    \textbf{Motivation:} Cash flow equivalence methods (i.e. PW and AW) require us to know the discount rate before we start.
    \begin{itemize}
        \item Rate of return analysis instead starts by askign what is the rate of return on an investment. 
    \end{itemize}
\end{intuition}

\begin{definition}
    \textbf{Internal Rate of Return (IRR):} The discount rate at which the present worth of a project is equal to 0.
            \begin{itemize}
                \item \textbf{i.e.} $\text{PW}(\text{benefits}) = \text{PW}(\text{costs})$ or $\text{AW}(\text{benefits}) = \text{AW}(\text{costs})$
                \item \textbf{Note:} More profitable projects have higher IRR.
                \begin{itemize}
                    \item \textbf{Intuition:} Discount rate needs to be higher for $\text{PW}(\text{benefits}) = \text{PW}(\text{costs})$.
                \end{itemize}
            \end{itemize}
\end{definition}

\subsection{Simple investment}
\begin{definition}

    When all negative cash flows occur before all positive cash flows.
        \begin{itemize}
            \item \textbf{Note:} For simple investments (costs earlier than benefits), higher $i$ gives lower $PW$.
            \item Can have multiple IRR when cash flows are not simple.
        \end{itemize}
\end{definition}

\subsubsection{Not simple investment (limitation of IRR)}
\begin{definition}
    When cash flows change signs more than once, calculation of IRR becomes complicated because there can be multiple discount rates that make $PW=0$.
    \customFigure[0.75]{00_Images/IRR1.png}{IRR}
\end{definition}

\subsection{Evaluating Independent Projects using IRR}
\begin{definition}
    Assuming simple investments, then 
    \begin{itemize}
        \item If $IRR > MARR$, project is worthwhile (i.e. $PW>0$)
        \item If $IRR < MARR$, project is not worthwhile (i.e. $PW<0$)
    \end{itemize}

    \customFigure[0.75]{00_Images/IRR.png}{IRR}
\end{definition}

\subsubsection{Different methods}
\begin{process}

    \textbf{Analytical/Interpolation method}
        \begin{enumerate}
            \item Draw cash flow diagram
            \item 1st method
            \begin{enumerate}
                \item Write PW equation for all cash flows using explicit formulas for cash flow factors, leaving discount rate as $i$.
                \item Enter the equation into Desmos, iteratively solve for $i > 0$ that makes PW equal to 0.
            \end{enumerate}
            \item 2nd method
            \begin{enumerate}
                \item Find upper and lower bound for $i$ s.t. $PW=0$. 
                \item Or use Casio 991 trick for solve interpolations. 
            \end{enumerate}
            \item If $IRR > MARR$, project is worthwhile.
            \item If $IRR < MARR$, project is not worthwhile.
        \end{enumerate}

    \vspace{1em}
    \textbf{Excel method}
        \begin{enumerate}
            \item Enter cash flows into Excel starting at year 0.
            \item Use IRR function: $=IRR(\text{cashflows}, i\% \text{ guess})$
            \item If $IRR > MARR$, project is worthwhile.
            \item If $IRR < MARR$, project is not worthwhile.
        \end{enumerate}            
\end{process}

\subsection{Payback period calculation}
\begin{terminology}
    \begin{itemize}

        \item \textbf{Payback Period:} Time it takes for the sum of revenues/savings to equal the initial investment.
        \begin{itemize}
            \item i.e. when will I make my money back?
            \item \textbf{Note:} Not sound method in Engineering because fails to account for profitability of the project after investmnet is recovered and only focuses on the part that recovers the cost. 
        \end{itemize}
        
        \item \textbf{Discounted Payback Period:} Time it takes for the sum of present worth (PW) of revenues/savings to equal the initial investment.
    \end{itemize}
\end{terminology}

\subsubsection{Non-Discounted}
    \begin{process}
        \begin{enumerate}
            \item Find the year when the sum of revenues/savings equals the initial investment.
            \item If between periods, interpolate:
            
            \begin{equation}
                \frac{N - y_1}{y_2 - y_1} = \frac{FC - c_1}{c_2 - c_1} \Rightarrow N = (y_2 - y_1) \frac{FC - c_1}{c_2 - c_1} + y_1
            \end{equation}
        \end{enumerate}

        \begin{itemize}
            \item $N$: Payback period (a decimal value when interpolating)
            \item $FC$: First cost/initial investment (positive)
            \item $y_1$: Lower bound of the period of interpolation $(y_1 \leq y_2)$
            \item $y_2$: Upper bound of the period of interpolation
            \item $c_1$: Cumulative sum of revenues/savings at period $y_1 (c_1 \leq c_2)$
            \item $c_2$: Cumulative sum of revenues/savings at period $y_2$
        \end{itemize}
    \end{process}

    \begin{example}
        \customFigure[0.75]{00_Images/PBP.png}{Payback period.}
    \end{example}

\subsubsection{Discounted}
    \begin{process}
        \begin{enumerate}
            \item Discount revenue at year $y_N$ by $(P/F, i, N)$.
            \item Add discounted revenue to cumulative discounted revenue of the previous year; the sum is the cumulative discounted revenue in year $y_N$.
            \item Visually identify the discounted payback period or interpolate as seen in the Non-Discounted section.
        \end{enumerate}
    \end{process}

    \begin{example}
        \customFigure[0.75]{00_Images/DPB.png}{Discounting Payback period.}
    \end{example}

\subsection{Incremental IRR}
\begin{terminology}
    \begin{itemize}
        \item \textbf{Incremental IRR:} Evaluates the difference (increment) between two mutually exclusive alternatives.
        \begin{itemize}
            \item \textbf{Motivation:} Since IRR is an efficiency (rate) of earning money, but doesn't take into account the initial size of the investmnet, we have to use incremental IRR.
        \end{itemize}
    \end{itemize}
\end{terminology}

\subsubsection{IRR to Compare Mutually Exclusive Options}
\begin{process}
    \begin{enumerate}
        \item Order alternatives in increasing order of $FC$ (First cost).
        
        \item Start with the "do-nothing" alternative.
        
        \item Using $\Delta FC$ and $\Delta A$ between the current choice and the option being evaluated as the $FC$ and $A$, if the resultant $IRR > MARR$, switch to that alternative as the reference.
        
        \item Repeat step 3 for the rest of the options.
        
        \item Final choice is the most profitable project.
    \end{enumerate}
\end{process}

\begin{warning}
    Simply comparing IRRs of mutually exclusive options may lead you to the wrong conclusion. 
\end{warning}

\begin{example}
    \customFigure[0.75]{00_Images/ME.png}{IRR for Mutually Exclusive Example}
\end{example}

\subsection{De Facto Marr}
\begin{definition}
    \textbf{De Facto MARR}: The IRR of the project that, when summing the FC of all projects by highest IRR, is the last to be taken on before exceeding the allotted budget.
\end{definition}

\subsection{Table analysis:}
\begin{example}
    \customFigure[0.75]{00_Images/TA.png}{Table analysis.}
\end{example}