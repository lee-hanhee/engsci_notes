\subsection{Terminology}
\begin{terminology}
    \begin{itemize}
        \item \textbf{Independent:} Expected costs and benefits of each project do not depend on whether or not the other one is chosen.
        
        \item \textbf{Mutually Exclusive:} When one project is chosen, all the others are excluded.
        
        \item \textbf{Related but not Mutually Exclusive:}
            \begin{itemize}
                \item Not mutually exclusive: You can select more than one (budget permitting).
                \item Related: Selecting one may affect the selection of another option.
            \end{itemize}
        
        \item \textbf{MARR:} Minimum acceptable rate of return/hurdle rate
            \begin{itemize}
                \item The "do nothing" option, i.e. the rate of return if you were to not invest in a project.
                \item Type of discount rate.
            \end{itemize}
        
        \item \textbf{Present Worth (PW):} Present value of benefits minus costs, discounted at MARR.
            \begin{itemize}
                \item The amount by which a project is beating the best alternative expressed in today's value.
                \item $PW > 0$: Acceptable
                \item $PW < 0$: Unacceptable
            \end{itemize}
        
        \item \textbf{Annual Worth (AW):} The equivalent annuity of PW, with MARR as discount rate.
            \begin{itemize}
                \item $AW > 0$: Acceptable
                \item $AW < 0$: Unacceptable
            \end{itemize}
    \end{itemize}            
\end{terminology}

\subsection{Evaluating mutually exclusive projects}
\begin{process}
    \begin{enumerate}
        \item Define the time horizon.
        
        \item Develop cash flows for each alternative.
        
        \item Calculate the PW using MARR.
        
        \item Compare the PWs and pick the best.
            \begin{itemize}
                \item Higher PW is better.
            \end{itemize}
    \end{enumerate}
\end{process}

\subsection{Comparing different lives}
\begin{example}
    \subsubsection{Repeated lives - PW}
        \begin{itemize}
            \item Assume project repeats itself, and use least common multiple as time horizon. 
            \begin{itemize}
                \item Repeats with same cash flow.
            \end{itemize}
            \item Compare PW at end of time horizon
        \end{itemize}

    \subsubsection{Repeated lives - AW}
        \begin{itemize}
            \item Compare equivalent annuity of PW for individual lives.
            \begin{itemize}
                \item AW is equivalent annuity, so magnitude of annuity stays the same when the project is repeated.
            \end{itemize}
        \end{itemize}
        
    \subsubsection{Study period}
        \begin{itemize}
            \item Specify a time period for comparison. 
            \item For projects that last longer than study period, assume can terminate them early and adjust salvage value if necessary.
            \begin{itemize}
                \item Calculate PW for new period of affected projects. 
            \end{itemize}
            \item Uses fixed time horizon. 
        \end{itemize}
\end{example}