\subsection{Learning objectives}
\begin{definition}
    \begin{itemize}
        \item Understand the relationship between projects: independent, mutually exclusive and related but not mutually exclusive
        \item Construct mutually exclusive alternatives from combination of related projects
        \item Calculate the present worth and annual worth of a project given discrete cash flows
        \begin{itemize}
            \item Describe the discount rate used for cash flows (MARR)
            \item Calculate PW and AW
            \item Evaluate independent projects using PW or AW.
            \item Evaluate mutually exclusive projects using 
            \begin{itemize}
                \item PW with repeated lives
                \item PW with study period
                \item AW
            \end{itemize}
        \end{itemize}
    \end{itemize}
\end{definition}

\subsection{Background}
\begin{intuition}
    \begin{itemize}
        \item \textbf{Projects:} Exchange of resources for expected benefits in the future. 
        \item \textbf{Objective:} Evaluate and compare projects when there are multiple solutions or opportunities presented. 
    \end{itemize}
\end{intuition}

\subsection{Project relationships}
\begin{definition}
    \begin{itemize}
        \item \textbf{Independent:} Expected costs and benefits of each project \textbf{do not depend} on whether or not the other one is chosen.
        \item \textbf{Mutually Exclusive:} When one project is chosen, all the others are excluded.
        \item \textbf{Related but not mutually exclusive:}
        \begin{itemize}
            \item \textbf{Related:} Selecting one may affect selection of another option.
            \item \textbf{Not mutually exclusive:} You can select more than one (budbget permitting).
        \end{itemize}
    \end{itemize}
\end{definition}

\begin{intuition}
    How to decide on projects if they are\dots
    \begin{itemize}
        \item Independent: Consider each option separately.
        \item Mutually exclusive: Find the best option.
        \item Related but not mutually exclusive: Form mutually exclusive alternatives.
    \end{itemize}
\end{intuition}

\subsection{What do project cash flows look like?}
\begin{intuition}
    \customFigure[0.5]{00_Images/PCF.png}{Project cash flows.}    
\end{intuition}

\subsection{MARR, PW, AW}
\begin{terminology}
    Minimum acceptable rate of return/hurdle rate
        \begin{itemize}
            \item Based on risk and/or expected rate of return from best alternative.
            \item The "do nothing" option, i.e. the rate of return if you were to not invest in a project.
            \begin{itemize}
                \item Assume that money can ALWAYS be invested at MARR.
            \end{itemize}
            \item Type of discount rate.
            \item Projects that earn less than MARR are not acceptable (i.e. doing nothing is better)
        \end{itemize}          
\end{terminology}

\subsection{Project comparison methods}
\begin{definition}

    \textbf{Present Worth (PW):} Present value of benefits minus costs, discounted at MARR.
    \begin{itemize}
        \item $PW:$ The amount by which a project is beating the best alternative expressed in today's value.
        \item For independent projects:
            \begin{itemize}
                \item $PW > 0$: Acceptable
                \item $PW < 0$: Unacceptable.
            \end{itemize}
        \begin{itemize}
            \item \textbf{Note:} Just because it is unacceptable doesn't mean its not profitable. It means that doing nothing is better.
        \end{itemize}
    \end{itemize}
    \vspace{1em}

    \textbf{Annual Worth (AW):} The equivalent annuity of PW, with MARR as discount rate.
    \begin{itemize}
        \item For independent projects:
        \begin{itemize}
            \item $PW > 0$: Acceptable
            \item $PW < 0$: Unacceptable.
        \end{itemize}
    \end{itemize}
\end{definition}

\begin{example}
    \customFigure[0.75]{00_Images/PCM1.png}{Project comparison methods for PW}
\end{example}

\begin{example}
    \customFigure[0.75]{00_Images/PCM2.png}{Project comparisons for AW.}
\end{example}

\subsection{Evaluating mutually exclusive projects}
Can only choose 1 option, so must choose the best.
\begin{process}
    \begin{enumerate}
        \item Define the time horizon.
        
        \item Develop cash flows for each alternative.
        
        \item Calculate the PW using MARR.
        
        \item Compare the PWs and pick the best.
            \begin{itemize}
                \item Higher PW is better.
            \end{itemize}
    \end{enumerate}
\end{process}

\begin{warning}
    Do nothing is always an alternative when no better choice is available (i.e. $PW=0$) (i.e. when all options have $PW<0$).
\end{warning}

\begin{example}
    \customFigure[0.5]{00_Images/EX.png}{Example of mutually exclusive projects with same lives.}
\end{example}

\subsection{Comparing different lives}

\subsubsection{Repeated lives - PW}
\begin{definition}
    \begin{itemize}
        \item Assume project repeats itself, and use least common multiple as time horizon. 
        \begin{itemize}
            \item Repeats with same cash flow.
        \end{itemize}
        \item Compare PW at end of time horizon
    \end{itemize}
\end{definition}

\begin{example}
    \customFigure[0.75]{00_Images/RL.png}{Repeated lives PW}
\end{example}

\subsubsection{Repeated lives - AW}
\begin{definition}
    \begin{itemize}
        \item Compare equivalent annuity of PW for individual lives.
        \begin{itemize}
            \item AW is equivalent annuity, so magnitude of annuity stays the same when the project is repeated (i.e. same annuity regardless how many times it is repeated).
        \end{itemize}
    \end{itemize}
\end{definition}

\begin{example}
    \customFigure[0.75]{00_Images/RLAW.png}{Repeated lives AW}
\end{example}
    
\subsubsection{Study period}
\begin{definition}
    \begin{itemize}
        \item Specify a time period for comparison. 
        \item For projects that last longer than study period, assume can terminate them early and adjust salvage value if necessary.
        \begin{itemize}
            \item Calculate PW for new period of affected projects. 
        \end{itemize}
        \item Uses fixed time horizon. 
    \end{itemize}
\end{definition}

\begin{example}
    \customFigure[0.75]{00_Images/SP.png}{Study period}
\end{example}