\subsection{Mortgages}
\begin{terminology}
    \begin{enumerate}
        \item \textbf{Mortgage loan:} A loan secured by \textbf{real property} (i.e. borrow money and use real property as a collateral)
        \item \textbf{Principle:} Fraction of the real property's cost that you borrow to pay. (Usually 80\%)
        \item \textbf{Down Payment:} Fraction of the real property's cost that you pay upfront yourself. (Usually 20\%)
        \item \textbf{Total cost:} Sum of the principle and down payment (i.e. property's cost)
        \item \textbf{Loan-to-Value Ratio (LTV):} Ratio of mortgage loan to value of the property.
        \item \textbf{Mortgage Rate:} The interest rate charged on the mortgage loan. Compounding period usually matches frequency of payments.
        \item \textbf{Amortization Period:} Time horizon for mortgage payment (i.e. repayment)
        \begin{itemize}
            \item \textbf{Note:} You should be regularly paying back your mortgage (e.g. monthly).
            \item \textbf{Note:} The amount of each regularly payment is calculated based on mortgage rate and amortization period.
        \end{itemize} 
        \item \textbf{Term:} Duration of time where the mortgage rate is fixed.
        \begin{itemize}
            \item \textbf{Key:} When term ends, re-evaluate how much you still owe, then use new interest rate to calculate monthly payment based on time left in amortization period. 
            \item \textbf{Note:} Fixed for a shorter period of time compared to amortization period.
        \end{itemize}
    \end{enumerate}
\end{terminology}

\begin{warning}
    The payments are calculated based on the amortization period, NOT the term.
\end{warning}

\subsubsection{Net amount owed at end of term:}
\begin{definition}
    \begin{equation}
        \begin{aligned}
            \text{Net} &= P \left(F/P, \, \frac{i}{N}, \, t \times N\right) - A \left(F/A, \, \frac{i}{N}, \, t \times N\right) \\
                       &= P (1 + i)^{t \times N} - A \left[\frac{(1 + i)^{t \times N} - 1}{i}\right]
        \end{aligned}
    \end{equation}
    \begin{itemize}
        \item $P$: Mortgage principle
        \item $A$: Regular mortgage payment (usually per month)
        \item $i$: Mortgage rate per annum based
        \item $N$: Number of payment periods per year
        \item $t$: Number of years in term
    \end{itemize}
\end{definition}

\begin{intuition}
    The next set of mortgage payments are determined by:
    \begin{itemize}
        \item The amount still owed.
        \item New mortgage rate.
        \item Remaining amortization period.
    \end{itemize}
\end{intuition}

\subsubsection{Net monthly payment:}
\begin{definition}
    \begin{equation}
        A = P \left(A/P, \, \frac{i}{N}, \, t \times N\right) = P \left[\frac{i (1+i)^{t \times N}}{(1+i)^{t \times N} - 1}\right]
    \end{equation}
    \begin{itemize}
        \item $P$: Mortgage principal (or what is left)
        \item $A$: Regular mortgage payment (usually per month)
        \item $i$: Mortgage rate per annum
        \item $N$: Number of payment periods per year
        \item $t$: Number of years in amortization (or what is left)
    \end{itemize}
\end{definition}

\subsubsection{How to determine monthly payments?}
\begin{process}
    \begin{enumerate}
        \item Identify the principle, mortgage rate, and amortization period.
        \item Change the mortgage rate to be in terms of the periods you want (e.g. want months then $r_{m/m}$)
        \item Change amortization period to the appropriate periods (e.g. years to months).
        \item Use net monthly payment to calculate.
        \item If need to calculate another term's monthly payment, then first calculate the net amount owed at end of period (i.e. principal at year ? - payments at year ?)
        \item Then repeat steps 1-4.
    \end{enumerate}
\end{process}

\begin{warning}
    Interest rates are semi-annual compounding per year.
\end{warning}

\subsection{Bonds}
\begin{terminology}
    \begin{itemize}
        \item \textbf{Bond:} A type of loan where the creditor pays
        \begin{enumerate}
            \item \textbf{Coupon Payments:} A stated amount at specified intervals for a defined period.
            \item \textbf{Face value:} A final amount at a specified date (i.e. maturity date)
        \end{enumerate}
        \item \textbf{Coupon Rate:} The rate used to calculate coupon payments.
        \item \textbf{Coupon Payments:} Regular payments made over the course of a bond’s lifetime. 
        \begin{equation}
            \text{Coupon Amount} = (\text{Coupon Rate}) \times \frac{\text{Face Value}}{\text{Payment Frequency}}
        \end{equation}
        \begin{itemize}
            \item \textbf{Note:} Amount is determined by coupon rate and frequency of payment (per the same time unit as the coupon rate).
        \end{itemize}
    \end{itemize}
\end{terminology}

\subsubsection{Yield}
\begin{definition}
    Hypothetical interest rate of a bond given a purchase price. Solved using interpolation.
        \begin{itemize}
            \item \textbf{Bond yield inversely proportional to bond price:} The higher the yield, the less you value the future payments the bond promises, and the price you are willing to pay is lower.
            \item \textbf{What is yield determined by?} (1) Market expectations of risk (i.e. less stable, means higher bond yields), (2) Longer time to maturity, higher bond yields.
        \end{itemize}
\end{definition}

\begin{intuition}
    \begin{itemize}
        \item \textbf{Intuition:} The yield reflects the total return that the investor expects, including both the fixed coupon payments and face value, which should be higher for risker investments (i.e. they will buy a bond for lower price) 
        \item \textbf{Risk proportional to reward:} A creditor (i.e. investor) must be compensated for the risk of default by demanding a lower price for the bondy when there is a higher preceived risk. 
    \end{itemize}
\end{intuition}

\begin{warning}
    Yield reflects the time value of money for bond payments.
\end{warning}

\subsubsection{Bond price}
\begin{definition}
    \begin{equation}
        \begin{aligned}
            P &= A \left(P/A, \, \frac{i}{m}, \, N\right) + F \left(P/F, \, \frac{i}{m}, \, N\right) \\
            &= A \left[\frac{(1 + \frac{i}{m})^{N} - 1}{\frac{i}{m} (1 + \frac{i}{m})^{N}}\right] + F \left[\frac{1}{(1 + \frac{i}{m})^{N}}\right]
        \end{aligned}
    \end{equation}

    \begin{itemize}
        \item $i$: Yield
        \item $m$: Frequency of coupon payments per time unit (e.g. year)
        \item $N$: Number of periods to maturity ($m \times \text{time unit}$)
        \item $A$: Value of coupon payment
    \end{itemize}
\end{definition}

\begin{intuition}
    Bond prices are calculated by discounting payments using the bond yield.
\end{intuition}

\subsubsection{How to calculate bond price?}
\begin{process}
    \begin{enumerate}
        \item Draw the cash-flow diagram (same everytime)
        \item Identify appropriate variables, making sure that $N$ and $i/m$ are in the same time unit.
        \item Use bond price formula (know why it works intuitively)
    \end{enumerate}
\end{process}

\subsubsection{How to calculate bond yield?}
\begin{process}
    \begin{enumerate}
        \item Draw the cash-flow diagram (same everytime)
        \item Identify appropriate variables, making sure that $N$ and $i/m$ are in the same time unit.
        \item Use bond price formula to plug in all known values. 
        \item Make guesses for the yield, until you get a lower and upper bound for $P$ 
        \item Interpolate: $\frac{Y_{\text{avg}} - Y_1}{X_{\text{avg}} - X_1} = \frac{Y_2 - Y_1}{X_2 - X_1}$
        \begin{itemize}
            \item Y: yield
            \item X: P: Bond price
        \end{itemize}
    \end{enumerate}
\end{process}

\begin{warning}
    Interest rates are semi-annual compounding per year.
\end{warning}
