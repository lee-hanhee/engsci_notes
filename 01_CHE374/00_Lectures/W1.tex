\subsection{Interest}
\begin{definition}
    Money that is earned by investors (creditors/lenders) for allowing others (borrowers) to use their money.
\end{definition}

\subsection{Interest rate}
\begin{definition}
    The rate at which interest is earned (determined by risks):
    \begin{equation}
        i = \frac{I}{P}
    \end{equation}
    \begin{itemize}
        \item \(P\): Principle amount (amount of money borrowed today)
        \item \(I\): Total interest amount 
    \end{itemize}
\end{definition}

\subsection{Simple interest}
\begin{definition}
    \begin{equation}
        F_N = P + NPi = P(1 + Ni)
    \end{equation}
    \begin{itemize}
        \item \(F_N\) : Future amount in (time unit) N
        \begin{itemize}
            \item \(N\): Number of periods (e.g. years)
        \end{itemize}
        \item \textbf{Key:} Applies only to the original principal.
    \end{itemize}
\end{definition}

\customFigure[0.5]{00_Images/Simple_Interest_v2.png}{Simple interest table.}

\customFigure[0.5]{00_Images/Derivation_Simple_Int.png}{Derivation of simple interest.}

\subsection{Compound interest}
\begin{definition}
    \begin{equation}
        F_N = P(1 + i)^N
    \end{equation}
    \begin{itemize}
        \item \textbf{Key:} Applies to the principal and to all interest already accrued, so that you can earn interest on both.
    \end{itemize}
\end{definition}

\begin{warning}
    Assume compound interest unless stated otherwise.
\end{warning}

\customFigure[0.5]{00_Images/Compound_Interest.png}{Compound interest table.}

\customFigure[0.5]{00_Images/Derivation_Compound_Int.png}{Derivation of compound interest.}

\subsection{Subperiod interest rate}
\textbf{Motivation:} What if you can compound multiple times per year?
\begin{definition}
    Fraction of the nominal interest rate:
    \begin{equation}
        i_s = \frac{r}{m}
    \end{equation}
    \begin{itemize}
        \item \(r\): Nominal interest rate (usually for 1 year), which doesn't take compounding into account and is stated annually.
        \item \(m\): Number of times compounded (subperiods) per year
    \end{itemize}
\end{definition}

\subsection{Effective interest rate}
\textbf{Motivation:} How would you compare investments with different compounding periods?
\begin{definition}
    The equivalent interest rate if compounded only once over the stated time period (usually 1 year).
    \begin{equation}
        i_e = (1 + i_s)^m - 1
    \end{equation}
    \begin{itemize}
        \item \textbf{Key:} Provides a measure of the annual interest cost, regardless of the compounding frequency. Whether interest is compounded monthly, quarterly, or continuously, the total amount of interest per year will be $i_e$. 
        \begin{itemize}
            \item $r$ will be adjusted to ensure that the effective annual rate remains consistent. 
        \end{itemize}
    \end{itemize}
\end{definition}

\customFigure[0.5]{00_Images/Effective_Interest_Rate.png}{(Top) Subperiod interest rate. (Bot.) Equivalent interest rate if compounded only once per year. This is used to solve for the actual effective interest rate.}

\subsection{Continuous compound interest}
\begin{definition}
    The finite limit of \(i_e\) as the compounding period over one year becomes infinitesimally small:
    \begin{equation}
        i_e = \lim_{m \to \infty} \left(1 + \frac{r}{m}\right)^m - 1 = e^r - 1
    \end{equation}
    \begin{itemize}
        \item \textbf{Key:} \(i_e\) increases as the compounding period decreases, but it reaches the finite limit eventually.
        \item \textbf{Careful:} Know when to use the continuous compounding and "regular" compounding formulas. 
    \end{itemize}
    \vspace{1em}
    
    The general version over $t$ years:
    \begin{equation}
        i_e = e^{rt} - 1
    \end{equation}
\end{definition}

\subsection{Compound interest with subperiods}
\begin{definition}
    \begin{equation}
        F = P(1 + i_s)^{m} = P(1 + i_e) 
    \end{equation}
    \begin{itemize}
        \item \(F\): Future amount 
        \item \textbf{Note:} For the same nominal interest rate, the more frequently you compound, the more you earn at the end of the year.
        \begin{itemize}
            \item \textbf{Intuition:} You are collecting some of the interest along the way and reinvesting that back.
        \end{itemize}
    \end{itemize}
\end{definition}

\customFigure[0.5]{00_Images/Compound_Interest_Subperiods.png}{Compound interest in which there is interest that can occur within the nominal interest rate over m times.}