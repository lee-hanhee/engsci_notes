\subsection{Interest}
\begin{definition}
    Money that is earned by investors (creditors/lenders) for allowing others (borrowers) to use their money.
\end{definition}

\begin{intuition}
    \begin{itemize}
        \item Interest is needed to be a reward for either putting money somewhere to earn more money or being given money and that having an interest rate. 
        \item If there was no interest rate, then the future value would be equal to the present value. So we need interest rates. This is the basis for the whole course.
    \end{itemize}
\end{intuition}

\subsection{Interest rate}
\begin{definition}
    The rate at which interest is earned (determined by risks):
    \begin{equation}
        i = \frac{I}{P}
    \end{equation}
    \begin{itemize}
        \item \(P\): Principle amount (amount of money borrowed today)
        \item \(I\): Total interest amount 
    \end{itemize}
\end{definition}

\subsection{Simple interest}
\begin{definition}
    \begin{equation}
        F_N = P + NPi = P(1 + Ni)
    \end{equation}
    \begin{itemize}
        \item \(F_N\) : Future amount in (time unit) N
        \begin{itemize}
            \item \(N\): Number of periods (e.g. years)
        \end{itemize}
        \item \textbf{Key:} Applies only to the original principal.
    \end{itemize}
\end{definition}

\customFigure[0.5]{00_Images/Simple_Interest_v2.png}{Simple interest table.}

\customFigure[0.5]{00_Images/Derivation_Simple_Int.png}{Derivation of simple interest.}

\subsection{Compound interest}
\begin{definition}
    \begin{equation}
        F_N = P(1 + i)^N
    \end{equation}
    \begin{itemize}
        \item \textbf{Key:} Applies to the principal and to all interest already accrued, so that you can earn interest on both.
    \end{itemize}
\end{definition}

\begin{warning}
    Assume compound interest unless stated otherwise.
\end{warning}

\customFigure[0.5]{00_Images/Compound_Interest.png}{Compound interest table.}

\customFigure[0.5]{00_Images/Derivation_Compound_Int.png}{Derivation of compound interest.}

\subsection{Subperiod interest rate}
\textbf{Motivation:} What if you can compound multiple times per year?
\begin{definition}
    Fraction of the nominal interest rate:
    \begin{equation}
        i_s = \frac{r}{m}
    \end{equation}
    \begin{itemize}
        \item \(r\): Nominal interest rate (usually for 1 year), which doesn't take compounding into account and is stated annually.
        \item \(m\): Number of times compounded (subperiods) per year
    \end{itemize}
\end{definition}

\subsection{Effective interest rate}
\textbf{Motivation:} How would you compare investments with different compounding periods?
\begin{definition}
    The equivalent interest rate if compounded only once over the stated time period (usually 1 year).
    \begin{equation}
        i_e = (1 + i_s)^m - 1
    \end{equation}
    \begin{itemize}
        \item \textbf{Key:} Provides a measure of the annual interest cost, regardless of the compounding frequency. Whether interest is compounded monthly, quarterly, or continuously, the total amount of interest per year will be $i_e$. 
        \begin{itemize}
            \item $r$ will be adjusted to ensure that the effective annual rate remains consistent. 
        \end{itemize}
    \end{itemize}
\end{definition}

\customFigure[0.5]{00_Images/Effective_Interest_Rate.png}{(Top) Subperiod interest rate. (Bot.) Equivalent interest rate if compounded only once per year. This is used to solve for the actual effective interest rate.}

\subsection{Continuous compound interest}
\begin{definition}
    The finite limit of \(i_e\) as the compounding period over one year becomes infinitesimally small:
    \begin{equation}
        i_e = \lim_{m \to \infty} \left(1 + \frac{r}{m}\right)^m - 1 = e^r - 1
    \end{equation}
    \begin{itemize}
        \item \textbf{Key:} \(i_e\) increases as the compounding period decreases, but it reaches the finite limit eventually.
        \item \textbf{Careful:} Know when to use the continuous compounding and "regular" compounding formulas. 
    \end{itemize}
    \vspace{1em}
    
    The general version over $t$ years:
    \begin{equation}
        i_e = e^{rt} - 1
    \end{equation}
\end{definition}

\subsection{Compound interest with subperiods}
\begin{definition}
    \begin{equation}
        F = P(1 + i_s)^{m} = P(1 + i_e) = Pe^{r_{cc}t}
    \end{equation}
    \begin{itemize}
        \item \(F\): Future amount 
        \item \textbf{Note:} For the same nominal interest rate, the more frequently you compound, the more you earn at the end of the year.
        \begin{itemize}
            \item \textbf{Intuition:} You are collecting some of the interest along the way and reinvesting that back.
        \end{itemize}
    \end{itemize}
\end{definition}

\customFigure[0.5]{00_Images/Compound_Interest_Subperiods.png}{Compound interest in which there is interest that can occur within the nominal interest rate over m times.}

\subsection{L1 Takeaways}

\begin{intuition}
    \begin{itemize}
        \item Rates are always happening, analogous to $1m/s,km/h,mi/h$
        \item The only difference between the rates is just language. We are converting into different rates based on compounding and the period (but they are all the same).
    \end{itemize}
\end{intuition}

\subsection{How to solve for equivalent interest rates?}
\begin{process}
    \begin{enumerate}
        \item Identify the variables and values using the $r_{x/y}$ notation. 
        \item Set the effective interest rates equal to each other using variables using the general formula or CC version.
        \begin{itemize}
            \item If using the continuous compound version and are looking for something other than $\%/year$, then you convert it by doing a simple unit change in the exponent
        \end{itemize}
        \item Solve for the variable of interest. 
    \end{enumerate}
\end{process}

\begin{intuition}
    $r_{x/y}$
    \begin{itemize}
        \item $x$: How much is interest compounded (e.g. monthly compounding, daily compounding, quarterly compounding, etc)
        \item $y$: How long of a period (e.g. per year, monthly, per month, etc)
    \end{itemize}
\end{intuition}

\begin{warning}
    \begin{itemize}
        \item Assume for any rate to be per year unless stated otherwise. 
        \item Assume interest is compounded once per period (unless specified)
    \end{itemize}
\end{warning}

\subsection{General formula}
\begin{definition}
    \begin{equation}
        r_{y/y} = \left(1 + \frac{r_{n/m}}{n_m}\right)^{n_y} - 1 
    \end{equation}
    \begin{itemize}
        \item $n_m$: \# of compounding periods $n$ in time $m$.
        \item $n_y$: \# of compounding periods $n$ per year
    \end{itemize}
\end{definition}

\begin{intuition}
    \begin{itemize}
        \item Dividing by $n_m$ gives you the interest rate applied at each compounding period within a certain time frame $m$.
        \item Raising to the power of $n_y$ accumulates that periodic interest over a year, resulting in the effective interest rate. This will always be due to the first part (i.e. if $r_{x/y}$ then it is based on x to determine how many times we compound it over that year)
        \item If $\left(1 + r_{x/x}\right)^{n_x} -1$, then $n_x$ is the number of compounding periods in one year. You don't need to divide since it is already the interest rate per compounding period. Therefore, the only thing you need to do is find the accumulative interest over a year by raising it to the number of compounding periods per year.
    \end{itemize}
\end{intuition}