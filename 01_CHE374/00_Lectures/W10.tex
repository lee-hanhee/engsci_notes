\subsection{Learning Objectives}
\begin{summary}
    \begin{itemize}
        \item Explain common causes and effects of inflation. 
        \item Understand how inflation is measured. 
        \item Understand real and actual interest rates. 
        \item Know how to account for inflation when evaluating cash-flows.
        \item Know how to account for inflation when using the tax benefit factors. 
    \end{itemize}
\end{summary}

\subsection{Inflation terms}
\begin{terminology}
    \begin{itemize}
        \item \textbf{Inflation:} A rise in the average price of goods and services over time, reflecting a decline in the purchasing power of the dollar.
        \item \textbf{Deflation:} A decrease in the average price of goods and services over time.
    \end{itemize}
\end{terminology}

\subsection{What causes inflation?}
\begin{definition}
    \begin{itemize}
        \item \textbf{Demand-pull inflation:} When demand for goods and services exceeds supply causing prices to rise.
        \item \textbf{Cost-push inflation:} When production costs increase, causing firms to raise prices to maintain profit margins.
        \item \textbf{Expanding money supply:} When the government prints more money or banks lend more money, causing the value of money to decrease.
        \item \textbf{Expectations:} When people expect prices to rise, they buy more now, causing businesses to raise prices, causing people to buy more now. This creates a cycle of inflation.
    \end{itemize}
\end{definition}

\subsection{What are the effects of inflation?}
\begin{definition}
    \begin{itemize}
        \item \textbf{Benefits borrowers:} Borrowers repay loans with money that is worth less than when they borrowed it.
        \item \textbf{Hurts lenders, savers, unemployed, retirees, etc:} Money they have saved or invested is worth less.
        \item \textbf{Menu costs, and tax distortions:} Cost of changing prices and taxes to keep up with inflation.
        \item \textbf{Short Run:} Higher inflation can lead to lower unemployment because people spend more causing businesses to hire more.
        \item \textbf{Low levels of inflation:} Stimulates the economy by encouraging spending.
        \item \textbf{High levels of inflation:} Can lead to hyperinflation, which can destroy an economy. Hyperinflation is when prices rise more than 50\% a month.
    \end{itemize}
\end{definition}

\subsection{Why don't we target zero inflation?}
\begin{intuition}
    \begin{itemize}
        \item \textbf{Deflation:} Can lead to a recession because people expect prices to fall, so they wait to buy things, causing businesses to lay off workers.
        \item \textbf{Low inflation:} Encourages spending and investment.
        \item \textbf{Moderate inflation:} Encourages people to spend and invest, but not so much that it causes hyperinflation.
    \end{itemize}
\end{intuition}

\subsection{CPI}
\begin{terminology}
    \begin{itemize}
        \item \textbf{Consumer Price Index (CPI):} A measure that examines the weighted average of prices of a basket of goods and services which are of primary consumer needs.
        \item \textbf{CPI Base Year:} 2002
        \item \textbf{CPI Base Year Index:} 100
        \begin{itemize}
            \item Index for any other year indicates the number of dollars needed in that year to buy the basket of goods that cost \$100 in 2002.
        \end{itemize}
    \end{itemize}
\end{terminology}

\begin{warning}
    Inflation is determined by changes in CPI. 
\end{warning}

\subsection{Calculating CPI \& inflation}
\subsubsection{CPI index}
    \begin{definition}
        \begin{equation}
            \text{CPI Index} = \left( \frac{\text{Basket Value in Year } N}{\text{Basket Value in Base Year}} \right) \times 100
        \end{equation}
    \end{definition}

\subsubsection{Inflation rate from CPI index}
    \begin{definition}
        \begin{equation}
            f_{N_1 \rightarrow N_2} = \frac{\text{CPI Index in Year } N_2 - \text{CPI Index in Year } N_1}{\text{CPI Index in Year } N_1};\: N_2 > N_1
        \end{equation}
        \begin{itemize}
            \item \(f\): Inflation from year $N_1$ to $N_2$.
        \end{itemize}
        \begin{equation}
            f_N = \frac{\text{CPI Index}_{N} - \text{CPI Index}_{N-1}}{\text{CPI Index}_{N-1}}
        \end{equation}
        
        \begin{itemize}
            \item $f_N$: Inflation in year N from CPI.
        \end{itemize}
    \end{definition}

\subsubsection{Average inflation from CPI}
    \begin{definition}
        \begin{equation}
            1 + f_{N_1 \rightarrow N_2} = \frac{\text{CPI Index}_{N_2}}{\text{CPI Index}_{N_1}}
        \end{equation}
        
        \begin{equation}
            (1 + f_{\text{avg}})^{N_2 - N_1} = 1 + f_{N_1 \rightarrow N_2}
        \end{equation}
    \end{definition}

\subsection{Real value}
\subsubsection{Terminology}
\begin{terminology}
    \begin{itemize}
        \item \textbf{Actual (current, nominal) dollars:} Expressed in the monetary units at the time the cash flow occurs.
        \item \textbf{Real (constant) dollars:} Expressed in the monetary units of constant purchasing power, and must always be associated with a particular date.
        \item \textbf{Purchasing Power Ratio:} Ratio of actual investment value over price of good when base values are identical.
        \item \textbf{Actual Interest Rate} (\(i, i_A\)): Observed interest rate based on actual dollars.
        \item \textbf{Real Interest Rate} (\(i', i_R\)): Interest rate based on dollars of constant purchasing power.
        \begin{equation}
            1 + i_R = \frac{1 + i_A}{1 + f}
        \end{equation}
        \item \(f\): Inflation rate
    \end{itemize}
\end{terminology}

\subsubsection{Real rate}
    \begin{definition}
        \begin{equation}
            1 + r_{real} = \frac{1 + r_{actual}}{1 + f}
        \end{equation}
        \begin{itemize}
            \item \(f\): Inflation rate
            \item \(r_{actual}\): Actual rate of growth (e.g. of investment)
        \end{itemize}
        \vspace{1em}

        If continuously compounding: 
        \begin{align}
            e^{r_{real}} = e^{r_{actual} - f} \\
            \therefore r_{real} = r_{actual} - f
        \end{align}
        
    \end{definition}

\subsubsection{Real value from purchasing power}
    \begin{definition}
        \begin{equation}
            \text{Real Value}_N = \text{CPI}_o \times \text{PP}_N = \text{CPI}_o \times \left (\frac{1 + r_{actual}}{1 + f} \right)^N = \text{CPI}_o \times (1 + r_{real})^N
        \end{equation}        
    \end{definition}

\subsection{Economic valuation with inflation}
\subsubsection{Actual vs. real values}
    \begin{definition}

        \textbf{Actual values}
            \begin{itemize}
                \item Must adjust for inflation. 
                \item Use actual MARR: \(i_A\)
                \item Most market interest rates given with actual rates
            \end{itemize}
        \textbf{Real values}
            \begin{itemize}
                \item Do not adjust for inflation. 
                \item Use real MARR: \(i_R\) 
            \end{itemize}
    \end{definition}
\subsubsection{Cash flow with inflation}
    \begin{definition}
        \begin{itemize}
            \item If \(A\) is given in actual dollars:
            \begin{equation}
                \text{PW} = -\text{FC} + A (P/A, i_A, N)
            \end{equation}
        
            \item If \(A\) is given in real dollars:
            \begin{equation}
                \text{PW} = -\text{FC} + A (P/A, i_R, N)
            \end{equation}
        \end{itemize}
    \end{definition}

\subsubsection{Loan with inflation}
    \begin{process}
        \begin{enumerate}
            \item Convert loan interest rate to effective annual rate:
            \begin{equation}
                1 + i_e = \left(1 + \frac{r}{m}\right)^m
            \end{equation}
        
            \item Find real effective interest rate:
            \begin{equation}
                1 + i_R = \frac{1 + i_A}{1 + f}
            \end{equation}
        \end{enumerate}
    \end{process}

\subsubsection{Bond with inflation}
    \begin{process}
        \begin{enumerate}
            \item Convert to actual effective rate \( i_A \) (yield):
            \begin{equation}
                1 + i_R = \frac{1 + i_A}{1 + f}
            \end{equation}
        
            \item Convert to interest rate with compounding period matching coupon amounts:
            \begin{equation}
                1 + i_A = \left(1 + \frac{r}{m} \right)^m
            \end{equation}
        
            \item Calculate bond price:
            \begin{equation}
                \text{Coupon amount} = \frac{\text{Coupon rate} \times \text{Face Value}}{\text{Payment Frequency}}
            \end{equation}
            \begin{align}
                P = A\left(P/A, \frac{i}{m}, N\right) + F\left(P/F, \frac{i}{m}, N\right) \\
                P = A \left( \frac{(1+\frac{i}{m})^{N} - 1}{\frac{i}{m} (1 + \frac{i}{m})^N} \right) + F \left( \frac{1}{(1+\frac{i}{m})^N} \right)
            \end{align}
        \end{enumerate}

        \begin{itemize}
            \item \( i \): Yield (\( i_A \)).
            \item \( m \): Frequency of coupon payments per time unit (e.g., year).
            \item \( N \): Number of periods to maturity (\( m \times \text{time unit} \)).
            \item \( A \): Value of each coupon payment.
        \end{itemize}
    \end{process}

\subsection{Inflation with tax benefits}
Depreciation factors are in actual dollars, so actual interest rates must be used in tax benefit factors.

\subsubsection{Tax benefit factors with inflation}
    \begin{process}
        \begin{enumerate}
            \item Calculate the actual rate:
            \begin{equation}
                1 + i_A = (1 + i_R)(1 + f)
            \end{equation}
            
            \item If \( S \) is given in today's dollars, convert to actual dollars. Alternatively, use the shortcut:
            \begin{equation}
                S_A = S(1 + f)^N
            \end{equation}
            
            \item Calculate present worth (PW):
            \begin{align}
                PW = -FC \times CTF + S_A \times CSF \left(P/A,i_A,N \right) \\ 
                = -FC \times CTF + S \frac{(1 + f)^N}{(1 + i_A)^N} \times CSF \\ 
                = -FC \times CTF + S \frac{1}{(1 + i_R)^N} \times CSF
            \end{align}
            \begin{itemize}
                \item If \( S \) is given in real dollars, only need to discount at real MARR:
            \end{itemize} 
            \begin{equation}
                PW = -FC \times CTF + S_R (P/F, i_R, N) \times CSF
            \end{equation}
        
            \begin{itemize}
                \item Always use actual values for tax benefit factors:
            \end{itemize}
            \begin{equation}
                \tau_{db} = \frac{td}{i + d};\: CSF = 1-\tau_{db}
            \end{equation}
            \begin{itemize}
                \item $t$: Tax rate
                \item $d$: Depreciation rate
                \item $i$: After-tax MARR
            \end{itemize}

            \begin{equation}
                \tau_{1/2} = \frac{td}{i + t_d} \cdot \frac{1 + i/2}{1+i}; \: CTF = 1 - \tau_{1/2}
            \end{equation}
            \vspace{1em}
        
            \item If benefits are present (annuity or geometric sequence):
            \vspace{1em}

            Multiply further by $(P/\text{geom}, i_A, f, N)$ if using actual values.
            \begin{itemize}
                \item Find \( i_R \) if \textbf{geometric}, discount at \( i_R \) increasing/decreasing at \(+/-\: g\% \):
                \begin{equation}
                    PW = G \left({P/\text{geom}, i_R, g_R, N} \right)
                \end{equation}
                \begin{itemize}
                    \item Discount if value given for end of year and is actual:
                \end{itemize}
                \begin{equation}
                    PW = \frac{G}{1+f} \left(P/\text{geom}, i_R, g_R, N\right)
                \end{equation}
                \begin{itemize}
                    \item Taxes if revenue:
                \end{itemize}
                \begin{equation}
                    PW = G \left(P/\text{geom}, i_R, g_R, N\right) (1-t)
                \end{equation}

                \item If \textbf{annuity}, find PW using \( i_R \):
                \begin{equation}
                    PW = A \left(P/A, i_R, N \right)
                \end{equation}
                \begin{itemize}
                    \item Discount if value given for end of year and is actual:
                \end{itemize}
                \begin{equation}
                    PW = \frac{A}{1+f} \left(P/A, i_R, N \right)
                \end{equation}
                \begin{itemize}
                    \item Taxes if revenue:
                \end{itemize}
                \begin{equation}
                    PW = \left(P/A, i_R, N \right) (1-t)
                \end{equation}
                \item Sum to PW of FC and SV to get total PW.
            \end{itemize}
        \end{enumerate}
    \end{process}