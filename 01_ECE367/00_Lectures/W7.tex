\subsection{Low-rank approximation}
\begin{definition}
    A rank-$k$ approximation of $A$ is:
    \[
    \sum_{i=1}^k \sigma_i u^{(i)} v^{(i)\top}.
    \]
\end{definition}
\begin{derivation}
    Let $A \in \mathbb{R}^{m \times n}$:
    \[
    A = U \Sigma V^\top
    \]
    \[
    = \begin{bmatrix} u^{(1)} & \cdots & u^{(r)} \end{bmatrix}
    \begin{bmatrix} 
    \sigma_1 & 0 & \cdots & 0 \\
    0 & \sigma_2 & \cdots & 0 \\
    \vdots & \vdots & \ddots & \vdots \\
    0 & 0 & \cdots & \sigma_r
    \end{bmatrix}
    \begin{bmatrix}
    v^{(1)\top} \\
    \vdots \\
    v^{(r)\top}
    \end{bmatrix}
    \]
    \[
    = \sum_{i=1}^r \sigma_i u^{(i)} v^{(i)\top}.
    \]

    Assume $\sigma_1 \geq \sigma_2 \geq \cdots \geq \sigma_r \geq 0$.

    A rank-$k$ approximation of $A$ is:
    \[
    \sum_{i=1}^k \sigma_i u^{(i)} v^{(i)\top}.
    \]

    It turns out that this is the \textbf{best rank-$k$ approximation}. It minimizes the Frobenius norm:
    \[
    \min_X \|A - X\|_F^2 \quad \text{s.t. } \text{rank}(X) = k.
    \]

    The solution is:
    \[
    X^* = \sum_{i=1}^k \sigma_i u^{(i)} v^{(i)\top}
    \]
    \[
    = \begin{bmatrix} u^{(1)} & \cdots & u^{(k)} \end{bmatrix}
    \begin{bmatrix} 
    \sigma_1 & 0 & \cdots & 0 \\
    0 & \sigma_2 & \cdots & 0 \\
    \vdots & \vdots & \ddots & \vdots \\
    0 & 0 & \cdots & \sigma_k
    \end{bmatrix}
    \begin{bmatrix}
    v^{(1)\top} \\
    \vdots \\
    v^{(k)\top}
    \end{bmatrix}.
    \]
\end{derivation}