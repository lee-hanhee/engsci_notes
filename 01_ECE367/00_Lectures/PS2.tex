\subsection{Calculating inverse of a matrix}
\begin{process}
    Given an \( n \times n \) matrix \( A \), the inverse matrix \( A^{-1} \) can be found through the following steps:

    \begin{enumerate}
        \item \textbf{Check if the matrix is invertible}:
        \begin{itemize}
            \item To be invertible, the matrix must have a non-zero determinant: \( \det(A) \neq 0 \).
            \item If \( \det(A) = 0 \), the matrix is singular and does not have an inverse.
        \end{itemize}

        \item \textbf{Find the determinant of the matrix}:
        \begin{itemize}
            \item The determinant of an \( n \times n \) matrix \( A \) is a scalar value that can be computed using cofactor expansion (also known as Laplace's expansion) along any row or column of the matrix.
            \item \textbf{e.g.} For the first row, the determinant of a matrix \( A = [a_{ij}] \) is given by:
            \[
            \det(A) = \sum_{j=1}^{n} (-1)^{1+j} a_{1j} \det(A_{1j})
            \]
            \item Here, \( A_{1j} \) is the \((n-1) \times (n-1)\) matrix obtained by removing the first row and \( j \)-th column of \( A \).
        \end{itemize}
        
        \item \textbf{Find the cofactors of the matrix}:
        \begin{itemize}
            \item The cofactor of an element \( a_{ij} \) in the matrix \( A \) is denoted by \( C_{ij} \).
            \item The cofactor is given by:
            \[
            C_{ij} = (-1)^{i+j} \det(A_{ij})
            \]
            \item Here, \( A_{ij} \) is the \((n-1) \times (n-1)\) matrix obtained by deleting the \( i \)-th row and the \( j \)-th column from \( A \).
            \item You will compute the cofactor for each element in the matrix \( A \), creating a cofactor matrix.
        \end{itemize}
        
        \item \textbf{Form the cofactor matrix (matrix of cofactors)}:
        \begin{itemize}
            \item Construct the cofactor matrix by placing each cofactor \( C_{ij} \) at the corresponding position in a new matrix:
            \[
            \text{Cofactor Matrix} = \begin{pmatrix}
            C_{11} & C_{12} & \cdots & C_{1n} \\
            C_{21} & C_{22} & \cdots & C_{2n} \\
            \vdots & \vdots & \ddots & \vdots \\
            C_{n1} & C_{n2} & \cdots & C_{nn}
            \end{pmatrix}
            \]
        \end{itemize}
        
        \item \textbf{Form the adjugate (or adjoint) matrix}:
        \begin{itemize}
            \item The adjugate matrix, denoted as \( \text{adj}(A) \), is the transpose of the cofactor matrix.
            \item In other words, swap the rows and columns of the cofactor matrix:
            \[
            \text{adj}(A) = \begin{pmatrix}
            C_{11} & C_{21} & \cdots & C_{n1} \\
            C_{12} & C_{22} & \cdots & C_{n2} \\
            \vdots & \vdots & \ddots & \vdots \\
            C_{1n} & C_{2n} & \cdots & C_{nn}
            \end{pmatrix}
            \]
        \end{itemize}
        
        \item \textbf{Calculate the inverse matrix}:
        \begin{itemize}
            \item The inverse of the matrix \( A \) is given by the formula:
            \[
            A^{-1} = \frac{1}{\det(A)} \text{adj}(A)
            \]
            \item Here, \( \det(A) \) is the determinant of the matrix \( A \), and \( \text{adj}(A) \) is the adjugate matrix.
            \item Multiply each element of the adjugate matrix by \( \frac{1}{\det(A)} \).
        \end{itemize}
        
        \item \textbf{Verify the inverse (optional)}:
        \begin{itemize}
            \item Multiply the matrix \( A \) by its inverse \( A^{-1} \). You should get the identity matrix \( I_n \), where:
            \[
            A \times A^{-1} = A^{-1} \times A = I_n
            \]
            \item The identity matrix \( I_n \) has 1's along the diagonal and 0's elsewhere:
            \[
            I_n = \begin{pmatrix}
            1 & 0 & \cdots & 0 \\
            0 & 1 & \cdots & 0 \\
            \vdots & \vdots & \ddots & \vdots \\
            0 & 0 & \cdots & 1
            \end{pmatrix}
            \]
        \end{itemize}
        
    \end{enumerate}
\end{process}

\subsection{Finding rank and dimension/basis for range and null space of A and A transpose}
\begin{process}
    \begin{enumerate}
        \item Find the REF of the matrix (i.e.  each leading entry in a row is to the right of the leading entry in the previous row, with all entries below leading entries being zeros).
        \item $\text{rank}(A)$ is the number of pivots. 
        \item $\dim (N(A^T)) = \text{Total num. of rows} - \text{rank}(A)$
        \item $\dim (N(A)) = \text{Total num. of cols} - \text{rank}(A)$
        \begin{itemize}
            \item \textbf{Note:} This makes sense from the FTLA, since the total is basically the universal space. 
        \end{itemize}
        \item $R(A)$: Span of column vectors where there is a pivot (can use REF or original matrix's vectors).
        \item $R(A^T)$: Span of row vectors where there is a pivot (can use REF or original matrix's vectors).
        \item $N(A):$ Solve $Ax=0$ and find $x$, where there is probably a free variable. Check by subbing $x$ into $Ax=0$
        \begin{itemize}
            \item You can use the REF of $A$ for this. 
        \end{itemize}
        \item $N(A^T):$ Solve $A^T x=0$ and find $x$, where there is probably a free variable. Check by subbing $x$ into $A^Tx=0$
        \begin{itemize}
            \item \textbf{Key:} Have to find the REF of $A^T$, which is not equivalent to the REF of $A$.
        \end{itemize}
    \end{enumerate}
\end{process}