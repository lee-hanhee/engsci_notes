\subsection{Non-Euclidean projection}
\subsection{Projection onto affine sets}
        \subsubsection{Affine spaces}
        \begin{definition}
            An affine space (or affine set) is a translation (or shift) of a subspace $S$.
        \end{definition}

        \begin{example}
            Consider a vector $x^{(0)}$ (not necessarily in $S$). The affine space $\mathcal{A}$ is defined as:
            \[
            \mathcal{A} = \{ u + x^{(0)} \mid u \in S \}
            \]
            where $x^{(0)}$ is the shifting vector and $S$ is the original subspace. This represents a shifted version of the subspace $S$.

            \customFigure[0.5]{00_Images/AS.png}{Affine space of a 2D space.}
        \end{example}

        \subsubsection{Projection of Affine space defined in terms of basis vectors of corresponding subspace}
        \begin{derivation}
            \begin{enumerate}
                \item The affine space is described by:
                \[
                \mathcal{A} = \left\{ x \mid x = \sum_{i=1}^{d} \alpha_i v^{(i)} + c \right\}
                \]
                \begin{itemize}
                    \item $\{v^{(1)}, \dots, v^{(d)}\}$: Basis vectors of the subspace $S$
                    \item $c$: Vector (i.e. shift).
                \end{itemize}
            
                \item Using the orthogonality principle, we must have:
                \[
                \langle x - x^*, v^{(j)} \rangle = 0 \quad \forall j = 1, \dots, d
                \]
                where $x^* \in \mathcal{A}$. Therefore:
                \[
                x^* = \sum_{i=1}^{d} \alpha_i v^{(i)} + c
                \]
            
                \item This leads to the condition:
                \[
                \left\langle x - \sum_{i=1}^{d} \alpha_i v^{(i)} - c, v^{(j)} \right\rangle = 0 \quad \forall j = 1, \dots, d
                \]
            
                \item Simplifying this expression using the linearity in first argument for inner product, we obtain:
                \[
                \langle x - c, v^{(j)} \rangle = \sum_{i=1}^{d} \alpha_i \langle v^{(i)}, v^{(j)} \rangle \quad \forall j = 1, \dots, d
                \]
            
                \item To solve for $\alpha_1, \dots, \alpha_d$, we set up the following system of linear equations in matrix form:
                \[
                \begin{bmatrix}
                \langle v^{(1)}, v^{(1)} \rangle & \cdots & \langle v^{(1)}, v^{(d)} \rangle \\
                \vdots & \ddots & \vdots \\
                \langle v^{(d)}, v^{(1)} \rangle & \cdots & \langle v^{(d)}, v^{(d)} \rangle
                \end{bmatrix}
                \begin{bmatrix}
                \alpha_1 \\
                \vdots \\
                \alpha_d
                \end{bmatrix}
                =
                \begin{bmatrix}
                \langle x - c, v^{(1)} \rangle \\
                \vdots \\
                \langle x - c, v^{(d)} \rangle
                \end{bmatrix}
                \]
            
                \item Solving this system gives us the values for $\alpha_1, \dots, \alpha_d$. Finally, the projection $x^*$ onto the affine space $\mathcal{A}$ is:
                \[
                x^* = \sum_{i=1}^{d} \alpha_i v^{(i)} + c
                \]
            \end{enumerate}
        \end{derivation}

        \subsubsection{Projection of Affine space defined in terms of orthogonal vectors to corresponding subspace}
        \begin{derivation}
            \begin{enumerate}
                \item The affine set $\mathcal{A}$ is defined as:
                \[
                \mathcal{A} = \left\{ x \mid \langle x, a^{(i)} \rangle = d_i, \; i = 1, \dots, m \right\}
                \]
                \begin{itemize}
                    \item $d_i$: Scalars
                    \item $\{a^{(1)}, \dots, a^{(m)}\}$: A set of vectors spanning the affine space. (Check why this is equivalent to the previous definition of an affine set.)
                \end{itemize}
                
                \item Since $x-x^*$ lies in the span of $\{a^{(1)}, \dots, a^{(m)}\}$:
                \[
                x - x^* = \sum_{i=1}^{m} \beta_i a^{(i)}
                \]
                where $\beta_1, \dots, \beta_m$ are the coefficients to be determined.
                
                \item Since $x^* \in \mathcal{A}$, we also have:
                \[
                \langle x^*, a^{(j)} \rangle = d_j \quad \forall j = 1, \dots, m
                \]
                This implies the orthogonality condition for the projection:
                \[
                \langle x - \sum_{i=1}^{m} \beta_i a^{(i)}, a^{(j)} \rangle = d_j \quad \forall j = 1, \dots, m
                \]
            
                \item Expanding the above expression using the linearity in first argument for inner product, we get:
                \[
                \langle x, a^{(j)} \rangle - \sum_{i=1}^{m} \beta_i \langle a^{(i)}, a^{(j)} \rangle = d_j \quad \forall j = 1, \dots, m
                \]
                
                \item This leads to the system of linear equations:
                \[
                \langle x, a^{(j)} \rangle - d_j = \sum_{i=1}^{m} \beta_i \langle a^{(i)}, a^{(j)} \rangle \quad \forall j
                \]
                
                \item We now solve this system of linear equations for the coefficients $\beta_1, \dots, \beta_m$. The system can be written in matrix form as:
                \[
                \begin{bmatrix}
                \langle a^{(1)}, a^{(1)} \rangle & \cdots & \langle a^{(1)}, a^{(m)} \rangle \\
                \vdots & \ddots & \vdots \\
                \langle a^{(m)}, a^{(1)} \rangle & \cdots & \langle a^{(m)}, a^{(m)} \rangle
                \end{bmatrix}
                \begin{bmatrix}
                \beta_1 \\
                \vdots \\
                \beta_m
                \end{bmatrix}
                =
                \begin{bmatrix}
                \langle x, a^{(1)} \rangle - d_1 \\
                \vdots \\
                \langle x, a^{(m)} \rangle - d_m
                \end{bmatrix}
                \]
                
                \item Solving this system gives the values for $\beta_1, \dots, \beta_m$. Once the $\beta_i$ values are known, the projection $x^*$ is given by:
                
                \[
                x^* = x - \sum_{i=1}^{m} \beta_i a^{(i)} + c
                \]
                \begin{itemize}
                    \item WHAT WOULD BE THE FINAL PROJECTION
                \end{itemize}
            \end{enumerate}            
        \end{derivation}

        \begin{example}
            \begin{enumerate}
                \item Consider the case where \( m = 1 \). The affine set \( \mathcal{A} \) is defined as:
                \[
                \mathcal{A} = \{ x \mid a^T x = d \}
                \]
                where \( a \) is a vector and \( d \) is a scalar.
            
                \item To project \( x \) onto the affine subspace, we start by using the orthogonality condition:
                \[
                \langle x, a \rangle - d = \beta \langle a, a \rangle
                \]
                This ensures that the difference between \( x \) and its projection \( x^* \) lies in the direction of \( a \).
            
                \item Solving for \( \beta \), we get:
                \[
                \beta = \frac{\langle x, a \rangle - d}{\langle a, a \rangle} = \frac{a^T x - d}{\| a \|_2^2}
                \]
                This provides the scalar \( \beta \), which tells us how much of the vector \( a \) needs to be subtracted from \( x \).
            
                \item The projection \( x^* \) onto the affine subspace is then:
                \[
                x^* = x - \beta a = x - \left( \frac{a^T x - d}{\| a \|_2^2} \right) a
                \]
                This gives the final expression for the projection of \( x \) onto the affine set \( \mathcal{A} \).
            \end{enumerate}            
        \end{example}
\subsection{Functions}
\subsection{Gradients}
\subsection{Hessians}