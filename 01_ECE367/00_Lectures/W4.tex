\subsection{Matrices}
\begin{definition}
    Matrices are two-dimensional arrays of numbers. A general matrix \( A \) is denoted as:
    \begin{equation}
        A = \begin{bmatrix}
            a_{11} & \cdots & a_{1n} \\
            \vdots & \ddots & \vdots \\
            a_{m1} & \cdots & a_{mn}
            \end{bmatrix} 
            = \begin{bmatrix}
                a^{(1)} & \cdots & a^{(n)}
                \end{bmatrix}
            \in \mathbb{R}^{m \times n} \quad \text{(or $\mathbb{C}^{m \times n}$ for complex numbers)}
    \end{equation}
\end{definition}

    \subsubsection{Matrix Transpose}
    \begin{definition}
        Given a matrix \( A \) with columns \( a^{(i)} \), its transpose \( A^\top \) is:
        \begin{equation}
            A^\top = \begin{bmatrix}
                (a^{(1)})^\top \\
                \vdots \\
                (a^{(n)})^\top
                \end{bmatrix}
        \end{equation}
    \end{definition}

    \subsubsection{Matrix Multiplication}
    \begin{definition}
        The multiplication of \( A^\top \) and a matrix \( B \) is given by:
        \begin{equation}
            A^\top B = \begin{bmatrix}
                (a^{(1)})^\top \\
                \vdots \\
                (a^{(n)})^\top
                \end{bmatrix}
                \begin{bmatrix}
                b^{(1)} & \cdots & b^{(r)}
                \end{bmatrix}
                = \begin{bmatrix}
                (a^{(1)})^\top b^{(1)} & \cdots & (a^{(1)})^\top b^{(r)} \\
                \vdots & \ddots & \vdots \\
                (a^{(n)})^\top b^{(1)} & \cdots & (a^{(n)})^\top b^{(r)}
                \end{bmatrix}
        \end{equation}

        For regular matrix multiplication \( AB \), we have:
        \begin{equation}
            AB = \begin{bmatrix}
                a^{(1)} & \cdots & a^{(n)}
                \end{bmatrix}
                \begin{bmatrix}
                (b^{(1)})^\top \\
                \vdots \\
                (b^{(n)})^\top
                \end{bmatrix}
                = \sum_{i=1}^{n} a^{(i)} (b^{(i)})^\top
        \end{equation}
        (Note: Try to think why this is the same as before.)
    \end{definition}

    \subsubsection{Block Matrix Product}
    \begin{definition}
        For a block matrix product:
        \begin{equation}
            \begin{bmatrix}
                A & B
                \end{bmatrix}
                \begin{bmatrix}
                X \\ Y
                \end{bmatrix}
                = AX + BY
        \end{equation}
        (Analogous to vectors: \( [a \ b] \begin{bmatrix} x \\ y \end{bmatrix} = a x + b y \))
    \end{definition}

    \subsubsection{Types of Matrices}
    \begin{definition}
        \begin{itemize}
            \item \textbf{Square matrix}: \(m = n\)
            \item \textbf{Diagonal matrix}: \(a_{ij} = 0 \quad \forall i \neq j\)
            \item \textbf{Identity matrix}
            \item \textbf{Triangular matrix}
            \item \textbf{Orthogonal matrix}: \((A^\top A = A A^\top = I)\) (i.e. inverse of A is $A^T$)
            \item \textbf{Symmetric matrix}: \(A = A^\top\)
        \end{itemize}
    \end{definition}

    \subsubsection{Matrices as Linear Maps}
    \begin{definition}
        Matrices are used as linear maps, where:
        \[
        y = A x
        \]
        \[
        \begin{bmatrix} y \end{bmatrix} \in \mathbb{R}^m \quad = \quad \begin{bmatrix} A \end{bmatrix} \in \mathbb{R}^{m \times n} \quad \begin{bmatrix} x \end{bmatrix} \in \mathbb{R}^n
        \]
        This represents a \textbf{linear function}.
        \vspace{1em}

        For an \textbf{affine function}, the equation becomes:
        \[
        y = A x + b
        \]
    \end{definition}
\subsection{Range}
\subsection{Null Space}
\subsection{Eigenvalues and eigenvectors}
\subsection{Matrices diagonalization}