\subsection{Matrices}
\begin{definition}
    Matrices are two-dimensional arrays of numbers. A general matrix \( A \) is denoted as:
    \begin{equation}
        A = \begin{bmatrix}
            a_{11} & \cdots & a_{1n} \\
            \vdots & \ddots & \vdots \\
            a_{m1} & \cdots & a_{mn}
            \end{bmatrix} 
            = \begin{bmatrix}
                a^{(1)} & \cdots & a^{(n)}
                \end{bmatrix}
            \in \mathbb{R}^{m \times n} \quad \text{(or $\mathbb{C}^{m \times n}$ for complex numbers)}
    \end{equation}
\end{definition}

    \subsubsection{Matrix Transpose}
    \begin{definition}
        Given a matrix \( A \) with columns \( a^{(i)} \), its transpose \( A^\top \) is:
        \begin{equation}
            A^\top = \begin{bmatrix}
                (a^{(1)})^\top \\
                \vdots \\
                (a^{(n)})^\top
                \end{bmatrix} \in \mathbb{R}^{n \times m}
        \end{equation}
        \begin{itemize}
            \item \textbf{Note:} $(A^T A)^T = A^T A$
        \end{itemize}
    \end{definition}

    \subsubsection{Matrix Multiplication}
    \begin{definition}
        The multiplication of \( A^\top \) and a matrix \( B \) is given by:
        \begin{equation}
            A^\top B = \begin{bmatrix}
                (a^{(1)})^\top \\
                \vdots \\
                (a^{(n)})^\top
                \end{bmatrix}
                \begin{bmatrix}
                b^{(1)} & \cdots & b^{(n)}
                \end{bmatrix}
                = \begin{bmatrix}
                (a^{(1)})^\top b^{(1)} & \cdots & (a^{(1)})^\top b^{(n)} \\
                \vdots & \ddots & \vdots \\
                (a^{(n)})^\top b^{(1)} & \cdots & (a^{(n)})^\top b^{(n)}
                \end{bmatrix} \in \mathbb{R}^{n \times n}
        \end{equation}

        For regular matrix multiplication \( AB \), we have:
        \begin{equation}
            AB = \begin{bmatrix}
                a^{(1)} & \cdots & a^{(n)}
                \end{bmatrix}
                \begin{bmatrix}
                (b^{(1)})^\top \\
                \vdots \\
                (b^{(n)})^\top
                \end{bmatrix}
                = \sum_{i=1}^{n} a^{(i)} (b^{(i)})^\top \in \mathbb{R}^{m \times m}
        \end{equation}
        (Note: Try to think why this is the same as before.)
    \end{definition}

    \subsubsection{Matrix vector multiplication}
    \begin{definition}
        \begin{equation*}
            Ax = \begin{bmatrix}
                a_{11} & \cdots & a_{1n} \\
                \vdots & \ddots & \vdots \\
                a_{m1} & \cdots & a_{mn} 
            \end{bmatrix}
            \begin{bmatrix}
                    x_{1} \\
                    \vdots \\ 
                    x_{n}
            \end{bmatrix}
            = \sum_{i=1}^{n} x_i a^{(i)} \in \mathbb{R}^{m \times 1}
        \end{equation*}
    \end{definition}

    \subsubsection{Block Matrix Product}
    \begin{definition}
        For a block matrix product:
        \begin{equation}
            \begin{bmatrix}
                A & B
                \end{bmatrix}
                \begin{bmatrix}
                X \\ Y
                \end{bmatrix}
                = AX + BY
        \end{equation}
        (Analogous to vectors: \( [a \ b] \begin{bmatrix} x \\ y \end{bmatrix} = a x + b y \))
    \end{definition}

    \subsubsection{Types of Matrices}
    \begin{definition}
        \begin{itemize}
            \item \textbf{Square matrix}: \(m = n\)
            \item \textbf{Diagonal matrix}: \(a_{ij} = 0 \quad \forall i \neq j\)
            \item \textbf{Identity matrix}: $a_{ij} = 0 \; \forall i \neq j, \; a_{ii} = 1 \; \forall i$
            \item \textbf{Triangular matrix}: $A=QR$ ($R$ is upper triangular)
            \item \textbf{Orthogonal matrix}: \(A^\top A = A A^\top = cI\) 
            \item \textbf{Orthonormal matrix}: \(A^\top A = A A^\top = I\) (i.e. inverse of A is $A^T$)
            \item \textbf{Symmetric matrix}: \(A = A^\top\)
        \end{itemize}
    \end{definition}

    \subsubsection{Matrices as Linear Maps}
    \begin{definition}
        Matrices are used as linear maps, where:
        \[
        y = A x
        \]
        \[
        \begin{bmatrix} y \end{bmatrix} \in \mathbb{R}^m \quad = \quad \begin{bmatrix} A \end{bmatrix} \in \mathbb{R}^{m \times n} \quad \begin{bmatrix} x \end{bmatrix} \in \mathbb{R}^n
        \]
        This represents a \textbf{linear function}.
        \vspace{1em}

        For an \textbf{affine function}, the equation becomes:
        \[
        y = A x + b
        \]
    \end{definition}

\subsection{Range Space}
    \begin{definition}
        \begin{align}
            \mathcal{R}(A) &= \{Ax \mid x \in \mathbb{R}^n\} \subseteq \mathbb{R}^m \\
            &= \left\{ \sum_{i=1}^{n} x_i a^{(i)} \mid x \in \mathbb{R}^n \right\} \\ 
            &= \text{span} \{a^{(1)}, a^{(2)}, \dots, a^{(n)}\}
        \end{align}
        \begin{itemize}
            \item $\mathcal{R}(A) \text{ is a subspace (closed under addition and scalar multiplication).}$
            \item \textbf{Note:} $A$ can consist of different column vectors that still span the space. E.g. In 2D, $(1,0)$ and $(0,1)$ can be used, but $(1,1)$ and $(1,0)$ can be used as well, which would have different $A$. 
            \item \textbf{Note:} $R(A) = R(A^T)$ iff $A=A^T$ (i.e. symmetric).
        \end{itemize}

        \customFigure[0.25]{00_Images/RS.png}{Range space.}
    \end{definition}

\subsection{Null Space}
    \begin{definition} 
    Set of all vectors that map to zero.

    \begin{equation}
        \mathcal{N}(A) = \{x \in \mathbb{R}^n \mid Ax = 0\} \subseteq \mathbb{R}^n
    \end{equation}

    \begin{itemize}
        \item $\mathcal{N}(A) \text{ is also a subspace (closed under addition and scalar multiplication).}$
    \end{itemize}
    \end{definition}

\subsection{Fundamental theorem of algebra}
\begin{definition}
    $\text{For any matrix } A \in \mathbb{R}^{m \times n}$, the subspaces are orthogonal to each other (i.e. orthogonal complement)
    \begin{equation}
        \mathcal{R}(A) \perp \mathcal{N}(A^T)
    \end{equation}

    \begin{equation}
    \mathcal{R}(A^T) \perp \mathcal{N}(A)
    \end{equation}

    Therefore, since $S \oplus S^\perp = V$, then the direct sum of the two subspaces (i.e. if you take one vector from one subspace, and another vector from the other subspace, the sum of these vectors would be in $\mathbb{R}^n$ or $\mathbb{R}^m$)

    \begin{equation}
    \mathcal{R}(A) \oplus \mathcal{N}(A^T) = \mathbb{R}^m
    \end{equation}
    \begin{itemize}
        \item $\dim(\mathcal{R}(A)) + \dim(\mathcal{N}(A^T)) = m$
    \end{itemize}

    \begin{equation}
    \mathcal{R}(A^T) \oplus \mathcal{N}(A) = \mathbb{R}^n
    \end{equation}
    \begin{itemize}
        \item $\dim(\mathcal{R}(A^T)) + \dim(\mathcal{N}(A)) = n$
    \end{itemize}
\end{definition}

\begin{derivation}
    \textbf{Why Does This Hold?} Recall the definition of orthogonal complement:

    \begin{itemize}
        \item $\forall x \in S, \, y \in S^\perp$, we have that $\langle x, y \rangle = 0$
        \item $x \in V$ can be decomposed as $x = x^* + v$, where $x^* \in S$ and $v \in S^\perp$
    \end{itemize}

    Therefore, 
    \begin{equation*}
        \dim V = \dim S + \dim S^\perp
    \end{equation*}
    \customFigure[0.5]{00_Images/OC.png}{Orthogonal complement.}
    \vspace{1em}

    Now using this definition of orthogonal complement, let's apply it to the range space and null space of A:
    Consider 
    \begin{equation*}
        \mathcal{R}(A) = \left\{ y \mid y = \sum_{i=1}^{n} \alpha_i a^{(i)} \right\}
    \end{equation*}

    \begin{equation*}
        \mathcal{N}(A^\top) = \left\{ x \mid \left(a^{(i)}\right)^\top x = 0, \ i = 1, \ldots, n \right\}
    \end{equation*}

    \[
    A^\top = \begin{bmatrix}
    (a^{(1)})^\top \\
    \vdots \\
    (a^{(n)})^\top
    \end{bmatrix}
    \]

    Consider $y \in \mathcal{R}(A)$ so $y = \sum_{i=1}^{n} \alpha_i a^{(i)}$, and $x \in \mathcal{N}(A^\top)$,

    \[
    \langle y, x \rangle = \left\langle \sum_{i=1}^{n} \alpha_i a^{(i)}, x \right\rangle = \sum_{i=1}^{n} \alpha_i \langle a^{(i)}, x \rangle
    \]

    Since $x \in \mathcal{N}(A^\top)$, this implies that $\langle a^{(i)}, x \rangle = 0$, so:

    \[
    \langle y, x \rangle = 0
    \]

    Thus, any $y \in \mathcal{R}(A)$ is orthogonal to any $x \in \mathcal{N}(A^\top)$.
    \vspace{1em}

    Therefore, $\mathcal{R}(A) \oplus \mathcal{N}(A^\top) = \mathbb{R}^m$, and $\dim(\mathcal{R}(A)) + \dim(\mathcal{N}(A^\top)) = m$.

\end{derivation}

\begin{example}
    \customFigure[0.5]{00_Images/FTA.png}{Example of fundamental theorem of linear algebra.}
    \begin{itemize}
        \item Given the range of A, we can find the null space of A by finding what is orthogonal to it. 
    \end{itemize}
\end{example}

\subsubsection{Rank of A}
\begin{definition}
    \begin{equation*}
        \text{Rank}(A) = \dim(\mathcal{R}(A^T)) = \dim(\mathcal{R}(A)) 
    \end{equation*}
    \begin{itemize}
        \item i.e. number of independent rows or independent columns. 
    \end{itemize}
\end{definition}

\subsection{Determinant}
\begin{definition}
    \begin{equation}
    A = \begin{pmatrix} a_{11} & a_{12} \\ a_{21} & a_{22} \end{pmatrix}
    \end{equation}

    \begin{equation}
    \det(A) = a_{11}a_{22} - a_{12}a_{21}
    \end{equation}

    The volume of the region transformed by matrix $A$ is given by:
    \begin{equation}
    \det(A)
    \end{equation}
\end{definition}

\begin{intuition}
    Graphically, the determinant represents the volume of the parallelogram spanned by the column vectors of $A$. 
    \begin{itemize}
        \item Starting from a unit square, the matrix $A$ transforms this square into a parallelogram. 
        \item The area of the transformed parallelogram, denoted as $P$, equals the determinant:
    \end{itemize}

    \[
    \text{vol}(P) = \det(A)
    \]

    \textbf{Why? It is easy to see this for a diagonal matrix, where:}

    \[
    A = \begin{bmatrix} a_{11} & 0 \\ 0 & a_{22} \end{bmatrix}
    \]

    The matrix $A$ scales the unit square by $a_{11}$ along one axis and by $a_{22}$ along the other axis. This results in a rectangle with area:

    \[
    \text{vol}(P) = a_{11}a_{22} = \det(A)
    \]

    For instance, applying $A$ to the basis vectors:

    \[
    A \begin{bmatrix} 1 \\ 0 \end{bmatrix} = \begin{bmatrix} a_{11} \\ 0 \end{bmatrix}, \quad A \begin{bmatrix} 0 \\ 1 \end{bmatrix} = \begin{bmatrix} 0 \\ a_{22} \end{bmatrix}
    \]

    \customFigure[0.5]{00_Images/US.png}{Diagonal}

    \textbf{What if the matrix is not diagonal? Consider an upper triangular matrix:}

    \[
    A = \begin{bmatrix} a_{11} & a_{12} \\ 0 & a_{22} \end{bmatrix}
    \]

    The matrix $A$ still transforms the unit square, but now the parallelogram $P$ is slanted due to the non-zero $a_{12}$. The area of the parallelogram is still given by the determinant:

    \[
    \text{vol}(P) = a_{11} a_{22} = \det(A)
    \]

    Again, applying $A$ to the basis vectors:

    \[
    A \begin{bmatrix} 1 \\ 0 \end{bmatrix} = \begin{bmatrix} a_{11} \\ 0 \end{bmatrix}, \quad A \begin{bmatrix} 0 \\ 1 \end{bmatrix} = \begin{bmatrix} a_{12} \\ a_{22} \end{bmatrix}
    \]

    \customFigure[0.5]{00_Images/UT.png}{Upper triangular}
    \textbf{For a general matrix $A$, we can use the $QR$ factorization, where:}

    \[
    A = QR
    \]

    $R$ is an upper triangular matrix and $Q$ is an orthogonal matrix (i.e., $Q^\top Q = I$). 
    \begin{itemize}
        \item The orthogonal matrix $Q$ represents a rotation, which preserves volume (the area remains unchanged). 
        \item Therefore, the volume is determined by the determinant of $R$.
    \end{itemize}

    \[
    \text{vol}(P) = \det(R) = r_{11} r_{22} = \det(A)
    \]

    \customFigure[0.5]{00_Images/QR.png}{General matrix}
\end{intuition}

\subsection{Matrix inner product and norm}
\begin{definition}
    Set of matrices $A \in \mathbb{R}^{m \times n}$ is a vector space. We can define an inner product over this space:
    \begin{equation}
        \langle A, B \rangle = \text{tr}(B^T A) = \text{tr}(A^T B)
    \end{equation}
    \begin{itemize}
        \item $\text{tr}:$ Trace is the sum of diagonal elements of a square matrix.
    \end{itemize}
\end{definition}

\subsubsection{Frobenius Norm}
\begin{definition}
    For $A, B \in \mathbb{R}^{m \times n}$, then 
    \begin{equation}
        \| A \|_F = \sqrt{\langle A, A \rangle} = \sqrt{\text{tr}(A^T A)} = \sqrt{\sum_{i,j} a_{ij}^2}
    \end{equation}
\end{definition}

\begin{intuition}
    Let
    \[
    A = \begin{bmatrix} 
    a^{(1)} & \cdots & a^{(n)} 
    \end{bmatrix}
    \]
    then
    \[
    A^T A = \begin{bmatrix} 
    (a^{(1)})^T \\ \vdots \\ (a^{(n)})^T 
    \end{bmatrix} 
    \begin{bmatrix} 
    a^{(1)} & \cdots & a^{(n)} 
    \end{bmatrix}
    = 
    \begin{bmatrix} 
    (a^{(1)})^T a^{(1)} & \cdots & (a^{(1)})^T a^{(n)} \\
    \vdots & \ddots & \vdots \\
    (a^{(n)})^T a^{(1)} & \cdots & (a^{(n)})^T a^{(n)} 
    \end{bmatrix}
    \]
    \begin{itemize}
        \item \textbf{Note:} The matrix multiplication shows us why the trace can be written as $\sum_{i,j} a_{ij}^2$. 
        \begin{itemize}
            \item e.g. $a^{(1)})^T a^{(1)} = a_{11}^2 + a_{21}^2 + \ldots + a_{n1}^2$
        \end{itemize}
    \end{itemize}
    \vspace{1em}

    If we think of the matrix as a long column vector of the column vectors of $A$, then 
    \[
    \begin{bmatrix}
    a^{(1)} \\
    \vdots \\
    a^{(n)}
    \end{bmatrix}
    \]
    so we can think of the Frobenius norm as the $l_2$ norm exactly since it's $\sqrt{\sum_{i,j} a_{ij}^2}$, which is identical to the expression for l2-norm.
\end{intuition}

\subsubsection{Operator norm (Motivation for eigenvalues and eigenvectors):}
\begin{intuition}
    \textbf{Let} $A \in \mathbb{R}^{m \times n}$ \textbf{be a linear map from} $\mathbb{R}^n$ \textbf{to} $\mathbb{R}^m$.
    \begin{itemize}
        \item Think of the matrix as a mapping from $\mathbb{R}^n$ to $\mathbb{R}^m$.
    \end{itemize}

    \[
    v \quad \longrightarrow \quad Av \quad \text{(ellipse)}
    \]

    \customFigure[0.75]{00_Images/OMN.png}{Matrix norm tht maximizes $Av$ given $v$ with unit norm.}

    \textbf{Question:} Which direction of $v$ on the unit circle results in $Av$ with the largest magnitude? \textbf{Operator norm}, which can defined for all p valiues.

    \[
    \|A\|_2 = \max_{\|v\|_2 = 1} \|Av\|_2 
    \]

    \[
    \|A\|_1 = \max_{\|v\|_1 = 1} \|Av\|_1
    \]

    \[
    \|A\|_\infty = \max_{\|v\|_\infty = 1} \|Av\|_\infty
    \]

    \textbf{"Interesting" directions (i.e. directions of interest):}
    \begin{enumerate}
        \item  \textbf{Max/Min amplification direction:}
        \[
        \max_{\|v\|_2 = 1} \|Av\|_2 = \|A\|_2, \quad \min_{\|v\|_2 = 1} \|Av\|_2
        \]
        \begin{itemize}
            \item Later, we will see that this is related to the \textbf{singular values} of the matrix.
        \end{itemize}
    
        \item \textbf{Invariant direction:} (square matrix $A \in \mathbb{R}^{n \times n}$)
    
        \[
        A v = \lambda v
        \]
        \begin{itemize}
            \item $v$ is the eigenvector and $\lambda$ is the eigenvalue.
            \item \textbf{Note:} In 2D, there are 2 eigenvectors (i.e. parallel with same and opposite direction)
        \end{itemize}
    \end{enumerate}
\end{intuition}

\subsection{Eigenvalues and Eigenvectors}
\begin{definition}
    For $A \in \mathbb{R}^{n \times n}$, a non-zero vector $v$ is an eigenvector of $A$ with associated eigenvalue $\lambda$ if 
    \[
    A v = \lambda v
    \]
    \begin{itemize}
        \item \textbf{Consequence:} $(A - \lambda I) v = 0$, therefore, $v \in \mathcal{N}(A - \lambda I)$, where $\mathcal{N}(A - \lambda I)$ is the null space of $(A - \lambda I)$.    
    \end{itemize}
\end{definition}

\subsubsection{Characteristic polynomial}
\begin{definition} The above leads to a polynomial equation in $\lambda$ of degree $n$. The roots are the eigenvalues of $A$. 
    \[
    \det(A - \lambda I) = 0
    \]
    \begin{itemize}
        \item \textbf{Roots can be real or complex.} If complex, they occur in conjugate pairs.
        \item \textbf{Note:} The determinant is equal to 0 because $Av=\lambda v$ (i.e. linearly dependent), and LD vectors have determinant of $0$ by definition.
    \end{itemize}
\end{definition}

\subsubsection{Geometric and algebraic multiplicity}
\begin{definition}

    \textbf{Algebraic multiplicity:} $\text{AM}(\lambda)$ is the number of times a given eigenvalue appears as a root of the characteristic equation $\det(A - \lambda I) = 0$.
    \begin{itemize}
        \item For example, $(\lambda - 1)^2$ has root 1 with multiplicity 2.
    \end{itemize}
    \vspace{1em}

    \textbf{Geometric multiplicity:} $\text{GM}(\lambda) = \dim \mathcal{N}(A - \lambda I)$
\end{definition}

\begin{theorem}
    In general, for any matrix $A \in \mathbb{R}^{n \times n}$ (i.e. MUST BE SQUARE),
    \[
    GM(\lambda) \leq AM(\lambda).
    \]
\end{theorem}

\subsubsection{Defective and non-defective}
\begin{definition}

    \textbf{Defective Matrix:} If $GM(\lambda) < AM(\lambda)$ for some $\lambda$
    \vspace{1em}

    \textbf{Non-defective Matrix:} If $GM(\lambda) = AM(\lambda)$, then the matrix is \textbf{diagonalizable} (i.e. non-defective).
    \begin{itemize}
        \item \textbf{"Most" matrices are non-defective.} In other words, the set of non-defective matrices is dense (a random matrix is almost surely non-defective).
    \end{itemize}

\end{definition}

\subsubsection{Eigenvectors corresponding to different eigenvalues are l.i.}
\begin{theorem}
    \textbf{Lemma:} If $u^{(1)}, \ldots, u^{(k)}$ are eigenvectors then,

    \begin{equation}
        u^{(i)} \notin \phi_j = \mathcal{N}(A-\lambda I) \text{ for } i \neq j
    \end{equation} 

    \begin{itemize}
        \item i.e. An eigen vector of $i$ of one null space cannot be in another eigenvector's null space. 
    \end{itemize}
\end{theorem}

\begin{derivation}
    \textit{Why?} Prove by contradiction, assume $u^{(i)}$ is in the null space of $\phi_j$.
    \begin{itemize}
        \item If $A u^{(i)} = \lambda_i u^{(i)}$ and $u^{(i)} \in \mathcal{N}_j$, then $A u^{(i)} = \lambda_j u^{(i)}$. 
        \item Therefore, $(\lambda_i - \lambda_j) u^{(i)} = 0$, which is impossible since $\lambda_i \neq \lambda_j$ and $u^{(i)}$ is a non-zero vector by definition of eigenvectors. 
    \end{itemize}   
\end{derivation}

\begin{theorem}
    Eigenvectors corresponding to different eigenvalues are linearly independent.
\end{theorem}

\begin{derivation}
    Let $\lambda_1, \ldots, \lambda_k$ be distinct eigenvalues, $u^{(1)}, \ldots, u^{(k)}$ be the corresponding eigenvectors, and $\phi_i = \mathcal{N}(A - \lambda_i I)$, the null space corresponding to eigenvalue $\lambda_i$.
    \vspace{1em}

    Proof by contradiction, assume that $u^{(1)}, \ldots, u^{(k)}$ are linearly dependent, i.e.,
    \[
    u^{(1)} = \sum_{i=2}^{k} \alpha_i u^{(i)}.
    \]
    \begin{itemize}
        \item i.e. The first eigenvector can be written as a linear combination of the other eigenvectors. 
    \end{itemize}
    \vspace{1em}

    \textbf{Part 1:}
    \[
    A u^{(1)} = \sum_{i=2}^{k} \alpha_i A u^{(i)} = \sum_{i=2}^{k} \alpha_i \lambda_i u^{(i)}.
    \]
    \begin{itemize}
        \item 2nd part: By definition of $u^{(1)}$
        \item 3rd part: Since $u^{(i)}$ is an eigenvector with corresponding $\lambda_i$
    \end{itemize}

    \textbf{Part 2:}
    \[
    \lambda_1 u^{(1)} = \sum_{i=2}^{k} \alpha_i \lambda_1 u^{(i)}.
    \]
    \begin{itemize}
        \item \textbf{Key:} $\lambda_1$ is inside the summation, while for part 1, it was $\lambda_i$
    \end{itemize}
    \vspace{1em}

    \textbf{Combining both part 1 and part 2:}
    But $A u^{(1)} = \lambda_1 u^{(1)}$, so
    \[
    A u^{(1)} - \lambda_1 u^{(1)} =\sum_{i=2}^{k} \alpha_i (\lambda_1 - \lambda_i) u^{(i)} = 0,
    \]
    which implies that $\{u^{(2)}, \ldots, u^{(k)}\}$ is linearly dependent because $\lambda_1 - \lambda_i \neq 0$ since they are distinct, so the only $\alpha$'s to make the linear combination of FINISH LATER
    \vspace{1em}

    Continue with the same argument to show that $\{u^{(3)}, \ldots, u^{(k)}\}$ is linearly dependent, and so on. 
    \vspace{1em}

    Eventually, $\{u^{(k-1)}, u^{(k)}\}$ are linearly dependent, so:
    \[
    u^{(k)} = \alpha u^{(k-1)}.
    \]
    But this is impossible by the previous lemma since an eigenvector of one null space cannot be in another eigenvector's null space (i.e. cannot write on as a l.c. of the other). Therefore, $\{u^{(1)}, \ldots, u^{(k)}\}$ is linearly independent.    
\end{derivation}

\subsection{Matrix Diagonalization}
\begin{intuition}
    The previous theorem leads us to the diagonalization of a non-defective square matrix $A \in \mathbb{R}^{n \times n}$. 
    \vspace{1em}

    The idea is to look at the null space $\mathcal{N}(A - \lambda_i I)$. 
    \begin{itemize}
        \item When $\lambda_1, \ldots, \lambda_n$ are distinct, then $u^{(1)}, \ldots, u^{(n)}$ are linearly independent, so the set $\{u^{(1)}, \ldots, u^{(n)}\}$ forms a basis for $\mathbb{R}^n$.
        \item If some $\lambda_i$ is a repeated eigenvalue. For example, if $AM(\lambda_i) = 2$, then $GM(\lambda_i) = 2$ because it's non-defective matrix so $\dim \mathcal{N}(A - \lambda_i I) = 2$. 
        \begin{itemize}
            \item This means we can find $\{u^{(i,1)}, u^{(i,2)}\}$ that span this null space.
            \item So, if we assemble all these $u^{(i,j)}$'s, then we get a basis for $\mathbb{R}^n$.
        \end{itemize}
    \end{itemize}
\end{intuition}

\begin{theorem}
    Let $\lambda_i$ for $i = 1, \dots, k$, be distinct eigenvalues of a non-defective matrix $A \in \mathbb{R}^{n \times n}$. Let $\mu_i$ be the multiplicity of $\lambda_i$ (i.e., $AM(\lambda_i) = GM(\lambda_i) = \mu_i$).
    \vspace{1em}

    Let
    \[
    U^{(i)} = \begin{bmatrix} 
    u^{(i,1)} & \dots & u^{(i,\mu_i)}
    \end{bmatrix}
    \]
    be a matrix of eigenvectors that form a basis for $\mathcal{N}(A - \lambda_i I)$.
    \vspace{1em}

    Then 
    \[
    U = \begin{bmatrix} 
    U^{(1)} & \dots & U^{(k)}
    \end{bmatrix}
    \]
    is a complete set of basis vectors for $\mathbb{R}^n$.
    \vspace{1em}

    Then the \textbf{Eigendecomposition} is
    \[
    A = U \Lambda U^{-1}
    \]
    where 
    \[
    \Lambda = \begin{bmatrix}
    \lambda_1 & 0 & 0 & 0 & \dots & 0 \\
    0 & \lambda_1 & 0 & 0 & \dots & 0 \\
    0 & 0 & \lambda_2 & 0 & \dots & 0 \\
    0 & 0 & 0 & \lambda_2 & \dots & 0 \\
    \vdots & \vdots & \vdots & \vdots & \ddots & \vdots \\
    0 & 0 & 0 & 0 & \dots & \lambda_n
    \end{bmatrix}
    \]
    is a diagonal matrix with the eigenvalues of $A$ on the diagonal.
    \begin{itemize}
        \item Where $\lambda_1$ appears $\mu_1$ times on the diagonal (indicating multiplicity), followed by $\lambda_2$ appears $\mu_2$ and so on. 
    \end{itemize}
\end{theorem}

\begin{derivation}
    $A u^{(i,j)} = \lambda_i u^{(i,j)}$, where $u^{(i,j)}$ is an eigenvector with eigenvalue $\lambda_i$, then this implies that

    \[
    A \begin{bmatrix}
    u^{(1,1)} & \dots & u^{(1,\mu_1)} & u^{(2,1)} & \dots & u^{(2,\mu_2)} & \dots & u^{(n,\mu_n)}
    \end{bmatrix}
    = U
    \begin{bmatrix}
        \lambda_1 & 0 & 0 & 0 & \dots & 0 \\
        0 & \lambda_1 & 0 & 0 & \dots & 0 \\
        0 & 0 & \lambda_2 & 0 & \dots & 0 \\
        0 & 0 & 0 & \lambda_2 & \dots & 0 \\
        \vdots & \vdots & \vdots & \vdots & \ddots & \vdots \\
        0 & 0 & 0 & 0 & \dots & \lambda_n
        \end{bmatrix}
    \]
    \begin{itemize}
        \item i.e. Generalizing to all the eigenvectors with their corresponding eigenvalues. 
    \end{itemize}
    \vspace{1em}

    Thus, we have:
    \[
    A U = U \Lambda
    \]
    which implies:
    \[
    A = U \Lambda U^{-1}
    \]
    where \( U \) is invertible because its columns (the eigenvectors) are linearly independent, and \( U \) is a square matrix.
\end{derivation}

\subsubsection{What does this diagonalization mean?}
\begin{intuition}
    We start with the eigendecomposition of a matrix \( A \):
    \[
    A = U \Lambda U^{-1}
    \]
    \begin{itemize}
        \item $U = \begin{bmatrix} u^{(1)} & \dots & u^{(n)} \end{bmatrix}.$
    \end{itemize}
    \vspace{1em}

    Let \( x \in \mathbb{R}^n \) be any vector. Then:
    \[
    A x = U \Lambda U^{-1} x.
    \]
    Let \( \tilde{x} = U^{-1} x \), so that:
    \[
    A x = U \Lambda \tilde{x} \quad \text{and} \quad x = U \tilde{x} = \begin{bmatrix} u^{(1)} & \dots & u^{(n)} \end{bmatrix} \begin{bmatrix} \tilde{x}_1 \\ \vdots \\ \tilde{x}_n \end{bmatrix} = \sum_{i=1}^{n} \tilde{x}_i u^{(i)}.
    \]
    \begin{itemize}
        \item The coordinates of \( \tilde{x} \) are the components (i.e. coefficients) of \( x \) in the new basis \( \{u^{(1)}, \dots, u^{(n)}\} \).
        \item \( \Lambda \tilde{x} \) means that we are scaling \( \tilde{x}_i \) by the corresponding eigenvalue \( \lambda_i \).
        \item \( U \Lambda \tilde{x} \) means that we are changing the coordinates back to the original basis.
    \end{itemize}
    \vspace{1em}

    This applies to any non-defective (or diagonalizable) matrix \( A \in \mathbb{R}^{n \times n} \).
\end{intuition}
