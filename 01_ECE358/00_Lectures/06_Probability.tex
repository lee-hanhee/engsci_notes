\subsection{Sample space}
    \begin{definition}
        The set of \emph{all possible outcomes} of a statistical experiment, denoted by $S$.
    \end{definition}

\subsection{Event}
    \begin{definition}
        A subset of a sample space $S$. An event is any outcome or combination of outcomes.
    \end{definition}

\subsection{Probability axioms}
    \begin{definition}
        For $A$, $B \subseteq S$
        \begin{enumerate}
            \item $0 \leq P(A) \leq 1$
            \item $P(S) = 1$
            \item $P(A \cup B) = P(A) + P(B)$ for two mutually exclusive events $A$ and $B$. 
        \end{enumerate}
    \end{definition}

\subsection{Mutually exclusive events}
\begin{definition}
    $\text{If } A_1, A_2, A_3, \dots \text{ is a sequence of mutually exclusive events, then}$
    \begin{equation}
        P(A_1 \cup A_2 \cup A_3 \cup \dots) = P(A_1) + P(A_2) + P(A_3) + \dots    
    \end{equation}
    \begin{itemize}
        \item \textbf{Key}: Independent is \textbf{NOT} mutually exclusive.
        \item \textbf{Key Implication}: For independent events, we can multiply their probabilities to get their intersection (useful for \textbf{calculating favorable outcomes}).
        \begin{itemize}
            \item \( P(A \cap B \cap C \cap D) = P(A)P(B)P(C)P(D) \) \text{ as long as they are \textbf{INDEPENDENT}.}
            \item \textbf{Note}: Can also be applied to a sequence of dependent events as long as you adjust for changing conditions.
        \end{itemize}
    \end{itemize}
\end{definition}

\subsection{Additive rule}
    \begin{definition}
        \begin{equation}
            P(A \cup B) = P(A) + P(B) - P(A\cap B)
        \end{equation}
    \end{definition}

\subsection{Uniform distribution}
    \begin{definition}
        If $\forall s\in S$ has probability $P(s) = \frac{1}{\abs{S}}$, then it is a uniform distribution.
    \end{definition}

\subsection{Independence}
    \begin{definition}
        $P(A \cap B) = P(A)P(B)$ if $A$, $B$ independent.
    \end{definition}

\subsection{Bayes theorem}
    \begin{definition}
        For events with $P(A)>0$ and $P(B)>0$, the probability $A$ happens given $B$ happens is:
        \begin{equation}
            P(B|A) = \frac{P(B\cap A)}{P(A)} = \frac{P(A|B)P(B)}{P(A)}
        \end{equation}
    \end{definition}

\subsection{Bayes' rule with total probability}
    \begin{definition}
        Suppose $C_1,\ldots,C_k$ is a partition. Then 
        \begin{equation}
            P(B|A) = \frac{P(B)P(A|B)}{\sum_{i=1}^{k} P\left(C_i\right)P\left(A|C_i\right)}
        \end{equation}
        Often $B$ is an element of $C_1,\ldots,C_k$, say $B=C_n$. Then 
        \begin{equation}
            P\left(C_n|A\right) = \frac{P\left(C_n\right)P\left(A|C_n\right)}{\sum_{i=1}^{k} P\left(C_i\right)P\left(A|C_i\right)}
        \end{equation}
    \end{definition}

    \begin{process}
        \begin{enumerate}
            \item Write down all the probabilities. 
            \item Try solving the problem directly using definitions. 
        \end{enumerate}
    \end{process}

    \begin{intuition}
        If given P(A|B) and want P(B|A), then automatically use Bayes’ Rule. 
    \end{intuition}

\subsection{Discrete random variable}
    \begin{definition}
        An RV is a function that associates a real number with each element of the sample space. Denote RVs with capital letters.    
    \end{definition}

\subsection{Probability mass function}
    \begin{definition}
        The set of ordered pairs $(x,f(x))$ of the discrete RV X if, for each possible outcome $x$, 
        \begin{enumerate}
            \item $f(x) \geq 0$ for each outcome $X=x$ 
            \item $\sum_{x} f(x) = 1$ (i.e. total probability sums to 1)
            \item $f(x) = P(X=x)$ (i.e. probability of each outcome)
        \end{enumerate}
    \end{definition}
    
\subsection{Expectation}
    \begin{definition}
        Let $X$ be an RV with distribution $f(x)$, then 
        \begin{equation}
            E[X] = \sum_{x\in X} xf(x)
        \end{equation}
        where the sum is taken over all possible values of $X$.
    \end{definition}

\subsection{Properties of expectation}
    \begin{definition}
        
        \begin{enumerate}
            \item $E[X+Y] = E[X] + E[Y]$ (linearity)
            \item $E[\alpha X] = \alpha E[X]$ (linearity)
            \item $E[XY] = E[X]E[Y]$ if independent
        \end{enumerate}
    \end{definition}