\subsection{Greedy Algorithms Intro and Class Scheduling}

\begin{intuition}
    \begin{itemize}
        \item "Greed works, greed is right, greed is good." 
        \item Greedy algorithms make locally optimal choices at each stage with the goal of finding a global optimum. This approach can approximate NP-complete problems.
    \end{itemize}
\end{intuition}

\subsubsection{Characteristics}
\begin{definition}
        Greedy algorithms are good for max/min problems. 
        \begin{itemize}
            \item \textbf{Optimal substructure:} Same as DP
            \item \textbf{Greedy principle:} At every step do greedy choice that looks at the best option at the time begin (i.e. aims to identify and select the most favorable choice at each of the given stagesaims to identify and select the most favorable choice at each of the given stages). 
        \end{itemize}
\end{definition}

\begin{warning}
    You will need to do a proof of correctness for DP and GA.
\end{warning}

\subsubsection{Class Scheduling Example}
\begin{example}
    Given one room and many classes to schedule, the goal is to maximize the number of classes scheduled in one day. The greedy aprooach is that
    \begin{enumerate}
        \item Sort by finish time (i.e. start with the class that starts more early)
        \item Schedule classes in order based on sorted times.
        \item The time complexity for sorting is $O(n\log(n))$.
    \end{enumerate}
    \customFigure[0.75]{00_Images/CSE.png}{Class scheduling example.}

\end{example}

\subsubsection{Proof of Correctness of Class Scheduling (On exams)}
\begin{derivation}

    Prove $g_1,\ldots,g_m$ is optimal:
    \begin{itemize}
        \item ATaC: Given that greedy choices $g_1, g_2, \dots, g_m$ are not optimal, but some other optimal choices $o_1, o_2, \dots, o_m$ exist $(m > n)$. 
        \item By the way the greedy algorithm works, $f(g_i) \leq f(o_i)$ (f is the finish time) because it replaces based on the finish time. 
        \customFigure[0.75]{00_Images/POC.png}{WHAT IS THIS EXAMPLE SHOWING.}
        \begin{itemize}
            \item You can replace $o_1$ with $g_1$ by construction because $g_1$ finishes first. 
            \item You can replace $o_2$ with $g_2$ by constrcution 
            \item Up to $g_m$
            \item $o_{m+1} \ldots o_k$ cannot exist because the greedy algorithm didn't pick any. So we proved that the optimal solution is $G$
        \end{itemize}
        \item $o_{n+1},\dots, o_m$ cannot exist as greedy algorithm would take these into account, therefore, this implies a contradiction since the greedy algorithm would take these into account.
        \item Therefore, $O=G$ and $G=\text{optimal}$.
    \end{itemize}
\end{derivation}

\begin{warning}
    Usually use proof by contradiction. 
\end{warning}

\subsection{Knapsack Problem (Show Difference B/W DP and GA)}

\subsubsection{Problem setup}
\begin{intuition}
A thief breaks into a store and the thief's bag can carry up to $\omega$ weight. The store has $n$ items, each with weight $w_i$ and value $v_i$ of item $i$.
\end{intuition}

\subsubsection{Types of Knapsack Problems}
\begin{definition}
    \begin{itemize}
        \item \textbf{Fractional:} Can take part of an item (Greedy solution).
        \item \textbf{0-1:} Can take or not take an item (Dynamic programming).
    \end{itemize}
\end{definition}

\subsubsection{Fractional Knapsack}
\begin{definition}
For fractional knapsack:
\begin{itemize}
    \item Sort $\frac{v_i}{w_i}$ and keep taking items in decreasing order.
    \begin{itemize}
        \item \textbf{Intuition:} Take the most valuable per weight items until you reach the max weight that you can hold.
    \end{itemize}
\end{itemize}
\end{definition}

\subsubsection{0-1 Knapsack}
\begin{definition}
    For 0-1 knapsack, fix some order $1, \dots, n$ of the items. The recursive formula for 0-1 knapsack is:
    \[
    C[i, \omega] = \begin{cases} 
        0 & \text{if } i = 0 \text{ and } \omega = 0 \\
        C[i-1, \omega] & \text{if } \omega_i > \omega \\
        \max \{C[i-1, \omega - \omega_i] +C_i, \; C[i-1, \omega]\} & \text{if } \omega_i \leq \omega
    \end{cases}
    \]
    \begin{itemize}
        \item $C[i, \omega]$ denotes the optimal (most valuable) selection in $1\ldots i$ with $\omega$ as the leftover weight.
        \item First case is an artificial base case.
    \end{itemize}
\end{definition}

\begin{definition}
    $O(n\cdot \omega)$
\end{definition}

\subsection{Intro to Huffman Encoding}
\begin{intuition}
    \customFigure[0.75]{00_Images/I.png}{WHAT IS THIS SHOWING ME }
    \begin{itemize}
        \item 
    \end{itemize}
\end{intuition}

\subsubsection{Problem Statement}
\begin{intuition}
Huffman coding minimizes the size of the encoded file by assigning fewer bits to more frequent characters.
\begin{equation*}
    \sum_c f(c)d(c) = \text{total number of bits needed.}
\end{equation*}
\end{intuition}

\begin{example}
    \customFigure[0.75]{00_Images/CHARS.png}{WHAT IS THIS SHOWING ME?}
    \begin{itemize}
        \item 
    \end{itemize}
\end{example}

\subsubsection{Steps in Huffman Coding}
\begin{itemize}
    \item Write two letters with the lowest frequency in a tree-like fashion.
    \item Replace the two letters with a new character that sums their frequencies.
    \item Label the resulting tree.
    \customFigure[0.75]{00_Images/LG.png}{Tree}
\end{itemize}

\subsubsection{Proof of Correctness}
\begin{theorem}
Every optimal tree can have the two lowest frequency keys being siblings, that differ only on one bit, with the greatest depth.

\end{theorem}

IS THIS A PROOF?
\begin{derivation}
    Assume two lowest frequency keys $x$ and $y$ and $T''$ remains optimal. 
    \customFigure[0.75]{00_Images/E.png}{T optimal, T', and T''}
    \vspace{1em}

    \begin{enumerate}
        \item Let $B(T)$ be the sum of the frequencies and their bit-lengths:
        \[
        B(T) = \sum_c f(c) \cdot d(c)
        \]
        We want to show that $B(T') \leq B(T_{\text{opt}})$, where $T_{\text{opt}}$ is the optimal tree and $T'$ is a different tree (similar proof for $B(T'') \leq B(T')$)
        \vspace{1em}
    
        \item By the greedy principle:
        \[
        B(T_{\text{opt}}) - B(T') = \sum_c f(c)(d(c_{opt}) - \sum_c f(c) d(c_{gr}) \geq 0
        \]
        Thus, $B(T_{\text{opt}}) \geq B(T')$.
    \end{enumerate}
\end{derivation}

\subsubsection{Optimal Substructure}
WHAT IS THE WORKFLOW, HOW DID WE GET HERE. IS THIS A THEOREM?
\begin{theorem}
    \begin{enumerate}
        \item Let \( x \) $\leftarrow z \rightarrow$ \( y \) be part of \( T_{\text{opt}} \) for character set \( C \). Then
        \[
        T_{\text{opt}} = T_{\text{opt}} - \{x, y\} + \{z\}
        \]
        is optimal prefix code for new set \( C' = C - \{x, y\} + \{z\} \), where \( f(z') = f(x) + f(y) \).
    
        \item We have
        \[
        f(x) \cdot d_{T_{\text{opt}}}(x) + f(y) \cdot d_{T_{\text{opt}}}(y) = f(z) \cdot d_{T_{\text{opt}}}(z) + [f(x) + f(y)]
        \]
        because 
        \[
        d_{T_{\text{opt}}}(x) = d_{T_{\text{opt}}}(y) = d_{T_{\text{opt}}}(z) + 1.
        \]
    
        \item Every time we "move" from \( T_{\text{opt}} \) to \( T'_{\text{opt}} \) by merging 2 keys, we have
        \[
        \mathcal{B}(T_{\text{opt}}) = \mathcal{B}(T'_{\text{opt}}) + (f(x) + f(y)).
        \]
    
        \item ATaC \( T'_{\text{opt}} \) is not optimal for \( C' \), but \( T'' \) is. 
    
        \[
        \mathcal{B}(T'') + (f(x) f(y)) < \mathcal{B}(T'_{\text{opt}}) + (f(x) f(y)),
        \]
        which is more optimal than \( T_{\text{opt}} \), a contradiction.
    \end{enumerate}

\end{theorem}