\subsection{Problem vs. Problem Instance}
\begin{definition}
    \begin{itemize}
        \item A \textbf{problem} is a general question that we want to answer.
        \item A \textbf{problem instance} is a specific input to the problem.
    \end{itemize}
\end{definition}

\subsubsection{Hamiltonian Cycle Problem}
\begin{definition}
    A \textbf{Hamiltonian Cycle Problem} is defined as follows:
    \begin{quote}
    Given a graph, is there a simple path (i.e., no vertices are repeated) that traverses all vertices of the graph?
    \end{quote}
\end{definition}

\subsubsection{Godelization}
\begin{definition}
    \textbf{Gödelization} maps a problem into a set of strings $\{0, 1\}^*$. For example, it can map strings to represent graphs without cycles.
\end{definition}

\begin{example}
    The Hamiltonian Cycle problem can be represented as a set of strings $L_{HC}$
    \customFigure[0.75]{00_Images/NP.png}{Transformation to Godelization}
    \begin{itemize}
        \item \textbf{Note:} The circled part represents the Hamiltonian Cycle problem. Otherwise, $G$ with no cycles is not part of the Hamiltonian Cycle problem.
    \end{itemize}
\end{example}

\subsubsection{Optimization vs. Decision Problems}
\begin{definition}
    \begin{itemize}
        \item An \textbf{optimization problem} asks for the best solution.
        \item A \textbf{decision problem} asks for a yes/no answer.
    \end{itemize}
    \vspace{1em}
    \[
    \text{Decision} \leq \text{Optimization}
    \]
    \begin{itemize}
        \item Given a solution to an optimization problem, we can immediately solve the corresponding decision problem.
        \item If the decision problem is hard to solve, then the optimization problem is at least as hard.
    \end{itemize}
\end{definition}

\begin{example} \textbf{Max Flow Problem}:
    \begin{itemize}
        \item Decision version: Does the graph have a flow of size $\geq 20$?
        \item Optimization version: Find the maximum flow in the graph.
    \end{itemize}
\end{example}

\subsection{Turing Machines and Language Recognition}
\begin{definition}
    A Turing machine (TM) can be conceptualized equivalently as:
    \[
    \text{Turing Machine} \equiv \text{Software (sw)} \equiv \text{Hardware (hw)} \equiv \text{Algorithm (algo)} \equiv \text{Complexity}.
    \]
\end{definition}

\begin{example}
    The goal of the TM is to recognize strings $w$:
    \[
    w \in L_{HC} \, ? \overset{\text{TM}_{\text{HC}}}{\rightarrow} 
    \quad
    \begin{cases}
    y & \text{if } w \in L_{HC}, \\
    n & \text{if } w \notin L_{HC}.
    \end{cases}
    \]    
    where $L_{HC}$ is the Hamiltonian Cycle problem (i.e., the set of strings that represent graphs with Hamiltonian cycles since we used Goedelization).
    \begin{itemize}
        \item \textbf{TM Decides a Language $L$:} A TM \textbf{decides} $L$ if, after a finite number of steps, it halts with the correct answer.
        
        \item \textbf{TM Accepts a Language $L$:} A TM \textbf{accepts} $L$ if:
        \begin{itemize}
            \item Whenever the TM halts on input $w$, the answer is correct.
            \item But as long as the TM rns, there is no guarantee if it will ever halt or not (i.e. it may run forever).
        \end{itemize}
    \end{itemize}
\end{example}

\subsubsection{The Halting Problem}
\begin{definition}
    The \textbf{Halting Problem} is undecidable:
    \begin{quote}
    It is not possible to build a TM that decides whether another TM halts or not on a given input.
    \end{quote}
    (Turing, 1936)
\end{definition}

\subsection{Complexity Classes}
\subsubsection{Complexity Class P}
\begin{definition}
    \[
P = \{ L \subseteq \{0,1\}^* \mid \text{$\exists$ a polynomial-time algorithm to decide } L \}.
\]
\end{definition}

\subsubsection{Problem Verification}
\begin{definition}
    An algorithm $A$ \textbf{verifies} a problem $L$ if, given an instance $x \in L$ of the problem, there exists a certificate (or witness or candidate solution) $y$ such that $A(x, y) = 1$. The language verified is:
    \[
    L = \{ x \in \{0, 1\}^* \mid \exists y \in \{0, 1\}^* \text{ s.t. } A(x, y) = 1 \}.
    \]
\end{definition}

\subsubsection{Complexity Class NP}
\begin{definition}
    The class \textbf{NP} consists of problems that can be verified in polynomial time. Formally:
    \[
    L \in NP \iff \exists \text{ a polynomial-time algorithm } A \text{ and a constant } c \text{ s.t. } L \text{ contains the strings}
    \]
    \[
    L = \{ x \in \{0,1\}^* \mid \exists \text{ certificate } y \text{ with } |y| = O(|x|^c) \text{ such that } A(x, y) = 1 \}.
    \]
\end{definition}

\subsection{Complement of NP}
\begin{definition}
    If $L \in NP$, then the complement $\bar{L} \in \text{co-NP}$.
\end{definition}

\subsubsection{Relationships Between Complexity Classes}
\begin{theorem}
    \textbf{Theorem:}
\[
P \subseteq NP
\]
\begin{itemize}
    \item If a problem can be solved in polynomial time, it can also be verified in polynomial time.
\end{itemize}

\textbf{Theorem:}
\[
P = \text{co-}P.
\]

\customFigure[0.5]{00_Images/NP1.png}{Relationships Between Complexity Classes}
\end{theorem}

\subsection{Reducibility}
\begin{intuition}
    NP-complete languages/problems belong to NP and have the property that if we can solve one NP-complete problem in polynomial time, we can solve every problem in NP in polynomial time.
\end{intuition}
\begin{definition}
    \begin{itemize}
        \item \textbf{Informally:} If any instance of a problem $Q$ can be transformed into an instance of $Q'$ in polynomial time, the solution to which (i.e. for Q') provides a solution to the problem $Q$, then $Q$ is ``no more difficult'' than $Q'$.
    
        \item \textbf{Formally:} $L_1$ is polynomially reducible to $L_2$, denoted as $L_1 \leq_p L_2$, if there exists a polynomial-time algorithm $F$ such that:
        \[
        x \in L_1 \iff F(x) \in L_2
        \]
        If we can answer whether or not $x \in L_2$, we can answer whether or not $x \in L_1$ in polynomially more time.
    
        \customFigure[0.5]{00_Images/NP2.png}{Reducibility}
    \end{itemize}
\end{definition}

\begin{example}
    \begin{itemize}
        \item Let $Q = $ difference constraints.
        \item Let $Q' = $ single-source shortest path (SSSP).
    \end{itemize}
    In polynomial time, we can transform a difference constraints problem to an SSSP problem. Since SSSP is solvable in polynomial time, difference constraints are also solvable in polynomial time.
\end{example}

\subsection{NP-Complete Problems}
\begin{definition}
    \begin{itemize}
        \item $L \in NP-C \iff L \in NP$ and $\forall L' \in NP, L' \leq_p L$.
        \item \textbf{Theorem:} If $\exists L \in NP-C \cap P$, then $P = NP$.
        \item \textbf{Theorem:} $NP = CO-NP \iff \exists L \in NP-C \text{ such that } L \in NP$.
        \item \textbf{Proof Outline:}
        \begin{itemize}
            \item $(\Rightarrow)$ Straightforward.
            \item $(\Leftarrow)$ Assume $L \in NP-C$ and $\bar{L} \in NP$. Pick any $L' \in NP$ ($\bar{L'} \in CO-NP$).
            \begin{itemize}
                \item $L' \leq_p L$ and $\bar{L'} \leq_p \overline{L}$ implies that $L' \in NP$.
                \item Since $\bar{L'} \in CO-NP$, $\overline{L} \in NP$, so $\overline{L'}$ is no harder than $L$, which is NP, so $L'$ is NP.
            \end{itemize}
        \end{itemize}
    \end{itemize}
\end{definition}

\subsubsection{How to Prove $L \in NP-C$}
\begin{process} Every problem in NP reduces to $L'$, which reduces to $L$.
    \begin{enumerate}
        \item Show $L \in NP$: give certificate/verifier (can be verified in poly-time).
        \item Select a known $L' \in NP-C$ (if only 2, NP-hard).
        \begin{itemize}
            \item Show $L' \leq_p L$ ($x \in L'$ iff $f(x) \in L$).
            \item Show $f$ is polynomial-time.
        \end{itemize}
    \end{enumerate}
\end{process}

\begin{example}
\begin{enumerate}
    \item \textbf{Cook's Theorem: CIRCUIT-SAT is NPC}
    \begin{enumerate}
        \item Common Mistake:
        \begin{itemize}
            \item To show \(L \in \text{NP-C}, L \leq_p L'\), \(L' \in \text{NP-C}\).
            \item Reduction Chain:
            \[
            \text{3-SAT} \rightarrow \text{Clique} \rightarrow \text{Vertex Cover}
            \]
        \end{itemize}

        \item CIRCUIT-SAT reduces to Formula-SAT:
        \[
        \text{Formula-SAT} \rightarrow \text{Ham-Cycle} \rightarrow \text{TSP}
        \]
    \end{enumerate}

    \item \textbf{Formula-SAT}

    A formula \(\phi\) contains \(n\)-Boolean variables and connectives:
    \[
    \land, \lor, \lnot, \Rightarrow, \Leftrightarrow
    \]
    Example:
    \[
    \phi = (x_1 \land x_2) \lor (x_3 \Leftrightarrow x_4 \lor x_5)
    \]

    \item \textbf{Decision Problem}
    Given \(\phi\), is there a satisfying assignment to \(x_1, \ldots, x_n\) such that \(\phi = 1\)?

    \begin{enumerate}
        \item Show that Formula-SAT \(\in \text{NP}\):
        \begin{itemize}
            \item Just ``plug'' the binary variable values and evaluate \(\phi\) in polynomial time.
        \end{itemize}
        
        \item Show that CIRCUIT-SAT \(\leq_p\) Formula-SAT:
        \begin{itemize}
            \item Example: Circuit given in the diagram.
            \item \(\phi_d = \omega \Leftrightarrow (x_1 \land x_2) \land (z \Leftrightarrow (x_3 \lor x_4)) \land (y \Leftrightarrow (\omega \land z)) \land y\).
            \item If any \(x_1, \ldots, x_4 \Rightarrow y = 1\), the assignment also makes \(\phi = 1\) and vice versa.
            \item For each \(\lor, \land, \lnot\), introduce polynomial-time variables/directives, so time complexity is polynomial.
        \end{itemize}
    \end{enumerate}

    \item \textbf{Reduction to 3-SAT}

    \begin{itemize}
        \item Formula \(\phi\) is given in 3-Conjunctive Normal Form (CNF):
        \[
        \phi = (x_1 \lor x_2 \lor \lnot x_4) \land (x_2 \lor x_3 \lor \lnot x_5) \land (\lnot x_1 \lor x_4 \lor x_7) \land \ldots
        \]
        \item Each literal represents a variable or its negation.
        \item Each clause contains at most 3 literals.
    \end{itemize}

    \item \textbf{Conclusion}
    \begin{enumerate}
        \item 2-SAT \(\in \phi\).
        \item 2.5-SAT \(\in \text{NP-C}\).
    \end{enumerate}
    \end{enumerate}
\end{example}