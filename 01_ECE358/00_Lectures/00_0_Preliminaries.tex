\subsection{Algorithm}
\begin{definition}
    An algorithm is any well-defined computational procedure tht takes some value, or set of values, as input and produces some value, or set of values, as output in a finite amount of time. 
\end{definition}

\subsection{Pseudocode}
\begin{definition}
    A description of the steps in an algorithm using a mix of conventions of programming languages (like assignment operator, conditional operator, loop) with informal, usually self-explanatory, notation of actions and conditions.
\end{definition}

\begin{warning}
    \begin{itemize}
        \item Do not write C/Python/Java codes.
        \item Do not copy entire block of known codes (e.g. build-heap, heapsort) form textbook. You should abstract out the details.
    \end{itemize}
\end{warning}

\subsection{Exam}
\begin{warning}
    \begin{itemize}
        \item \textbf{Explanation:} Clearly explain your algorithms using English or Pseudocode + English. 
        \begin{itemize}
            \item You can reference your algorithm/comments for your proof of correctness.
        \end{itemize}
        \item \textbf{Algorithms from class:} You can assume its correctness and use it to prove other algorithms. 
        \begin{itemize}
            \item However, if you modify it, then you need to prove the new algorithm from scratch.
        \end{itemize}
        \item \textbf{Loops:} State what will happen before and after each loop execution is sufficient. 
    \end{itemize}
\end{warning}