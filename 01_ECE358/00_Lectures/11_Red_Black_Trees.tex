\subsection{Properties}
\begin{definition}
    Structurally balanced, i.e., \( h = O(\log n) \). RBT is a BST with the following properties:
    \begin{itemize}
        \item Every node is either red or black.
        \item The root is always black.
        \item A red node has black children.
        \item Every path from root to leaf has the same number of black nodes.
    \end{itemize}
\end{definition}

\subsubsection{Black height}
\begin{definition}
    The black height \( bh(x) \) is the number of black nodes on any path to a leaf, excluding \( x \).
    \customFigure[0.5]{00_Images/RBT.png}{Red black trees.}
\end{definition}

\begin{theorem}
    A Red-Black Tree with \( n \) internal nodes has 
    \begin{equation}
        h \leq 2 \log(n+1) = O(\log n)
    \end{equation}
\end{theorem}

\begin{theorem}
    \textbf{Lemma:} A subtree in a Red-Black Tree rooted at \( x \) has at least 
    \begin{equation}
        2^{bh(x)} - 1
    \end{equation}
    internal nodes.
\end{theorem}

\begin{derivation}
    Use induction on \( h \) to prove the theorem

    \begin{enumerate}
        \item  \textbf{Basis:} \( h = 0 \). Only one node, which is black, and \( bh(\text{node}) = 0 \), so we have 0 internal nodes:
        \[
        2^0 - 1 = 0 
        \]
    
        \item \textbf{Hypothesis:} Assume the lemma holds for all heights \( \leq h-1 \).
        
        \item \textbf{Step:} Suppose \( x \) is a red node. Then 
            \[
            bh(x) = bh(x.\text{child}) - 1
            \]
            If \( x \) is black, then
            \[
            bh(x) = bh(x.\text{child})
            \]
            Thus, \( x \) has internal nodes from both children.
            \vspace{1em}
            
            At least how many internal nodes does \( x \) have?
        
            \[
            \begin{aligned}
                & (2^{bh(x)-1} - 1) \quad \text{from the left subtree}, \\
                & (2^{bh(x)-1} - 1) \quad \text{from the right subtree}, \\
                & + 2 \quad \text{for the root and two children}.
            \end{aligned}
            \]
        
            Thus,
            \[
            2(2^{bh(x)-1}) - 1 = 2^{bh(x)} - 1 \quad \checkmark
            \]
        
        \item \textbf{Return to the theorem:}
            \[
            n \geq 2^{bh(\text{root})} - 1 \quad \text{(by the lemma)}.
            \]
        
            Thus,
            \[
            h \leq 2 \log(n+1) = O(\log n)
            \]
    \end{enumerate}
\end{derivation}

\subsection{Balance proof}

\subsection{Operations}