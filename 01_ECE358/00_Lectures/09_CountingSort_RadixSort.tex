\subsection{Lower bound on sorting and counting sort}
   
    \subsubsection{Comparison based sorting}
    \begin{definition}
        Only using comparisons between elements to gain information about the sequence. \\
        \hspace{0.5cm} $\hookrightarrow$ Can only use operations: $=$, $<$, $>$, $\leq$, $\geq$    
    \end{definition}
    \subsubsection{Lower bound on comparison-based sorting}
    \begin{definition}
        No \textbf{comparison-based} sorting algorithm on \textbf{unrestricted} range (i.e. any numbers) can do better than $\Omega(n\log(n))$.    
    \end{definition}

    \subsubsection{Decision tree}
    \begin{definition}
        Any sorting algorithm can be written as a decision tree. (Goes both ways)
    \end{definition}

    \begin{example}
        \customFigure[0.5]{00_Images/Insertion_Sort.png}{Insertion sort.}
    \end{example}

    \begin{theorem}
        Any decision tree for an n-element sorting algorithm has $h=\Omega (n\log n)$
    \end{theorem}

    \begin{derivation}
        \[
        \log(n!) \approx n \log n - n
        \]
        Using Stirling's approximation:
        \[
        n! \approx \sqrt{2\pi n} \left( \frac{n}{e} \right)^n
        \]
        Taking the logarithm of both sides:
        \[
        \log(n!) = n \log n - n + O(\log n)
        \]
        Therefore, \(\log(n!)\) is asymptotically bounded by:
        \[
        \log(n!) = \Theta(n \log n)
        \]
    \end{derivation}

\subsection{Counting sort}
\begin{definition}
    \begin{lstlisting}[language=Python, caption={Counting Sort Pseudocode}]
        C[i] = 0 for all i in [0...k]
        for j = 1 ... length(A) do  # O(n)
            C[A[j]] = C[A[j]] + 1
        for i = 1 ... k /* prefix sums */ # O(k)
            C[i] = C[i] + C[i-1]
        for j = length(A) ... 1 do # O(n)
            B[C[A[j]]] = A[j]
            C[A[j]] = C[A[j]] - 1 
    \end{lstlisting}
    \begin{itemize}
        \item $A$: array to be sorted. 
        \item \textbf{Sorting range:} Assume numbers in range $[0...k]$.
        \item \textbf{Stable sorting:} Not in-place but stable sorting. 
        \begin{itemize}
            \item \textbf{Stable:} Stable sorting in counting sort ensures that elements with equal values retain their relative order from the original input when sorted.
        \end{itemize}
        \item \textbf{Time complexity:} Time $O(n+k)$ if $O(k) = O(n)$ implies $O(n)$
        \item \textbf{Auxilliary arrays:} $C [0\ldots k]$ and $B [1 \ldots n]$.
    \end{itemize}
\end{definition}

\begin{example}
    \begin{enumerate}
        \item \textbf{Create the Input Array \( A \)}:
            \[
            A = [2, 5, 3, 0, 2, 3, 0, 3]
            \]
        
        The elements of \( A \) are integers within a known range (e.g., \( 0 \) to \( 5 \)).

        
        \item \textbf{Count the Occurrences (Frequency Array) (i.e. 1st loop)}:
        \begin{itemize}
            \item Create an auxiliary array \( C \), where each index represents a possible value in \( A \), and the value at each index in \( C \) represents the number of times that value appears in \( A \).
            \item Initialize \( C \) with zeros:
            \[
            C = [0, 0, 0, 0, 0, 0]
            \]
            \item Count the occurrences of each number in \( A \), updating \( C \):
            \[
            C = [2, 0, 2, 3, 0, 1]
            \]
            This tells us:
            \begin{itemize}
                \item \( 0 \) appears 2 times.
                \item \( 1 \) appears 0 times.
                \item \( 2 \) appears 2 times.
                \item \( 3 \) appears 3 times.
                \item \( 5 \) appears 1 time.
            \end{itemize}
        \end{itemize}
        
        \item \textbf{Modify \( C \) to Store the Cumulative Count (i.e. 2nd loop)}:
        \begin{itemize}
            \item Modify the array \( C \) such that each index contains the sum of the counts for all values less than or equal to that index.
            \item The cumulative count helps to determine the final positions of each element in the sorted array.
            \item After modification, \( C \) becomes:
            \[
            C = [2, 2, 4, 7, 7, 8]
            \]
            This means:
            \begin{itemize}
                \item Values \( \leq 0 \) will be placed before index 2.
                \item Values \( \leq 2 \) will be placed before index 4.
                \item Values \( \leq 3 \) will be placed before index 7.
                \item Values \( \leq 5 \) will be placed before index 8.
            \end{itemize}
        \end{itemize}
        
        \item \textbf{Place the Elements in the Sorted Array \( B \) (i.e. 3rd loop)}:
        \begin{itemize}
            \item Iterate through the elements in \( A \), using the cumulative counts in \( C \) to place each element in the correct position in \( B \).
            \item Decrement the corresponding count in \( C \) for each placement to ensure the stability of sorting.
            \item The sorted array \( B \) is:
            \[
            B = [0, 0, 2, 2, 3, 3, 3, 5]
            \]
            \item \textbf{Example of the third loop:} $B(C(A(8)))=B(C(3))=B(7)=A(8)$, so we place the 8th element in A (e.g. 8) into B(7) (i.e. 7th index)
            \vspace{1em}

            $B(C(A(7)))=B(C(0))=B(2)=A(7)$, so we place the 7th element in A (e.g. 0) into B(2) (i.e. 2nd index), 
        \end{itemize}
    \end{enumerate}
    \customFigure[0.5]{00_Images/Counting_Sort.png}{Example of counting sort}
\end{example}

\subsection{Radix sort}
\begin{definition}
    \begin{lstlisting}[language=Python, caption={Radix Sort Pseudocode}]
        for i = Least significant bit (LSB) -> Most significant bit (MSB)
            counting_sort(digit) # or any stable sorting algorithm on digit i
    \end{lstlisting}
\end{definition}

\begin{example}
    The numbers in the image are sorted digit by digit, starting with the least significant digit. 

    \begin{enumerate}
        \item \textbf{Sort by the Least Significant Digit (LSD)}:
        \begin{itemize}
            \item Start by examining the rightmost digit (the units place) of each number.
            \item In the image, the numbers:
            \[
            [512, 413, 242, 375, 695, 112]
            \]
            are sorted by their least significant digits:
            \begin{align*}
            512 & \rightarrow 2, \\
            413 & \rightarrow 3, \\
            242 & \rightarrow 2, \\
            375 & \rightarrow 5, \\
            695 & \rightarrow 5, \\
            112 & \rightarrow 2.
            \end{align*}
            Sorting by the rightmost digits results in:
            \[
            [512, 242, 112, 413, 375, 695]
            \]
        \end{itemize}
        
        \item \textbf{Sort by the Next Significant Digit (Tens place)}:
        \begin{itemize}
            \item After sorting by the least significant digit, now sort the numbers based on the second digit (the tens place).
            \item The current sequence of numbers:
            \[
            [512, 242, 112, 413, 375, 695]
            \]
            is sorted by the tens digit:
            \begin{align*}
            512 & \rightarrow 1, \\
            242 & \rightarrow 4, \\
            112 & \rightarrow 1, \\
            413 & \rightarrow 1, \\
            375 & \rightarrow 7, \\
            695 & \rightarrow 9.
            \end{align*}
            Sorting by the tens place results in:
            \[
            [512, 112, 413, 242, 375, 695]
            \]
        \end{itemize}
        
        \item \textbf{Sort by the Most Significant Digit (Hundreds place)}:
        \begin{itemize}
            \item Finally, sort by the hundreds digit. The current sequence:
            \[
            [512, 112, 413, 242, 375, 695]
            \]
            is sorted by the hundreds digit:
            \begin{align*}
            512 & \rightarrow 5, \\
            112 & \rightarrow 1, \\
            413 & \rightarrow 4, \\
            242 & \rightarrow 2, \\
            375 & \rightarrow 3, \\
            695 & \rightarrow 6.
            \end{align*}
            Sorting by the hundreds place results in the final sorted sequence:
            \[
            [112, 242, 375, 413, 512, 695]
            \]
        \end{itemize}
        
    \end{enumerate}

    \customFigure[0.5]{00_Images/Radix_Sort.png}{Radix sort example}
\end{example}

\subsubsection{Runtime complexity analysis of Radix sort}
\begin{definition}
    \begin{itemize}
        \item \textbf{Variables:}
        \begin{itemize}
            \item n: \# numbers 
            \item r: range of numbers 
            \item d: \# digits
        \end{itemize}

        \item \textbf{One pass complexity:} $=O(n+r)$ (i.e. the time complexity of counting sort)
        \item \textbf{All passes complexity:} $dO(n+r) = O(dn + dr)$ (i.e. sorting all digits, so it's the time complexity of counting sort times the number of digits)
        \item \textbf{r=d complexity:} If $r=d=O(1)$, then $O(n)$ true for all passes.
    \end{itemize}
\end{definition}

\subsubsection{Example of QS vs. RS}
\begin{example}
    Sorting $1000\#$'s, each represented using 64 bits. Each digit is 16 bits wide. 
    \begin{itemize}
        \item \textbf{Quicksort:} $O(n \log(n)) = \frac{O(1000 \log(1000))}{1000}$ (i.e. divided by 1000 to show the average time compared to Radix sort)
        \begin{itemize}
            \item $\log(1000) \approx 10$ passes./number
        \end{itemize} 
        \item \textbf{Radix sort:} $4 \text{ passes/number}$ (i.e. one for each digit)
    \end{itemize}
    \customFigure[0.5]{00_Images/Bits.png}{Number of passes for 1000\#'s 64 bits.}
\end{example}

\subsection{Tournament tree selection}
    \subsubsection{What is the kth largest element in unsorted sequence}
    \begin{definition}
        How many comparisons to find max element? $(n-1)$ 
        \begin{itemize}
            \item \textbf{Max Comparisons:} $n-1$: It's the winner (i.e. root) of the tournament tree. Don't need to check the $nth$ element because it is the biggest.
            \begin{itemize}
                \item \textbf{Note:} Don't need to check the $nth$ element because it is the biggest.
            \end{itemize}
            \item \textbf{Min Comparisons:} $n/2 - 1$: Take the losers (i.e. leaves) of the tournament to find the minimum (assume that we found the maximum already). Since $n/2$ only contains the leaves and we know that the last loser will be the minimum, therefore, $n/2 - 1$ comparisons.
            \item \textbf{2nd Max Comparisons:} $\log n - 1$ additional comparisons after finding the maximum by comparing the max elements battles with other elements (i.e. 2nd biggest loses to max so follow max route). Since the height of the tree is $\log(n)$ and we know the max already, therefore, $\log(n) - 1$
            \item \textbf{Kth Max comparisons:} Total comparisons is the $O((n-1) + k\log n)=O(n+k\log(n))$
            \begin{itemize}
                \item $O(n-1)$: Cost of building the tournament tree by comapring all elements. 
                \item $O(k\log(n))$: Additional comparisons required to identify subsequent maximums. At each level, there are $\log(n)$ comparisons.
            \end{itemize}
        \end{itemize}
        \customFigure[0.5]{00_Images/TT.png}{Tournament tree.}
    \end{definition}

    \subsubsection{Selecting the kth element in O(n) expected time}
    \begin{example}
        How to find the $kth$ element without sorting the array for finding a certain element. For finding a specific element, it is $O(n)$ expected time. Quickselect is a variant of Quicksort used to find the
        k-th smallest or largest element without fully sorting the array.            
        \begin{enumerate}
            \item \textbf{Initial Array:} The original unsorted array is:
            \[
            [8, 1, 5, 9, 2, 10, 4, 17, 20]
            \]
            We are tasked with finding the 7th smallest element when the array is sorted.
            
            \item \textbf{Choosing a Pivot:} In the Quickselect algorithm, the first pivot element chosen is `8`.
            
            \item \textbf{Partitioning Around Pivot:} The array is partitioned into two parts:
            \begin{itemize}
                \item Elements smaller than or equal to `8` are moved to the left.
                \item Elements greater than `8` are moved to the right.
            \end{itemize}
            After partitioning, the array becomes:
            \[
            [1, 5, 2, 4, 8, 9, 10, 17, 20]
            \]
            The pivot `8` is in its correct sorted position (5th position in the array).
            
            \item \textbf{Comparing Pivot Position with Target:} Since we are looking for the 7th smallest element, and `8` is in the 5th position, the desired element must be in the subarray to the right of `8`:
            \[
            [9, 10, 17, 20]
            \]
            Now we search for the 2nd smallest element in this subarray because it corresponds to the 7th smallest element in the overall array.
            
            \item \textbf{Recursive Partitioning:} Another pivot is chosen from the remaining subarray, let’s assume it’s `9`. After partitioning:
            \[
            [9, 10, 17, 20]
            \]
            The pivot `9` is in the 1st position of this subarray (6th position in the overall array).
            
            \item \textbf{Final Step:} Now, we are looking for the 1st smallest element in the subarray:
            \[
            [10, 17, 20]
            \]
            After partitioning, the smallest element is `10`, which is the 7th smallest element in the entire array.
        \end{enumerate}        
    \end{example}