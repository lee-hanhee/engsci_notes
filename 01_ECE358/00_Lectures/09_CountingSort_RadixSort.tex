\subsection{Lower bound on sorting and counting sort}
    \subsubsection{Lower bound on comparison-based sorting}
    \begin{definition}
        No \textbf{comparison-based} sorting algorithm on \textbf{unrestricted} range (i.e. any numbers) can do better than $\Omega(n\log(n))$.    
    \end{definition}

\subsection{Counting sort}
\begin{definition}
    \begin{lstlisting}[language=Python, caption={Counting Sort Pseudocode}]
        C[i] = 0 for all i in [0...k]
        for j = 1 ... length(A) do  # O(n)
            C[A[j]] = C[A[j]] + 1
        for i = 1 ... k /* prefix sums */ # O(k)
            C[i] = C[i] + C[i-1]
        for j = length(A) ... 1 do # O(n)
            B[C[A[j]]] = A[j]
            C[A[j]] = C[A[j]] - 1 
    \end{lstlisting}
    \begin{itemize}
        \item $A$: array to be sorted. 
        \item Assume numbers in range $[0...k]$.
        \item \textbf{Stable sorting:} Not in-place but stable sorting. 
        \item \textbf{Time complexity:} Time $O(n+k)$ if $O(k) = O(n)$ implies $O(n)$
        \item \textbf{Additional arrays:} $C [0\ldots k]$ and $B [1 \ldots n]$.
    \end{itemize}
\end{definition}

\begin{example}
    \begin{enumerate}
        \item \textbf{Create the Input Array \( A \)}:
            \[
            A = [2, 5, 3, 0, 2, 3, 0, 3]
            \]
        
        The elements of \( A \) are integers within a known range (e.g., \( 0 \) to \( 5 \)).

        
        \item \textbf{Count the Occurrences (Frequency Array)}:
        \begin{itemize}
            \item Create an auxiliary array \( C \), where each index represents a possible value in \( A \), and the value at each index in \( C \) represents the number of times that value appears in \( A \).
            \item Initialize \( C \) with zeros:
            \[
            C = [0, 0, 0, 0, 0, 0]
            \]
            \item Count the occurrences of each number in \( A \), updating \( C \):
            \[
            C = [2, 0, 2, 3, 0, 1]
            \]
            This tells us:
            \begin{itemize}
                \item \( 0 \) appears 2 times.
                \item \( 1 \) appears 0 times.
                \item \( 2 \) appears 2 times.
                \item \( 3 \) appears 3 times.
                \item \( 5 \) appears 1 time.
            \end{itemize}
        \end{itemize}
        
        \item \textbf{Modify \( C \) to Store the Cumulative Count}:
        \begin{itemize}
            \item Modify the array \( C \) such that each index contains the sum of the counts for all values less than or equal to that index.
            \item The cumulative count helps to determine the final positions of each element in the sorted array.
            \item After modification, \( C \) becomes:
            \[
            C = [2, 2, 4, 7, 7, 8]
            \]
            This means:
            \begin{itemize}
                \item Values \( \leq 0 \) will be placed before index 2.
                \item Values \( \leq 2 \) will be placed before index 4.
                \item Values \( \leq 3 \) will be placed before index 7.
                \item Values \( \leq 5 \) will be placed before index 8.
            \end{itemize}
        \end{itemize}
        
        \item \textbf{Place the Elements in the Sorted Array \( B \)}:
        \begin{itemize}
            \item Iterate through the elements in \( A \), using the cumulative counts in \( C \) to place each element in the correct position in \( B \).
            \item Decrement the corresponding count in \( C \) for each placement to ensure the stability of sorting.
            \item The sorted array \( B \) is:
            \[
            B = [0, 0, 2, 2, 3, 3, 3, 5]
            \]
        \end{itemize}
    \end{enumerate}
    \customFigure[0.5]{00_Images/Counting_Sort.png}{Example of counting sort}
\end{example}

\subsection{Radix sort}
\begin{definition}
    \begin{lstlisting}[language=Python, caption={Radix Sort Pseudocode}]
        for i = Least significant bit (LSB) -> Most significant bit (MSB)
            counting_sort(digit)
    \end{lstlisting}
\end{definition}

\begin{example}
    The numbers in the image are sorted digit by digit, starting with the least significant digit. 

    \begin{enumerate}
        \item \textbf{Sort by the Least Significant Digit (LSD)}:
        \begin{itemize}
            \item Start by examining the rightmost digit (the units place) of each number.
            \item In the image, the numbers:
            \[
            [512, 413, 242, 375, 695, 112]
            \]
            are sorted by their least significant digits:
            \begin{align*}
            512 & \rightarrow 2, \\
            413 & \rightarrow 3, \\
            242 & \rightarrow 2, \\
            375 & \rightarrow 5, \\
            695 & \rightarrow 5, \\
            112 & \rightarrow 2.
            \end{align*}
            Sorting by the rightmost digits results in:
            \[
            [512, 242, 112, 413, 375, 695]
            \]
        \end{itemize}
        
        \item \textbf{Sort by the Next Significant Digit (Tens place)}:
        \begin{itemize}
            \item After sorting by the least significant digit, now sort the numbers based on the second digit (the tens place).
            \item The current sequence of numbers:
            \[
            [512, 242, 112, 413, 375, 695]
            \]
            is sorted by the tens digit:
            \begin{align*}
            512 & \rightarrow 1, \\
            242 & \rightarrow 4, \\
            112 & \rightarrow 1, \\
            413 & \rightarrow 1, \\
            375 & \rightarrow 7, \\
            695 & \rightarrow 9.
            \end{align*}
            Sorting by the tens place results in:
            \[
            [112, 512, 413, 242, 375, 695]
            \]
        \end{itemize}
        
        \item \textbf{Sort by the Most Significant Digit (Hundreds place)}:
        \begin{itemize}
            \item Finally, sort by the hundreds digit. The current sequence:
            \[
            [112, 512, 413, 242, 375, 695]
            \]
            is sorted by the hundreds digit:
            \begin{align*}
            112 & \rightarrow 1, \\
            512 & \rightarrow 5, \\
            413 & \rightarrow 4, \\
            242 & \rightarrow 2, \\
            375 & \rightarrow 3, \\
            695 & \rightarrow 6.
            \end{align*}
            Sorting by the hundreds place results in the final sorted sequence:
            \[
            [112, 242, 375, 413, 512, 695]
            \]
        \end{itemize}
        
    \end{enumerate}

    \customFigure[0.5]{00_Images/Radix_Sort.png}{Radix sort example}
\end{example}

\subsubsection{Runtime complexity analysis of Radix sort}
\begin{definition}
    \begin{itemize}
        \item \textbf{Variables:}
        \begin{itemize}
            \item n: \# numbers 
            \item r: range of numbers 
            \item d: \# digits
        \end{itemize}

        \item \textbf{One pass complexity:} $=O(n+r)$ (i.e. the time complexity of counting sort)
        \item \textbf{All passes complexity:} $dO(n+r) = O(dn + dr)$ (i.e. sorting all digits, so it's the time complexity of counting sort times the number of digits)
        \item \textbf{r=d complexity:} If $r=d=O(1)$, then $O(n)$ true.
    \end{itemize}
\end{definition}

\subsubsection{Example of QS vs. RS}
\begin{example}
    Sorting $1000\#$'s, each represented using 64 bits. Each digit is 16 bits wide. 
    \begin{itemize}
        \item \textbf{Quicksort:} $O(n \log(n)) = \frac{O(1000 \log(1000))}{1000}$ (i.e. divided by 1000 to show the average time compared to Radix sort)
        \begin{itemize}
            \item $\log(1000) \approx 10$ passes./number
        \end{itemize} 
        \item \textbf{Radix sort:} $4 \text{ passes/number}$ (i.e. one for each digit)
    \end{itemize}
    \customFigure[0.5]{00_Images/Bits.png}{Number of passes for 1000\#'s 64 bits.}
\end{example}

\subsection{Find kth largest element in sorted sequence}
\begin{definition}
    \begin{enumerate}
        \item How many comparisons to find max element? $(n-1)$ 
        \begin{itemize}
            \item \textbf{Note:} Don't need to check the $nth$ element because it is the biggest.
            \item \textbf{Max:} $n-1$
            \item \textbf{Min:} $n/2 - 1$
        \end{itemize}
        
    \end{enumerate}
\end{definition}