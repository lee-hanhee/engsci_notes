\subsection{Definition of Single Source Shortest Paths (SSSPs)}
\begin{definition}
    Given a connected, directed, weighted (positive or negative) graph, find the shortest path (i.e. SP) from start vertex \( S \) (i.e. source) to \(\forall\) vertex in $V$. (e.g., cities and air mileage).

    \begin{itemize}
        \item \textbf{SPs} (Shortest Paths) may not be unique.
    \end{itemize}

    \customFigure[0.75]{00_Images/SP.png}{Shortest Path}
    \begin{itemize}
        \item \textbf{Weights:} Cost, distance, penalities, delays, profits, etc, which can be positive or negative, but no negative cycles.  
        \item \textbf{Source to Top Right Node:} $3 + 6 = 9$
    \end{itemize}
\end{definition}

\subsubsection{Graph-based Formulation}
\begin{definition}
    Given a weighted (+ or -), directed graph G, find $\delta(u,v)$ s.t.
    \[
    \delta(u, v) = \begin{cases} 
        \min \{ w(p) : u  \overset{P}{\rightarrow} v \} & \text{if } \exists \, \text{path } p \, u \overset{p}{\rightarrow} v \\
        \infty & \text{if } \nexists \, \text{path } p \, u \overset{p}{\rightarrow} v
    \end{cases}
    \]
\end{definition}

\subsection{Topics of Study}
\begin{summary}
    We will study:
    \begin{itemize}
        \item \textbf{Dijkstra's Algorithm} (for graphs with non-negative weights).
        \item \textbf{SPs on DAGs}
        \item \textbf{Bellman-Ford Algorithm} (handles negative weights and reports negative cycles).
        \item \textbf{Application to Difference Constraints}.
    \end{itemize}
\end{summary}

\subsubsection{Variations}
\begin{summary}
    \begin{itemize}
        \item Single source SP (What we study).
        \item Single destination SP (Solution: reverse edges and run single source SP).
        \item Single pair SP (No better than SSSP).
        \item All pairs SP.
    \end{itemize}
    \begin{enumerate}
        \item First 3 are greedy 
        \item Last one is dynamic programming
    \end{enumerate}
\end{summary}

\subsection{Edge Relaxation}
\begin{definition}
    Both Dijkstra's and Bellman-Ford algorithms use \textbf{edge relaxation}.
    \begin{itemize}
        \item \textbf{Intuition:} Both algorithms keep an estimate $d[u]$ of $\delta(s,u)$ and iteratively they improve that estimate until they converge \& $d[u] = \delta(s,u)$.
    \end{itemize}
    \vspace{1em}

    \textbf{RELAX Methodology:} Keep track of this quantity \( d[v] \), which is the best estimate you got for the shortest path (SP) from \( s \overset{P}{\rightarrow} v \). They \textbf{relax} edges, trying to improve that estimate.

\begin{lstlisting}[mathescape=true]
Relax($u, v, w(u, v)$)   # Note: $(u, v) \in E$
    if $d[v] > d[u] + w(u, v)$ then
        $d[v] = d[u] + w(u, v)$
        $\pi(v) = u$   # parent of $v$ in SP is $u$
\end{lstlisting}
        
    \begin{itemize}
        \item \textbf{Note:} Always \( d[V] \geq \delta(S, V) \).
    \end{itemize}
    \begin{itemize}
        \item Edge relaxation is a process used in shortest path algorithms.
        \item For each edge \((u, v)\) with weight \(w(u, v)\), we check if going through \(u\) offers a shorter path to \(v\) than previously known.
        \item If \(d[v] > d[u] + w(u, v)\):
        \begin{itemize}
            \item We update \(d[v]\) to \(d[u] + w(u, v)\), reflecting the shorter path estimate to \(v\).
            \item We set \(\pi(v) = u\), making \(u\) the predecessor of \(v\) in the shortest path tree.
        \end{itemize}
        \item This process is repeated for all edges, gradually refining shortest path estimates.
    \end{itemize}
    

    \customFigure[0.75]{00_Images/SP1.png}{Edge relaxation.}
    \begin{itemize}
        \item \textbf{Note:} The idea is to relax edges until you get the shortest path.
    \end{itemize}
    
    \customFigure[0.75]{00_Images/SP2.png}{Example of edge relaxation: Relaxation from 10 to 8 and non-relaxation cases.}
\end{definition}

\subsection{Dijkstra's Algorithm (no negative weights)}
\begin{definition}
    \textbf{Time Complexity:} \( O(E \log(V)) \) due to topological sort. 

    \begin{enumerate}
        \item Initialize \( d[V] = \infty \) for all \( v \in V \).
        \item Set \( d[s] = 0 \).
        \item \( Q = V \) \quad (\( Q \) is a priority queue).
        \item While \( Q \neq \emptyset \): $O(v \log v)$
        \begin{enumerate}
            \item \( u = \text{extract\_min}(Q) \). $O( \log v)$
            \item \( S = S \cup \{ u \} \).
            \item For each vertex adjacent to \( u \), do: $O(E \log v)$
            \begin{itemize}
                \item Relax \( (u, v, w(u, v)) \) $O(\log v)$
            \end{itemize}
        \end{enumerate}
    \end{enumerate}
    \begin{itemize}
        \item \textbf{Extract Min:} $O(\log v)$
        \item \textbf{While Loop:} $O(v \log v)$
        \item \textbf{Relax:} $O(\log v)$
        \item \textbf{For loop:} $O(E \log v)$
        \item \textbf{Fibonnaci Heaps Amortized:} $O(v \log v)$
        \item \textbf{Note:} Implement with a min-heap (priority queue, extract min).
    \end{itemize}
\end{definition}

\subsection{SSSPs on DAGs (Single Source Shortest Paths on Directed Acyclic Graphs)}
\begin{definition}
    \textbf{Weights:} Can be positive or negative \\
    \textbf{Note:} No cycles; guarantees shortest path.

    \customFigure[0.75]{00_Images/SP3.png}{Example of single source shortest paths on DAG with positive and negative weights.}

\begin{lstlisting}[mathescape=true]
# Shortest Path Algorithm using Relaxation in Topological Order

1. Initialize $\forall u \in V, d[u] = \infty$, $d[s] = 0$.
2. Topologically sort vertices.
3. From left to right, relax the edges.
4. For each $(u, v)$ in the topological order, Relax $(u, v, w(u, v))$.
\end{lstlisting}
\begin{itemize}
    \item \textbf{Time Complexity:} \( O(V + E) \) due to topological sort.
\end{itemize}
\end{definition}

\subsection{Bellman-Ford Algorithm (Negative Weights)}
\begin{summary}
    \begin{itemize}
        \item \( \delta(u, v) \): The true shortest path length from \( u \) to \( v \).
        \item \( d(u, v) \): The current estimate of \( \delta(u, v) \).
        \item \( d(u, v) \geq \delta(u, v) \), meaning the estimate is initially greater than or equal to the true shortest path.
    \end{itemize}        
\end{summary}

\begin{definition}
\begin{lstlisting}[mathescape=true]
for $1 \dots (|V| - 1)$:
    for $\forall (u, v) \in E$ with weight $w$:
        Relax($u, v, w$)

for $\forall (u, v) \in E$:
    if $d[v] > d[u] + w(u, v)$:
        return False  # negative cycle detected
\end{lstlisting} 
\begin{itemize}
    \item \textbf{Time Complexity:} \( O(VE) \) due to nested loops.
\end{itemize}
\end{definition}

\begin{example}
    Add an example for intuition
\end{example}

\begin{intuition}
    Let us say the shortest path from \( v_1 \) to \( v_n \) is \( v_1, v_2, \dots, v_n \).

    \begin{itemize}
        \item This path consists of edges: \( (v_1, v_2), (v_2, v_3), \dots, (v_{n-1}, v_n) \).
        
        \item If we relax these edges in that order (even with other relaxations in between), we obtain the correct shortest path lengths.
        
        \item The first iteration relaxes every edge (including \( (v_1, v_2) \)).
        
        \item The \( i \)-th iteration relaxes every edge (including \( (v_i, v_{i+1}) \)).
        
        \item Thus, after \( |V| - 1 \) iterations, the needed sequence has occurred.
    \end{itemize}
\end{intuition}

\subsubsection{Theorem}
\begin{theorem}
    Bellman-Ford (BF) correctly reports no solution if there is a negative cycle.
\end{theorem}

\begin{derivation}
    \textbf{ATaC:} Suppose there exists a negative cycle \( v_1, \dots, v_k \) where \( v_1 = v_k \). In other words, 
    \[
    \sum_{i=2}^{k} w(v_{i-1}, v_i) < 0.
    \]

    However, if BF does not report \texttt{FALSE}, this means the "if" condition for detecting a negative cycle didn't trigger.

    Using telescoping:

    \[
    d[v_2] \leq d[v_1] + w(v_1, v_2)
    \]
    \[
    d[v_3] \leq d[v_2] + w(v_2, v_3)
    \]
    \[
    d[v_4] \leq d[v_3] + w(v_3, v_4)
    \]
    \[
    \vdots
    \]
    \[
    d[v_k] \leq d[v_{k-1}] + w(v_{k-1}, v_k)
    \]

    Adding these inequalities together, we get:
    \[
    0 \leq \sum_{i=2}^{k} w(v_{i-1}, v_i).
    \]

    This contradicts the assumption that the cycle has a negative weight. Thus, BF correctly reports \texttt{FALSE} if there is a negative cycle.
    \customFigure[0.75]{00_Images/NC.png}{Negative cycle.}
\end{derivation}

\subsection{Application to Difference Constraints Through Integer Programming}
\begin{definition}
    Given matrices \( A \in \mathbb{R}^{m \times n} \), \( B \in \mathbb{R}^{m \times 1} \), and \( C \in \mathbb{R}^{1 \times n} \), we want to find \( x_1, \dots, x_n \) (i.e. integer solutions) such that it satisfies the constraint map:
    \[
    \begin{bmatrix}
    a_{11} & a_{12} & \cdots & a_{1n} \\
    a_{21} & a_{22} & \cdots & a_{2n} \\
    \vdots & \vdots & \ddots & \vdots \\
    a_{m1} & a_{m2} & \cdots & a_{mn}
    \end{bmatrix}
    \begin{bmatrix}
    x_1 \\
    x_2 \\
    \vdots \\
    x_n
    \end{bmatrix}
    \leq
    \begin{bmatrix}
    b_1 \\
    b_2 \\
    \vdots \\
    b_m
    \end{bmatrix}
    \]
    while maximizing (or minimizing) the objective function:
    \[
    \begin{bmatrix}
    c_1 & c_2 & \cdots & c_n
    \end{bmatrix}
    \begin{bmatrix}
    x_1 \\
    x_2 \\
    \vdots \\
    x_n
    \end{bmatrix}
    = c_1 x_1 + c_2 x_2 + \dots + c_n x_n.
    \]
    \begin{itemize}
        \item \textbf{Algorithm:} Simplix(exp)
        \item \textbf{Algorithm:} Ellipsoid(Poly)
        \item \textbf{Note:} In integer programming, all \( x \in \mathbb{Z} \).
    \end{itemize}
\end{definition}


\subsubsection{Difference Constraints}
\begin{example}
Difference constraints are often tested on exams. They take the form:
\begin{align*}
x_1 - x_2 &\leq 5, \\
x_1 - x_3 &\leq 6, \\
x_2 - x_4 &\leq -1, \\
x_3 - x_4 &\leq -2, \\
x_4 - x_1 &\leq -3.
\end{align*}
\begin{itemize}
    \item \textbf{Note:} No objective function, if there is a solution, then there are infinite solutions.
\end{itemize}

\textbf{Algorithm: Constraint Graph}
    To solve these constraints:
    \begin{itemize}
        \item Build a "constraint graph:" Create vertices \( v_0, \dots, v_n \), where each vertex represents a variable, plus a pseudo source \( v_0 \).
        \item For each constraint, add one edge per constraint. The edge set \( E = \{ (v_i, v_j) \} \) with weight \( b_k \) is defined such that \( x_j - x_i \leq b_k \).
        \item Add edges with zero weight from the pseudo source \( v_0 \) to all other vertices.
        \item Run the Bellman-Ford algorithm from \( v_0 \).
        \begin{itemize}
            \item If there is a negative cycle, then there is no solution.
        \end{itemize}
    \end{itemize}
    \vspace{1em}

    We \textbf{reduce DC to SSSP} such that if there is a solution to SSSP, we obtain a solution to DC.

    \begin{itemize}
        \item Introduce one vertex for each variable.
        \item Add a pseudo source vertex \( v_0 \).
        \item For each inequality constraint, add an edge with the corresponding weight.
        \item Add pseudo edges with weight zero from \( v_0 \) to all other vertices.
        \item Relax all edges in the graph.
    \end{itemize}
    \vspace{1em}

    An example of the constraint graph is shown below:

    \customFigure[0.75]{00_Images/BF.png}{Example of a constraint graph.}

    This builds the graph based on difference constraints, setting up the edges and weights to represent inequalities and running the Bellman-Ford algorithm to find feasible solutions.
\end{example}