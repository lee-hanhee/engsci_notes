\subsection{Intro}
\begin{summary}
    \begin{itemize}
        \item Let \( G = (V, E) \) represent a graph, where \( V \) is the set of vertices and \( E \) is the set of edges.
        \item Edge complexity: \( E = O(V^2) \).
        \item Types of Graphs:
        \begin{itemize}
            \item Directed / Undirected Graphs
            \item Weighted / Unweighted Graphs
            \item Cliques (Complete Graphs)
            \item Directed Acyclic Graphs (DAGs)
            \item Paths / Cycles (Simple — no repeated vertex except for start, perhaps)
        \end{itemize}
    \end{itemize}
\end{summary}

\subsection{Breadth-first search}
\begin{definition}
    Traverse vertices by sending a wave and visit what the wave passes through. 
    \begin{itemize}
        \item \textbf{Time complexity:} $O(V + E)$
        \begin{itemize}
            \item \textbf{Note:} Edge or vertices will dominate.
        \end{itemize}
        \item \textbf{Implementation:} Implement with first-in-first-out.
    \end{itemize}
\end{definition}

\begin{example}
    \customFigure[0.75]{00_Images/BFS.png}{BFS.}
\end{example}

\subsection{Depth-first search}
\begin{definition}
\begin{lstlisting}[mathescape=true]
DFS(G=(V, E)):
    for each $u \in V$
        color(u) = white
    time = 0
    
    for each $u \in V$
        if color(u) = white
            DFS_VISIT(u)

DFS_VISIT(u):
    color(u) = gray
    time += 1
    d[u] = time # discovery
    
    for each $v \in Adjacent(u)$
        if color(v) = white
            DFS_VISIT(v)
    
    color(u) = black
    time += 1
    f(u) = time # finish
\end{lstlisting}
\begin{itemize}
    \item \textbf{Time compelxity:} $O(V+E)$
    \item \textbf{d/f:} discovery and finish time. 
\end{itemize}

\end{definition}

\subsubsection{Well Parenthesis Theorem}
\begin{theorem}
    In a DFS of an undirected graph, every edge is tree and back edge (i.e. no cross edges and forward edges)
    \begin{itemize}
        \item No forward edge: It is a back edge. 
        \item No cross edge: It is a tree edge.
    \end{itemize}
\end{theorem}

\begin{example}
    \customFigure[0.75]{00_Images/DFS.png}{DFS}
    \vspace{1em}

    \textbf{DFS Forest}
    \customFigure[0.75]{00_Images/DFS_Forest.png}{DFS Forest}

    \vspace{1em}
    \customFigure[0.75]{00_Images/WP.png}{Well-parenthesis theorem}
    \begin{itemize}
        \item \textbf{Note:} Descendants nested with ancestors in terms of $d/f$ times. Look in the graph above to grasp how this works.
    \end{itemize}
\end{example}

\subsection{Topological Sort}
\begin{definition}
    Given a DAG, create a linear (total) order out fo the partial order $\rightarrow$ "serialize" these events (may not be unique).
    \begin{itemize}
        \item \textbf{Note:} Doesn't work on non-DAG graphs since a cycle cannot be serialized, therefore, problem is undefined if graph is not DAG.
    \end{itemize}
    
\begin{lstlisting}
Topological_Sort(G):
    Call DFS from a source (no incoming edges)
    Report vertice order in descending finish time. 
\end{lstlisting}
\begin{itemize}
    \item \textbf{Time Complexity:} $O(V+E)$
    \item \textbf{Report Reverse Order:} Shorts, socks, skates, books, t-shirt, sweater, bag.
    \begin{itemize}
        \item \textbf{Note:} These change order depending on starting source (hence not unique).
    \end{itemize}
\end{itemize}
\end{definition}

\begin{example}
    \customFigure[0.75]{00_Images/TS.png}{Topological sort.}
\end{example}

\subsubsection{Lemma}
\begin{theorem}
    A directed graph is a DAG iff DFS yields no back edges.
\end{theorem}

\begin{derivation}
    \begin{enumerate}
        \item If G is a DAG $\Rightarrow$ no back edges. 
        \begin{itemize}
            \item ATaC that if G is a DAG $\Rightarrow$ back edges, then G becomes a cycle, but G is a DAG, so contraction.
        \end{itemize}
        \item If no backedges $\Rightarrow$ G is a DAG. 
        \begin{itemize}
            \item ATaC that no backedges $\Rightarrow$ G is not a DAG, then G becomes a cycle, but a backedge is formed, so contradiction.
        \end{itemize}
    \end{enumerate}
\end{derivation}

\begin{derivation} \textbf{Proof of Correctness}
    We need to prove that if edge \( (u, v) \in E \), then \( f(u) > f(v) \). When DFS enters \( u \), it is colored gray. What is the color of \( v \)? It can either be:
    \begin{itemize}
        \item \textbf{Black:} \( v \) has finished already, but not \( u \), thus \( f(u) > f(v) \).
        
        \item \textbf{White:} \( v \) must be fully traversed before \( u \) completes, therefore \( f(u) > f(v) \) (descendant).
        
        \item \textbf{Gray:} This is impossible, as it would imply a back edge. Since gray means \( v \) was visited before but not fully completed, \( v \) would have been visited before \( u \), which is impossible by the lemma above.
    \end{itemize}
\end{derivation}
