\begin{definition}
    Splay trees are binary search trees (BSTs) without any structural balancing condition. 
    \begin{itemize}
        \item \textbf{Weighted dictionary problem:} Key and frequencies accessed. You want to insert, delete, and search.
    \end{itemize}
\end{definition}

\subsection{Online vs. Offline:}
\begin{definition}
    \begin{itemize}
        \item \textbf{Offline:} A prior know the frequencies.
        \item \textbf{Online:} Splay trees achieve the info $\rightarrow$ achieves theoretical optimality (amortized sense) 
    \end{itemize}
\end{definition}


\subsection{Splay operation}
\begin{definition}
    \begin{lstlisting}
        Splay(x):
            while x != root:
                if p(x) = root: # zig
                    rotate p(x)
                if p(x) and x both left or both right child: # zigzig
                    rotate p(p(x))
                    rotate p(x)
                else: # zigzag
                    rotate p(x)
                    rotate p(p(x))
    \end{lstlisting}
\end{definition}

\begin{intuition}
    \begin{enumerate}
        \item \textbf{Zig Case}: 
        \begin{lstlisting}
        Occurs when x's parent is the root.
        - Intuition: x is one step from the root, so a single rotation is enough to bring x to the root.
        \end{lstlisting}
    
        \item \textbf{Zig-Zig Case}: 
        \begin{lstlisting}
        Occurs when both x and its parent are on the same side (both left or both right).
        - Intuition: Two aligned nodes, so we perform two rotations in the same direction to bring x closer to the root.
        \end{lstlisting}
    
        \item \textbf{Zig-Zag Case}:
        \begin{lstlisting}
        Occurs when x and its parent are on opposite sides (one left, one right).
        - Intuition: Zig-zag structure is flattened by alternating two rotations in opposite directions.
        \end{lstlisting}
        
    \end{enumerate}  
\end{intuition}

\begin{example}
    \customFigure[0.75]{00_Images/EX2.png}{Example 2}
\end{example}

\subsection{Operations}
\begin{definition}
    \begin{itemize}
        \item \textbf{Search(x):} Like BST and splay $x$. $O(\log n)$.
        \item \textbf{Delete(x):} Like BST and splay $p(x)$. $O(\log n)$.
        \item \textbf{Insert(x):} Like BST and splay $x$. $O(\log n)$.
        \customFigure[0.75]{00_Images/INS.png}{Reason for time complexity for insert.}
        \item \textbf{Split(x):} $\text{splay}(x)$ and disconnect components. $O(\log n)$
        \customFigure[0.75]{00_Images/SPLIT.png}{Split}
        \item \textbf{Join(x,R,L):} $O(\log n)$
        \customFigure[0.75]{00_Images/JOIN.png}{Join.}
        % FIGURE
    \end{itemize}
\end{definition}

\subsection{Amortized Cost of Splay}

\subsubsection{Rank}
\begin{definition}
    $WT(x)$ of a node $x$ is the number of nodes in the subtree of $x$ plus $x$, therefore, the rank of $x$ is 
    \begin{equation*}
        \text{rank}(x) = \lceil \log(WT(x)) \rceil
    \end{equation*}
\end{definition}

\subsubsection{Credit invariant}
\begin{definition}
    Every node has $\$(\text{rank}(x))$ on it. 
\end{definition}

\subsubsection{Lemma}
\begin{theorem}
    Every zig, zigzig, zigzag costs 
    \begin{equation*}
        3(\text{nr}(x) - \text{or(x)})
    \end{equation*}    
    except the zig that may require an extra \$1.
    \begin{itemize}
        \item nr: new rank
        \item or: old rank
    \end{itemize}
\end{theorem}

\begin{derivation}
    \customFigure[0.75]{00_Images/P1.png}{Proof of zig zig}
    \customFigure[0.75]{00_Images/P2.png}{Proof of zig zag and zig}
\end{derivation}

\subsubsection{Theorem}
\begin{theorem}
    Amortized cost of splay is $O(\log n)$ dollars. 
\end{theorem}

\begin{derivation}
    \customFigure[0.75]{00_Images/PROOF.png}{PROOF by telescoping.}
\end{derivation}


