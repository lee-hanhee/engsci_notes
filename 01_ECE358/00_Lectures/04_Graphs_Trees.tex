\subsection{Graphs}
    \begin{definition}
        \begin{itemize}
            \item \textbf{Graph:} $G = (V, E)$, where $V = \{\text{vertices}\}$ and $E = \{\text{edges}\}$.
            \item \textbf{Subgraph:} $G' = (V', E')$ is a subgraph of a graph $G$ iff:
            \begin{itemize}
                \item $V' \subseteq V$
                \item $E' \subseteq E$
                \item If $e = (v, u) \in E'$, then $v \in V'$ and $u \in V'$
            \end{itemize}
            \item \textbf{Adjacent vertices:} $N(v) = \{u : (v, u) \in E\}$
            \item \textbf{Incident edges:} $I(v) = \{(u, v) : (u, v) \in E\}$
            \item \textbf{Degree of a vertex:} Number of neighbors that a vertex has.
        \end{itemize}            
    \end{definition}

    \subsubsection{Directed and Undirected, Weighted Graphs}
        \begin{definition}
            \begin{itemize}
                \item \textbf{Directed graph (digraph):} Each edge has a direction from one vertex to another. 
                \begin{itemize}
                    \item \textbf{Edges:} $(v_1,v_2)$ and $(v_2,v_1)$ are different.
                    \item \textbf{Self-loop:} Edges from a vertex to itself.
                    \item \textbf{In/Out Degree of V:} Out-degree is the \# of edges leaving it, while in-degree is the \# of edges entering it.
                    \item \textbf{Degree of V:} In-degree plus out-degree.
                \end{itemize}
                \item \textbf{Undirected graph:} Each edge does not have a specific direction.
                \begin{itemize}
                    \item \textbf{Edges:} $(v_1,v_2)$ and $(v_2,v_1)$ are indifferent.
                    \item \textbf{Self-loop:} Forbidden.
                    \item \textbf{Degree of V:} Number of edges incident on it.
                \end{itemize}
                \item \textbf{Weighted graph:} A graph where each edge is associated with a value (e.g. distance, profit, penalty).
            \end{itemize}
        \end{definition}

        \customFigure[0.75]{00_Images/Directed_Undirected.png}{(a) Directed graph, (b) Undirected graph.}

    \subsubsection{Paths and cycles}
    \begin{terminology}
        \begin{itemize}
            \item \textbf{Path:} Going from one vertex to another. 
            
            \item \textbf{Simple Path:} A path with no repetition of vertices.
            
            \item \textbf{Cycle:} A path that begins and ends at the same vertex, which can pass through the same vertex multiple times.

            \item \textbf{Simple Cycle:} A path that begins and ends at the same vertex, with no other repeated vertices.
        \end{itemize}
    \end{terminology}

    \subsubsection{AG, DAG}
    \begin{terminology}
        \begin{itemize}
            \item \textbf{Acyclic Graph:} A graph with no cycles is an acyclic graph.
            \item \textbf{Directed Acyclic Graph:} A DAG is a directed acyclic graph.

            \item \textbf{Connected:} Two vertices are connected if there is a path between them.
            \item \textbf{Connected graph:} $\exists$ path between $\forall \; 2$ vertices. 
            
            \item \textbf{Simple graph:} Graph with no self-loops, or multi-edges.
            \item \textbf{Spanning subgraph:} Let $G' = (V',E')$ be a subgraph of $G=(V,E)$, $G'$ is a spanning subgraph if $V'=V$.

            \item \textbf{Degrees of all V}: $\sum_{v\in V} degree(v) = 2\abs{E}$

            \item \textbf{Bipartite Gs:} $V$ can be partitioned into $2$ sets $V_1$ and $V_2$ s.t. $V_1 \cap V_2 = \emptyset$ and $V_1 \cup V_2 = V$ and adjacencies only between elements of $V_1$ and $V_2$.
            \customFigure[0.25]{00_Images/Bipartite.png}{Bipartite graph.}
        \end{itemize}
    \end{terminology}

    \subsubsection{Clique (complete G)}
    \begin{definition}
        $\exists$ edge between $\forall \; 2$ vertices. 
        \begin{equation*}
            \# edges = \frac{V(V-1)}{2}
        \end{equation*}
    \end{definition}

    \subsubsection{Graph representation}
    \begin{definition}
        
        \textbf{Adjacency matrix (AM):} An $n\times n$ matrix where $M[i][j]=1$ if there is an edge between $v_i$ and $v_j$, and $0$ otherwise (undirected) or $M[i][j]=1$ if there is an edge from $v_i$ to $v_j$, and $0$ otherwise (directed).
        \begin{itemize}
            \item \textbf{Time to search for E:} $O(1)$
            \item \textbf{Memory space:} $O(V^2)$
            \item Good for dense $G$ (i.e. $E>>V^2$)
        \end{itemize}
        \vspace{1em}

        \textbf{Adjacency list (AL):} For $n=\abs{V}$ vertices, $n$ linked lists. The $ith$ linked list, $L[i]$ is a list of all the vertices that are adjacent to vertex $i$.
        \begin{itemize}
            \item \textbf{Time to search for E:} $O(V)$
            \item \textbf{Memory space:} $O(V+E)$ 
            \item Good for sparse $G$ (i.e. $E<<V^2$).
        \end{itemize}
        \vspace{1em}
    \end{definition}
    \customFigure[1]{00_Images/DG_AM_AL.png}{(i) Directed graph, (ii) Adjacency matrix, (iii) Adjacency list}

    \subsubsection{Graph examples}
    \begin{example}
        \textbf{In a graph with } $n$ \textbf{ vertices, how many edges are there?}

        \textbf{At least:}
        \begin{enumerate}
            \item connected: $n - 1$
            \begin{itemize}
                \item This refers to a tree, which is the simplest connected graph with $n$ vertices. In a tree, there is exactly one path between any two vertices, and the minimum number of edges is $n-1$. Adding fewer edges would result in a disconnected graph.
            \end{itemize}
            
            \item not connected: $0$
            \begin{itemize}
                \item A graph with no edges at all is completely disconnected. This represents the trivial case where no vertices are connected to each other.
            \end{itemize}
        \end{enumerate}

        \textbf{At most:}
        \begin{enumerate}
            \item simple: 
            \[
            \binom{n}{2} = \frac{n(n-1)}{2}
            \]
            \begin{itemize}
                \item In a simple graph, there are no loops (edges from a vertex to itself) and no multiple edges between the same pair of vertices. The maximum number of edges occurs when every vertex is connected to every other vertex. This is a complete graph, and the number of edges in a complete graph is given by the binomial coefficient $\binom{n}{2}$, which counts all possible pairs of distinct vertices.
            \end{itemize}
            
            \item not simple: No upper bound
            \begin{itemize}
                \item For graphs that allow loops and multiple edges, there is no upper bound on the number of edges. You can add as many edges between two vertices as you like, including loops (edges that connect a vertex to itself). Thus, the number of edges is unbounded.
            \end{itemize}
        \end{enumerate}

        \textbf{Number of edges and degrees of vertices (undirected graph):}
        \[
        \sum_{v \in V} \deg(v) = 2m, \quad \text{where } m \text{ is the number of edges.}
        \]
        \begin{itemize}
            \item In an undirected graph, the sum of the degrees of all the vertices equals twice the number of edges. This is because each edge contributes to the degree of two vertices (its endpoints), so when counting the degrees, each edge is counted twice.
        \end{itemize}
    \end{example}

\subsection{Trees}
    \begin{definition}
        A tree is a connected, acyclic, undirected graph. 
    \end{definition}

    \subsubsection{Properties}
    \begin{definition}
        Let $G=(V,E)$ be an undirected graph. The following statements are equivalent:
        \begin{enumerate}
            \item $G$ is a tree. 
            \item Any two vertices in $G$ are connected by a unique simple path. 
            \item $G$ is connected, but if any edge is removed from $E$, the resulting graph is disconnected. 
            \item $G$ is connected, and $\abs{E} = \abs{V} - 1$.
            \item $G$ is acyclic, and $\abs{E} = \abs{V} - 1$.
            \item $G$ is acyclic, but if any edge is added to $E$, the resulting graph contains a cycle.
        \end{enumerate}
        \begin{itemize}
            \item \textbf{Note:} There's a proof to show each of these statements are equivalent by proving 1 -> 2 -> ... -> 6 -> 1
        \end{itemize}
    \end{definition}

    \subsubsection{Terminology}
    \begin{terminology}
        \begin{itemize}            
            \item \textbf{Parent:} A node \( y \) is the parent of node \( x \) if \( y \) is directly connected to \( x \) on the path from the root.
            
            \item \textbf{Child:} A node \( x \) is a child of node \( y \) if \( y \) is the parent of \( x \).
            
            \item \textbf{Siblings:} Nodes are siblings if they share the same parent.
            
            \item \textbf{Leaf (or External Node):} A leaf is a node with no children.
            
            \item \textbf{Internal Node:} An internal node is a nonleaf node, which means it has at least one child.
            
            \item \textbf{Degree:} The degree of a node \( x \) is the number of children it has.
            
            \item \textbf{Depth:} The depth of a node \( x \) is the length of the path from the root to \( x \).
            
            \item \textbf{Level:} A level of a tree consists of all nodes at the same depth.
            
            \item \textbf{Height:} The height of a node is the number of edges in the longest path from that node to a leaf.
            \begin{itemize}
                \item \textbf{Height of tree:} From root to any leaf.
            \end{itemize}
            
        \end{itemize}
    \end{terminology}

\subsection{Binary and positional trees}
    \subsubsection{Binary trees}
    \begin{definition}
        A \textbf{binary tree} $T$ is a structure defined on a finite set of nodes that either
        \begin{itemize}
            \item contains no nodes, or 
            \item is composed of three disjoint sets of nodes: a \textbf{root} node, a \textbf{left subtree}, and a \textbf{right subtree}.
        \end{itemize}
    \end{definition}

    \subsubsection{Terminology}
    \begin{terminology}
        \begin{itemize}
            \item \textbf{Empty Tree:} A binary tree with no nodes.
            \item \textbf{Left and Right Child:} The roots of the non-empty left and right subtrees of the root.
            \item \textbf{Full Binary Tree:} Every node is either a leaf or has exactly two children.
            \item \textbf{Position Matters:} The distinction between left and right children is crucial in a binary tree, unlike in general ordered trees.
        \end{itemize}
    \end{terminology}

    \subsubsection{Positional trees}
    \begin{definition}
        The children of a node are labeled with distinct positive integers.
    \end{definition}

    \subsubsection{K-ary trees}
    \begin{definition}
            A positional tree in which each $\text{node} \leq k$ children. A binary tree has \( k = 2 \).
            \vspace{1em}

            A \textbf{complete k-ary tree} is a k-ary tree in which all leaves have the same depth, and every internal node has exactly \( k \) children. 
    \end{definition}

    

