\subsection{Graphs}
    \begin{definition}
        $G = (V, E)$, where $V = \{\text{vertices}\}$ and $E = \{\text{edges}\}$.
    \end{definition}

    \subsubsection{Directed and Undirected, Weighted Graphs}
        \begin{terminology}
            \begin{itemize}
                \item \textbf{Directed graph (digraph):} Each edge has a direction from one vertex to another. 
                \begin{itemize}
                    \item \textbf{Edges:} $(v_1,v_2)$ and $(v_2,v_1)$ are different.
                    \item \textbf{Self-loop:} Edges from a vertex to itself.
                    \item \textbf{In/Out Degree of V:} Out-degree is the \# of edges leaving it, while in-degree is the \# of edges entering it.
                    \item \textbf{Degree of V:} In-degree plus out-degree.
                \end{itemize}
                \item \textbf{Undirected graph:} Each edge does not have a specific direction.
                \begin{itemize}
                    \item \textbf{Edges:} $(v_1,v_2)$ and $(v_2,v_1)$ are indifferent.
                    \item \textbf{Self-loop:} Forbidden.
                    \item \textbf{Degree of V:} Number of edges incident on it.
                \end{itemize}
                \item \textbf{Weighted graph:} A graph where each edge is associated with a value (e.g. distance, profit, penalty).
                \item \textbf{Simple graph:} Graph with no self-loops, or multi-edges.
                \item \textbf{Induced graph:} Subset of $G$ and the associated edges.
                \item \textbf{Spanning subgraph:} Let $G' = (V',E')$ be a subgraph of $G=(V,E)$, $G'$ is a spanning subgraph if $V'=V$.
            \end{itemize}
        \end{terminology}

        \customFigure[0.75]{00_Images/Directed_Undirected.png}{(a) Directed graph, (b) Undirected graph.}

    \subsubsection{Paths and cycles}
    \begin{terminology}
        \begin{itemize}
            \item \textbf{Path:} Going from one vertex to another. 
            
            \item \textbf{Simple Path:} A path with no repetition of vertices.
            
            \item \textbf{Cycle:} A path that begins and ends at the same vertex, which can pass through the same vertex multiple times.

            \item \textbf{Simple Cycle:} A path that begins and ends at the same vertex, with no other repeated vertices.
        \end{itemize}
    \end{terminology}

    \subsubsection{AG, DAG}
    \begin{terminology}
        \begin{itemize}
            \item \textbf{Acyclic Graph:} A graph with no cycles is an acyclic graph.
            \item \textbf{Directed Acyclic Graph:} A DAG is a directed acyclic graph. 
        \end{itemize}
    \end{terminology}

    \subsubsection{Connected, disconnected graph}
    \begin{terminology}
        \begin{itemize}
            \item \textbf{Connected:} Two vertices are connected if there is a path between them.
            \item \textbf{Connected graph:} $\exists$ path between $\forall \; 2$ vertices.
            \item \textbf{Disconnected graph:} $\exists$ at least one pair of vertices such that no path exists between them.
        \end{itemize}
    \end{terminology}

    \subsubsection{Bipartite Gs}
    \begin{definition}
        $V$ can be partitioned into $2$ sets $V_1$ and $V_2$ s.t. $V_1 \cap V_2 = \emptyset$ and $V_1 \cup V_2 = V$ and adjacencies only between elements of $V_1$ and $V_2$.
        \customFigure[0.25]{00_Images/Bipartite.png}{Bipartite graph.}
    \end{definition}

    \subsubsection{Clique (complete G)}
    \begin{definition}
        $\exists$ edge between $\forall \; 2$ vertices. 
        \begin{equation}
            \# edges = \frac{V(V-1)}{2} = \binom{V}{2}
        \end{equation}
        \customFigure[0.75]{00_Images/Clique.png}{Clique for 3 and 4.}
    \end{definition}

    \subsubsection{Degrees of all V}
    \begin{definition}
        \begin{equation}
            \sum_{v\in V} degree(v) = 2\abs{E}
        \end{equation}
    \end{definition}

    \subsubsection{Graph representation}
    \begin{definition}
        
        \textbf{Adjacency matrix (AM):} An $n\times n$ matrix where $M[i][j]=1$ if there is an edge between $v_i$ and $v_j$, and $0$ otherwise (undirected) or $M[i][j]=1$ if there is an edge from $v_i$ to $v_j$, and $0$ otherwise (directed).
        \begin{itemize}
            \item \textbf{Time to search for E:} $O(1)$ (i.e. since in matrix format)
            \item \textbf{Memory space:} $O(V^2)$ (i.e. matrix has $x^2$ entries)
            \item Good for dense $G$ (i.e. $E>>V^2$)
            \item For a directed graph, when you take the transpose of $G$, you get the complement of the original graph.
        \end{itemize}
        \vspace{1em}

        \textbf{Adjacency list (AL):} For $n=\abs{V}$ vertices, $n$ linked lists. The $ith$ linked list, $L[i]$ is a list of all the vertices that are adjacent to vertex $i$.
        \begin{itemize}
            \item \textbf{Time to search for E:} $O(V)$ (i.e. may be on the last vertex)
            \item \textbf{Memory space:} $O(V+E)$ (i.e. store all the vertices and edges once)
            \item Good for sparse $G$ (i.e. $E<<V^2$).
        \end{itemize}
        \vspace{1em}
    \end{definition}
    \customFigure[1]{00_Images/DG_AM_AL.png}{(i) Directed graph, (ii) Adjacency matrix, (iii) Adjacency list}

    \subsubsection{Graph examples}
    \begin{example}
        \textbf{In a graph with } $n$ \textbf{ vertices, how many edges are there?}

        \textbf{At least:}
        \begin{enumerate}
            \item connected: $n - 1$
            \begin{itemize}
                \item This refers to a tree, which is the simplest connected graph with $n$ vertices. In a tree, there is exactly one path between any two vertices, and the minimum number of edges is $n-1$. Adding fewer edges would result in a disconnected graph.
            \end{itemize}
            
            \item not connected: $0$
            \begin{itemize}
                \item A graph with no edges at all is completely disconnected. This represents the trivial case where no vertices are connected to each other.
            \end{itemize}
        \end{enumerate}

        \textbf{At most:}
        \begin{enumerate}
            \item simple: 
            \[
            \binom{n}{2} = \frac{n(n-1)}{2}
            \]
            \begin{itemize}
                \item In a simple graph, there are no loops (edges from a vertex to itself) and no multiple edges between the same pair of vertices. The maximum number of edges occurs when every vertex is connected to every other vertex. This is a complete graph, and the number of edges in a complete graph is given by the binomial coefficient $\binom{n}{2}$, which counts all possible pairs of distinct vertices.
            \end{itemize}
            
            \item not simple: No upper bound
            \begin{itemize}
                \item For graphs that allow loops and multiple edges, there is no upper bound on the number of edges. You can add as many edges between two vertices as you like, including loops (edges that connect a vertex to itself). Thus, the number of edges is unbounded.
            \end{itemize}
        \end{enumerate}

        \textbf{Number of edges and degrees of vertices (undirected graph):}
        \[
        \sum_{v \in V} \deg(v) = 2m, \quad \text{where } m \text{ is the number of edges.}
        \]
        \begin{itemize}
            \item In an undirected graph, the sum of the degrees of all the vertices equals twice the number of edges. This is because each edge contributes to the degree of two vertices (its endpoints), so when counting the degrees, each edge is counted twice.
        \end{itemize}
    \end{example}

\subsection{Trees}
    \begin{definition}
        A tree is a connected, acyclic, undirected graph. 
        \customFigure[0.5]{00_Images/T.png}{Choosing a node as the root, then making it a tree.}
    \end{definition}

    \subsubsection{Properties}
    \begin{definition}
        Let $G=(V,E)$ be an undirected graph. The following statements are equivalent:
        \begin{enumerate}
            \item $G$ is a tree. 
            \item Any two vertices in $G$ are connected by a unique simple path. 
            \item $G$ is connected, but if any edge is removed from $E$, the resulting graph is disconnected. 
            \item $G$ is connected, and $\abs{E} = \abs{V} - 1$.
            \item $G$ is acyclic, and $\abs{E} = \abs{V} - 1$.
            \item $G$ is acyclic, but if any edge is added to $E$, the resulting graph contains a cycle.
        \end{enumerate}
        \begin{itemize}
            \item \textbf{Note:} Prove 1 -> 2 -> ... -> 6 -> 1
        \end{itemize}
    \end{definition}

    \begin{derivation}
        \begin{enumerate}
            \item 1 $\implies$ 2:
            \textbf{Proof:} Assume that \( G \) is a free tree, i.e., it is connected and acyclic. Let \( u \) and \( v \) be any two vertices in \( G \).
            \begin{enumerate}
                \item Since \( G \) is connected, there exists at least one simple path between \( u \) and \( v \).
                \item Suppose, for contradiction, that there are two distinct simple paths between \( u \) and \( v \).
                \item These two paths must diverge at some vertex \( w \) and reconverge at another vertex, forming a cycle, which contradicts the assumption that \( G \) is acyclic.
                \item Therefore, there is exactly one simple path between any two vertices in \( G \).
            \end{enumerate}
            
            \item 2 $\implies$ 3:
            \textbf{Proof:} Assume that any two vertices in \( G \) are connected by a unique simple path. Then \( G \) is connected.
            \begin{enumerate}
                \item Let \( e = (u, v) \in E \) be any edge in \( G \). Since there is a unique simple path between \( u \) and \( v \), this path must include \( e \).
                \item Suppose we remove edge \( e \).
                \item Since \( e \) was the only path between \( u \) and \( v \), removing it disconnects \( G \), as there would be no remaining path between \( u \) and \( v \).
            \end{enumerate}
            Thus, removing any edge from \( G \) disconnects the graph.
        
            \item 3 $\implies$ 4:
            \textbf{Proof:} Assume that \( G \) is connected, and removing any edge disconnects \( G \) (i.e., \( G \) is minimally connected).
            
            We will prove by induction on the number of vertices \( |V| \) that \( |E| = |V| - 1 \).
            \begin{enumerate}
                \item \textbf{Base case:} For \( |V| = 1 \), the graph has no edges, so \( |E| = 0 = |V| - 1 \).
                \item \textbf{Inductive step:} Assume the statement holds for all graphs with fewer than \( n \) vertices. Let \( G \) have \( n \) vertices.
                \begin{enumerate}
                    \item Removing a leaf vertex \( v \) and its incident edge from \( G \) results in a graph \( G' \) with \( n - 1 \) vertices.
                    \item By the induction hypothesis, \( G' \) has \( |E'| = (n - 1) - 1 \) edges.
                    \item Hence, \( G \) has \( |E| = |E'| + 1 = (n - 2) + 1 = n - 1 \) edges.
                \end{enumerate}
            \end{enumerate}
            Therefore, \( |E| = |V| - 1 \).
        
            \item 4 $\implies$ 5: 
            \textbf{Proof:} Assume that \( G \) is connected and \( |E| = |V| - 1 \).
            \begin{enumerate}
                \item Suppose, for contradiction, that \( G \) contains a cycle.
                \item Removing any edge from this cycle would not disconnect the graph, since a cycle has more than one path between its vertices.
                \item This contradicts the assumption that \( G \) is minimally connected, so \( G \) cannot contain a cycle.
            \end{enumerate}
            Therefore, \( G \) is acyclic.
        
            \item 5 $\implies$ 6:
            \textbf{Proof:} Assume that \( G \) is acyclic and \( |E| = |V| - 1 \).
            \begin{enumerate}
                \item Since \( G \) is acyclic, it contains no cycles.
                \item Suppose we add any edge \( e = (u, v) \) to \( G \). Since \( G \) already has \( |E| = |V| - 1 \), adding any edge will create a cycle by closing a path between \( u \) and \( v \) that already exists.
                \item Therefore, adding any edge creates a cycle in \( G \).
            \end{enumerate}
            
            \item 6 $\implies$ 1:
            \textbf{Proof:} Assume that \( G \) is acyclic, and adding any edge creates a cycle.
            \begin{enumerate}
                \item We need to show that \( G \) is connected.
                \item Suppose, for contradiction, that \( G \) is disconnected. Let \( C_1 \) and \( C_2 \) be two disconnected components of \( G \).
                \item Select vertices \( u \in C_1 \) and \( v \in C_2 \). Adding the edge \( (u, v) \) would connect the components, but it would not create a cycle, as no path exists between \( u \) and \( v \) in the original graph.
                \item This contradicts the assumption that adding any edge creates a cycle.
                \item Therefore, \( G \) must be connected.
            \end{enumerate}
            Since \( G \) is both connected and acyclic, \( G \) is a free tree.
        \end{enumerate}
    \end{derivation}

    \subsubsection{Terminology}
    \begin{terminology}
        \begin{itemize}            
            \item \textbf{Parent:} A node \( y \) is the parent of node \( x \) if \( y \) is directly connected to \( x \) on the path from the root.
            
            \item \textbf{Child:} A node \( x \) is a child of node \( y \) if \( y \) is the parent of \( x \).
            
            \item \textbf{Siblings:} Nodes are siblings if they share the same parent.
            
            \item \textbf{Leaf (or External Node):} A leaf is a node with no children.
            
            \item \textbf{Internal Node:} An internal node is a nonleaf node, which means it has at least one child.
            
            \item \textbf{Degree:} The degree of a node \( x \) is the number of children it has.
            
            \item \textbf{Level:} A level of a tree consists of all nodes at the same depth.
            
        \end{itemize}
    \end{terminology}

    \subsubsection{Height, depth}
    \begin{definition}
        \begin{itemize}
            \item \textbf{Depth:} The depth of a node \( x \) is the number of edges from the root to \( x \).
            
            \item \textbf{Height:} The height of a node is the number of edges in the longest path from that node to a leaf.
            \begin{itemize}
                \item \textbf{Height of tree:} From root to any leaf.
            \end{itemize}
        \end{itemize}

    \end{definition}

    \subsubsection{K-ary trees}
    \begin{definition}
            Each $\text{node} \leq k$ children. A binary tree has \( k = 2 \).
            \vspace{1em}

            A \textbf{complete k-ary tree} is a k-ary tree in which all leaves have the same depth, and every internal node has exactly \( k \) children. 
    \end{definition}

    

