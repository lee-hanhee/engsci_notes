\textbf{Motivation:} 45\% induction, 45\% contradiction, 10\% math

\subsection{Direct proof}
\begin{process}
    \begin{enumerate}
        \item Start with the givens
        \item Mathematically manipulate the givens and/or reason about the givens to arrive at the conclusion.
    \end{enumerate}
\end{process}

\begin{example}
    Prove that \( |a + b| \leq |a| + |b| \).
    
    \begin{enumerate}
    
        \item \textbf{Express \( |a + b|^2 \).}
        To begin, we square both sides of the inequality. This allows us to work with non-negative quantities:
        \[
        (a + b)^2 = a^2 + b^2 + 2ab
        \]
        
        \item \textbf{Use the fact that \( ab \leq |a||b| \).}
        Since \( ab \leq |a||b| \), we can write:
        \[
        a^2 + b^2 + 2ab \leq a^2 + b^2 + 2|a||b|
        \]
        Therefore, we have:
        \[
        (a + b)^2 \leq (|a| + |b|)^2
        \]
        
        \item \textbf{Take square roots.}
        Taking the square root of both sides gives us:
        \[
        |a + b| \leq |a| + |b|
        \]
        
    \end{enumerate}
\end{example}

\begin{example}
    \[
    \sum_{i=0}^{n-1} i a^i = \frac{a - a^n}{(1 - a)^2} - \frac{(n - 1) a^n}{1 - a}
    \]
        
    \begin{enumerate}
        \item \textbf{Write out the summation.}
        We begin by expressing the summation explicitly:
        \[
        \sum_{i=0}^{n-1} i a^i = 0 \cdot 1 + 1 \cdot a + 2 \cdot a^2 + 3 \cdot a^3 + \cdots + (n - 1) \cdot a^{n-1}
        \]
        We will refer to this summation as equation \( (1) \).
        
        \item \textbf{Multiply the summation by \( a \).}
        Multiply both sides of the summation by \( a \), which results in:
        \[
        a \sum_{i=0}^{n-1} i a^i = 0 \cdot a + 1 \cdot a^2 + 2 \cdot a^3 + \cdots + (n-1) \cdot a^n
        \]
        This is equation \( (2) \).
        
        \item \textbf{Subtract equation \( (2) \) from equation \( (1) \).}
        Now subtract equation \( (2) \) from equation \( (1) \), which gives:
        \[
        (1 - a) \sum_{i=0}^{n-1} i a^i = a + a^2 + a^3 + \cdots + a^{n-1} - (n-1) \cdot a^n
        \]
        The right-hand side is a geometric series, and the sum of the geometric series can be written as:
        \[
        a + a^2 + \cdots + a^{n-1} = \frac{a(1 - a^{n-1})}{1 - a}
        \]
        Thus, we now have:
        \[
        (1 - a) \sum_{i=0}^{n-1} i a^i = \frac{a - a^n}{1 - a} - (n-1) a^n
        \]
        
        \item \textbf{Solve for the summation.}
        Divide both sides by \( 1 - a \) to isolate the summation:
        \[
        \sum_{i=0}^{n-1} i a^i = \frac{a - a^n}{(1 - a)^2} - \frac{(n - 1) a^n}{1 - a}
        \]
    \end{enumerate}
\end{example}

\subsection{Disprove by counterexample}
\begin{process}
    \begin{enumerate}
        \item Provide a case where the proposition is not true.
    \end{enumerate}
\end{process}

\subsection{Weak Induction (Ap. A.2)}
    
    \begin{process}
        Given predicate $P(n)$: $[P(1) \land \; \forall n \; P(n) \rightarrow P(n+1)] \rightarrow \forall n P(n)$
        \begin{enumerate}
            \item \textbf{Basis Step:} Prove $P(n_0)$ for some value $n_0$. 
            \item \textbf{Hypothesis:} Assume true for $P(n)$ for a $n=k$.
            \item \textbf{Inductive step:} Use the hypothesis to show its true for $P(n=k) \implies P(n+1=k+1)$.
        \end{enumerate}
        Therefore, $\forall n \geq c \text{, } P(n)$.
    \end{process}

    \begin{intuition}
        You don't always need to guess the exact value of a summation in order to use induction. Instead, use induction to prove an upper or lower bound on a summation.
    \end{intuition}

    \begin{example}
        Prove $\sum_{k=1}^{n} k = 1 + 2 + \ldots + n = \frac{n(n+1)}{2}$
        \begin{enumerate}
            \item \textbf{Basis:} $n=1 \text{, } \frac{1(1+1)}{2} = 1$
            \item \textbf{Inductive hypothesis:} Assume true for n, $1 + 2 + \ldots + n = \frac{n(n+1)}{2}$
            \item \textbf{Inductive step:} Prove for $n+1$: $1 + 2 + \ldots + n + (n+1) = \frac{n(n+1)}{2} + (n+1) = \frac{(n+1)(n+2)}{2}$
        \end{enumerate}
        Therefore, we proved by induction that the formula, $\sum_{k=1}^{n} k = \frac{n(n+1)}{2} \text{ for } n+1$ is true for $n+1$.
    \end{example}

    \begin{example}
        Prove the asymptotic upper bound $\sum_{k=0}^{n} 3^k = O\left(3^n\right)$ or $\sum_{k=0}^{n} 3^k \leq c3^n$ for some constant $c$.
        \begin{enumerate}
            \item \textbf{Basis:} $n = 0 \text{: } \sum_{k=0}^{0} 3^k = 1 \leq c \text{ as long as } c \geq 1$
            \item \textbf{Inductive hypothesis:} Assume that the bound holds for $n$.
            \item \textbf{Inductive step:} Prove for $n+1$: 
            \begin{align*}
                \sum_{k=0}^{n+1} 3^k &= \sum_{k=0}^{n} 3^k + 3^{n+1} \\
                                    &\leq c3^n + 3^{n+1} \text{ by the inductive hypothesis} \\ 
                                    &= \left(\frac{1}{3} + \frac{1}{c}\right) c3^{n+1} \text{ by factoring out } c3^{n+1}\\
                                    &\leq c3^{n+1} \text{ since we are using the inequality it still holds true}
            \end{align*}
        \end{enumerate} 
        Therefore, as long as $\left(\frac{1}{3} + \frac{1}{c}\right) \leq 1 \text{ or } c \geq \frac{3}{2}. \text{ Thus, } \sum_{k=0}^{n} 3^k = O(3^n)$.
    \end{example}

    \begin{example}
        Prove that for all integers \( n \geq 1 \), the following inequality holds: $n < 2^n$

        \begin{enumerate}
            \item \textbf{Basis:} 
            We check the case where \( n = 1 \):
            \[
            1 < 2^1 = 2.
            \]
            Since this is true, the base case holds.

            \item \textbf{Inductive Hypothesis:} 
            Assume that the inequality holds for some integer \( n \geq 1 \), i.e.,
            \[
            n < 2^n.
            \]
            
            \item \textbf{Inductive Step:} 
            We need to prove that the inequality holds for \( n = n+1 \), i.e.,
            \[
            n+1 < 2^{n+1}.
            \]
            Starting with the left-hand side:
            \[
            n+1 < \boxed{2^n} + 1 < 2^n + 2^n = 2 \cdot 2^n = 2^{n+1}.
            \]
            Therefore, the inequality holds for \( n = n+1 \).
            \begin{itemize}
                \item \textbf{Note:} The boxed in part is applying the hypothesis.
            \end{itemize}

        \end{enumerate}

        \textbf{Conclusion:} By the principle of mathematical induction, the inequality \( n < 2^n \) holds for all \( n \geq 1 \).
    \end{example}

    \begin{example}
        Prove that the sum of the first \( n \) odd positive integers is \( n^2 \), i.e.,
        \[
        1 + 3 + 5 + \dots + (2n-1) = n^2.
        \]

        \begin{enumerate}
            \item \textbf{Basis:} 
            For \( n = 1 \), the sum of the first odd positive integer is just \( 1 \):
            \[
            1 = 1^2.
            \]
            Therefore, the formula holds for \( n = 1 \).
            
            \item \textbf{Inductive Hypothesis:} 
            Assume that the formula holds for some integer \( n \), i.e.,
            \[
            1 + 3 + 5 + \dots + (2n-1) = n^2.
            \]
            
            \item \textbf{Inductive Step:} 
            We need to prove that the formula holds for \( n+1 \), i.e., 
            \[
            \boxed{1 + 3 + 5 + \dots + (2n-1)} + (2n+1) = (n+1)^2.
            \]
            Using the inductive hypothesis, we know:
            \[
            1 + 3 + 5 + \dots + (2n-1) = n^2.
            \]
            Now, adding the next odd integer \( 2n + 1 \) to both sides, we get:
            \[
            n^2 + (2n + 1) = (n+1)^2.
            \]
            Therefore, both sides are equal, proving that the formula holds for \( n+1 \).

        \end{enumerate}

        \textbf{Conclusion:} By the principle of mathematical induction, the formula for the sum of the first \( n \) odd positive integers, \( 1 + 3 + 5 + \dots + (2n-1) = n^2 \), holds for all \( n \geq 1 \).
    \end{example}

    \begin{example}
        Prove that the power set of a set \( S \), where \( |S| = n \), has \( 2^n \) elements.

        \begin{enumerate}
            \item \textbf{Basis:} 
            For \( n = 0 \), let \( S = \emptyset \) (the empty set). The power set of \( S \) is:
            \[
            2^S = \{\emptyset\}.
            \]
            This power set contains exactly 1 element, which matches \( 2^0 = 1 \). 
            
            Now consider \( n = 1 \), where \( S = \{A\} \). The power set of \( S \) is:
            \[
            2^S = \{\emptyset, \{A\}\}.
            \]
            This power set contains 2 elements, which matches \( 2^1 = 2 \). Thus, the base case holds for \( n = 0 \) and \( n = 1 \).

            \item \textbf{Inductive Hypothesis:} 
            Assume that for a set \( S \) with \( |S| = n \), the power set of \( S \) has \( 2^n \) elements. 
            
            \item \textbf{Inductive Step:} 
            We need to prove that if \( |S| = n+1 \), then the power set of \( S \), \( 2^S \), contains \( 2^{n+1} \) elements.
            \vspace{1em}

            Let \( S \) be a set with \( n+1 \) elements, and let \( a \) be an additional element such that \( S = S' \cup \{a\} \), where \( S' \) is a set of \( n \) elements. 
            
            \vspace{1em}
            The power set of \( S \) consists of:
            \begin{itemize}
                \item Subsets of \( S' \) that do not include \( a \). By the inductive hypothesis, there are \( 2^n \) such subsets.
                \item Subsets of \( S' \) that include \( a \). There are \( 2^n \) such subsets as well, since each subset of \( S' \) can either include or exclude \( a \).
            \end{itemize}
            
            Therefore, the total number of subsets in \( 2^S \) is:
            \[
            2^n + 2^n = 2 \cdot 2^n = 2^{n+1}.
            \]
            This proves that the power set of a set with \( n+1 \) elements contains \( 2^{n+1} \) elements.
            
        \end{enumerate}

        \textbf{Conclusion:} By the principle of mathematical induction, the power set of a set \( S \) with \( |S| = n \) contains \( 2^n \) elements for all \( n \geq 0 \).
    \end{example}

    \begin{example}
        Show that any \( 2^n \times 2^n \) chessboard with one square missing can be tiled using L-shaped tiles.

        \begin{enumerate}
            \item \textbf{Basis:} 
            For \( n = 1 \), the chessboard is a \( 2 \times 2 \) grid. With one square removed, it is clear that the remaining three squares can be covered by a single L-shaped tile, as shown in the diagram below:
            \[
            \begin{matrix}
            \square & \square \\
            \square & \text{x} 
            \end{matrix}
            \]
            This satisfies the base case for \( n = 1 \).

            \item \textbf{Inductive Hypothesis:} 
            Assume that for a \( 2^n \times 2^n \) chessboard with any single square removed, the board can be tiled using L-shaped tiles. That is, the tiling is possible for a board of size \( 2^n \times 2^n \) where one square is missing.

            \item \textbf{Inductive Step:} 
            We need to prove that a \( 2^{n+1} \times 2^{n+1} \) chessboard with one square removed can also be tiled with L-shaped tiles.

            Divide the \( 2^{n+1} \times 2^{n+1} \) chessboard into four \( 2^n \times 2^n \) quadrants. One of these quadrants will have the missing square. To apply the inductive hypothesis, place an L-shaped tile at the center of the chessboard to cover one square from each of the other three quadrants, as shown below:
            
            \[
            \begin{matrix}
            \square & \square & \square & \square \\
            \square & \text{x} & \text{L} & \square \\
            \square & \text{L} & \text{L} & \square \\
            \square & \square & \square & \square 
            \end{matrix}
            \]
            
            Now, each of the four \( 2^n \times 2^n \) quadrants either has one square removed (the original missing square or the square covered by the L-shaped tile in the other three quadrants). By the inductive hypothesis, each \( 2^n \times 2^n \) quadrant can be tiled using L-shaped tiles. Hence, the entire \( 2^{n+1} \times 2^{n+1} \) board can be tiled.

        \end{enumerate}

        \textbf{Conclusion:} By the principle of mathematical induction, any \( 2^n \times 2^n \) chessboard with one square removed can be tiled with L-shaped tiles for all \( n \geq 1 \).
    \end{example}

    \begin{example}
        Prove that \( 11^n - 6 \) is divisible by 5 for all \( n \geq 1 \).

        \begin{enumerate}
            \item \textbf{Base Case:} Let \( n = 1 \).
            \[
            11^1 - 6 = 11 - 6 = 5
            \]
            Clearly, 5 divides 5, so the base case holds. Thus, \( P(1) \) is true.

            \item \textbf{Inductive Hypothesis:} Assume that for some \( n = k \), \( 11^k - 6 \) is divisible by 5. That is, assume:
            \[
            P(k): \quad 5 \mid (11^k - 6)
            \]
            This means there exists some integer \( m \) such that:
            \[
            11^k - 6 = 5m
            \]

            \item \textbf{Inductive Step:} Now we need to show that \( P(k+1) \) is true, i.e., \( 11^{k+1} - 6 \) is divisible by 5. Start by expressing \( 11^{k+1} \) in terms of \( 11^k \):
            \[
            11^{k+1} - 6 = 11 \cdot 11^k - 6
            \]
            Using the inductive hypothesis, substitute \( 11^k - 6 = 5m \) into the equation:
            \[
            11^{k+1} - 6 = 11 \cdot 11^k - 6 = 11 \cdot 5m 
            \]
            Simplifying the expression:
            \[
            11^{k+1} - 6 = 5(11m)
            \]
            Since \( 11^{k+1} - 6 = 5(11m) \), it follows that \( 11^{k+1} - 6 \) is divisible by 5.
        \end{enumerate}
    \end{example}

    \begin{warning}
        Consider the following fallacious proof that \( \sum_{k=1}^n k = O(n) \). 
        \begin{enumerate}
            \item \textbf{Basis:} \( \sum_{k=1}^1 k = O(1) \) 
            \item \textbf{Inductive hypothesis:} Assume that the bound holds for \( n \).
            \item \textbf{Inductive step:} Prove for $n+1$:
            \begin{align*}
                \sum_{k=1}^{n+1} k &= \sum_{k=1}^{n} k + (n + 1) \\
                                    &= O(n) + (n + 1) \\
                                    &= O(n + 1) \quad \text{(wrong!)}
            \end{align*}
        \end{enumerate}

        The bug in the argument is that the “constant” hidden by the “big-O” grows with \( n \) and thus is not constant. We have not shown that the same constant works for all \( n \).
    \end{warning}

\subsection{Strong Induction}
\begin{process}
    \begin{enumerate}
        \item \textbf{Basis:} Show \( P(n_0), P(n_1), \ldots \) are true.
        \item \textbf{Hypothesis:} Assume \( P(k) \) is true, \( \forall k \leq n \).
        \item \textbf{Step:} Show \( P(n_0) \land \cdots \land P(k) \land \cdots \land P(n) \Rightarrow P(n+1) \).
    \end{enumerate}    
\end{process}

\begin{warning}
    \begin{itemize}
        \item \textbf{Weak:} Assume for a specific case $n=k$
        \item \textbf{Strong:} Assume for all cases up to $n=k+1$
    \end{itemize}
\end{warning}

\begin{example}
    Prove the Fundamental Theorem of Arithmetic, which states that every integer \( n \geq 2 \) can be expressed as the product of one or more prime numbers.

    \begin{enumerate}

        \item \textbf{Base Case:} \( n = 2 \)
        \[
        2 \text{ is a prime number, and it can be written as the product of itself.}
        \]
        Hence, the base case is true for \( n = 2 \).

        \item \textbf{Inductive Hypothesis:} Assume that for all integers \( k \) such that \( 2 \leq k \leq n \), the number \( k \) can be expressed as the product of one or more prime numbers.
        \[
        \text{That is, assume that all numbers in the range } [2, n] \text{ can be written as the product of primes.}
        \]

        \item \textbf{Inductive Step:} Now, we need to prove that \( n + 1 \) can also be written as the product of one or more prime numbers.
        
        \begin{enumerate}
            \item \textbf{Case 1:} If \( n + 1 \) is prime, then it can be written as the product of itself, since a prime number is, by definition, a product of primes (in this case, just one prime).
            \[
            n + 1 \text{ is a prime } \Rightarrow n + 1 = n + 1 \text{ (a product of itself)}.
            \]
            
            \item \textbf{Case 2:} If \( n + 1 \) is not prime, then \( n + 1 \) can be written as the product of two smaller integers, say \( k_1 \) and \( k_2 \), where \( 2 \leq k_1, k_2 \leq n \).
            \[
            n + 1 = k_1 \cdot k_2 \quad \text{for some } k_1, k_2 \text{ where } k_1, k_2 < n+1.
            \]
            By the inductive hypothesis, both \( k_1 \) and \( k_2 \) can be written as the product of one or more primes (since \( k_1, k_2 \leq n \)).
            Therefore, \( n + 1 \) is also the product of primes:
            \[
            n + 1 = \text{(product of primes)} \times \text{(product of primes)}.
            \]
        \end{enumerate}
    \end{enumerate}

\end{example}

\begin{example}
    Prove that using \$2 and \$5 bills, we can make any amount of money greater than or equal to \$4.
    \begin{enumerate}

        \item \textbf{Base Cases:}
        \begin{enumerate}
            \item For \( n = 4 \), we can make 4 dollars using two \$2 bills:
            \[
            4 = 2 + 2
            \]
            \item For \( n = 5 \), we can make 5 dollars using one \$5 bill:
            \[
            5 = 5
            \]
        \end{enumerate}
        Thus, the base cases hold for \( n = 4 \) and \( n = 5 \).

        \item \textbf{Inductive Hypothesis:} Assume that for any amount \( k \) such that \( 4 \leq k \leq n \), we can make \( k \) dollars using \$2 and \$5 bills. That is, assume we can form any amount from 4 dollars to \( n \) dollars with the given denominations.

        \item \textbf{Inductive Step:} We now need to show that we can make \( n + 1 \) dollars using \$2 and \$5 bills.

        \begin{enumerate}
            \item If \( n + 1 \geq 6 \), then:
            \[
            n + 1 = (n - 1) + 2
            \]
            Since by the inductive hypothesis we know that \( n - 1 \) dollars can be made using \$2 and \$5 bills (because \( n - 1 \geq 4 \)), we can add a \$2 bill to make \( n + 1 \).

            \item If \( n + 1 = 6 \), we can make 6 dollars using three \$2 bills:
            \[
            6 = 2 + 2 + 2
            \]
        \end{enumerate}
        Therefore, we can make \( n + 1 \) dollars using \$2 and \$5 bills for any \( n \geq 4 \).
    \end{enumerate}

\end{example}

\subsection{Contradiction}
    \begin{process}
        Predicate $P(n)$ either true or false.
        \begin{enumerate}
            \item Assume toward a contradiction $\neg P(n)$.
            \item Make some argument by working with the expression $\neg P(n)$ to get to a contradiction.
            \item Arrive at a contradiction
            \item If this resulted in a contradiction then $P(n)$ is true. 
        \end{enumerate}
        
    \end{process}

    \begin{example}
        Prove that if $x^2 - 5x + 4 < 0, \text{ then } x >0$
        \begin{enumerate}
            \item \textbf{ATaC:} Assume towards a contradiction (ATaC) that \( x^2 - 5x + 4 < 0 \) but \( x \leq 0 \).
            \vspace{1em}
            \item Analyze the quadratic expression:
            \[
            x^2 - 5x + 4 = (x - 1)(x - 4)
            \]
            Thus, the inequality becomes:
            \[
            (x - 1)(x - 4) < 0
            \]
            
            \item This inequality implies that \( x \) must lie between the roots 1 and 4, i.e., \( 1 < x < 4 \).
            \vspace{1em}
            \item \textbf{Contradiction:} However, the assumption \( x \leq 0 \) contradicts this because there are no values of \( x \leq 0 \) that satisfy \( 1 < x < 4 \).
            \end{enumerate}
            \vspace{1em}
            Therefore, the contradiction shows that the assumption \( x \leq 0 \) cannot be true if \( x^2 - 5x + 4 < 0 \). Hence, if \( x^2 - 5x + 4 < 0 \), it must be that \( x > 0 \).
    \end{example}

    \begin{example}
        Prove $\sqrt{2}$ is irrational.
        \begin{enumerate}
            \item \textbf{ATaC:} Suppose \(\sqrt{2}\) is rational. Then we can write:
            
            \[
            \sqrt{2} = \frac{a}{b}
            \]
            
            where \( a \) and \( b \) are integers with no common divisors other than 1 (i.e., the fraction is in its simplest form).
            \vspace{1em}
            \item \textbf{Square Both Sides:}
            
            \[
            2 = \frac{a^2}{b^2}
            \Rightarrow a^2 = 2b^2
            \]
            
            This implies that \( a^2 \) is even (since it is twice an integer). Therefore, \( a \) must also be even (by a lemma which states that if \( a^2 \) is even, then \( a \) is even).
            \vspace{1em}

            \item \textbf{Express \( a \) as an Even Number:}
            
            \[
            a = 2k \text{ for some integer } k
            \]
            
            Substitute \( a = 2k \) into the equation:
            
            \[
            (2k)^2 = 2b^2 \Rightarrow 4k^2 = 2b^2 \Rightarrow 2k^2 = b^2
            \]
            
            This implies that \( b^2 \) is even, and thus \( b \) must also be even.
            \vspace{1em}
            \item \textbf{Contradiction:} Since both \( a \) and \( b \) are even, they have a common factor of 2. This contradicts our initial assumption that \( \frac{a}{b} \) is in its simplest form.
            \vspace{1em}
        
        \end{enumerate}
        The contradiction shows that our assumption that \(\sqrt{2}\) is rational is false. Therefore, \(\sqrt{2}\) is irrational.
    \end{example}

    \begin{example}
            Show that if \( a^2 \) is even, then \( a \) is even.

        \begin{enumerate}
            \item \textbf{Assume the opposite:} 
            Suppose \( a^2 \) is even, but \( a \) is odd.
            
            \item \textbf{Write the form of an odd number:} 
            If \( a \) is odd, then we can express \( a \) as:
            \[
            a = 2k + 1 \quad \text{for some integer } k.
            \]
            
            \item \textbf{Square both sides:} 
            Now, compute \( a^2 \):
            \[
            a^2 = (2k + 1)^2 = 4k^2 + 4k + 1 = 2(2k^2 + 2k) + 1.
            \]
            
            \item \textbf{Analyze the result:} 
            From the equation, we can see that \( a^2 = 2(2k^2 + 2k) + 1 \), which is an odd number (because it is of the form \( 2m + 1 \), where \( m \) is an integer).

            \item \textbf{Contradiction:} 
            This contradicts the assumption that \( a^2 \) is even, as we have just shown that \( a^2 \) must be odd if \( a \) is odd.

        \end{enumerate}

        Since assuming \( a \) is odd leads to a contradiction, it must be the case that \( a \) is even. 

        \textbf{Conclusion:} If \( a^2 \) is even, then \( a \) must be even.
    \end{example}

    \begin{example}
        Prove that there are infinitely many prime numbers.

        \begin{enumerate}
            \item \textbf{Assume the opposite.} 
            Assume, for the sake of contradiction, that there are only finitely many prime numbers. Let the set of all prime numbers be:
            \[
            S = \{p_1, p_2, p_3, \dots, p_k\}
            \]
            where \( p_1, p_2, \dots, p_k \) are all the primes. This implies that no other primes exist.
        
            \item \textbf{Construct a number from all primes.} 
            Consider the number \( P \) constructed by multiplying all primes in \( S \) and adding 1:
            \[
            P = \left( \prod_{x \in S} x \right) + 1 = (p_1 \cdot p_2 \cdot \dots \cdot p_k) + 1
            \]
            By construction, \( P \) is greater than any prime in \( S \).
        
            \item \textbf{Show that \( P \) is not divisible by any prime in \( S \).} 
            Now, consider dividing \( P \) by any prime \( p_i \) from the set \( S \). Since \( P \) is the product of all primes in \( S \) plus 1, the remainder when \( P \) is divided by any prime \( p_i \) is:
            \[
            P \mod p_i = (p_1 \cdot p_2 \cdot \dots \cdot p_k + 1) \mod p_i = 1
            \]
            This means that \( P \) leaves a remainder of 1 when divided by any prime \( p_i \), i.e., \( P \) is \textbf{not divisible} by any prime in \( S \).
        
            \item \textbf{\( P \) must either be a prime or divisible by some other prime.} 
            Since \( P \) is not divisible by any of the primes in \( S \), either:
            \begin{enumerate}
                \item \( P \) is itself a prime, or
                \item \( P \) is divisible by some other prime that is not in \( S \).
            \end{enumerate}
        
            \item \textbf{Reach a contradiction.}
            Both cases lead to a contradiction:
            \begin{enumerate}
                \item If \( P \) is a prime, it should be in the set \( S \) (since we assumed \( S \) contains all primes). But \( P \) is greater than any prime in \( S \), so \( P \notin S \).
                \item If \( P \) is divisible by some other prime not in \( S \), then we have found a prime outside of \( S \), contradicting the assumption that \( S \) contains all primes.
            \end{enumerate}
            
            \item \textbf{Step 6: Conclusion.}
            Therefore, there are infinitely many prime numbers.
        \end{enumerate}
    \end{example}