\subsection{Induction (Ap. A.2)}
    \textbf{Motivation:} The most basic way to evaluate a series is to use induction.
    \begin{process}
        Given proposition $P(n)$
        \begin{enumerate}
            \item Basis: Prove the base case $P(1)$
            \item Inductive hypothesis: Assume true for $P(n)$
            \item Inductive step: Use the hypothesis to show its true for $P(n) \rightarrow P(n+1)$
        \end{enumerate}
        Therefore, $\forall \; n P(n)$.
    \end{process}

    \begin{intuition}
        You don't always need to guess the exact value of a summation in order to use induction. Instead, use induction to prove an upper or lower bound on a summation.
    \end{intuition}

    \begin{example}
        Prove $\sum_{k=1}^{n} k = 1 + 2 + \ldots + n = \frac{n(n+1)}{2}$
        \begin{enumerate}
            \item \textbf{Basis:} $n=1 \text{, } \frac{1(1+1)}{2} = 1$
            \item \textbf{Inductive hypothesis:} Assume true for n, $1 + 2 + \ldots + n = \frac{n(n+1)}{2}$
            \item \textbf{Inductive step:} Prove for $n+1$: $1 + 2 + \ldots + n + (n+1) = \frac{n(n+1)}{2} + (n+1) = \frac{(n+1)(n+2)}{2}$
        \end{enumerate}
        Therefore, we proved by induction that the formula, $\sum_{k=1}^{n} k = \frac{n(n+1)}{2} \text{ for } n+1$ is true for $n+1$.
    \end{example}

    \begin{example}
        Prove the asymptotic upper bound $\sum_{k=0}^{n} 3^k = O\left(3^n\right)$ or $\sum_{k=0}^{n} 3^k \leq c3^n$ for some constant $c$.
        \begin{enumerate}
            \item \textbf{Basis:} $n = 0 \text{: } \sum_{k=0}^{0} 3^k = 1 \leq c \text{ as long as } c \geq 1$
            \item \textbf{Inductive hypothesis:} Assume that the bound holds for $n$.
            \item \textbf{Inductive step:} Prove for $n+1$: 
            \begin{align*}
                \sum_{k=0}^{n+1} 3^k &= \sum_{k=0}^{n} 3^k + 3^{n+1} \\
                                    &\leq c3^n + 3^{n+1} \text{ by the inductive hypothesis} \\ 
                                    &= \left(\frac{1}{3} + \frac{1}{c}\right) c3^{n+1} \text{ by factoring out } c3^{n+1}\\
                                    &\leq c3^{n+1} \text{ since we are using the inequality it still holds true}
            \end{align*}
        \end{enumerate} 
        Therefore, as long as $\left(\frac{1}{3} + \frac{1}{c}\right) \leq 1 \text{ or } c \geq \frac{3}{2}. \text{ Thus, } \sum_{k=0}^{n} 3^k = O(3^n)$.
    \end{example}

    \begin{warning}
        Consider the following fallacious proof that \( \sum_{k=1}^n k = O(n) \). 
        \begin{enumerate}
            \item \textbf{Basis:} \( \sum_{k=1}^1 k = O(1) \) 
            \item \textbf{Inductive hypothesis:} Assume that the bound holds for \( n \).
            \item \textbf{Inductive step:} Prove for $n+1$:
            \begin{align*}
                \sum_{k=1}^{n+1} k &= \sum_{k=1}^{n} k + (n + 1) \\
                                    &= O(n) + (n + 1) \\
                                    &= O(n + 1) \quad \text{(wrong!)}
            \end{align*}
        \end{enumerate}

        The bug in the argument is that the “constant” hidden by the “big-O” grows with \( n \) and thus is not constant. We have not shown that the same constant works for all \( n \).
    \end{warning}

\subsection{Contradiction}
    \begin{process}
        Property $P(n)$ which you want to prove true, and it can be true or false. 
        \begin{enumerate}
            \item If want to prove true, assume $\neg P(n)$.
            \item Work towards a contradiction by working with the expression $\neg P(n)$ and prove this to be false.
            \item If this resulted in a false statement then $P(n)$ is true. 
        \end{enumerate}
        
    \end{process}

    \begin{example}
        Prove that if $x^2 - 5x + 4 < 0, \text{ then } x >0$
        \begin{enumerate}
            \item \textbf{ATaC:} Assume towards a contradiction (ATaC) that \( x^2 - 5x + 4 < 0 \) but \( x \leq 0 \).
            \vspace{1em}
            \item Analyze the quadratic expression:
            \[
            x^2 - 5x + 4 = (x - 1)(x - 4)
            \]
            Thus, the inequality becomes:
            \[
            (x - 1)(x - 4) < 0
            \]
            
            \item This inequality implies that \( x \) must lie between the roots 1 and 4, i.e., \( 1 < x < 4 \).
            \vspace{1em}
            \item \textbf{Contradiction:} However, the assumption \( x \leq 0 \) contradicts this because there are no values of \( x \leq 0 \) that satisfy \( 1 < x < 4 \).
            \end{enumerate}
            \vspace{1em}
            Therefore, the contradiction shows that the assumption \( x \leq 0 \) cannot be true if \( x^2 - 5x + 4 < 0 \). Hence, if \( x^2 - 5x + 4 < 0 \), it must be that \( x > 0 \).
    \end{example}

    \begin{example}
        Prove $\sqrt{2}$ is irrational.
        \begin{enumerate}
            \item \textbf{ATaC:} Suppose \(\sqrt{2}\) is rational. Then we can write:
            
            \[
            \sqrt{2} = \frac{a}{b}
            \]
            
            where \( a \) and \( b \) are integers with no common divisors other than 1 (i.e., the fraction is in its simplest form).
            \vspace{1em}
            \item \textbf{Square Both Sides:}
            
            \[
            2 = \frac{a^2}{b^2}
            \Rightarrow a^2 = 2b^2
            \]
            
            This implies that \( a^2 \) is even (since it is twice an integer). Therefore, \( a \) must also be even (by a lemma which states that if \( a^2 \) is even, then \( a \) is even).
            \vspace{1em}

            \item \textbf{Express \( a \) as an Even Number:}
            
            \[
            a = 2k \text{ for some integer } k
            \]
            
            Substitute \( a = 2k \) into the equation:
            
            \[
            (2k)^2 = 2b^2 \Rightarrow 4k^2 = 2b^2 \Rightarrow 2k^2 = b^2
            \]
            
            This implies that \( b^2 \) is even, and thus \( b \) must also be even.
            \vspace{1em}
            \item \textbf{Contradiction:} Since both \( a \) and \( b \) are even, they have a common factor of 2. This contradicts our initial assumption that \( \frac{a}{b} \) is in its simplest form.
            \vspace{1em}
        
        \end{enumerate}
        The contradiction shows that our assumption that \(\sqrt{2}\) is rational is false. Therefore, \(\sqrt{2}\) is irrational.
    \end{example}