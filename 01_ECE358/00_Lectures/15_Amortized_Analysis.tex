\begin{definition}
A worst-case guarantee (bound) for the average case. We view/analyze every operation as part of the sequence of $n$ operations. If $n$ operations take $T(n)$ time, the operation has $T(n)/n$ amortized costs, which may be different from the actual cost.
\end{definition}

\subsubsection{Analysis Methods}
\begin{itemize}
    \item Aggregate method (brute force)
    \item Accounting method (splay).
    \item Potential method.
\end{itemize}

\subsubsection{Stack Operations}
\begin{itemize}
    \item \textbf{Push:} $O(1)$
    \item \textbf{Pop:} $O(1)$ (naive: $O(n)$ if we call multipop $n$ times)
    \item \textbf{Multipop:} $O(n)$
\end{itemize}

\begin{warning}
    You cannot pop more than what you push, hence $n$ operations can take at most $O(n)$ time or $O(n)/n$ amortized time.
\end{warning}

\subsubsection{Binary Counter: Increment}
\begin{lstlisting}
Increment(A, k)
    i = 0
    while i < k and A[i] == 1:
        A[i] = 0
        i = i + 1
    if i < k:
        A[i] = 1
\end{lstlisting}

\begin{warning}
    Let $k = n$. Assume $O(1)$ to flip $1 \rightarrow 0$ or $0 \rightarrow 1$. Increment is $O(n)$ time $\Rightarrow$ calling $n$ times is $O(n^2)$.
\end{warning}

\subsubsection{Aggregate Analysis of Flips}
\begin{definition}
    \[
\# \text{ bit flips} = \sum_{i=0}^{n} \frac{n}{2^i}
\]
The total bit flips is $\leq 2n$, and the amortized cost of increment is $O(1)$.
\end{definition}


\subsubsection{Accounting Method}
\begin{definition}
We charge \$ for every operation, and the data structure behaves like a ``bank".
\begin{itemize}
    \item \textbf{Amortized cost} $=$ what we charge.
    \item \textbf{Actual cost} $=$ how expensive the operation actually is.
    \item If amortized cost $>$ actual cost $\Rightarrow$ save/deposit on the data structure.
    \item If amortized cost $<$ actual cost $\Rightarrow$ use credit from the data structure.
    \item We cannot run during analysis on negative credit.
\end{itemize}
\end{definition}

\subsubsection{Stack Operations Table}

\begin{table}[h!]
\centering
\begin{tabular}{|c|c|c|}
\hline
Operation & Actual Cost & Amortized Cost \\
\hline
Push & $O(1)$ & \$2 \\
Pop & $O(1)$ & \$1 \\
Multipop & $O(n)$ & \$0 \\
\hline
\end{tabular}
\caption{Cost table for stack operations}
\end{table}

For $n$ operations, we need $2n$ or $O(n)/n = O(1)$ in an amortized sense.