\begin{definition}
    A worst-case guarantee (bound) for the average case. 
    \begin{itemize}
        \item We view/analyze every operation as part of the sequence of $n$ operations. 
        \item If $n$ operations take $T(n)$ time, the operation has $T(n)/n$ amortized costs, which may be different from the actual cost.
    \end{itemize}
    \end{definition}
    
    \subsection{Analysis Methods}
    \begin{definition}
        \begin{itemize}
            \item Aggregate method (brute force)
            \item Accounting method (splay)
            \item Potential method
        \end{itemize}
    \end{definition}
    
    \subsection{Aggregate Method}
    \begin{process}
        The aggregate method works by computing the total cost of \( n \) operations and dividing by \( n \) to find the average, or amortized cost, per operation.

        \begin{enumerate}
            \item \textcolor{black}{We begin by analyzing the total cost $T(n)$ of a sequence of operations.}
            \item \textcolor{black}{Next, we calculate the amortized cost per operation as $\frac{T(n)}{n}$.}
            \item \textcolor{black}{In this method, the expensive operations are averaged out over a large number of operations, ensuring that the average cost per operation is small.}
        \end{enumerate}
    \end{process}
    
    \subsection{Accounting Method}
    \begin{process}
        We charge \$ for every operation, and the data structure behaves like a ``bank".
        \begin{itemize}
            \item \textbf{Amortized cost} $=$ what we charge.
            \item \textbf{Actual cost} $=$ how expensive the operation actually is.
            \item If amortized cost $>$ actual cost $\Rightarrow$ save/deposit on the data structure.
            \item If amortized cost $<$ actual cost $\Rightarrow$ use credit from the data structure.
            \item \textbf{Note:} We cannot run during analysis on negative credit (use as a check).
            \begin{itemize}
                \item Most likely fix the set amortized cost. 
            \end{itemize}
        \end{itemize}
    \end{process}
    
    \subsection{Stack Operations Example}
    \begin{example}
    \begin{tabbing}
        \hspace*{4cm} \= \hspace*{5cm} \= \kill
        \textbf{Operation} \> \textbf{Cost} \> \textbf{Naive Approach} \\
        \textbf{push} \> $O(1)$ \\
        \textbf{pop} \> $O(1)$ \>  \\
        \textbf{multipop} \> $O(n)$
    \end{tabbing}
    \end{example}

    \subsubsection{Naive analysis:}
    \begin{example}
        $n \cdot O(n)=O(n^2)$ if we call \textit{multipop} $n$ times
    \end{example}
    
    \subsubsection{Aggregate Method}
    \begin{example}
        You cannot pop more than what you push, hence $n$ operations can take at most $O(n)$ time or $O(n)/n = O(1)$ per increment in amortized time.
    \end{example}
    
    \subsubsection{Accounting Method}
    \begin{example}
        \customFigure[0.75]{00_Images/OPS.png}{Accounting method.}
        \begin{itemize}
            \item For n operations, need $\$ 2n$ or $\frac{O(n)}{n}= O(1)$ per increment in an amortized sense. 
        \end{itemize}
    \end{example}
    
    \subsection{Binary Counter Example}
    \begin{example}
    \begin{lstlisting}
        Increment(A, k)
            i = 0
            while i < k and A[i] == 1:
                A[i] = 0
                i = i + 1
            if i < k:
                A[i] = 1
    \end{lstlisting}
    \end{example}

    \subsubsection{Naive analysis:}
    \begin{example}
        Let $k = n$. Assume $O(1)$ to flip $1 \rightarrow 0$ or $0 \rightarrow 1$. Increment is $O(n)$ time $\Rightarrow$ calling $n$ times is $O(n^2)$.
    \end{example}
    
    \subsubsection{Aggregate Method}
    \begin{example}
        \customFigure[0.75]{00_Images/AGR.png}{Aggregate method}
    \end{example}
    
    \subsubsection{Accounting Method}
    \begin{example}
        \customFigure[0.75]{00_Images/BINCTR.png}{Derivation of amoritzed sense.}
    \end{example}