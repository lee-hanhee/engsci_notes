\subsection{Logarithms (Ch. 3.3 pg. 66-7)}
    \subsubsection{Definition and notation}
        \begin{definition}
            \begin{equation}
                a = b^c \iff \log_b a = c
            \end{equation}

            \textbf{Notation:}
            \begin{itemize}
                \item $lg \: n = log_2 n$
                \item $ln \: n = log_e n$
                \item $lg^k \: n = (lg \: n)^k$
                \item $lg^{(2)} n = lg \: lg \: n = lg(lg \: n)$
            \end{itemize}
        \end{definition}

    \subsubsection{Properties}
        \begin{definition}
            $\forall \text{ real } a>0 \text{, } b>0 \text{, } c>0, \text{ and } n, \text{ we have}$
            \begin{enumerate}
                \item \( a = b^{\log_b a} \)
                \item \( \log_c(ab) = \log_c a + \log_c b \)
                \item \( \log_b a^n = n \log_b a \)
                \item \( \log_b a = \frac{\log_c a}{\log_c b} \)
                \item \( \log_b \left(\frac{1}{a}\right) = -\log_b a \)
                \item \( \log_b a = \frac{1}{\log_a b} \)
                \item \( a^{\log_b c} = c^{\log_b a} \)
                \item \( \log_b \frac{a}{c} = \log_b a - \log_b c\)
            \end{enumerate}
        \end{definition}

    \subsection{Logarithm iteration}
        \begin{definition}
            $log(n)$ iteratively applied $i$ times to an initial value of $n$.
            \begin{equation}
                log^{(i)}(n) = 
                \begin{cases}
                    n & \text{iff } i = 0, \\
                    log\left(log^{(i-1)}(n)\right) & \text{if } i > 0.
                \end{cases}
            \end{equation}
        \end{definition}
        
    \subsection{Iterated logarithm function}
        \begin{definition}
            The minimum number of times \( i \) that the logarithm function must be applied to \( n \) for the result to be less than or equal to 1:
            \begin{equation}
                \lg^{*} n = \min \left\{ i \geq 0 \; : \; \lg^{(i)} n \leq 1 \right\}
            \end{equation}
        \end{definition}
        \begin{intuition}
            \begin{itemize}
                \item \textbf{Definition of \( \lg^{(i)} n \):} The expression \( \lg^{(i)} n \) denotes the logarithm function applied \( i \) times in succession. 
                \begin{itemize}
                    \item If \( i = 1 \), then \( \lg^{(1)} n = \lg n \). If \( i = 2 \), then \( \lg^{(2)} n = \lg(\lg n) \), and so on. 
                    \item This is different from \( \lg^i n \), which would mean \( (\lg n)^i \), i.e., raising \( \lg n \) to the power \( i \). 
                \end{itemize}

                \item \textbf{Conditions for Definition:} The iterated logarithm \( \lg^{(i)} n \) is only defined if \( \lg^{(i-1)} n > 0 \). This constraint exists because the logarithm of a non-positive number is undefined in real numbers. 
             \end{itemize}
        \end{intuition}
        \begin{example}
            The iterated logarithm is a \emph{very} slowly growing function:

            \begin{itemize}
                \item \(\lg^{*} 2 = 1\) because one application of the logarithm to 2 results in a value less than or equal to 1.
                \item \(\lg^{*} 2^2 = 1 + lg^{*} 2 = 2\)
                \item \(\lg^{*} 2^{2^2} = 3\) because three applications of the logarithm to reach a value less than or equal to 1.
                \item \(\lg^{*} 2^{2^{2^2}} = 4\)
                \item \(\lg^{*} (2^{65536}) = 5\)
            \end{itemize}
        \end{example}

        \begin{intuition}
            
            \textbf{Useful formula:} $O(nlg^* n) \approx O(n)$
        \end{intuition}

\subsection{Fibonacci Numbers (Ch. 3.3)}
    \subsubsection{Definition}
        \begin{definition}
            \begin{equation}
                F_i = 
                \begin{cases}
                    0 & \text{if } i = 0, \\
                    1 & \text{if } i = 1, \\
                    F_{i-1} + F_{i-2} & \text{if } i \geq 2.
                \end{cases}
            \end{equation}
        \end{definition}
    
    \subsubsection{Golden ratio and its conjugate}
        \begin{definition}
            \begin{equation}
                \phi = \frac{1 + \sqrt{5}}{2} \approx 1.61803\ldots \label{eq:phi}
            \end{equation}
            
            and its conjugate, by
            
            \begin{equation}
                \hat{\phi} = \frac{1 - \sqrt{5}}{2} \approx -0.61803\ldots \label{eq:phihat}
            \end{equation}
            
            Specifically, we have
            
            \begin{equation}
                F_i = \frac{\phi^i - \hat{\phi}^i}{\sqrt{5}} \label{eq:fibonacci}
            \end{equation}
            
        \end{definition}

\subsection{Summations (Ap. A.1 pg. 1140-51)}
    \subsubsection{Arithmetic series}
        \begin{definition}
            \begin{equation}
                \sum_{k=1}^{n} k = 1 + 2 + \ldots + n  = \frac{n(n+1)}{2} = \Theta(n^2)
            \end{equation}
        \end{definition}

    \subsubsection{General arithmetic series}
        \begin{definition}
            For $a \geq 0 \text{ and } b > 0$,
            \begin{equation}
                \sum_{k=1}^{n} (a + bk) = \Theta(n^2)    
            \end{equation}
        \end{definition}

    \subsubsection{Sums of squares and cubes}
        \begin{definition}

            \textbf{Sums of squares:}
            \begin{equation}
                \sum_{k=0}^{n} k^2 = \frac{n(n+1)(2n+1)}{6}
            \end{equation}
            \vspace{1em}

            \textbf{Sums of cubes:}
            \begin{equation}
                \sum_{k=0}^{n} k^3 = \frac{n^2(n+1)^2}{4}
            \end{equation}
        \end{definition}

    \subsubsection{Finite geometric series}
        \begin{definition}
            For $x \neq 1$, 
            \begin{equation}
                \sum_{k=0}^{n} x^k = 1 + x + \ldots + x^n  = \frac{x^{n+1} - 1}{x-1} 
            \end{equation}
        \end{definition}

    \subsubsection{Infinite decreasing geometric series}
        \begin{definition}
            For $\abs{x} < 1$, 
            \begin{equation}
                \sum_{k=0}^{\infty} x^k = \frac{1}{1-x}
            \end{equation}
        \end{definition}

    \subsubsection{Harmonic series}
        \begin{definition}
            For positive integers $n$, the $nth$ harmonic number is
            \begin{equation}
                H_n = 1 + \frac{1}{2} + \frac{1}{3} + \frac{1}{4} + \cdots + \frac{1}{n} = \sum_{k=1}^{n} \frac{1}{k} = \ln n + O(1)
            \end{equation}                
        \end{definition}

    \subsubsection{Telescoping series}
        \begin{definition}
            For any sequence $a_0, a_1, \text{ ... }, a_n$,

            \begin{equation}
                \sum_{k=1}^{n} (a_k - a_{k-1}) = a_n - a_0 \quad \text{ OR } \quad \sum_{k=0}^{n-1} (a_k - a_{k+1}) = a_0 - a_n
            \end{equation}

            \begin{itemize}
                \item Each of the terms is added in exactly once and subtracted out exactly once.
            \end{itemize}

        \end{definition}

    \subsubsection{Reindexing summations}
        \begin{intuition}
            \begin{equation}
                \sum_{k=0}^{n} a_{n-k} = \sum_{j=0}^{n} a_j 
            \end{equation}
            \begin{itemize}
                \item $j = n-k$
            \end{itemize}

            \begin{itemize}
                \item If the summation index appears in the body of the sum with a minus sign, it's worth thinking about reindexing.
            \end{itemize}
        \end{intuition}

    \subsubsection{Products}
        \begin{definition}
            The finite product $a_1 a_2 \cdots a_n$ can be expressed as: 
            \begin{equation}
                \prod_{k=1}^{n} a_k
            \end{equation}
        \end{definition}

    \subsubsection{Product to summation}
        \begin{definition}
            \begin{equation}
                lg \left( \prod_{k=1}^{n} a_k \right) = lg \left( a_1 \cdot a_2 \cdots a_n \right) = lg(a_1) + lg(a_2) + \ldots + lg(a_n) = \sum_{k=1}^{n} lg(a_k)
            \end{equation}
        \end{definition}

    \begin{example}
        We want to prove that

        \[
        \sum_{k=0}^{\infty} k x^k = \frac{x}{(1-x)^2}, \quad \text{for} \quad |x| < 1.
        \]

        \begin{enumerate}
            \item Start with the geometric series:
            \[
            \sum_{k=0}^{\infty} x^k = \frac{1}{1-x}, \quad |x| < 1.
            \]
            This is a known result for the sum of an infinite geometric series.

            \item Differentiate the series term by term:
            \[
            \frac{d}{dx} \left( \sum_{k=0}^{\infty} x^k \right) = \sum_{k=1}^{\infty} k x^{k-1} = \frac{1}{(1-x)^2}.
            \]
            Differentiating each term in the geometric series results in this new sum, where the term \( k x^{k-1} \) emerges.

            \item Multiply the differentiated series by \( x \):
            \[
            \sum_{k=1}^{\infty} k x^k = x \sum_{k=1}^{\infty} k x^{k-1} = \frac{x}{(1-x)^2}.
            \]
        \end{enumerate}

        Thus, we have the desired result:

        \[
        \sum_{k=0}^{\infty} k x^k = \frac{x}{(1-x)^2}.
        \]
    \end{example}

    \begin{example}
        We want to simplify the following sum into a closed form.

        \[
        \sum_{i=1}^{n} (a_i - a_{i-1}).
        \]

        \begin{enumerate}
            \item Start with the term at \( i = n \):
            \[
            a_n - a_{n-1}.
            \]

            \item Next, the term at \( i = n-1 \):
            \[
            + a_{n-1} - a_{n-2}.
            \]
            Here, the term \( a_{n-1} \) from the previous step cancels out with the \( a_{n-1} \) in this expression.

            \item Continue with the next terms:
            \[
            + a_{n-2} - a_{n-3},
            \]
            where \( a_{n-2} \) cancels out.

            \item Proceed until the last terms, where \( i = 1 \):
            \[
            + \cdots + a_1 - a_0.
            \]
            After expanding the full sum, all intermediate terms cancel, leaving:
            \[
            a_n - a_0.
            \]
        \end{enumerate}

        Thus, the simplified result is:

        \[
        \sum_{i=1}^{n} (a_i - a_{i-1}) = a_n - a_0.
        \]

        Equivalently, If we reverse the index and express the sum from \( k = 0 \), we can write it as:

        \[
        \sum_{k=0}^{n-1} (a_k - a_{k+1}) = a_0 - a_n.
        \]

        This equivalent expression also simplifies in the same manner, where intermediate terms cancel, leaving \( a_0 - a_n \).
    \end{example}

    \begin{example}
        We are given the sum:

        \[
        \sum_{k=1}^{n-1} \frac{1}{k(k+1)}.
        \]

        We want to express this sum in a simplified form using partial fractions and recognize the telescoping nature of the series.

        \begin{enumerate}
            \item \textbf{Rewrite the fraction:} First, decompose the fraction into partial fractions:
            \[
            \frac{1}{k(k+1)} = \frac{1}{k} - \frac{1}{k+1}.
            \]
            This allows us to rewrite the sum in a more convenient form for cancellation.

            \item \textbf{Write the sum:} Now, substitute the partial fraction decomposition into the original sum:
            \[
            \sum_{k=1}^{n-1} \frac{1}{k(k+1)} = \sum_{k=1}^{n-1} \left( \frac{1}{k} - \frac{1}{k+1} \right).
            \]

            \item \textbf{Expand the sum:} Expanding this sum term-by-term or using the previous example, we get:
            \[
            \left( \frac{1}{1} - \frac{1}{2} \right) + \left( \frac{1}{2} - \frac{1}{3} \right) + \cdots + \left( \frac{1}{n-1} - \frac{1}{n} \right).
            \]

            \item \textbf{Observe the cancellation:} Notice that most terms cancel out, leaving only:
            \[
            1 - \frac{1}{n}.
            \]

            \item \textbf{Conclusion:} Thus, the sum simplifies to:
            \[
            \sum_{k=1}^{n-1} \frac{1}{k(k+1)} = 1 - \frac{1}{n}.
            \]

            This result illustrates the telescoping nature of the series, where most intermediate terms cancel out, leaving only the first and the last terms.
        \end{enumerate}
    \end{example}