\subsection{4 main classes of signals}
\begin{definition}
    \begin{enumerate}
        \item $\mathbb{R}^\mathbb{Z}$ (i.e. real-valued, discrete time)
        \item $\mathbb{C}^\mathbb{Z}$ (i.e. complex-valued, discrete time)
        \item $\mathbb{R}^\mathbb{R}$ (i.e. real-valued, continuous time)
        \item $\mathbb{C}^\mathbb{R}$ (i.e. complex-valued, continuous time)
    \end{enumerate}

    \begin{itemize}
        \item \textbf{Assumption:} Complex unless told otherwise.
    \end{itemize}
\end{definition}

\subsection{Support}
\begin{definition}
    The support of a CT signal \( x \in \mathbb{C}^{\mathbb{R}} \), $x(t) \neq \text{zero}$ is the smallest interval $[a,b]$ s.t.:
    \[
    x(t) = 0 \; \text{for} \; t \notin [a, b] 
    \]
    \customFigure[0.5]{00_Images/Support.png}{Support of a nonzero signal.}
    The support of a DT signal \( x \in \mathbb{C}^{\mathbb{Z}} \), $x[n] \neq \text{zero}$ is the smallest interval $\{a,a+1,\ldots,b\}$ s.t.:
    \[
    x[n] = 0 \; \text{for} \; n \notin \{a, a+1,\ldots,b\} 
    \]
\end{definition}

\begin{process}
    \textbf{DT:}
    \begin{enumerate}
        \item Understand the support of the original signal: $\text{Support of } x[n] = {n_1,\ldots,n_k}$
        \item Time shift by $k$:
        \begin{enumerate}
            \item Right shift: $\text{Support of } x[n-k] = \left\{ n_1 + k, \ldots, n_k + k \right\}$
            \item Left shift: $\text{Support of } x[n+k] = \left\{ n_1 - k, \ldots, n_k - k \right\}$
        \end{enumerate}
        \item Time reversal: Reflects the signal across the vertical axis s.t. $\text{Support of } x[-n] = \left\{ -n_1 , \ldots, -n_k \right\}$
        \item Time scaling: Scaling by $a$ (keep only integers) s.t. $\text{Support of } x[an] = \left\{ \frac{n_1}{a}, \ldots, \frac{n_k}{a} \right\}$
        \begin{enumerate}
            \item If $a>1$, then compression
            \item If $0<a<1$, then expanded
        \end{enumerate}
    \end{enumerate}
    \vspace{1em}

    \textbf{CT:}
    \begin{enumerate}
        \item Understand the support of the original signal:
        \begin{itemize}
            \item Identify the range of \( t \) for which the signal \( x(t) \neq 0 \). This range is known as the support of the signal.
        \end{itemize}
        \item Set the argument (e.g. if $x(1-t)$, then the argument is $1-t$) as an inequality to the support. 
        \item Solve for $t$.
        \item If it is a product or a sum, then you must use logic to see which function will take priority to include all cases. 
        \begin{enumerate}
            \item Product: The lowest bound should take priority b/c the product will be zero as soon as either signal is zero (i.e. only non-zero when both signals are non-zero)
            \item Sum: The highest bound should take priority b/c a sum will be zero when both signals are zero. 
        \end{enumerate}
    \end{enumerate}
\end{process}

\begin{warning}
    You might look for the values s.t. it is guaranteed to be 0.
\end{warning}

\subsection{Signal energy and power}
    \subsubsection{Total energy}
    \begin{definition}
        
        \textbf{Continuous:} Total energy from $t_1 \leq t \leq t_2$ is 
        \begin{equation}
            E_{[t_1,t_2]} = \int_{t_1}^{t_2} \abs{x(t)}^2 dt
        \end{equation}
        \begin{itemize}
            \item $x(t)$: Continuous-time signal
            \item $\abs{x}$: Magnitude of the number
        \end{itemize}

        \textbf{Discrete:} Total energy from $n_1 \leq n \leq n_2$ is
        \begin{equation}
            E_{[t_1,t_2]} = \sum_{n=n_1}^{n_2} \abs{x[n]}^2
        \end{equation}
        \begin{itemize}
            \item $x[t]$: Discrete-time signal
        \end{itemize}
    \end{definition}

    \subsubsection{Average power}
    \begin{definition}

        \textbf{Continuous:} Average power from $t_1 \leq t \leq t_2$ is 
        \begin{equation}
            P_{[t_1,t_2]} = \frac{E_{[t_1,t_2]}}{t_2 - t_1} 
        \end{equation}

        \textbf{Discrete:} Average power from $n_1 \leq n \leq n_2$ is
        \begin{equation}
            P_{[t_1,t_2]} = \frac{E_{[t_1,t_2]}}{n_2 - n_1 + 1} 
        \end{equation}
    \end{definition}

    \subsubsection{Total energy over infinite time interval}
    \begin{definition}

        \textbf{Continuous:}
        \begin{equation}
            E_{\infty} \triangleq \lim_{T \to \infty} \int_{-T}^{T} |x(t)|^2 \, dt = \int_{-\infty}^{\infty} |x(t)|^2 \, dt
        \end{equation}

        \textbf{Discrete:}
        \begin{equation}
            E_{\infty} \triangleq \lim_{N \to \infty} \sum_{n=-N}^{+N} |x[n]|^2 = \sum_{n=-\infty}^{\infty} |x[n]|^2
        \end{equation}
    \end{definition}

    \subsubsection{Average power over infinite time interval}
    \begin{definition}

        \textbf{Continuous:}
        \begin{equation}
            P_{\infty} \triangleq \lim_{T \to \infty} \frac{1}{2T} \int_{-T}^{T} |x(t)|^2 \, dt
        \end{equation}

        \textbf{Discrete:}
        \begin{equation}
            P_{\infty} \triangleq \lim_{N \to \infty} \frac{1}{2N + 1} \sum_{n=-N}^{+N} |x[n]|^2
        \end{equation}
    \end{definition}

    \begin{intuition}
        A finite energy signal has zero average power. This is because the energy is finite and spread out over an infinite time, causing the power to approach zero.
    \end{intuition}