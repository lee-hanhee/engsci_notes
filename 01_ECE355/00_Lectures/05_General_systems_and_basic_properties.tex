\subsection{Systems}
\begin{definition}
    A process that transforms input signals into output signals.
    \begin{itemize}
        \item \textbf{Note:} Signals can be discrete or continuous.
        \begin{itemize}
            \item \textbf{CT:} $x(t)\rightarrow y(t)$
            \item \textbf{DT:} $x[n] \rightarrow y[n]$
        \end{itemize} 
    \end{itemize}
\end{definition}

\subsection{Interconnection of systems}
    \begin{definition}
        \begin{itemize}
            \item \textbf{Series:}
            \customFigure[0.5]{00_Images/Series.png}{Series system.}
            \item \textbf{Parallel:}
            \customFigure[0.5]{00_Images/Parallel.png}{Parallel system.}
            \item \textbf{Series and parallel:}
            \customFigure[0.5]{00_Images/Series_Parallel.png}{Series and parallel system.}
            \item \textbf{Feedback:}
            \customFigure[0.5]{00_Images/Feedback.png}{Feedback system.}
        \end{itemize}
    \end{definition}

\subsection{Basic properties}
    \subsubsection{Memory}
    \begin{definition}
        A system is said to be \emph{memoryless} if $y(t_0)$ depends only on $x(t_0)$, $\forall t_0$ 
    \end{definition}

    \subsubsection{Invertibility}
    \begin{definition}
        A system is invertible if distinct input leads to distinct output.
        \customFigure[0.5]{00_Images/Inverse.png}{Inverse system concept.}
    \end{definition}

    \subsubsection{Causality}
    \begin{definition}
        A system is causal if the output at any time depends only on values of the input at the present time and in the past.
        \vspace{1em}

        \textbf{DT:} $y[n_0]$ does not depend on values of $x[n]$ for $n > n_0$.

        \vspace{1em}
        
        \textbf{CT:} $y(t_0)$ does not depend on values of $x(t)$ for $t > t_0$.
        
    \end{definition}

    \subsubsection{Stability}
    \begin{definition}
        A system is bounded-input-bounded-output (BIBO) stable if bounded input signals (i.e. $\abs{x(t)} < \infty \; \forall t$), leads to bounded output signals (i.e. $\abs{y(t)} < \infty \; \forall t$)
        
    \end{definition}

    \subsubsection{Time invariance}
    \begin{definition}
        A system is time invariant if a time shift in the input signal results in an identical time shift in the output signal.
        \begin{itemize}
            \item \textbf{DT:} If $y[n]$ is the output of a time-invariant system when $x[n]$ is the input, then $y[n - n_0]$ is the output when $x[n - n_0]$ is applied.
            \item \textbf{CT:} If $y(t)$ is the output of a time-invariant system when $x(t)$ is the input, then $y(t - t_0)$ is the output when $x(t - t_0)$ is applied.
        \end{itemize}        
    \end{definition}

    \subsubsection{Linearity}
    \begin{definition}

        Suppose for inputs $x_1$, $x_2$ correspond to outputs $y_1$, $y_2$ respectively, then a system is linear if:
        \begin{enumerate}
            \item \textbf{Additivity:} The response to $x_1 + x_2$ is $y_1 + y_2$.
            \item \textbf{Homogeneity:} The response to $ax_1$ is $ay_1$, where $a \in \mathbb{C}$.
        \end{enumerate}
        \vspace{1em}

        \textbf{Superposition:}
        If $x_k$, $k = 1, 2, 3, \ldots$, are a set of inputs to a linear system with corresponding outputs $y_k$, $k = 1, 2, 3, \ldots$, then the response to a linear combination of these inputs given by

        \[
        x = \sum_k a_k x_k \text{ is } y = \sum_k a_k y_k 
        \]
        \begin{itemize}
            \item \textbf{Consequence:} $x = 0 \rightarrow y = 0$
        \end{itemize}
    \end{definition}