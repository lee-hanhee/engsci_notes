\subsection{DT Sifting property}
\begin{definition}
    Any DT signal can be written as 
    \begin{equation}
        x[n] = \sum_{k=-\infty}^{+\infty} x[k]\delta [n-k]
    \end{equation}
    \begin{itemize}
        \item \textbf{Key:} $\delta [n-k]$ is nonzero only when $k=n$, so it preserves only that value. 
    \end{itemize}
\end{definition}

\subsection{DT Unit impulse response}
\begin{definition}
    $h[n]$ is the output of the LTI system when $\delta [n]$ is the input.
    \begin{equation}
        h[n] = h_0 [n]
    \end{equation}
\end{definition}

\begin{intuition}
    $\delta[n] \rightarrow \text{LTI} \rightarrow h[n]$
\end{intuition}

\subsection{CT Sifting property}
\begin{definition}
    Any CT signal can be written as
    \begin{equation}
        x(t) = \int_{-x}^{+x} x(\tau) \delta(t - \tau) \, d\tau
    \end{equation}
\end{definition}

\subsection{CT Unit impulse response}
\begin{definition}
    $h(t)$ is the output of the LTI system when $\delta (t)$ is the input.
    \begin{equation}
        h(t) = h_0 (t)
    \end{equation}
\end{definition}

\begin{intuition}
    $\delta(t) \rightarrow \text{LTI} \rightarrow h(t)$
\end{intuition}