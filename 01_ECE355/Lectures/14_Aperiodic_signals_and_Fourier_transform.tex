\subsection{Fourier Transform of Rectangle Function}
\begin{intuition}
    \textbf{Recall this \( T \)-periodic signal, \( x_T(t) \)} represented by the rectangular function \( \text{rect}(t) \):
    \customFigure[0.75]{00_Images/RECT1.png}{Periodic rectangular function}
    \[
    x_T(t) \underset{T}{\overset{FS}{\leftrightarrow}} c_k = \frac{1}{T} \, \text{sinc} \left( \frac{k}{T} \right)
    \]
    where \( \text{sinc}(x) = \frac{\sin(\pi x)}{\pi x} \).
    \vspace{1em}

    \textbf{What happens as $T \to \infty$?}
    \begin{itemize}
        \item As \( T \to \infty \), we observe that \( c_k \) terms vanish for all \( k \neq 0 \), however, the limit of \( T c_k \) converges to \( \text{sinc}(f) \) as \( f = \frac{k}{T} \).
        \item Each bar represents a \( c_k \) value at frequency \( \frac{k}{T} \), and these values vibrate at frequencies defined by \( k \).
    \end{itemize}
    \customFigure[0.5]{00_Images/CK.png}{Fourier series coefficients}
    \vspace{1em}
    \textbf{Transition to Aperiodic Signal as \( T \to \infty \):} In the limit as \( T \to \infty \), only the central rectangle remains while the others vanish, indicating a transition from periodic to aperiodic. 
    Thus, as \( T \to \infty \),
    \[
    x_T(t) \to \text{rect}(t) = 
    \begin{cases}
    1, & -\frac{1}{2} \leq t \leq \frac{1}{2} \\
    0, & \text{otherwise}
    \end{cases}
    \]

    \textbf{Fourier Transform of the \( \text{rect}(t) \) function:} In the frequency domain, as \( T \to \infty \), \( \frac{1}{T} \to 0 \), therefore, the bars become a continuum and are not just at multiples of $\frac{1}{T}$. 
    \customFigure[0.75]{00_Images/RECT2.png}{Continuum of the FS coefficients.}
\end{intuition}

\subsection{Arbitrary Signal with Finite Support Fourier Transform}
\begin{derivation}
    Let \( x(t) \) be an arbitrary signal with finite support \( \left[ -\frac{S}{2}, \frac{S}{2} \right] \). For \( T > S \), let \( x_T(t) \) be the \( T \)-periodic extension of \( x(t) \):
    \[
    x_T(t) = \sum_{k=-\infty}^{\infty} x(t - kT)
    \]
    \customFigure[0.75]{00_Images/PERIODIC.png}{Periodic signal.}

    The Fourier Series of \( x_T(t) \) is:
    \begin{align*}
        x_T(t) &\underset{T}{\overset{FS}{\leftrightarrow}} c_k = \frac{1}{T} \int_{T} x_T(t) e^{-j 2 \pi \frac{k}{T} t} \, dt \\
        x_T(t) &\underset{T}{\overset{FS}{\leftrightarrow}} c_k = \frac{1}{T} \int_{-\frac{T}{2}}^{\frac{T}{2}} x_T(t) e^{-j 2 \pi \frac{k}{T} t} \, dt \\ 
        x_T(t) &\underset{T}{\overset{FS}{\leftrightarrow}} c_k = \frac{1}{T} \int_{-\frac{T}{2}}^{\frac{T}{2}} x(t) e^{-j 2 \pi \frac{k}{T} t} \, dt
    \end{align*}
    \begin{itemize}
        \item \textbf{Line 3:} Since \( x_T(t) = x(t) \) for \( t \in \left( -\frac{T}{2}, \frac{T}{2} \right) \).
    \end{itemize}

    Define \( T c_k \) as:
    \[
    T c_k = \int_{-\infty}^{\infty} x(t) e^{-j 2 \pi f t} \, dt = X(f) \quad \text{where} \quad f = \frac{k}{T}
    \]
    \begin{itemize}
        \item $X(f):$ Fourier transform of $x(t)$.
    \end{itemize}
    Thus,
    \[
    c_k = \frac{1}{T} X\left( \frac{k}{T} \right)
    \]
    \vspace{1em}

    Furthermore, 
    \begin{align*}
        x_T(t) &= \sum_{k \in \mathbb{Z}} c_k e^{j 2 \pi \frac{k}{T} t} \\
               &= \sum_{k \in \mathbb{Z}} \frac{1}{T} X\left( \frac{k}{T} \right) e^{j 2 \pi \frac{k}{T} t} \\
               &= \sum_{k \in \mathbb{Z}} X\left( k \Delta f \right) e^{j 2 \pi k \Delta f \, t} \, \Delta f \quad \text{where} \quad \Delta f = \frac{1}{T}
    \end{align*}    
    As \( T \to \infty \), \( \Delta f \to 0 \), we have the conversion from a Riemann sum to an integral:
    \begin{equation*}
        x_T(t) = \sum_{k} X\left( k \Delta f \right) e^{j 2 \pi (k \Delta f) t} \, \Delta f \to x(t) = \int_{-\infty}^{\infty} X(f) e^{j 2 \pi f t} \, df
    \end{equation*}
\end{derivation}

\subsubsection{Fourier Transform Synthesis and Analysis Equations}
\begin{definition}
    \begin{equation*}
        x(t) = \int_{-\infty}^{\infty} X(f) e^{j 2 \pi f t} \, df \quad \text{and} \quad X(f) = \int_{-\infty}^{\infty} x(t) e^{-j 2 \pi f t} \, dt
    \end{equation*}

    \begin{itemize}
        \item The first equation is the \textbf{synthesis equation} (Inverse Fourier Transform of \( X(f) \)).
        \begin{itemize}
            \item i.e. Synethesizing $x(t)$ as a LC of frequnecies in continuous summation, with $X(f)$ as the weights.
        \end{itemize}
        \item The second equation is the \textbf{analysis equation} (Fourier Transform of \( x(t) \)).
    \end{itemize}

    \textbf{Note:} We have that \( x(t) \overset{F}{\leftrightarrow} X(f) \), therefore \( x \) and \( X \) form a Fourier Transform pair.
\end{definition}

\begin{definition}
    \( \omega = 2 \pi f \), which implies \( d\omega = 2 \pi \, df \).
    \[
    x(t) = \frac{1}{2 \pi} \int_{-\infty}^{\infty} X(\omega) e^{j \omega t} \, d\omega \quad \text{and} \quad X(\omega) = \int_{-\infty}^{\infty} x(t) e^{-j \omega t} \, dt
    \]
    \begin{itemize}
        \item Same formulation, with a factor of \( 2 \pi \) for the \( \omega = 2 \pi f \) case.
    \end{itemize}
\end{definition}

\begin{warning}
    \begin{itemize}
        \item \textbf{Note:} These integrals may not converge in some cases, but for our purposes, all integrals of practical signals will converge.
        \item While these integrals are theoretically present, in practice we can often refer to tables of known Fourier Transform results to avoid calculating each integral directly.
    \end{itemize}
\end{warning}

\subsection{Non-Time-Limited Signals Fourier Transform}
\begin{derivation}
    \textbf{What happens if \( x(t) \) is not time-limited (i.e., not of finite support)?} To address this, we can approximate by limiting it and then allowing the truncation to extend to infinity. For example, suppose \( x(t) \) is some Gaussian function.

    Define \( x_S(t) \) as:
    \[
    x_S(t) = 
    \begin{cases}
        x(t), & -\frac{S}{2} \leq t \leq \frac{S}{2} \\
        0, & \text{otherwise}
    \end{cases}
    \]
    \customFigure[0.75]{00_Images/NTLS.png}{Non Time-limited Signals}

    By cutting off the tails of the function, we proceed with the Fourier Transform theory.

    \begin{align*}
        x_S(t) &= \int_{-\infty}^{\infty} X_S(f) e^{j 2 \pi f t} \, df \quad \text{and} \quad X_S(f) = \int_{-\frac{S}{2}}^{\frac{S}{2}} x_S(t) e^{-j 2 \pi f t} \, dt 
    \end{align*}

    Now take $S \to \infty$, then
    \begin{equation*}
        x_S(t) \to x(t) = \int_{-\infty}^{\infty} X(f) e^{j 2 \pi f t} \, df, \quad \text{and} \quad X_S(f) \to X(f) = \int_{-\infty}^{\infty} x(t) e^{-j 2 \pi f t} \, dt
    \end{equation*}

    \begin{itemize}
        \item \textbf{Note:} \( S \) grows to infinity more slowly than \( t \) does.
        \item \textbf{Key:} Converges to the same formula as the time-limited signals.
    \end{itemize}
\end{derivation}

\subsection{Example: Fourier Transform of rect(t)}
\begin{example}
    Consider \( x(t) = \text{rect}(t) \).

    Then,
    \begin{align*}
        X(f) &= \int_{-\infty}^{\infty} x(t) e^{-j 2 \pi f t} \, dt \\
             &= \int_{-\frac{1}{2}}^{\frac{1}{2}} e^{-j 2 \pi f t} \, dt \\
             &= \left. \frac{e^{-j 2 \pi f t}}{-j 2 \pi f} \right|_{-\frac{1}{2}}^{\frac{1}{2}} \\
             &= \frac{e^{-j \pi f} - e^{j \pi f}}{-j 2 \pi f} \\
             &= \frac{\sin(\pi f)}{\pi f} \\
             &= \text{sinc}(f)
    \end{align*}
    
    Thus,
    \[
    \text{rect}(t) \overset{F}{\leftrightarrow} \text{sinc}(f)
    \]
\end{example}

