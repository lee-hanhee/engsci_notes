\subsection{LTI Systems}
\begin{definition}
    Systems that are linear and time invariant.
\end{definition}

\subsection{General DT Signal}
\begin{definition}
    Any DT signal can be written as a linear combination of shifted impulses (i.e. $\delta$ series as a basis for the DT signals)
    \begin{equation}
        x[n] = \sum_{k \in \mathbb{Z}} x[k] \delta[n - k]
    \end{equation}
\end{definition}

\subsection{Impulse response}
\begin{intuition}
    Suppose \( S: \mathbb{C}^\mathbb{Z} \to \mathbb{C}^\mathbb{Z} \) is LTI. How does \( S \) respond to \( x \)?

    \textbf{Special cases:}
    \begin{enumerate}
        \item What if input signal was an impulse? $x[n]=\delta[n] \overset{S} \rightarrow h[n]$
        \customFigure[0.5]{00_Images/SC1.png}{Special case 1}
        \begin{itemize}
            \item \textbf{Analogy:} I hit a bell (i.e. input signal), and the ringing afterwards is the impulse response
        \end{itemize}
        \item What if input signal was a scaled impulse? $A\delta[n] \overset{S} \rightarrow Ah[n], \; A\in \mathbb{C}$
        \customFigure[0.5]{00_Images/SC2.png}{Special case 2}
        \item What if input signal was a shifted impulse? $\delta[n-k] \overset{S} \rightarrow h[n-k]$
        \customFigure[0.5]{00_Images/SC3.png}{Special case 3}
        \item What if input signal was a superposition of impulses? $x_0 \delta[n] + x_k \delta[n-k] \overset{S} \rightarrow x_0 h[n] + x_k h[n-k]$
        \customFigure[0.5]{00_Images/SC4.png}{Special case 4}
        \item What if input signal was a general DT signal? $\sum_{k \in \mathbb{Z}} x[k] \delta[n-k] \overset{S} \rightarrow \sum_{k \in \mathbb{Z}} x[k] h[n-k]$ 
        \customFigure[0.5]{00_Images/SC5.png}{Special case 5}
    \end{enumerate}
\end{intuition}

\begin{warning}
    You are replacing $\delta$ with $h$ for all of these cases.
\end{warning}

\subsubsection{General input-output relationship of impulse response}
\begin{definition}
    \begin{equation}
        y[n] = \sum_{k \in \mathbb{Z}} x[k] h[n-k]
    \end{equation}
    \customFigure[0.5]{00_Images/GC1.png}{General case}
\end{definition}

\begin{example}
    \customFigure[0.5]{00_Images/EX.png}{Example of impulse response systems}
    \begin{itemize}
        \item Basically you are applying the system onto the input which is changing the deltas to hs. 
        \item Since we are given what the impulse response looks like, we just have to plot it with its shifted versions and sum them together.
    \end{itemize}
\end{example}

\begin{warning}
    Will be harder on the HW compared to the test. 
\end{warning}

\begin{intuition}
    In summary, substituting \( x[n] = \delta[n] \) is valid because the impulse response \( h[n] \) of an LTI system is defined as the system's output when the input is \( \delta[n] \). This step allows us to solve for the specific output sequence that defines the system's fundamental behavior.
\end{intuition}

\subsection{Convolution}
\begin{definition}
    Let \( x \in \mathbb{C}^{\mathbb{Z}} \), \( h \in \mathbb{C}^{\mathbb{Z}} \) be DT signals. Define a new signal \( y \in \mathbb{C}^{\mathbb{Z}} \), called the convolution of \( x \) and \( h \), and denoted as

    \begin{equation}
        y = x * h, \, \text{via} \quad y[n] = (x * h)[n] = \sum_{k \in \mathbb{Z}} x[k]h[n-k]
    \end{equation}

    (\( x \) convolved with \( h \)).

    \customFigure[0.5]{00_Images/CLTI.png}{Convolution.}
\end{definition}

\begin{example}
    \customFigure[0.5]{00_Images/PM.png}{Polynomial multiplication.}
    \begin{itemize}
        \item Along the diagonals, they are of the same degree. 
        \item For each row, one of the terms is held constant. 
    \end{itemize}
    \vspace{1em}

    How can this be used?

    \customFigure[0.5]{00_Images/PMEX.png}{Polynomial multiplication example.}

    \begin{itemize}
        \item From degree 1 to degree 2, we take the coefficients from the polynomial multiplication and line that up as denoted. 
        \item Take the outer product (i.e. matrix multiplication) to find the numbers in the middle. 
        \item Take the diagonals and number the diagonals based on the degree.
        \item Sum the diagonals and multiply the appropriate degree term.
    \end{itemize}
\end{example}

\begin{example}
    \customFigure[0.5]{00_Images/EX1.png}{Multiplying numbers example.}
    \begin{itemize}
        \item Take the LSD and multiply it with each digit on the top, and place them side by side in the first row. 
        \item Take the 2nd LSD and put a 0 as the first digit as a placeholder, then multiply it with each digit on the top, and place them side by side in the second row. 
        \item Continue until all bottom digits have been taken account for. 
        \item Sum the columns 
        \item Shift the MSDs over until you are left with one digit in each column (exception is the last column)
        \item Then you found the final answer. 
    \end{itemize}
\end{example}



