\subsection{LTI Systems}
\begin{definition}
    Systems that are linear and time invariant.
\end{definition}

\subsubsection{DT}
\begin{definition}
    Any DT signal can be written as a linear combination of shifted impulses (i.e. $\delta$ series as a basis for the DT signals)
    \begin{equation}
        x[n] = \sum_{k \in \mathbb{Z}} x[k] \delta[n - k]
    \end{equation}
\end{definition}

\subsubsection{Impulse response}
\begin{definition}
    Suppose \( S: \mathbb{R}^\mathbb{Z} \to \mathbb{R}^\mathbb{Z} \) is LTI. How does \( S \) respond to \( x \)?
    
    \textbf{Special case:}
    \begin{enumerate}
        \item Impulse respnse
    \end{enumerate}
    
\end{definition}

