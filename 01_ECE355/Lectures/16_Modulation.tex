\subsection{Bandwidth}
\begin{definition}
    Length of support of a signal over non-negative frequencies.
\end{definition}

\subsection{Modulation Motivational Examples}
\subsubsection{Modulation with cosine wave}
\begin{example}
    \customFigure[0.75]{00_Images/M2.png}{Modulation with cosine wave}
    \begin{itemize}
        \item Since muiltiplication in the time domain is convolution in the frequency domain, the message signal is shifted to the carrier frequency in the frequency domain by $-f_c$ and $f_c$.
        \item \textbf{Note:} $W \gg f_c$, where $W$ is the bandwidth of the message signal.
        \item \textbf{Lower side band:} The range of frequencies that are below the carrier frequency and bounded by $W$ (i.e. the bandwidth of the message signal).
        \begin{itemize}
            \item Range: $f_c - W \leq f \leq f_c$
        \end{itemize}
        \item \textbf{Upper side band:} The range of frequencies that are above the carrier frequency and bounded by $W$.
        \begin{itemize}
            \item Range: $f_c \leq f \leq f_c + W$
        \end{itemize}
    \end{itemize}
\end{example}

\subsubsection{Modulation with square wave}
\begin{example}
    \customFigure[0.75]{00_Images/M3.png}{Modulation with square wave}
    \begin{itemize}
        \item Same idea here but the square wave has a lot of harmonics rather than 2, so the modulation with impulse functions creates copies at harmonics of $f_c$. 
    \end{itemize}
\end{example}

\subsection{Sinusoidal modulation}
\begin{definition}
    \begin{equation*}
        x(t) = A \cos(2 \pi f t + \Theta)
    \end{equation*}

\begin{itemize}
    \item $A$: \text{Amplitude}
    \item $f$: \text{Frequency}
    \item $\Theta$: \text{Phase}
\end{itemize}
\vspace{1em}

\textbf{Modulation}: Making one (or more) of these parameters vary in response to a message signal $m(t)$ so that $m(t)$ can be recovered (demodulated) by the receiver.
\customFigure[0.75]{00_Images/M4.png}{System diagram}
\end{definition}

\subsubsection{Types of Modulation}
\begin{definition}
    \begin{enumerate}
        \item $\textbf{If} \, x(t) = A(t) \cos(2 \pi f_c t + \Theta) \, \text{(Amplitude Modulation)}$
        \item $\textbf{If} \, x(t) = A \cos(\Psi(t)) \, \text{(Angle Modulation)}$

        \begin{itemize}
            \item $\Psi(t)$: \text{Instantaneous phase}
            \item \(\frac{1}{2\pi} \frac{d}{dt} \Psi(t)\): \text{Instantaneous frequency}
        \end{itemize}

        \item \textbf{In PM (Phase Modulation)}:
        \[
        \Psi(t) = 2 \pi f_c t + k_{p} m(t)
        \]
        \customFigure[0.75]{00_Images/M5.png}{System diagram}

    \item \textbf{In FM (Frequency Modulation)}:
        \[
        \frac{1}{2\pi} \frac{d}{dt} \Psi(t) = f_c + k_f m(t) \rightarrow \Psi(t)=2 \pi f_c t + 2 \pi k_f \int_{-\infty}^{t} m(\tau) d\tau + \Theta
        \]
        \[
        x(t) = A_c \cos\left(2 \pi f_c t + 2 \pi k_f \int_{-\infty}^{t} m(\tau) d\tau + \Theta\right)
        \]

    \end{enumerate}


\end{definition}

\subsubsection{Assumptions}
\begin{definition}
    \textbf{Typical Assumptions about } m(t):
    \begin{enumerate}
        \item Bandlimited
        \item Peak-limited: $|m(t)| \leq A_m, \, \forall t$
        \item Peak-limited derivative: $|m'(t)| \leq B_m$
    \end{enumerate}
\end{definition}

\subsubsection{Envelope}
\begin{definition}
    \textbf{Unmodulated Carrier}
    \[
    A \cos(2 \pi f_c t + \theta)
    \]
    \vspace{1em}

    \textbf{Modulating Signal}
    \[
    A(t) \cos(2 \pi f_c t + \theta)
    \]
    \begin{itemize}
        \item \textbf{Note:} Often, \( A(t) \) varies slowly relative to \( f_c \), i.e., \( W \ll f_c \).
    \end{itemize}
    \vspace{1em}

    \textbf{Envelope of $x(t)$:}
    \[
    |A(t)|
    \]
    \customFigure[0.75]{00_Images/M6.png}{System diagram}
\end{definition}

\subsection{Envelope Detector}
\begin{definition}
    \customFigure[0.75]{00_Images/ED.png}{Envelope detector}
    \begin{itemize}
        \item \textbf{Rectifier:} Rectifier is needed to create a periodic signal and then a low-pass filter is used to remove the high frequency components, leaving $|A(t)|$
        \item \textbf{Key:} Envelope detector doesn't recover the sign of the original signal. 
    \end{itemize}
\end{definition}

\subsubsection{Full Wave Rectifier Analysis}
\begin{derivation}
    \customFigure[0.75]{00_Images/ED2.png}{Full wave rectifier}
    \begin{itemize}
        \item The periodic is $T=\frac{1}{2f_c}$
    \end{itemize}
    \customFigure[0.75]{00_Images/ED1.png}{Full wave rectifier}
\end{derivation}

\subsubsection{Half Wave Rectifier Analysis}
\begin{derivation}
    \customFigure[0.75]{00_Images/ED3.png}{Half wave rectifier}
\end{derivation}

\subsection{Distortion}
\begin{definition}
    If $A(t)$ can go negative, then $|A(t) \neq A(t)$.
    \customFigure[0.75]{00_Images/D3.png}{Distortion}
    \begin{itemize}
        \item \textbf{Solution:} To avoid distortion, add a DC offset to the message signal. 
    \end{itemize}
\end{definition}

\subsection{Commericial AM Transmission}
\begin{definition}
    \begin{equation*}
        x(t) = A_c \left(1 + k_a m(t)\right) \cos(2 \pi f_c t)
    \end{equation*}
    \begin{itemize}
        \item $k_a$: Amplitude sensitivity $k_a \leq 1$
        \item $|m(t)| \leq 1$: Normalized message signal
    \end{itemize}

    \customFigure[0.75]{00_Images/D4.png}{Commercial AM Transmission}
    \customFigure[0.75]{00_Images/D5.png}{Commercial AM Transmission}
\end{definition}

\subsection{Double Sideband Suppressed Carrier (DSB-SC)}
\begin{definition}
    \customFigure[0.75]{00_Images/D6.png}{DSB-SC}
\end{definition}

