\subsection{FT Properties}
\begin{definition}
    We denote the Fourier Transform pair as:
    \[
    x(t) \overset{F}{\leftrightarrow} X(f)
    \]
    \vspace{1em}

    \textbf{Fourier Transform:}
    \[
    X(f) = \int_{-\infty}^{\infty} x(t) e^{-j 2 \pi f t} \, dt
    \]
    \[
    x(t) = \int_{-\infty}^{\infty} X(f) e^{j 2 \pi f t} \, df
    \]
    \vspace{1em}

    \textbf{Fourier Transform Conventions}
    \begin{itemize}
        \item Use lowercase letters for time domain functions (e.g., \( x(t) \)).
        \item Use uppercase letters for frequency domain functions (e.g., \( X(f) \)).
    \end{itemize}
\end{definition}

\subsubsection{Common Fourier Transform Pairs}
\begin{definition}
    \begin{enumerate}
    \item For the rectangular function:
    \[
    \text{rect}(t) \overset{F}{\leftrightarrow} \text{sinc}(f)
    \]
    \customFigure[0.75]{00_Images/RECT3.png}{Time domain to frequency domain.}

    \item For \( x(t) = e^{-at} u(t) \):
    \begin{equation*}
        e^{-at} u(t) \overset{F}{\leftrightarrow} \frac{1}{a + j 2 \pi f} \text{ for $a > 0$}
    \end{equation*}
    \begin{itemize}
        \item \textbf{Proof:} \begin{align*}
            X(f) &= \int_{0}^{\infty} e^{-at} e^{-j 2 \pi f t} \, dt \\
                &= \left. \frac{e^{-(a + j 2 \pi f)t}}{-(a + j 2 \pi f)} \right|_0^{\infty} = \frac{1}{a + j 2 \pi f} \quad \text{for } a > 0
        \end{align*}
    \end{itemize}

    \item For \( x(t) = \delta(t) \):
    \begin{equation*}
        \delta(t) \overset{F}{\leftrightarrow} 1 
    \end{equation*}
    \begin{itemize}
        \item \textbf{Proof:} $X(f) = \int_{-\infty}^{\infty} \delta(t) e^{-j 2 \pi f t} \, dt =  \int_{-\infty}^{\infty} \delta(t) e^{j0} \, dt = 1$
    \end{itemize}

    \item For \( X(f) = \delta(f) \):
    \begin{equation*}
        1 \overset{F}{\leftrightarrow} \delta(f)
    \end{equation*}
    \begin{itemize}
        \item \textbf{Proof:} $x(t) = \int_{-\infty}^{\infty} \delta(f) e^{j 2 \pi f t} \, df = \int_{-\infty}^{\infty} \delta(f) e^{j 0} \, df = 1$
    \end{itemize}

    \item For \( x(t) = \delta(t - t_0) \):
    \begin{equation*}
        \delta(t - t_0) \overset{F}{\leftrightarrow} e^{-j 2 \pi f t_0} 
    \end{equation*}
    \begin{itemize}
        \item \textbf{Proof:} $X(f) = \int_{-\infty}^{\infty} \delta(t - t_0) e^{-j 2 \pi f t} \, dt = e^{-j 2 \pi f t_0}$
    \end{itemize}
    \customFigure[0.5]{00_Images/P1.png}{.}

    \item For \( X(f) = \delta(f - f_0) \):
    \begin{equation*}
        e^{j 2 \pi f_0 t} \overset{F}{\leftrightarrow} \delta(f - f_0)
    \end{equation*}
    \begin{itemize}
        \item \textbf{Proof:} $x(t) = \int_{-\infty}^{\infty} \delta(f - f_0) e^{j 2 \pi f t} \, df = e^{j 2 \pi f_0 t}$
    \end{itemize}
    \customFigure[0.5]{00_Images/P2.png}{.}

    \item \textbf{Linearity property:} If \( x \overset{F}{\leftrightarrow} X \) and \( y \overset{F}{\leftrightarrow} Y \), then
    \[
    \alpha x + \beta y \overset{F}{\leftrightarrow} \alpha X + \beta Y \quad \text{for all } \alpha, \beta \in \mathbb{C}
    \]
    \begin{itemize}
        \item \textbf{Proof:}
        \begin{align*}
            \int_{-\infty}^{\infty} \left( \alpha x(t) + \beta y(t) \right) e^{-j 2 \pi f t} \, dt &= \alpha \int_{-\infty}^{\infty} x(t) e^{-j 2 \pi f t} \, dt + \beta \int_{-\infty}^{\infty} y(t) e^{-j 2 \pi f t} \, dt \\
            &= \alpha X + \beta Y
        \end{align*}
    \end{itemize}
    \item \textbf{General linearity property:} 
    \[
    \sum_{k \in \mathbb{Z}} \alpha_k x_k \overset{F}{\leftrightarrow} \sum_{k \in \mathbb{Z}} \alpha_k X_k
    \]
    \end{enumerate}
\end{definition}

\subsection{Fourier Series Transforms of Periodic Signals (4.2)}
\begin{definition}
    Suppose \( x \underset{T}{\overset{FS}{\leftrightarrow}} c_k \), then
    \[
    x(t) = \sum_{k \in \mathbb{Z}} c_k e^{j 2 \pi \frac{k}{T} t} \rightarrow X(f) = \sum_{k \in \mathbb{Z}} c_k \delta\left(f - \frac{k}{T}\right) \quad \text{(Line Spectrum)}
    \]
\end{definition}

\begin{example}
    \begin{enumerate}
        \item For \( x(t) = \cos(2 \pi f_0 t) \):
           \begin{align*}
               x(t) &= \frac{1}{2} e^{j 2 \pi f_0 t} + \frac{1}{2} e^{-j 2 \pi f_0 t} \\
               X(f) &= \frac{1}{2} \delta(f - f_0) + \frac{1}{2} \delta(f + f_0)
           \end{align*}
           \customFigure[0.5]{00_Images/COS.png}{.}
        
        \item For \( y(t) = \sin(2 \pi f_0 t) \):
           \begin{align*}
               y(t) &= \frac{1}{2j} e^{j 2 \pi f_0 t} - \frac{1}{2j} e^{-j 2 \pi f_0 t} \\
               X(f) &= \frac{1}{2j} \delta(f - f_0) - \frac{1}{2j} \delta(f + f_0)
           \end{align*}
           \customFigure[0.5]{00_Images/SINE.png}{.}
        
        \item For the periodic rectangle function.
        \[
        x(t) \underset{T}{\overset{FS}{\longleftrightarrow}} c_k = \frac{1}{T} \, \text{sinc} \left( \frac{k}{T} \right)
        \]
        \customFigure[0.5]{00_Images/RECT4.png}{.}
        \[
        X(f) = \sum_{k \in \mathbb{Z}} \frac{1}{T} \, \text{sinc} \left( \frac{k}{T} \right) \delta \left( f - \frac{k}{T} \right)
        \]
        \customFigure[0.5]{00_Images/REC5.png}{Also a line spectrum (note, in reality these lines diverge to infinity, but we have just drawn in this manner to scale the area to $1$)}

        \item For the rectangular pulse train:
           \[
           x(t) = \sum_{k \in \mathbb{Z}} \delta(t - kT) \underset{T}{\overset{FS}{\longleftrightarrow}} c_k = \frac{1}{T}
           \]
           \[
           X(f) = \sum_{k \in \mathbb{Z}} \frac{1}{T} \delta\left(f - \frac{k}{T}\right)
           \]
           This represents a \textbf{picket fence function}, which will be significant in sampling theory.
           \customFigure[0.75]{00_Images/PF.png}{Picket fence.}
    \end{enumerate}
\end{example}
