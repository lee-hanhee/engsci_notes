\subsection{Affine transformations of the Independent Variable}
    In general, $y(t) = x(at+b)$ for any $a,b \in \mathbb{R}$ (and usually $a \neq 0$)
    \subsubsection{Time dilation}
    \begin{definition}
        $x(t) \rightarrow x\left(\frac{t}{a}\right)$ then 
        \begin{enumerate}
            \item \textbf{Speed up:} If $a > 1$ (i.e. compressed)
            \item \textbf{Slow down:} If $0<a<1$ (i.e. stretched)
        \end{enumerate}
    \end{definition}

    \begin{example}
        \customFigure[0.75]{00_Images/TD_Speed.png}{Time dilation, which sped up compared to x}
        \customFigure[0.75]{00_Images/TD_Slow.png}{Time dilation, which slowed down compared to x}
    \end{example}

    \subsubsection{Time reversal}
    \begin{definition}
        $x(t)\rightarrow x(-t) = \tilde{x}(t)$ (i.e. reflect across y-axis)
    \end{definition}

    \begin{example}
        \customFigure[0.75]{00_Images/TR.png}{Time reversal, which reverses time compared to x}
    \end{example}

    \subsubsection{Time delay}
    \begin{definition}
        $x(t) \rightarrow x(t - a)$ for $a>0$ (i.e. right shift)
    \end{definition}

    \begin{example}
        \customFigure[0.75]{00_Images/TD.png}{Time delay, which delays time compared to x}
    \end{example}

    \subsubsection{Time advance}
    \begin{definition}
        $x(t) \rightarrow x(t + a)$ for $a>0$ (i.e. left shift)
    \end{definition}

    \begin{example}
        \customFigure[0.75]{00_Images/TA.png}{Time advance, which advances time compared to x}
    \end{example}

    \subsubsection{Combined transformations}
    \begin{example}
        \begin{enumerate}
            \item Time delay and shift
            \customFigure[0.75]{00_Images/TDS.png}{Time is stretched and delayed in time compared to x}
            \item Time reversal, dilation, and shift
            \customFigure[0.75]{00_Images/TRDS.png}{Time is reversal, dilated, and shifted compared to x}
        \end{enumerate}
    \end{example}

\subsection{Transformations of Discrete Time}
    In general, $y[n] = x[an+b]$ for any $a,b \in \mathbb{Z}$ (and usually $a \neq 0$)
    
    \begin{example}
        \customFigure[0.75]{00_Images/DT1.png}{Transformation of DT signal.}
        \customFigure[0.75]{00_Images/DT2.png}{Transformation of DT signal.}
    \end{example}

    \begin{warning}
        The same transformations in CT hold for DT, but we need to be careful. 
        \begin{itemize}
            \item When $|a|>1$, only one in every $|a|$ samples from $x$ is retained.
            \begin{itemize}
                \item For \(y[n] = x[an]\), the points of \(y\) at any \(n\) correspond to \(x\) evaluated at intervals of \(a\). If \(a = 3\), then:

                \[
                y[0] = x[0], \quad y[1] = x[3], \quad y[2] = x[6], \quad \dots
                \]
                
                This demonstrates how only every third sample is retained, compressing the original signal.
            \end{itemize}
            
            \item Defining $y[n] = x[n/2]$ does not make sense, since $x[-1/2], x[1/2],x[3/2],\ldots$ are undefined.
        \end{itemize}    
    \end{warning}

\subsection{Periodic Signals}
    \subsubsection{CT: T-periodic}
    \begin{definition}
        A CT signal \(x\) is $T$-periodic for some positive real number \(T\) if
        \begin{equation}
            x(t + T) = x(t) \quad \text{for all} \quad t \in \mathbb{R}.
        \end{equation}

        \begin{itemize}
            \item If $x$ is $T$-periodic, then \(x(t + kT) = x(t)\) for all \(k \in \mathbb{Z}\) and all \(t \in \mathbb{R}\). (i.e. if x is $T$-periodic, then x is also $kT$-periodic)
            \item Let \(y(t) = x(t + T)\), then \(x\) is \(T\)-periodic if $y \overset{a.e.}{=} x$.
        \end{itemize}
    \end{definition}

    \subsubsection{CT: Fundamental period}
    \begin{definition}
        The \textbf{fundamental period} (if it exists) of a CT periodic signal \(x\) is the smallest positive real number \(T_0\) such that \(x\) is \(T_0\)-periodic.
        \begin{itemize}
            \item \textbf{Fundamental frequency:} $T_0 = \frac{1}{f_0}$
        \end{itemize}
    \end{definition}

    \begin{warning}
        A constant signal \(x(t) = C\) is \(T\)-periodic for all \(T \in (0, \infty)\). Such a signal has no fundamental period since the set \((0, \infty)\) does not have a smallest element.
    \end{warning}

    \subsubsection{DT: N-Periodic}
    \begin{definition}
        A DT signal \(x\) is \(N\)-periodic for some positive integer \(N\) if
        \begin{equation}
            x[n + N] = x[n] \quad \text{for all} \quad n \in \mathbb{Z}    
        \end{equation}
            
        \begin{itemize}
            \item If \(x\) is \(N\)-periodic, then \(x[n + kN] = x[n]\) for all \(k, n \in \mathbb{Z}\) (i.e. If \(x\) is \(N\)-periodic, then \(x\) is also \(kN\)-periodic).
        \end{itemize}
    \end{definition}
    
    \begin{warning}
        A 1-periodic signal must be constant.
    \end{warning}

    \subsubsection{DT: Fundamental Period}
    \begin{definition}
        The \textbf{fundamental period} of a DT periodic signal \(x\) is the smallest positive integer \(N_0\) such that \(x\) is \(N_0\)-periodic.
    \end{definition}

    \begin{example}
        \customFigure[0.75]{00_Images/FP_DT}{Fundamental period of a DT signal}
    \end{example}

    \begin{warning}
        The fundamental period cannot include the same sample twice (i.e. don't pick the range inclusive of two peaks). However, this is fine in CT signals.
    \end{warning}

\subsection{Even and Odd Signals}
\begin{definition} 

    A signal \(x\) is said to be \textbf{even} if \(x = \tilde{x}\).
    \begin{itemize}
        \item An even signal has mirror-image symmetry about the time origin.
    \end{itemize}
    \vspace{1em}

    A signal \(x\) is said to be \textbf{odd} if \(x = -\tilde{x}\).

    \begin{itemize}
        \item An odd signal has reversed mirror-image symmetry about the time origin.
        \begin{itemize}
            \item Therefore an odd signal must have value 0 at the time origin.
        \end{itemize}
    \end{itemize}
\end{definition}

\begin{example}
    \customFigure[0.75]{00_Images/EO.png}{Even and odd examples.}
\end{example}

    \subsubsection{Even and odd parts of a signal}
    \begin{definition}

        The \textbf{even part} of a signal \(x\) is the signal
        \begin{equation}
            x_{\text{even}} = \frac{1}{2}(x + \tilde{x})    
        \end{equation}

        The \textbf{odd part} of a signal \(x\) is the signal
        \begin{equation}
            x_{\text{odd}} = \frac{1}{2}(x - \tilde{x})
        \end{equation}

        \begin{itemize}
            \item $x_{even} + x_{odd} = x$
        \end{itemize}
    \end{definition}

    \begin{example}
        Prove $x_{even} (-t) = x_{even} (t)$ and prove $x_{odd} (-t) = - x_{odd} (t)$

        \[
        x_{\text{even}}(-t) = \frac{1}{2}(x(-t) + \tilde{x}(-t)) = \frac{1}{2}(\tilde{x}(t) + x(t)) = x_{\text{even}}(t)
        \]

        \[
        x_{\text{odd}}(-t) = \frac{1}{2}(x(-t) - \tilde{x}(-t)) = \frac{1}{2}(\tilde{x}(t) - x(t)) = -x_{\text{odd}}(t)
        \]
    \end{example}

    \begin{example}
        \customFigure[0.5]{00_Images/EO_Ex.png}{Even and odd decomposition example.}
    \end{example}
    

