\subsection{Linear Algebra Primer}
\textbf{Motivation:} Getting rid of convolution and trying to use multiplication. 

\subsubsection{Eigenvalues and eigenvectors}
\begin{definition}
    Let \( V \) be a vector space over \( \mathbb{C} \) (or \( \mathbb{R} \)), let \( T: V \rightarrow V \) be a linear transformation. 
    \vspace{1em}

    A scalar \( \lambda \in \mathbb{C} \) (or \( \mathbb{R} \)) is called an eigenvalue of \( T \) if \( T(v) = \lambda v \) for some non-zero vector \( v \in V \) (v is then called an eigenvector associated with \( \lambda \)).
    \begin{itemize}
        \item \textbf{Note:} Helps understand \( T \): \( T \) acts on \( v \) by scaling them.
    \end{itemize}
\end{definition}

\begin{example}
    Suppose $V = \mathbb{R}^2$ with $T: \mathbb{R}^2 \to \mathbb{R}^2$ defined so that:
    \[
    \begin{pmatrix} x \\ y \end{pmatrix} \xrightarrow{T} \begin{pmatrix} 2 & -1 \\ 0 & 1 \end{pmatrix} \begin{pmatrix} x \\ y \end{pmatrix} = \begin{pmatrix} 2x - y \\ y \end{pmatrix}
    \]
    
    For the vector $\begin{pmatrix} 1 \\ 1 \end{pmatrix}$:
    \[
    T \begin{pmatrix} 1 \\ 1 \end{pmatrix} = \begin{pmatrix} 1 \\ 1 \end{pmatrix} = 1 \cdot \begin{pmatrix} 1 \\ 1 \end{pmatrix}
    \]
    Thus, $\lambda_1 = 1$ is an eigenvalue of $T$, and every non-zero multiple of $\begin{pmatrix} 1 \\ 1 \end{pmatrix}$ is a corresponding eigenvector.
    \vspace{1em}

    For the vector $\begin{pmatrix} 1 \\ 0 \end{pmatrix}$:
    \[
    T \begin{pmatrix} 1 \\ 0 \end{pmatrix} = \begin{pmatrix} 2 \\ 0 \end{pmatrix} = 2 \cdot  \begin{pmatrix} 1 \\ 0 \end{pmatrix}
    \]
    Thus, $\lambda_2 = 2$ is an eigenvalue of $T$, and every non-zero multiple of $\begin{pmatrix} 1 \\ 0 \end{pmatrix}$ is a corresponding eigenvector.
    \vspace{1em}

    Let $v_1 = \begin{pmatrix} 1 \\ 1 \end{pmatrix}, v_2 = \begin{pmatrix} 1 \\ 0 \end{pmatrix}$. If $v = \alpha_1 v_1 + \alpha_2 v_2$, then:
    \[
    T(v) = T(\alpha_1 v_1 + \alpha_2 v_2) = \alpha_1 T(v_1) + \alpha_2 T(v_2)
    \]
    \[
    = \alpha_1 \lambda_1 v_1 + \alpha_2 \lambda_2 v_2
    \]
\end{example}

\begin{intuition}
    This shows that a change of basis diagonalizes $T$, which helps simplify computations, especially for larger matrices.
\end{intuition}

\subsubsection{Key Fact!}
\begin{definition}
    Complex exponential signals are eigenfunctions of LTI systems.

    \customFigure[0.5]{00_Images/EF.png}{Eigenfunction.}
\end{definition}