\subsection{Cascading systems}
\begin{definition}
    The following systems are equivalent: 
    \customFigure[0.75]{00_Images/CS.png}{Equivelent systems in series.}
    \begin{itemize}
        \item \textbf{Certain order:} Good to pick $h_1$ and $h_2$ in a certain order to make the computation easier (i.e. the intermediate value may be easier such as a $\delta$).
        \item \textbf{Simplification:} Instead of applying the input \( x[n] \) through each individual system one after the other, you can first convolve the impulse responses of the systems to get a single equivalent system. 
        \item \textbf{Associative Property:} Convolution is also associative, which means you can group different sections of the cascaded systems and still obtain the same result. 
    \end{itemize}
\end{definition}

\subsection{Theorem of LTI System Properties}
\begin{theorem}
    Let $S$ be a DT LTI system with impulse response $h[n]$.
    \begin{enumerate}
        \item $S$ is memoryless if and only if:
        \[
        h[n] = k \delta[n]
        \]
        \begin{itemize}
            \item $k \in \mathbb{C}$ 
            \item \textbf{Amplifier:} Only an amplifier is an example of a memoryless LTI system.
            \item \textbf{Intuition:} $h[n]$ is a window (i.e. filter) of size 1 to only have the present value. 
        \end{itemize}
        \vspace{1em}
        
        \item $S$ is causal if and only if:
        \[
        h[n] = 0 \quad \text{for} \quad n < 0
        \]
        \begin{itemize}
            \item \textbf{i.e.}, the impulse response must be \textbf{right-sided}.
            \begin{itemize}
                \item \textbf{Implication:} Its support includes no negative values for a causal system.
            \end{itemize}
            \item \textbf{Intuition:} $h[n]$ window becomes $h[n-k]$ in the convolution (i.e. flips and shifts by n), so only looks in the past. 
            \begin{itemize}
                \item So any values in the past (i.e. $n<0$) would be in the future, making it non-causal.
            \end{itemize}
            \customFigure[0.75]{00_Images/ST.png}{Causal system since it cannot respond early if it's causal.}
        \end{itemize}
        \vspace{1em}
        
        \item $S$ is stable if and only if:
        \[
        \sum_{n \in \mathbb{Z}} |h[n]| < \infty
        \]
        \begin{itemize}
            \item \textbf{i.e.} Impulse response is absolutely summable.
        \end{itemize}
        \vspace{1em}
        
        \item $S$ is invertible if and only if:
        \[
        x[n] * h[n] \neq \text{zero}[n] \quad \text{when} \quad x[n] \neq \text{zero}[n]
        \]
        or equivalently, if there exists $h'[n]$ such that:
        \[
        h[n] * h'[n] = \delta[n]
        \]
        \begin{itemize}
            \item \textbf{Note:} To verify if $h'[n]$ is an inverse, convolve it with $h[n]$ and check if the result is $\delta[n]$.
        \end{itemize}
    \end{enumerate}
\end{theorem}

\begin{derivation}
    For stability, let's see why it must be absolutely summable:
    \begin{enumerate}
        \item The output of an LTI system is given by the convolution:
        \[
        y[n] = x[n] * h[n] = \sum_{k} x[k] h[n - k]
        \]
        \item Applying the definition of stability 
        \begin{align*}
            |y[n]| &= \left| \sum_{k} x[k] h[n - k] \right| \\
            &= \left| \sum_{k} h[k] x[n - k] \right| \quad \text{by commutativity} \\
            &\leq \sum_{k} |h[k]| |x[n - k]| \quad \text{by triangular inequality} \\
            &\leq B \sum_{k} |h[k]| \quad \text{since } |x(t)| \leq B \text{ is bounded}
        \end{align*}
        Therefore, $h[n]$ must be absolutely summable. 
    \end{enumerate}
\end{derivation}