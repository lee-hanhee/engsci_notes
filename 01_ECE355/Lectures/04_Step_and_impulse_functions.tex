\subsection{DT: Unit Impulse}
\begin{definition}
    \begin{equation}
        \delta[n] = 
        \begin{cases}
        1 & \text{if } n = 0 \\
        0 & \text{otherwise}
        \end{cases}
    \end{equation}

    \customFigure[0.5]{00_Images/DT_UI.png}{DT unit impulse.}
\end{definition}

    \subsubsection{Sampling}
    \begin{definition}
        \begin{enumerate}
            \item \textbf{No shift:}
            \begin{equation}
                x[n] \delta[n] = x[0] \delta[n]
            \end{equation}

            \customFigure[0.5]{00_Images/DTS.png}{Discrete time sampling}
            
            \item \textbf{Shift:}
            \begin{equation}
                x[n] \delta[n-k] = x[k] \delta[n-k]
            \end{equation}
        \end{enumerate}
    \end{definition}

\subsection{DT: Unit Step}
\begin{definition}
    \begin{equation}
        u[n] =
        \begin{cases} 
        1 & \text{if } n \geq 0 \\
        0 & \text{otherwise}
        \end{cases}
    \end{equation}     
    
    \customFigure[0.5]{00_Images/DT_US.png}{DT unit step.}
\end{definition}

    \subsubsection{First-order difference (Impulse as a function of steps)}
    \begin{definition}
        \begin{equation}
            u[n] - u[n-1] = \delta [n]
        \end{equation}
        \begin{itemize}
            \item Analogous to derivatives.
        \end{itemize}
    \end{definition}

    \subsubsection{Running sum (Step as a function of impulses)}
    \begin{definition}
        \begin{equation}
            u[n] = \sum_{m=-\infty}^{n} \delta[m] 
        \end{equation}

        \customFigure[0.5]{00_Images/RS.png}{Running sum with (a) $n<0$ and (b) $n>0$.}

        \noindent \textbf{Change of variables to} \(m = n - k \):
        \begin{equation}
            u[n] = \sum_{k=\infty}^{0} \delta[n-k] = \sum_{k=0}^{\infty} \delta[n-k]
        \end{equation}

        \begin{itemize}
            \item i.e. Unit step is a linear combination of the unit impulse functions.
            \item \textbf{Bottom:} $k=-\infty$ because $k = n - m = n - (-\infty) = \infty$
            \item \textbf{Top:} $n \rightarrow 0$ because $n = m + k = -\infty + \infty = 0$
        \end{itemize}

        \customFigure[0.5]{00_Images/RS_CV.png}{Running sum with (a) $n<0$ and (b) $n>0$.}
    \end{definition}

    \begin{process}
        To do the summation change of variables
        \begin{enumerate}
            \item \textbf{Start with the original sum or integral:}
            \[
            S = \sum_{k=a}^{b} f(k)
            \]
        
            \item \textbf{Introduce a new variable:}
            Let the new variable \(\ell\) be defined in terms of the old variable \(k\):
            \[
            \ell = g(k)
            \]
            where \(g(k)\) is a function of \(k\).
        
            \item \textbf{Substitute the new variable into the sum:}
            The original function \(f(k)\) must be expressed in terms of \(\ell\), so the substitution yields:
            \[
            f(k) = h(\ell)
            \]
            where \(k\) is replaced by its expression in terms of \(\ell\).
        
            \item \textbf{Adjust the limits of summation:}
            Update the summation limits by solving the boundary values for \(k\) in terms of \(\ell\):
            \[
            \text{When } k = a, \quad \ell = g(a), \quad \text{and when } k = b, \quad \ell = g(b)
            \]
            The summation now runs from \(\ell = g(a)\) to \(\ell = g(b)\).
        
            \item \textbf{Rewrite the summation in terms of the new variable:}
            Finally, the summation (or integral) is rewritten using the new variable:
            \[
            S = \sum_{\ell=g(a)}^{g(b)} h(\ell)
            \]
        \end{enumerate}
    \end{process}

\subsection{CT: Unit Step}
\begin{definition}
    \begin{equation}
        u(t) =
        \begin{cases}
        1, & \text{if } t \geq 0 \\
        0, & \text{if } t < 0
        \end{cases}
    \end{equation}

    \customFigure[0.5]{00_Images/US_CT.png}{CT Unit step}
\end{definition}

\subsection{CT: Unit Impulse}
\begin{definition}
    \begin{equation}
        u(t) = \int_{-\infty}^{t} \delta(\tau) \, d\tau
    \end{equation}
    \begin{itemize}
        \item Tempting to write $\delta(t) = \frac{d}{dt} u(t)$ but it isn't differentiable at $t=0$.
        \item \textbf{Note:} The impulse function represents a spike which has a total area of $1$.
    \end{itemize}

    \customFigure[0.5]{00_Images/DELTA.png}{The area is 1.}
\end{definition}

    \subsubsection{Approximatation}
    \begin{derivation}
        \begin{equation}
            \delta_{\Delta}(t) = \frac{d}{dt} u_{\Delta}(t) \rightarrow \delta(t) = \lim_{\Delta \to 0} \delta_{\Delta}(t)
        \end{equation}

        \customFigure[0.5]{00_Images/D.png}{Approximation where the RS figure has an area of 1.}
    \end{derivation}

    \begin{intuition}
        The Dirac delta function, denoted \(\delta(\tau)\), has the following key property:
    
        \[
        \int_{-\infty}^{\infty} \delta(\tau) d\tau = 1
        \]
    
        It is defined to be zero everywhere except at \(\tau = 0\), where it spikes to infinity in such a way that the total area under the curve is 1.
        \vspace{1em}
        
        Now, consider the integral of the Dirac delta function from \(-\infty\) to \(t\):
    
        \[
        u(t) = \int_{-\infty}^{t} \delta(\tau) d\tau
        \]
    
        This integral defines the Heaviside step function \(u(t)\), which is 0 for \(t < 0\) and 1 for \(t \geq 0\). The delta function \(\delta(\tau)\) can be seen as the derivative of the Heaviside step function:
    
        \[
        \frac{d}{dt} u(t) = \delta(t)
        \]
    
        Thus, integrating the Dirac delta function over time accumulates the effect of an impulse, resulting in a step increase from 0 to 1 at \(t = 0\).
    
    \end{intuition}

    \subsubsection{Sampling}
    \begin{definition}
        Let $x(t)$ be a signal and consider the product $x(t) \cdot \delta_{\Delta} (t)$

        \begin{enumerate}
            \item \textbf{No shift:}
            \begin{equation}
                \lim_{\Delta \to 0} x(t) \delta_{\Delta}(t) = x(0) \delta(t)
            \end{equation}

            \item \textbf{Shift:}
            \begin{equation}
                x(t) \delta(t - t_0) = x(t_0) \delta(t - t_0)                
            \end{equation}

            \begin{itemize}
                \item \textbf{Special case:} $\text{If } x(t_0) = 0, \text{ then } 0 \delta(t - t_0) = \text{zero}(t)$ 
                \begin{itemize}
                    \item \textbf{Note:} $\text{zero}(t)$ not $0$ because a vector scaled by a vector should be a vector (i.e. signal).
                \end{itemize}
            \end{itemize} 
        \end{enumerate}
    \end{definition}

    \subsection{Intuition from tutorial}
    \subsubsection{Two operators}
    \begin{definition}
        \begin{itemize}
            \item \textbf{CT:}
            \begin{itemize}
                \item \textit{Derivative:} $y(t) = \frac{dx(t)}{dt}$
                \item \textit{Running Integral:} $y(t) = \int_{-\infty}^{t} x(\tau) d\tau$
            \end{itemize}
            
            \item \textbf{DT:}
            \begin{itemize}
                \item \textit{Difference:} $y[n] = x[n] - x[n-1]$
                \item \textit{Running Sum:} $y[n] = \sum_{m=-\infty}^{n} x[m]$
            \end{itemize}
        \end{itemize}
    \end{definition}

    \begin{intuition}
        \textbf{CT:}
        \customFigure[0.5]{00_Images/PRO.png}{The relationship between the ramp, impulse and step functions}

        \textbf{DT:}
        \customFigure[0.5]{00_Images/PRO1.png}{The relationship between the ramp, impulse and step functions}
    \end{intuition}

