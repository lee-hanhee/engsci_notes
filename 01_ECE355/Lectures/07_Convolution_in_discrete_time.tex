\begin{definition}
    Let \( x \in \mathbb{C}^{\mathbb{Z}} \), \( h \in \mathbb{C}^{\mathbb{Z}} \) be DT signals. Define a new signal \( y \in \mathbb{C}^{\mathbb{Z}} \), called the convolution of \( x \) and \( h \) (if it exists), and denoted as

    \begin{equation}
        y = x * h, \, \text{via} \quad y[n] \equiv (x * h)[n] \equiv \sum_{k \in \mathbb{Z}} x[k]h[n-k]
    \end{equation}

    (\( x \) convolved with \( h \)).

    \customFigure[0.5]{00_Images/CLTI.png}{Convolution.}
\end{definition}

\begin{example}
    \customFigure[0.5]{00_Images/PM.png}{Polynomial multiplication.}
    \begin{itemize}
        \item Along the diagonals, they are of the same degree. 
        \item For each row, one of the terms is held constant. 
    \end{itemize}
    \vspace{1em}

    How can this be used?

    \customFigure[0.5]{00_Images/PMEX.png}{Polynomial multiplication example.}

    \begin{itemize}
        \item From degree 1 to degree 2, we take the coefficients from the polynomial multiplication and line that up as denoted. 
        \item Take the outer product (i.e. matrix multiplication) to find the numbers in the middle. 
        \item Take the diagonals and number the diagonals based on the degree.
        \item Sum the diagonals and multiply the appropriate degree term.
    \end{itemize}
\end{example}

\begin{example}
    \customFigure[0.5]{00_Images/EX1.png}{Multiplying numbers example.}
    \begin{itemize}
        \item Take the LSD and multiply it with each digit on the top, and place them side by side in the first row. 
        \item Take the 2nd LSD and put a 0 as the first digit as a placeholder, then multiply it with each digit on the top, and place them side by side in the second row. 
        \item Continue until all bottom digits have been taken account for. 
        \item Sum the columns 
        \item Shift the MSDs over until you are left with one digit in each column (exception is the last column)
        \item Then you found the final answer. 
    \end{itemize}
\end{example}

\subsection{Alternative expressions}

\subsubsection{Impulse expression}
\begin{definition}
    \begin{equation}
        (x \ast h)[n] = \sum_{k \in \mathbb{Z}} \sum_{l \in \mathbb{Z}} x[k] h[l] \delta[n - k - l]
    \end{equation}
\end{definition}

\begin{derivation}
    \[
    \sum_{k \in \mathbb{Z}} x[k] \sum_{l \in \mathbb{Z}} h[l] \delta[n - k - l]
    \]
    \begin{itemize}
        \item Since $\delta=1$ only when $n-k-l=0$, therefore, when $l=n-k$, so using this fact, then:
    \end{itemize}
    \[
    = \sum_{k \in \mathbb{Z}} x[k] h[n-k]
    \]
\end{derivation}

\begin{intuition}
    (similar to polynomial math): For simplicity, write $x_k$ for $x[k]$, $h_l$ for $h[l]$

    \textbf{k+l-plane: i.e.}
    \[
    \begin{array}{ccccccc}
    x_{-2} h_1 & x_{-1} h_1 & x_0 h_1 & x_1 h_1 & x_2 h_1 & x_3 h_1 & \cdots \\
    x_{-2} h_0 & x_{-1} h_0 & x_0 h_0 & x_1 h_0 & x_2 h_0 & x_3 h_0 & \cdots \\
    x_{-2} h_{-1} & x_{-1} h_{-1} & x_0 h_{-1} & x_1 h_{-1} & x_2 h_{-1} & x_3 h_{-1} & \cdots \\
    \end{array}
    \]
    \begin{itemize}
        \item $n=k+l=0$: Along the diagonal for $-1,1$, $0,0$, and $1,-1$. Similar results can be achieved for all the other diagonals. 
        \item Taking the sum along the diagonals get one component of $y[n]$
        \item Take the outer product.
    \end{itemize}
\end{intuition}

\subsubsection{Graphical expression}
\begin{definition}
    \customFigure[0.75]{00_Images/GE.png}{Graphical expression}
    \begin{itemize}
        \item $t=0$: Output at $h[0]$
        \item We are sliding this filter across and summing up the linear combinations of $x,h$, which is weighed by $h$.
    \end{itemize}
\end{definition}

\begin{intuition}
    \textbf{Diagram Explanation:} The process of convolution involves taking a filter \(h[n]\), reversing it, and sliding it across the signal \(x[k]\) to compute the output at each position \(n\). 
    
    \textbf{Visual Representation in the Diagram:} 
    \begin{itemize}
        \item As the filter slides, the overlapping values are summed along each diagonal, corresponding to different shifts of \(h[n-k]\).
        \item The small window highlights the part of \(x[k]\) being multiplied by the reversed and shifted filter \(h[n-k]\), and the resulting product is summed to give the final convolution output for each \(n\).
    \end{itemize}
    \vspace{1em}
    
    \textbf{Key Steps:}
    \begin{enumerate}
        \item Reverse the filter \(h[n]\) to obtain \(h[n-k]\).
        \item Slide the reversed filter along the \(k\)-axis.
        \item Multiply the overlapping elements of \(x[k]\) and \(h[n-k]\).
        \item Sum the products to compute the output at each \(n\).
    \end{enumerate}
\end{intuition}

\subsection{Properties of convolution}
\subsubsection{Commutativity}
\begin{definition}
    For all signals \(x[n]\) and \(h[n]\), the convolution is commutative, i.e.,
    
    \begin{equation}
    x * h = h * x
    \end{equation}
    \end{definition}
    
\begin{derivation}
    \[
    (x * h)[n] = \sum_{k \in \mathbb{Z}} \sum_{\lambda \in \mathbb{Z}} x[k] h[\lambda] \delta[n - k - \lambda]
    \]
    \[
    = \sum_{\lambda} h[\lambda] \left(\sum_{k} x[k] \delta[n - k - \lambda]\right)
    \]
    \[
    = \sum_{\lambda} h[\lambda] x[n - \lambda] = (h * x)[n]
    \]
\end{derivation}
    
\begin{intuition}
Linear Time-Invariant (LTI) systems exhibit commutative properties as shown by the interchangeable order of the systems in series.

\[
\text{Input } \to X \to Y \to \text{Output} = \text{Input } \to Y \to X \to \text{Output}
\]
\end{intuition}
    
\subsubsection{Associativity}
\begin{definition}
Convolution is associative, meaning $\forall x_1,x_2,x_3 \in \mathbb{C}^{\mathbb{Z}}$

\begin{equation}
(x_1 * x_2) * x_3 = x_1 * (x_2 * x_3)
\end{equation}

This holds because the order of operations does not matter.
\end{definition}

\subsubsection{Distributivity Over Addition}
\begin{definition}
Convolution is distributive over addition, i.e., for all \(x_1, x_2, h\),

\begin{equation}
(x_1 + x_2) * h = (x_1 * h) + (x_2 * h)
\end{equation}
\end{definition}

\begin{derivation}
    $\sum_k (x_1[k] + x_2[k]) h[n-k] = \sum_k x_1[k] h[n-k] + \sum_k x_2[k] h[n-k]$
\end{derivation}
    
\subsubsection{Time-Shifting Property}
\begin{definition}
If \(x[n] * h[n] = y[n]\), then for any integer shifts \(n_0, n_1\), the following holds:

\begin{enumerate}
    \item \(x[n - n_0] * h[n] = y[n - n_0]\)
    \item \(x[n] * h[n - n_1] = y[n - n_1]\)
    \item \(x[n - n_0] * h[n - n_1] = y[n - n_0 - n_1]\)
\end{enumerate}
\end{definition}

\begin{warning}
    \[
x[n - n_0] * h[n - n_0] = y[n - n_0 - n_0] = y[n - 2n_0]
\]
\end{warning}
    
\subsubsection{Scaling property}
\begin{definition}
For any scalar \(A\), convolution respects scaling:

\begin{equation}
A x[n] * h[n] = A y[n]
\end{equation}
\end{definition}
    
\subsubsection{Convolutional Identity}
\begin{definition}
The identity element for convolution is the delta function \(\delta[n]\), such that for all \(x[n]\):

\begin{equation}
x[n] * \delta[n] = \delta[n] * x[n] = x[n]
\end{equation}
    
\end{definition}

\begin{derivation}
    \begin{equation}
        \sum_k x[k] \delta[n - k] = x[n]
    \end{equation}
    
    \begin{itemize}
        \item \(\delta[n] * \delta[n] = \delta[n]\)
        \item \(\delta[n - n_0] * \delta[n] = \delta[n - n_0]\)
        \item \(\delta[n - n_0] * \delta[n - n_1] = \delta[n - n_0 - n_1]\)
    \end{itemize}
\end{derivation}

\begin{example}
\textbf{Convolution of Two Sequences}
\[
(a_0 \delta[n] + a_1 \delta[n-1] + a_2 \delta[n-2]) * (b_0 \delta[n] + b_1 \delta[n-1] + b_2 \delta[n-2])
\]

\[
= a_0 b_0 \delta[n] * \delta[n] + a_0 b_1 \delta[n] * \delta[n-1] + a_0 b_2 \delta[n] * \delta[n-2] 
\]
\[
+ a_1 b_0 \delta[n-1] * \delta[n] + a_1 b_1 \delta[n-1] * \delta[n-1] + a_1 b_2 \delta[n-1] * \delta[n-2]
\]
\[
+ a_2 b_0 \delta[n-2] * \delta[n] + a_2 b_1 \delta[n-2] * \delta[n-1] + a_2 b_2 \delta[n-2] * \delta[n-2] + \dots
\]
\end{example}